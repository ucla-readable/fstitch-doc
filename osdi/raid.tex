\section{RAID}
\label{sec:raid}

We implemented a RAID mirroring \module. The code is actually very
straightforward, and not very complicated. It attaches to two block
devices and presents a single block device. Either of the two
subordinate block devices can fail without the RAID \module\ reporting
error. Read requests are passed to both devices, alternating based on
the stride size. Write and sync requests are passed ideally to both
devices, but maybe to only one subordinate device if the other is
down.

The RAID \module\ also has the ability to attach a subordinate block
device and sync its contents so that it mirrors the existing
subordinate block device. In this way, RAID can be used to hot-swap
disks: start out with a RAID \module\ and only one subordinate block
device, attach a second subordinate disk, wait for synchronization to
complete, remove the original subordinate block device. A userspace
tool, provides the ability to force a subordinate device to go into
failed mode.
