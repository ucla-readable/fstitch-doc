\section{Implementation}
\label{sec:implementation}

\Kudos\ was originally implemented as a stand-alone file system server daemon
for a small operating system (called ``KudOS''), but now runs as either a
FUSE~\cite{fuse} file system server under Linux or BSD, or as a Linux kernel
module. The former is particularly useful for profiling and debugging, while the
latter is best for real-world performance testing and actual use. Additionally,
it supports \opgroups\ which are not supported by the FUSE version due to the
IPC that would be required to implement them.

The \Kudos\ Linux kernel module looks like a VFS file system to Linux, but
unlike traditional file system modules which are connected directly to block
devices, it is merely a front end to a \Kudos\ \module\ graph. The \Kudos\
server starts up a kernel thread which performs all the initialization to get
the \module\ graph set up, and \modules\ then can make file systems available to
mount though the VFS front end. Each file system is given a name which can be
used to mount it using the standard Linux \texttt{mount} program, using a
command like \mbox{\texttt{mount -t kfs kfs:\textit{name} /mnt/point}}.

There is a small change necessary to the standard Linux kernel in order to
provide support for \opgroups, however. As a kernel module, there is no way to
get notifications about when processes are created, or when they terminate. This
is necessary to clone the \opgroup\ scope (which is a lot like a file descriptor
table) from the parent process when the process is created, and to \abandon\ all
the \opgroups\ when it terminates. Ordinarily, this sort of state would be kept
as part of the Linux \texttt{task\_struct} (i.e. process) structure, but this is
not a viable option for a dynamically loadable kernel module. Instead, the
\Kudos\ module keeps this state internally, and it uses hooks we added to the
kernel to be notified when processes fork or exit. The patch to add these hooks
is based on the ``process events connector'' which was introduced in Linux
2.6.15, and is only about 250 lines long. Without it, \Kudos\ can still operate,
but it does not allow \opgroups.

At the other end of the \Kudos\ \module\ graph, the terminal block device
\modules\ are wrappers for Linux's normal block devices. By using Linux's
generic block layer, \Kudos\ can interface with any existing block device (e.g.
RAM disk, IDE disk, etc.). When the \module\ receives a read or write call, it
passes the request to Linux via \texttt{generic\_make\_request()}. Linux is
configured to use a `noop' elevator, meaning that the calls we make to
\texttt{generic\_make\_request()} are sent to the physical disk in the exact
order we make them. \Kudos\ takes advantage of DMA capabilities offered by
Linux to do disk reads and writes efficiently.

The \module\ also implements rudimentary read-ahead. When a read call needs to
be made to the physical device, the \module\ checks to see if any of the
following nine blocks are in memory already. If any of them are not, they are
also requested at the same time. After all blocks are read in, the read call is
complete.

Our interfaces bypass all Linux caches, because \Kudos\ provides its own
\chdesc-aware caches (\S\ref{sec:modules:wbcache}) in order to provide its
write-ordering guarantees.

In the current implementation, we are using Linux's queuing structures for
queuing I/O requests for block devices. Although this is working, there are a
couple subtle problems that we would like to solve in a future version. First of
all, while we can be sure that writes to a single block device are delivered in
our exact order, we cannot be sure that writes we make to multiple block devices
are made in the correct order relative to each other. Also, we would like to be
able to insert read and write requests to different sides of the I/O queue.
Right now a read requests will be satisfied only after all the pending writes
are completed, but a better queue design would allow reads to ``cut the line,''
since they are more urgent.
