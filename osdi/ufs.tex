% -*- mode: latex; tex-main-file: "paper" -*-

\subsection{UFS}
\label{sec:modules:ufs}

Kudos supports two base file system types, a simple inode-less
implementation called JOSFS and a version of UFS (Unix File System), the
modern incarnation of the Fast File System~\cite{mckusick84fast} used in
4.2 BSD. 
%
We describe UFS as an example of a complete and relatively complex file
system.

The general concepts for UFS, such as inodes, cylinder groups, and
indirect blocks, are well understood.  In \Kudos, we implemented the UFS1
file system as an LFS \module. UFS is of particular interest because it is
the only file system that has been extended with both soft updates and
journaling. ~\cite{seltzer00journaling} We chose UFS1 over UFS2, as UFS1 is
well established and more widely supported.

The UFS \module\ is implemented at the LFS interface. This keeps properties
specific to UFS (such as the UFS on-disk format and rules governing block
allocation) hidden within the file system \module. For instance, UFS uses
2KB \emph{fragments} to store small files efficiently. Once a file gets big
enough to require the use of indirect blocks, then UFS changes its allocation
policy and starts allocating 16KB \emph{blocks}, where a block is made up of 8
aligned and contiguous fragments. Our UFS \module\ implement this by using
fragments as the basic block size. For large files, our UFS \module\ internally
allocates a block, but returns only the first fragment in that block at the LFS
level. The next 7 allocation calls will be no-ops that simply return the
subsequent fragments in the allocated block. In this way, UFS can stay
consistent internally without special support from other \Kudos\ modules.

The UFS \module\ creates \chdescs\ for all its changes to the disk, and
connects them to form configurations that achieve soft updates consistency.
Unlike FreeBSD's soft updates implementation, once the dependencies have been
hooked up, UFS no longer needs to concern itself with the \chdescs\ it has
created. The block device subsystem will track and enforce the
\chdesc\ orderings.

In order to implement soft update ordering, we examined the places where UFS
created \chdescs. With the consistency rules for soft update in mind, we
carefully connected the \chdescs created within an LFS call. Assured that
all LFS calls in the UFS \module\ followed soft updates ordering, we then made
sure that the combinations of LFS calls do not violate soft updates
consistency.

\subsubsection {Modularity}
Although the UFS implementation must be somewhat monolithic in order to
understand the interlinked UFS-specific disk structures, we would like to take
advantage of modularity to allow for flexibility and encapsulation. UFS is
divided into modules at boundaries that expose naturally replaceable policies,
namely in resource allocation and directory entry traversal, as well as for
objects such as the superblock and cylinder groups. We use specialized
interfaces specific to UFS, because these interfaces do not apply to all file
systems in general.

\begin{figure}[htb]
  \centering
  \includegraphics[width=108pt]{fig/figures_4}
  \caption{\label{fig:ufsmodules} UFS and its submodules.}
\end{figure}

Within the UFS \module, the internal organization is shown in
Figure~\ref{fig:ufsmodules}. The \emph{base} module contains the core code for
the file system. Using object-oriented techniques, we encapsulated the code
for data structures in the file system, like the superblock and the cylinder
groups. We also recognized two locations where we can apply different search
algorithms. One is with respect to the allocation functions for blocks,
fragments, and inodes. The other is for searching and writing directory entries
(see Figure~\ref{fig:moduleinterface}). Currently we have two allocator modules
as a proof of concept for modularity.

\begin{figure}[htb]
Cylinder Group Interface:
\vspace{-0.5\baselineskip}
\begin{scriptsize}
\begin{alltt}
const UFS_cg_t * \textbf{read}(int num);
int \textbf{write_time}(int num, int value, chdesc_t ** head);
int \textbf{write_rotor}(int num, int value, chdesc_t ** head);
int \textbf{write}...
int \textbf{sync}(int num, chdesc_t ** head);
\end{alltt}
\end{scriptsize}

Allocator Interface:
\vspace{-0.5\baselineskip}
\begin{scriptsize}
\begin{alltt}
int \textbf{find_free_block}(fdesc_t * file, int purpose);
int \textbf{find_free_frag}(fdesc_t * file, int purpose);
int \textbf{find_free_inode}(fdesc_t * file, int purpose);
\end{alltt}
\end{scriptsize}
\vspace{-0.5\baselineskip}
\caption{\label{fig:moduleinterface} Example UFS Internal Module Interfaces}
\end{figure}

The UFS cylinder group module interface regulates access to the cylinder
groups. While there is unrestricted read access to cylinder groups, the
interface limits write access to certain fields within the \emph{UFS\_cg}
struct. This is because, under normal operations, fields like the number of
data blocks per cylinder group remain constant. The cylinder group module can
also define the policy for when changes to the cylinder group are written to
disk. It can, for example, make the policy ``write-through'' and have all
changes immediately go to disk. However, choosing this policy means that on
every block allocation, UFS needs to write \emph{cg\_rotor}, the position of
the last used block, to disk. To avoid performance hits like this, we
implemented the ``write-back'' policy instead. To support flushing dirty data
to disk, the cylinder group module interface also has a sync call. Similarly,
the super block module interface controls access to the super block.

The UFS allocator module interface has methods for allocating blocks,
fragments, and inodes. All three methods take in a \emph{file descriptor} and a
\emph{purpose} variable. The UFS \emph{file descriptor} contains all relevant
information pertaining to a given file, including an in-memory copy of the
file's inode. The intent of the \emph{purpose} variable is to provide hints to
the allocator about how the newly allocated resource will be used.  Given these
two pieces of information, as well as cylinder group information from the
previously mentioned module, UFS allocator modules can make informed decisions
to allocate resources in an efficient manner. For instance, given the status
for all cylinder groups, a block allocator function can examine the blocks
previously allocated to a given file and figure out what block to allocate
next based on the strategies used by FreeBSD.

\begin{figure}[htb]
  \centering
  \includegraphics[width=192pt]{fig/figures_6}
  \caption{\label{fig:chdescarrange} Change descriptor dependencies, when
  not strictly needed, restrict the possible choices for write ordering.
  This results in suboptimal write ordering and more scans through the
  \chdescs\ for \Kudos. On the left, \chdescs\ C1, C2, and C3 can be written
  in any order. Only one ordering is possible on the right.}
\end{figure}

\subsubsection {Change Descriptor Arrangements}
While optimizing our UFS implementation, we gained some insight and found a
couple practices critical to good performance. First, do not create unnecessary
dependencies between \chdescs. Doing so artificially limits the commit order
for \chdescs, which results in bad performance for several reasons. Not only
will unneeded dependencies force the disk to do more writes and seeks, but
\Kudos\ will have to scan through the \chdesc\ graph multiple times, since the
dependencies prevent the \chdescs\ from being flushed out to disk at once.

In Figure~\ref{fig:chdescarrange}, we have two possible arrangements for three
byte \chdescs. The noop \chdesc\ represents a root node that can reach all
other \chdescs. In the parallel arrangement on the left, \Kudos\ has the
freedom to write \chdescs\ $C_1$, $C_2$, and $C_3$ to disk in any order. All
three \chdescs\ can be marked writable with one graph traversal. In the serial
arrangement on the right, there exists only one valid write ordering. For
\Kudos\ to write this arrangement out to disk, it will have to scan through
the graph three times, since \chdesc\ $C_n$ cannot be marked as writable until
$C_{n-1}$ has been written.
An instance like this came up because our UFS implementation frequently writes
the \emph{cylinder group summaries} out to disk. By simply changing the
arrangement between three \chdescs\ created in a single function, UFS got a
33\% speed increase for several common file operations.

A corollary of this observation is to create \chdescs\ of the minimal size to
avoid accidental overlaps, which in turn, create unnecessary dependencies.
\Chdescs\ can represent changes to regions as small as one byte, and as large
as an entire block. Many times, it can be tedious for developers to calculate
exactly what parts in a large data structure have been modified and need to be
written to disk. As such, laziness will result in the creation of \chdescs\
for the entire data structure. Doing this can be detrimental to performance,
as shown in Figure~\ref{fig:overlap}.

An occurrence of this problem came up for inodes, where we make several
independent modifications to different fields within an inode. In principle,
the \chdescs\ created are conflict-free. Previously, because we did not take
the time to calculate the exact offsets and lengths for the fields that
changed, we just created a \chdesc\ for the entire inode every time we modified
any part of the inode. Thus all changes to any particular inode would always
overlap, causing unnecessary dependencies. Our solution to this problem is a
utility function called \texttt{chdesc\_create\_diff()}, which compares a
modified copy of a data structure to the original, and creates a minimal set of
\chdescs\ accordingly. Due to the frequent use of inodes, one simple use of
\texttt{chdesc\_create\_diff()} in the UFS inode functions reduced \chdesc\
graph traversal time significantly. This does introduce a slight
overhead, due to the need to \emph{diff} the two structures.

\begin{figure}[htb]
  \centering
  \includegraphics[width=64pt]{fig/figures_5}
  \caption{\label{fig:overlap} On inode 17, the gray regions represent
  modified fields that do not overlap. If \chdesc\ A and \chdesc\ B are
  exactly the size of the gray regions, then there is no implicit dependency.
  Making \chdescs\ for the entire inode data structure will, in turn, make
  one \chdesc\ depend on the other because they overlap.}
\end{figure}

