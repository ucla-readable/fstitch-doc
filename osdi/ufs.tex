\section{UFS implementation}
\label{sec:ufs}

\begin{itemize}
\item Motivation: Show KFS is capable of accommodating soft updates
  sematics by writing an implementation of UFS. (See \S\ref{sec:softupdate}.)
\item What is UFS / soft updates.
  \begin{itemize}
  \item UFS is the modern incarnation of the classic FFS used in BSD.
    The general concepts for UFS are well understood and implementations
    exist for many UNIX-like operating systems.
  \item Soft updates for UFS is one of the biggest innovations in file
    systems in terms of lowering the overhead required to provide file
    system consistency. By carefully ordering writes to disk, soft updates
    avoid the need for synchronous writes or duplicate writes to a journal.
  \end{itemize}
\item KFS UFS Implementation
  \begin{itemize}
  \item The LFS interface naturally divides the implementation into a set
    of simpler tasks. With an understanding of the UFS on-disk format and
    soft updates requirements, we added functionality to our UFS module
    in a progressive manner.
  \item Using change descriptors, we can control the ordering for writes
    to disk. This allow us to express the dependencies between different
    writes and achieve soft updates consistency.
  \item Applying our modularity philosophy to UFS, we refactored the file
    system module and separated many functions into sub-modules internal
    to UFS. Modularity makes it easy to exchange one piece of code for
    another. For example, we can change the inode allocation policy or
    the behavior when updating the superblock by swapping in different
    sub-module.
  \end{itemize}
\item Shortcomings
  \begin{itemize}
  \item Our UFS code is mostly complete, but some features are missing:
    i.e. handling triple indirect blocks and sparse files.
  \item Modularity means there exists places where we are less efficient
    than a more integrated implementation. These trouble spots present
    an opportunity for improvement.
  \end{itemize}
\end{itemize}
