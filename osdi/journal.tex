\section{Journaling}
\label{sec:journal}

Although change descriptors might initially seem to be specifically designed to
implement soft updates-like consistency semantics, they are in fact much more
flexible and can be used to implement journaling as well. What distinguishes
journaling from soft updates (from the point of view of change descriptors and
the write-back cache) is that with soft updates, change descriptors can always
be written to the disk to empty the cache, while journals can ``lock'' changes
into the cache while the transaction is in progress.

To allow this, it is possible to create a ``managed'' NO-OP change descriptor
that cannot be written at all, and other changes can be made to depend on it to
prevent them from being written. The component in the KFS system that created
the managed change descriptor can then later explicitly satisfy it, allowing all
the other change descriptors to be written. This is how the journal module can
set up dependencies to implement journaling: it makes all change descriptors
passing through it depend on a managed NO-OP change descriptor, and makes copies
of them to write to the journal. When the transaction is over, it can create a
commit record, set all the change descriptors in the transaction to depend on
it, and satisfy the managed NO-OP change descriptor.

A particularly nice property of this arrangement is that the journal component
is completely independent of any specific file system. It is a block device
component that automatically journals whatever file system is stored on it. Even
metadata-only journaling (as opposed to the full data journaling described
above) is possible, though we have not yet implemented it. The key to doing it
is to know which change descriptors are user data, and which are file system
metadata. This is already provided by the rest of the KFS system, and in
particular by the UHFS component, so the remaining work would be to allow user
data change descriptors to pass through the journal without being changed.
