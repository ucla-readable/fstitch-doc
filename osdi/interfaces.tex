\section {Module Interfaces}
\label{sec:interfaces}

A complete KFS configuration is composed of many components. By breaking file
system code into small, stackable components, we are able to significantly
increase code reuse. Between block devices and high-level file system
operations, we add an additional interface that helps to divide file system
implementations into common (reusable) code and file system-specific code. We
call this intermediate interface the ``Low-level File System'' (LFS). This new
interface is a departure from other stackable component systems, like
FiST~\cite{zadok00fist}, which stack higher-level operations.

The LFS interface has functions to allocate blocks, add blocks to files,
allocate file names, and other file system micro-ops. A component implementing
the LFS interface should define how bits are laid out on the disk, but doesn't
have to know how to combine the micro-ops into larger, more familiar file system
operations. A generic VFS-to-LFS component decomposes the larger file write,
read, append, and other standard operations into LFS micro-ops. This one
component can be used with many different LFS components implementing different
file systems.

\begin{figure}[tb]
  \centering
  \includegraphics[height=3in]{fig/figures_1}
  \caption{A running KFS configuration. {\it/} is a soft updated
    file system on an IDE drive; {\it/loop} is an externally journaled
    file system on loop devices.}
  \label{fig:kfs-graph}
\end{figure}

Figure~\ref{fig:kfs-graph} shows a contrived example taking advantage of the LFS
interface and change descriptors. A file system image is mounted with an
external journal, both of which are loop devices on the root file system, which
uses soft updates. The journaled file system's ordering requirements are sent
through the loop device as change descriptors, allowing dependency information
to be maintained across boundaries that might otherwise lose that information.
In contrast, without change descriptors and the ability to forward change
descriptors through loop devices, BSD cannot express soft updates' consistency
requirements through loop-back file systems.
