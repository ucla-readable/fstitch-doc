\subsection {\Module\ Interfaces}
\label{sec:design:interfaces}

A complete \Kudos\ configuration is composed of many \modules. By breaking file
system code into small, stackable \modules, we are able to significantly
increase code reuse. More importantly, new \modules\ are simple to write, and
can implement a variety of new features for the file system. Finally, by
changing the arrangement of the \modules, a broad range of behaviors can be
implemented -- and it is easy to tell what behavior a given arrangement will
give just by looking at the connections between the \modules.

% FIXME: cite other stackable file systems
% FIXME: state that we could do the ext2 block replication from SOSP 05 with LFS?

Stackable \module\ software for file systems is not a new idea (for instance, it
was proposed in the early 90s in \cite{rosenthal90evolving, heidemann91layered,
skinner93stacking, heidemann94filesystem}), but previous work like
FiST~\cite{zadok00fist} or GEOM~\cite{geom} generally focuses on a small portion
of the system and thus restrict both what a \module\ can do and how \modules\
can be arranged. FiST, for instance, does not provide a way to deal with
structures on the disk directly -- it provides only ``wrapper'' functionality
around existing file systems. GEOM, on the other hand, deals only with the block
device layer, and has no way to work with the file systems stored on those block
devices. Neither has a formal way of specifying or honoring complex
write-ordering information, which is what \chdescs\ in \Kudos\ provide.

\Kudos\ has three major types of \modules. First, there are block device
(``BD'') \modules, which have a fairly conventional interface. Next, there are
``common file system'' (``CFS'') \modules, which have an interface similar to
VFS~\cite{kleiman86vnodes}. In traditional systems, a CFS-like \module\ would be
connected directly to a BD-like \module\ in order to set up a file system to be
ready for mounting. In \Kudos, however, there is an intermediate interface which
can be used between block devices and the high-level CFS interface: the
``low-level file system'' (``LFS'') interface. LFS helps to divide file system
implementations into common (reusable) code and file system specific code. In
\Kudos, all three interfaces can be combined in many ways -- a departure from
other stackable \module\ systems, which generally allow stacking only a single
interface.

\begin{figure}[htb]
\vskip-14pt
\begin{tabular}{@{\hskip0.25in}p{2in}@{}}
\begin{scriptsize}
\begin{alltt}
int (*\textbf{get_root})(inode_t *inode);
uint32_t (*\textbf{allocate_block})(
    fdesc_t *file, int purpose,
    chdesc_t **head);
int (*\textbf{free_block})(
    fdesc_t *file, uint32_t block,
    chdesc_t **head);
bdesc_t *(*\textbf{lookup_block})(uint32_t number);
int (*\textbf{write_block})(
    bdesc_t *block, chdesc_t **head);
int (*\textbf{lookup_name})(
    inode_t parent, const char *name,
    inode_t *inode);
fdesc_t *(*\textbf{lookup_inode})(inode_t inode);
void (*\textbf{free_fdesc})(fdesc_t *fdesc);
uint32_t (*\textbf{get_file_numblocks})(fdesc_t *file);
uint32_t (*\textbf{get_file_block})(
    fdesc_t *file, uint32_t offset);
int (*\textbf{get_dirent})(
    fdesc_t *file, struct dirent *entry,
    uint16_t size, uint32_t *basep);
int (*\textbf{append_file_block})(
    fdesc_t *file, uint32_t block,
    chdesc_t **head);
uint32_t (*\textbf{truncate_file_block})(
    fdesc_t *file, chdesc_t **head);
fdesc_t *(*\textbf{allocate_name})(
    inode_t parent, const char *name,
    uint8_t type, fdesc_t *link,
    inode_t *newinode, chdesc_t **head);
int (*\textbf{remove_name})(
    inode_t parent, const char *name,
    chdesc_t **head);
int (*\textbf{rename})(
    inode_t oldparent, const char *oldname,
    inode_t newparent, const char *newname,
    chdesc_t **head);
int (*\textbf{get_metadata})(
    inode_t inode, uint32_t id,
    size_t size, void *data);
int (*\textbf{set_metadata})(
    inode_t inode, uint32_t id,
    size_t size, const void *data,
    chdesc_t **head);
\end{alltt}
\end{scriptsize}
\end{tabular}
\vspace{-10pt}
\caption{\label{fig:lfs} The LFS interface, simplified slightly.}
\end{figure}

The LFS interface (Figure~\ref{fig:lfs}\todo{Put inside a \texttt{struct LFS}?}) has functions to allocate blocks, add
blocks to files, allocate file names, and other file system micro-ops. A
\module\ implementing the LFS interface should define how bits are laid out on
the disk, but doesn't have to know how to combine the micro-ops into larger,
more familiar file system operations. A generic CFS-to-LFS \module\ decomposes
the larger file write, read, append, and other standard operations into LFS
micro-ops. This one \module, called the ``universal high-level file system''
or \emph{UHFS}, can be used with many different LFS \modules\ implementing
different file systems. (There is also a generic VFS-to-CFS layer, which has
two implementations: one for the Linux kernel module, and one for the
FUSE~\cite{fuse} version. See \S\ref{sec:implementation} for details.)

\begin{figure}[tb]
  \centering
  \includegraphics[height=3in]{fig/figures_1}
  \caption{A running \Kudos\ configuration. {\it/} is a soft updated
    file system on an IDE drive; {\it/loop} is an externally journaled
    file system on loop devices.}
  \label{fig:kfs-graph}
\end{figure}

Figure~\ref{fig:kfs-graph}\todo{Show multiple WB caches? Remove the mount selector?} shows a contrived example taking advantage of the LFS
interface and \chdescs. A file system image is mounted with an external journal,
both of which are loop devices on the root file system, which uses soft updates.
The journaled file system's ordering requirements are sent through the loop
device as \chdescs, allowing dependency information to be maintained across
boundaries that might otherwise lose that information. In contrast, without
\chdescs\ and the ability to forward \chdescs\ through loop devices, BSD cannot
express soft updates' consistency requirements through loop-back file systems.
