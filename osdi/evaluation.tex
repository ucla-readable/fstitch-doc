\section {Evaluation}
\label{sec:evaluation}

We will evaluate our prototype implementation of \Kudos\ in two ways: first, we
show that the overall performance is within reason, even though it is slower
than Linux or BSD by a nontrivial amount. We justify this difference, and argue
that it can be improved substantially. Second, we show specific block write
orderings in a variety of situations, demonstrating the effect and utility of
\opgroups\ and the potential for them to improve the efficiency of applications
using them.

\subsection {Overall Performance}

Our first benchmark test is to untar the Linux 2.6.15 source code, from
\texttt{linux-2.6.15.tar} which is already decompressed and cached in RAM. We
did this test on identical machines running Linux 2.6.12 (using ext3), FreeBSD
5.4 (using UFS, with and without soft updates), and \Kudos\ running as a Linux
2.6.12 kernel module (using UFS, with and without soft updates). As a second
test, we then delete the resulting source tree. The times below include time to
fully sync the changes to disk.

% make this into a table
\begin{figure}[htb]
%\centering
%\begin{tabular}{|l|r|r|} \\ \hline
System -- Untar time (sec) -- Delete time (sec) \\
Kudos (SU) -- 24.50 -- 11.69 \\ 
Kudos (no SU) -- -- \\ 
FreeBSD 5.4 (SU) -- 20.5 -- 4.7 \\ 
FreeBSD 5.4 (no SU) -- 17.75 -- 15.15 \\ 
% double line to separate linux
Linux 2.6.12 -- 12.90 -- 4.90 
%\end{tabular}
%\caption{\label{fig:macro} The macro foo}
\end{figure}

\Kudos\ is about 20\% slower than FreeBSD using soft updates. This shows
that, although it is slow, it is close to running in acceptable time. While
we do have CPU usage and I/O delay problems, we have only focused on
performance for a few weeks, in which we were able to speed up the system
enormously.  There are more optimization opportunities that can likely
bring us very close in performance to existing implementations of soft
updates (see \S\ref{sec:discussion}).

\subsection {Block Writes}
We have also looked at the number of block writes that \Kudos\ makes
relative to other systems for several operations. In one test, we create
100 small files in a directory and measure the number of blocks written to
disk. We do this with and without soft updates.

We also look at a very small benchmark: removing a file from a directory
and adding a new file to it. We compare the number of blocks written by
\Kudos\ with that of FreeBSD using soft updates.

\subsection {UW IMAP Case Study}
\label{sec:evaluation:uwimap}

Show fewer writes in UW IMAP using \opgroups\ (see \S\ref{sec:opgroup:uwimap})
