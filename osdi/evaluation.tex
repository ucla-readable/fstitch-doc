\section {Evaluation}
\label{sec:evaluation}

We will evaluate our prototype implementation of \Kudos\ in two ways: first, we
show that the overall performance is within reason, even though it is slower
than Linux or BSD by a nontrivial amount. We justify this difference, and argue
that it can be improved substantially. Second, we show specific block write
orderings in a variety of situations, demonstrating the effect and utility of
\opgroups\ and the potential for them to improve the efficiency of applications
using them.

\subsection {Overall Performance}

Our first benchmark test is to untar the Linux 2.6.15 source code, from
\texttt{linux-2.6.15.tar} which is already decompressed and cached in RAM. We
did this test on identical machines running Linux 2.6.12 (using ext3), FreeBSD
5.4 (using UFS, with and without soft updates), and \Kudos\ running as a Linux
2.6.12 kernel module (using UFS, with and without soft updates). As a second
test, we then delete the resulting source tree. The times below include time to
fully sync the changes to disk.

% make this into a table
Kudos (SU)
Kudos (no SU)
FreeBSD 5.4 (SU)
FreeBSD 5.4 (no SU)
Linux 2.6.12

\begin{itemize}
\item Show performance is not \emph{that} bad (macrobenchmarks)
\item Argue performance can be improved (see \S\ref{sec:discussion})
\end{itemize}

\subsection {Block Writes}
Demonstrate numbers of writes vs. BSD with and without soft updates.

\subsection {UW IMAP Case Study}
\label{sec:evaluation:uwimap}

Show fewer writes in UW IMAP using \opgroups\ (see \S\ref{sec:opgroup:uwimap})
