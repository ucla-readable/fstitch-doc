\section {Evaluation}
\label{sec:evaluation}

We will evaluate our prototype implementation of \Kudos\ in two ways: first, we
show that the overall performance is within reason, even though it is slower
than Linux or BSD by a nontrivial amount. We justify this difference, and argue
that it can be improved substantially. Second, we show specific block write
orderings in a variety of situations, demonstrating the effect and utility of
\opgroups\ and the potential for them to improve the efficiency of applications
using them.

\subsection {Overall Performance}

Our first benchmark test is to untar the Linux 2.6.15 source code, from
\texttt{linux-2.6.15.tar} which is already decompressed and cached in RAM. We
did this test on identical machines running Linux 2.6.12 (using ext3), FreeBSD
5.4 (using UFS, with and without soft updates), and \Kudos\ running as a Linux
2.6.12 kernel module (using UFS, with and without soft updates). As a second
test, we then delete the resulting source tree. The times below include time to
fully sync the changes to disk.

% make this into a table
\begin{figure}[htb]
%\centering
%\begin{tabular}{|l|r|r|} \\ \hline
System -- Untar time (sec) -- Delete time (sec) \\
Kudos (SU) -- 24.50 -- 11.69 \\ 
Kudos (no SU) -- -- \\ 
FreeBSD 5.4 (SU) -- 20.5 -- 4.7 \\ 
FreeBSD 5.4 (no SU) -- 17.75 -- 15.15 \\ 
Linux 2.6.12 -- 12.90 -- 4.90 
%\end{tabular}
%\caption{\label{fig:macro} The macro foo}
\end{figure}

\Kudos\ is about a factor of two slower than Linux. This shows that,
although it is slow, it is close to running in acceptable time. We
have only focused on performance for a couple weeks, in which we were
able to speed up the filesystem by literally many orders of
magnitude. There are more optimization opportunities that can likely
bring us acceptably close to Linux. (See \S\ref{sec:discussion}).

\subsection {Block Writes}
We've also looked at the number of block writes that \Kudos\ makes
relative to other filesystems. In one test, we create 100 small files
in a directory and measure the number of blocks written to disk. We do
this with and without soft updates.

We also look at a very small benchmark: removing a file from a
directory and adding a new file to it. We compare the number of blocks
written by soft updates and UFS and with \Kudos.

\subsection {UW IMAP Case Study}
\label{sec:evaluation:uwimap}

Show fewer writes in UW IMAP using \opgroups\ (see \S\ref{sec:opgroup:uwimap})
