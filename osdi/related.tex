\section{Related Work}
\label{sec:related}

\subsection{Stackable systems}

% \cite{webber93portable}

Stackable \module\ software for file systems is not a new idea. For instance, it
was proposed in the early 90s in \cite{rosenthal90evolving, heidemann91layered,
skinner93stacking, heidemann94filesystem}. Nevertheless, it remains an
interesting idea, and still attracts active research (\cite{zadok99extending,
zadok00fist}). However, previous systems like FiST~\cite{zadok00fist} or
GEOM~\cite{geom} generally focus on an individual portion of the system and
thus restrict both what a \module\ can do and how \modules\ can be arranged.
FiST, for instance, does not provide a way to deal with structures on the disk
directly -- it provides only ``wrapper'' functionality around existing file
systems. (Wrapfs~\cite{zadok99stackable, zadok99extending} is similar.) GEOM, on
the other hand, deals only with the block device layer, and has no way to work
with the file systems stored on those block devices. Neither has a formal way of
specifying or honoring complex write-ordering information, which is what
\chdescs\ in \Kudos\ provide. We imagine that systems like these could be
adapted to work with \chdescs, giving the benefits of both ideas.

\subsection{Consistency}

Soft updates~\cite{ganger00soft} is one of the biggest innovations in file
systems in terms of lowering the overhead required to provide file system
consistency. By carefully ordering writes to disk, soft updates avoids the need
for synchronous writes to disk or duplicate writes to a journal. Soft updates
also guarantees a strong level of consistency after a crash, enough so that the
system can avoid time-consuming file system consistency checks using a utility
like \emph{fsck}~\cite{mckusick94fsck}. \S\ref{sec:consistency:softupdate}
explains soft updates in greater detail.

Another approach to protecting the integrity of the file system is to write
upcoming operations to a journal first. The content and the layout of the
journal varies in each implementation, but in all cases, the system can use the
journal to play out or roll back the operations that did not complete as a
result of a crash. Thus, \emph{fsck} can be avoided by consulting the journal
when recovering from a crash. Section \ref{sec:consistency:journal} explains
journaling with \chdescs. For a comparison of journaling versus soft updates,
see ~\cite{seltzer00journaling}.

Customizable application-level consistency protocols have previously been
considered in the context of distributed, parallel file systems by
CAPFS~\cite{vilayannur05providing}. In such a system, enforcing an unnecessary
consistency protocol can be extremely expensive, and not providing the right
consistency protocol can cause unpredictable failures. However, this is also
true with a local file system -- and as a result, applications must use
expensive interfaces like \texttt{fsync()} when they require specific
consistency guarantees. \Kudos\ brings this sort of customizable consistency to
all applications, not just those using specialized distributed file systems.

% "allow the kernel to safely and efficiently handle any metadata layout without understanding the layout itself"
% \cite{kaashoek97application}

\subsection{Applications}

A variety of extensions to file systems have been proposed in recent work, like
the FS2 Free Space File System~\cite{...}, and encrypting file systems like
NCryptfs~\cite{wright01ncryptfs}. Although we have not implemented \Kudos\
\modules\ for these extensions, we believe that they would be not only possible
to implement but also not terribly difficult. NCryptfs has been written in a
stackable way already, allowing it to be easily adapted for new underlying file
systems -- but FS2 is currently specific to ext2, since it deals directly with
low-level disk structures. The \Kudos\ \module\ interface should allow such an
extension to be written in a portable way, giving it the same benefit.
