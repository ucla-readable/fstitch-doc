\subsection {Soft Updates}
\label{sec:consistency:softupdate}

Generally, a file system image is consistent if a program like \emph{fsck}
would report no errors -- that is, all the structures are in a completely
correct organization. It is not generally possible to maintain this invariant,
so in the ``soft updates''~\cite{ganger00soft} system this definition is
relaxed slightly by allowing some structures (like inodes or disk blocks) to be
marked as allocated even though there are no references to them. The key
observation is that this will not cause any harm beyond making the structures
unusable until they are discovered and reclaimed, which can be done safely in
the background while the file system is in use.

This relaxed consistency invariant can be implemented by carefully ordering the
writes to the disk. A simple set of rules governing the order for writes,
mostly consisting of the idea that a structure should never be marked free
while a reference to a structure still exists on the disk, accomplishes this
easily. In the FFS implementation of soft updates, each file system operation
is represented by a structure in memory which encapsulates the different
structural changes to the disk image necessary to implement that operation. As
a result, there exist many specialized data structures to represent the
different possible file system operations. The learning curve is high when it
comes to understanding these structures, their relationships to one another,
and their uses for tracking and enforcing dependencies.

In \Kudos, \chdescs\ are used to store all changes to the disk. So, rather than
a single structure which represents the whole file system operation, several
\chdescs\ are created -- one for each range of bytes on the disk which must be
changed. Then, these \chdescs\ can be connected to one another to specify the
order in which they must be written to disk. For instance, when a block is
removed from a file, we create (at least) two \chdescs\ in most file systems:
one that clears out the reference to that block number in the file's list of
blocks, and one that marks the block as free. By hooking up the second \chdesc\
to depend upon the first, we can implement the soft updates semantic
straightforwardly. Another example is depicted in Figure~\ref{fig:softupdate}.

\begin{figure}[htb]
  \centering
  \includegraphics[width=92pt]{fig/figures_3}
  \caption{\label{fig:softupdate} Soft updates \chdesc\ graph,
  including the dependencies for adding a newly allocated block to an
  inode. Writing the new block pointer to an inode (Attach) depends on
  initializing the block (Clear) and updating the free block map (Alloc).
  Updating the size of the inode (Size) depends on writing the block
  pointer.}
\end{figure}

The difference between soft updates and \Kudos\ is that soft updates tracks
these updates to the disk image at the level of file system operations, which
are specific to the file system in use (which, in practice, is FFS), while
\Kudos\ represents the changes in a file system independent way. Although it
uses more memory, the \Kudos\ approach allows the enforcement of the dependency
information to be separated from the specification. This makes the actual
implementation easier to read, and it also allows the dependency structure to be
examined and modified by other \modules\ of the system that may not have any
idea what the changes are actually doing in terms of the file system structures
they are changing.
