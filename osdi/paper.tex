\documentclass[10pt,twocolumn,letterpaper]{article}
\usepackage[nocopyright]{sigmin}
\usepackage{amsmath,amssymb,alltt,comment,times,mathptmx}
\usepackage[square,comma,numbers,sort&compress]{natbib}
\usepackage{multirow}
\frenchspacing

\usepackage{url,hyperref,ifpdf}
\ifpdf
  \usepackage[pdftex]{graphicx}
  \usepackage[monochrome]{color}
  \DeclareGraphicsExtensions{.jpg,.pdf,.mps,.png}
  \pdfinfo
  { 
    /Title (On The Key Advantages Of Modular File Systems Over Wedding Cakes)
    /Author (Andrew de los Reyes, Chris Frost, Mike Mammarella, Lei Zhang, {adlr,frost,mikem,leiz}@cs.ucla.edu)
  }
  \pdfcompresslevel=9
\else
  \usepackage{graphicx}
  \DeclareGraphicsExtensions{.eps,.jpg,.mps,.png}
\fi

% A list of stuff not to hyphenate
\hyphenation{KudOS}

%\renewcommand{\dbltopfraction}{1.00}
%\renewcommand{\topfraction}{1.00}
%\renewcommand{\textfraction}{0.10}

\pagestyle{empty}

%\renewcommand{\topfraction}{.8}
%\renewcommand{\bottomfraction}{.8}

\newcommand{\todo}[1]{\footnote{\textbf{TODO}: #1}}

\begin{document}

\normalsize

\title{\sffamily\textbf{On The Key Advantages Of Modular File Systems Over Wedding Cakes}}


\author{\sffamily Andrew de los Reyes, Chris Frost, Eddie Kohler, Mike
Mammarella, and Lei Zhang \\
\noalign{\vskip2pt}
\sffamily{\fontsize{10pt}{10pt}\selectfont University of California, Los Angeles} \\
\noalign{\vskip2pt}
\sffamily{\{adlr,frost,kohler,mikem,leiz\}@cs.ucla.edu} \\
\noalign{\vskip-.25in}
\null}
\date{}
\maketitle

\def\assast{\raise.2ex\hbox{$^\ast$}}

% -*- mode: latex; tex-main-file: "paper.tex" -*-

\begin{abstract}

%% The reliability of a file system to correctly store and later provide
%%  access to data is one of its most important properties.
%% %
%% File systems today deal with many challenges that make implementing this
%%  reliability difficult: power losses, software failures, and even user
%%  intervention all pose significant threats.
%% %
%% In order to avoid time-consuming and potentially ineffective manual
%%  checks like \emph{fsck} when recovering from failures, file systems use
%%  a variety of techniques.
%
Reliable storage systems depend in part on ``write-before''
 relationships, where some changes to stable storage are delayed until
 other changes commit.
%
A journaled file system, for example, must commit a
 journal transaction before applying that transaction's changes, and
 soft updates~\cite{ganger00soft} and other consistency enforcement
 mechanisms have similar constraints, implemented in each case in
 system-dependent ways.
%
%% These relationships are implemented in system-dependent ways.
%
We present a general abstraction, the \emph{\patch}, that makes write-before
 relationships
 explicit and file system agnostic.
%
A \patch-based file system implementation expresses dependencies among
 writes, leaving lower system layers to determine write orders
 that satisfy those dependencies.
%
Storage system \modules\ can examine and modify the dependency
 structure, and % the buffer cache writes blocks as constrained by that structure.
%
generalized file system dependencies are naturally exportable to
 user level.
%
Our patch-based storage system, \emph{\Featherstitch}, includes several
 important optimizations that reduce \patch\ overheads by orders of magnitude.
%
Our ext2 prototype runs in the Linux kernel and supports asynchronous
 writes, soft updates-like dependencies, and journaling.
%
It outperforms similarly reliable ext2 and ext3
 configurations on some, but not all, benchmarks.
%
It also supports unusual configurations, such as correct dependency
 enforcement within a loopback file system, and lets applications
 define consistency requirements without micromanaging how those
 requirements are satisfied.


\begin{comment}

File systems ensure that their data is kept consistent through careful
 write ordering, where certain disk blocks must be committed to stable
 storage before other blocks.
%
Previous file systems have enforced write orderings in system-dependent
 ways, either with rules specialized for each file system
 structure~\cite{ganger00soft} or with a journal, which enforces a
 particular consistency protocol.
%
We present a general \emph{\patch} abstraction that can represent any
 write ordering in a file system agnostic manner.
%
A \patch-based file system implementation expresses dependencies among
 writes, but does not enforce specific block write orders that satisfy
 those dependencies.
%
Storage system \modules\ can examine, preserve, and modify write
 orderings.
%
Generalized file system dependencies are naturally exportable to user
 level, allowing applications to specify their own consistency protocols
 for the storage system to follow.

We present the \patch\ abstraction, describe a number of important
 optimizations for \patch-based storage systems, and present a Linux kernel
 implementation of a storage subsystem that uses \patches\ to enforce
 consistency.
%
Our ext2 prototype is competitive with Linux ext2 and ext3 and allows
 several novel configurations, such as ext2 with soft updates or correct
 dependency enforcement within a loopback file system, and provides a
 simple interface for user applications to directly affect \patches.

\end{comment}


\begin{comment}

We propose a file system implementation architecture, called \emph{\Kudos},
where structures called \emph{\patches} represent any and all changes to
stable storage.
%
%%  File systems generate \patches\ for all writes, then
%% send them to block devices for eventual commit. Explicit dependencies between
%% \patches\ let \Kudos\ \modules\ preserve necessary file system
%% invariants without understanding the file system itself. \Patches\ can
%% implement many consistency mechanisms, including soft updates and journaling.
%
\Kudos\ is decomposed into fine-grained \modules\ which generate, consume,
 forward, and manipulate \patches.
%
The uniform abstraction of \patches\ allows modules to impose and
 follow arbitrary file system consistency policies: a collection of
 loosely-coupled modules cooperates to implement strong and possibly
 complex guarantees, even though each individual module does a relatively
 small part of the work.
%
%% A particular innovation of the
%% \module\ design is the separation of the low-level specification of on-disk
%% layout from higher-level file system-independent code, which operates on
%% abstract disk structures. 
%
For example, by observing and modifying \patch\ constraints, our
 journaling \module\ can automatically add journaling to any file system.
%
Additionally, a new system call interface gives applications some direct
 control over \patches. We have used this interface to
improve the UW IMAP server, removing inefficient and unnecessary calls to
\texttt{fsync()} while preserving the integrity of mail messages.
%
We have implemented \Kudos\ as a Linux kernel module. Our current
implementation is competitive with FreeBSD soft updates for number of
blocks written, and allows several novel configurations like ext2 with
soft updates or correct UFS soft updates over a loopback device.

\end{comment}

\end{abstract}


\section{Nuggets}

\begin{itemize}
\item soft updates related work
\item UHFS, LFS, CFS
\item chdesc patterns
\item write-back cache
\item RAID: low-level chdescs
\item cross-fs dependencies (cross-mount deps)
\item journal module
\item user-level chdesc iface (how we chose, guarantees)
\item UFS implementation
\item linux integration
\end{itemize}

\section{User-level Change Descriptors}
\begin{itemize}
\item Goal: Allow applications to specify ordering so that
  applications can implement custom consistency semantics and make
  custom performance requirements tradeoffs. It is desirable for the
  interface to fit in with existing calls naturally (eg write()) and
  the file system must be able to ensure userspace can not ``harm the
  system'' (eg non-satisfiable requirements).
\item interface
  \begin{itemize}
  \item Explain how we think applications might make use of opgroups
  \item We want to prevent apps from being able to screw the system up
    (grab huge amounts of memory that can't be released, stop other
    processes' data from going to disk, etc)
  \item Explain the ``core differences'' between chdescs and
    opgroups
  \item File systems operate at the block level whereas user-level
    programs (generally) operate at the file and file-offset level.
    So expose a higher level of change descritors. Also expose a
    higher level so that apps do not deal with file system
    implementation specifics; allows opgroups to better fit in with
    existing file system interactions (eg still use write(), don't
    create chdescs, create byte/bit differences, add deps). These are
    the reasons why why we chose to deal with groups of changes
    instead of individual changes.
  \item Opgroups are simpler than chdescs. Retaining not possible (app
    waits tomake write) and destruction not an issue (unless a
    transaction). Apply/rollback and dependency removal aren't seen as
    helpful since apps are not doing transformations, just specifying
    ordering requirements. (We should improve this reasoning.)
  \item We want to prevent cycles from complicating the standard fs
    calls; eg we don't want to apps to have to worry about write()
    causing a cycle and thus canceling the write (how would the app
    recover). Cycles from user data would are bad to have to worry
    about, cycles from internal-fs changes are horrible to have to
    worry about. Explain how we prevent cycles through cycle detection
    and allowed state transitions.
  \item Opgroups are designed to support different consistency
    requirements. ordering ``most fundamental'' (?). Also support
    atomic; more in the works.
  \item Explain why create and engage are distinct. Why disengage,
    release, and abandon are distinct. (Cycle prevention and for
    atomic/trancactional opgroups.) Give transitions for opgroups and
    atomic opgroups, or perhaps just a sampling, and the reasoning
    behind these.
  \end{itemize}
\item kernel-side implementation
  \begin{itemize}
  \item cycle detection
  \item chdesc implementation
  \item scopes and opgroups across forks (multithreaded todos?)
  \end{itemize}
\end{itemize}

\section{UFS implementation}
\begin{itemize}
\item Motivation: Show Kudos is capable of accommodating soft updates
  sematics by writing an implementation of UFS.
\item What is UFS / soft updates.
  \begin{itemize}
  \item UFS is the modern incarnation of the classic FFS used in BSD.
    The general concepts for UFS are well understood and implementations
    exist for many UNIX-like operating systems.
  \item Soft updates for UFS is one of the biggest innovations in file
    systems in terms of lowering the overhead required to provide file
    system consistency. By carefully ordering writes to disk, soft updates
    avoid the need for synchronous writes or duplicate writes to a journal.
  \end{itemize}
\item Kudos UFS Implementation
  \begin{itemize}
  \item The LFS interface naturally divides the implementation into a set
    of simpler tasks. With an understanding of the UFS on-disk format and
    soft updates requirements, we added functionality to our UFS module
    in a progressive manner.
  \item Using change descriptors, we can control the ordering for writes
    to disk. This allow us to express the dependencies between different
    writes and achieve soft updates consistency.
  \item Applying our modularity philosophy to UFS, we refactored the file
    system module and separated many functions into sub-modules internal
    to UFS. Modularity makes it easy to exchange one piece of code for
    another. For example, we can change the inode allocation policy or
    the behavior when updating the superblock by swapping in different
    sub-module.
  \end{itemize}
\item Shortcomings
  \begin{itemize}
  \item Our UFS code is mostly complete, but some features are missing:
    i.e. handling triple indirect blocks and sparse files.
  \item Modularity means there exists places where we are less efficient
    than a more integrated implementation. These trouble spots presents
    an opportunity for improvement.
  \end{itemize}
\end{itemize}

\bibliography{paper}
\bibliographystyle{plain}

\end{document}
