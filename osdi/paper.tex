\documentclass[10pt,twocolumn,letterpaper]{article}
\usepackage[nocopyright]{sigmin}
\usepackage{amsmath,amssymb,alltt,comment,times,mathptmx}
\usepackage[square,comma,numbers,sort&compress]{natbib}
\usepackage{multirow}
\frenchspacing

\usepackage{url,hyperref,ifpdf}
\ifpdf
  \usepackage[pdftex]{graphicx}
  \usepackage[monochrome]{color}
  \DeclareGraphicsExtensions{.jpg,.pdf,.mps,.png}
  \pdfinfo
  { 
    /Title (On The Key Advantages Of Modular File Systems Over Wedding Cakes)
    /Author (Andrew de los Reyes, Chris Frost, Mike Mammarella, Lei Zhang, {adlr,frost,mikem,leiz}@cs.ucla.edu)
  }
  \pdfcompresslevel=9
\else
  \usepackage{graphicx}
  \DeclareGraphicsExtensions{.eps,.jpg,.mps,.png}
\fi

% A list of stuff not to hyphenate
\hyphenation{KudOS}

%\renewcommand{\dbltopfraction}{1.00}
%\renewcommand{\topfraction}{1.00}
%\renewcommand{\textfraction}{0.10}

\pagestyle{empty}

%\renewcommand{\topfraction}{.8}
%\renewcommand{\bottomfraction}{.8}

\newcommand{\todo}[1]{\footnote{\textbf{TODO}: #1}}

\begin{document}

\normalsize

\title{\sffamily\textbf{On The Key Advantages Of Modular File Systems Over Wedding Cakes}}


\author{\sffamily Andrew de los Reyes, Chris Frost, Eddie Kohler, Mike
Mammarella, and Lei Zhang \\
\noalign{\vskip2pt}
\sffamily{\fontsize{10pt}{10pt}\selectfont University of California, Los Angeles} \\
\noalign{\vskip2pt}
\sffamily{\{adlr,frost,kohler,mikem,leiz\}@cs.ucla.edu} \\
\noalign{\vskip-.25in}
\null}
\date{}
\maketitle

\def\assast{\raise.2ex\hbox{$^\ast$}}

\documentclass[10pt,twocolumn]{article}
\usepackage[9pt]{sigmin}
\usepackage{graphicx,url,amsmath,amssymb,alltt,comment}
\usepackage[square,comma,numbers,sort&compress]{natbib}
\usepackage{multirow}
\usepackage{times,mathrmletter}
\frenchspacing

% A list of stuff not to hyphenate
\hyphenation{KudOS}

\renewcommand{\dbltopfraction}{1.00}
\renewcommand{\topfraction}{1.00}
\renewcommand{\textfraction}{0.10}

\pagestyle{empty}

%\renewcommand{\topfraction}{.8}
%\renewcommand{\bottomfraction}{.8}

% reduce enumeration spacing in a hacky way;
\newcommand{\itemvspace}{}%\vspace{-2mm}}
\newcommand{\postlistspacing}{}%\vspace{-0.08in}}

% reduce pre-pargraph spacing in a hacky way
\newcommand{\preparagraphspacing}{}%\vspace{-0.15in}}

%\usepackage{titlesec} % {pre,post}-[sub]section spacing adjustments
% how dare we use inches!
%\titlespacing{\section}    {0pt} {0.04in} {0.03in}
%\titlespacing{\subsection} {0pt} {0.10in} {0.03in}

\usepackage{url, hyperref, ifpdf}
\ifpdf
  \usepackage[pdftex]{graphicx}
  \usepackage[monochrome]{color}
  \DeclareGraphicsExtensions{.jpg,.pdf,.mps,.png}
  \pdfinfo
  { 
    /Title (The KudOS File Server Architecture)
    /Author (Andrew de los Reyes, Chris Frost, Mike Mammarella, Lei Zhang, {adlr,frost,mikem,leiz}@cs.ucla.edu)
  }
  \pdfcompresslevel=9
\else
  \usepackage{graphicx}
  \DeclareGraphicsExtensions{.eps,.jpg,.mps,.png}
\fi

\newcommand{\todo}[1]{\footnote{\textbf{TODO}: #1}}

\begin{document}

\title{\sffamily\textbf{Sympathy for the Sensor Network Debugger}}
%\title{A Debugging System for Sensor Networks}


\author{\sffamily Nithya Ramanathan, Kevin Chang, Rahul Kapur, Lewis Girod,
Eddie Kohler, and Deborah Estrin \\
\noalign{\vskip2pt}
\sffamily{\fontsize{10pt}{10pt}\selectfont UCLA Center for Embedded Network
Sensing} \\
\noalign{\vskip2pt}
\sffamily{\{nithya, kchang, rkapur, girod, kohler, destrin\}@cs.ucla.edu} \\
\noalign{\vskip-.25in}
\null}
\date{}
\maketitle

\begin{comment}
% this takes less space than \maketitle:
\begin{center}
{\LARGE The KudOS File Server Architecture}\\
\vspace{0.08in}
{\large Andrew de los Reyes,$^*$ Chris Frost,$^{*+}$ Eddie Kohler,$^+$ Mike Mammarella,$^{*+}$ Lei Zhang$^*$ }\\
{\large \{adlr,frost,kohler,mikem,leiz\}@cs.ucla.edu }\\
($^*$: students, $^+$: attending conference)
\end{center}
\end{comment}

\normalsize
\conferenceinfo{SenSys'05,} {November 2--4, 2005, San Diego, California, USA.}
\CopyrightYear{2005}
\crdata{1-59593-054-X/05/0011}

\section {Introduction}
\label{sec:intro}

Once upon a time, our research began as an attempt to decompose file system
software into small \modules, which would make the system as a whole more
configurable, extensible, and easier to understand -- as has been done in other
domains, like packet forwarding~\cite{kohler00click}. Upon examining existing
systems that accomplish similar tasks, and after some initial design for the new
system, it became clear that a key part of modern file system design is not
addressed by \modules\ alone: consistency. For robustness, stability, and
recovery speed, modern file system implementations must ensure that the file
system's stored image is kept consistent or easy to return to consistency.
Advanced consistency mechanisms such as soft updates~\cite{ganger00soft} and
journaling make this possible; unfortunately, they are generally tied to a
particular file system, and can't be ported or adapted without significant
engineering effort. Furthermore, existing module systems did not provide a way
to actually \emph{implement} (or even just \emph{change}) such consistency
mechanisms; the stacking is done either above or below where that part of the
system would need to be implemented.

These consistency mechanisms all hinge on one critical ability in the file
system software: the ability to place specific ordering requirements on writes
to the disk. What was needed, then, was a way to formalize these ordering
requirements in a file system independent way, so that all the \modules\ in our
system would be able to work with them. In this way, the loosely-coupled
\modules\ that make up the complete file system implementation would be able to
cooperate to implement strong and often complex consistency guarantees, even
though each individual \module\ does only a small and simple part of the work.

% We use "change descriptor" here on purpose. An abberviated form, if necessary,
% will only be introduced in the change descriptor section.

We propose a new file system implementation architecture, called \emph{\Kudos},
where \emph{change descriptor} structures represent any and all changes to
stable storage. File systems generate change descriptors for all writes, then
send them to block devices for eventual commit. Explicit dependencies between
change descriptors let \Kudos\ \modules\ preserve necessary file system
invariants without understanding the file system itself. Change descriptors can
implement many consistency mechanisms, including soft updates and journaling.
They can also be extended, in a carefully controlled way, into userspace --
enabling applications with custom consistency and performance requirements to
specify explicit write ordering restrictions to be honored by the file system.
We will show that this can give such applications several benefits over existing
interfaces like \texttt{fsync()} which provide only coarse control over
consistency, or which either impose high overhead (data journaling) or don't
guarantee data consistency (soft updates, for example, ensures metadata
consistency only).

\Kudos\ is decomposed into fine-grained \modules\ which generate, consume,
forward, and manipulate change descriptors. A particular innovation of the
\module\ design is the separation of the low-level specification of on-disk
layout from higher-level file system-independent code, which operates on
abstract disk structures. Our journaling \module\ can automatically add
journaling to any file system, and combinations of simple \modules\ can support,
for example, correct consistency on RAID over loop-back devices.

% -*- mode: latex; tex-main-file: "paper.tex" -*-

%% \subsection {Interfaces}
\label{sec:modules:interfaces}

\begin{comment}
New \modules\ are
simple to write, and by changing the \module\ arrangement, a broad range of
behaviors can be implemented. It's also easy to tell what behavior a given
arrangement will give just by looking at the connections between the \modules.
\end{comment}

A complete \Kudos\ configuration is composed of many \modules, making it
a finer-grained variant of a stackable file system.
%
There are has three major types of \modules.
%
Closest to the disk are block device (BD) \modules, which have a fairly
conventional block device interface with interfaces such as ``read block'' and
``flush''. 
%
Closest to the system call interface are \emph{common file system} (CFS)
\modules, which have an interface similar to VFS~\cite{kleiman86vnodes}. 
%
\Kudos\ also supports an intermediate interface between BD and CFS.
%
This \emph{low-level file system} (\LFS) interface helps divide file system
implementations into common (reusable) code and file system specific code. 
%
\begin{comment}
A
\Kudos\ file system designer combines modules with all three interfaces in many
ways -- a departure from stackable file systems, which act only at the VFS/CFS
layer. \Kudos\ \modules\ are implemented in C using structures of function
pointers to achieve object oriented behavior, very much like the rest of the
Linux kernel.
\end{comment}
%
The \LFS\ interface has functions to allocate blocks, add blocks to files,
allocate file names, and other file system micro-operations. A \module\ implementing
the \LFS\ interface should define how bits are laid out on the disk, but doesn't
have to know how to combine the micro-operations into larger, more familiar file system
operations. A generic CFS-to-\LFS\ \module\ decomposes the larger file write,
read, append, and other standard operations into \LFS\ micro-operations. This module,
called UHFS, is described in the next section.

Each \chdesc\ on a cached block may or may not be visible to a given \module.
For example, \modules\ that respond to user requests generally view the most
current state of every block -- the block with all \chdescs\ applied. However, a
write-back cache may choose to write some \chdescs\ on a block while reverting
others, since those others currently have outstanding dependencies. In this
case, \modules\ below the write-back cache (i.e. closer to the disk) should view
those \chdescs\ in the reverted state. \Kudos\ provides a block revisioning
library function that automatically rolls back those \chdescs\ that should not
be visible at a particular \module, and then rolls them forward again after that
\module\ is done with the block.

\section{Change Descriptors}
\label{sec:chdescs}

In contrast with traditional systems, where changes to filesystem data are
accomplished by changing in-memory copies of the disk blocks and then marking
the block as needing to be written to disk (thus losing all record of what was
changed), every change to filesystem data in the KudOS file server effects a
corresponding ``change descriptor'' that describes what was changed. Change
descriptors (``chdescs'' for short) allow individual changes to be undone and
redone, as well as allowing a dependency graph to be created describing which
changes ``depend'' on which other changes (i.e. which changes must be written to
disk after which other changes).

The ability to revert and re-apply chdescs is inspired by the ``soft updates''
system in BSD's FFS~\cite{ganger00soft}, but it is much more generalized. A
chdesc can describe a change as small as a single bit, or as large as a whole
disk block. Block data is almost secondary to the chdescs from the point of view
of the file system server --- chdescs are what really move around in the system.
This concept turns out to be very powerful, and it allows some interesting
configurations of modules as well as very simple implementations of
traditionally complicated features. For instance, journalling is implemented as
a single BD module, and it can automatically add journalling --- even
metadata-only journalling --- to {\it any} LFS filesystem. Other block device
layering systems, like GEOM~\cite{geom}, need special hooks into filesystem code
in order to get the necessary hints (i.e. what is metadata and what is not) to
do metadata-only journalling. Change descriptors and the LFS/CFS division allow
us to do this automatically.

For many complex operations, like RAID or journalling, the chdesc graph is
changed in specific ways which can be broken down into sequences of simple graph
transformations. For example, in the (mirroring) RAID module it is necessary to
duplicate a change descriptor, including all its dependents and dependencies.
For this purpose, we provide a library of such transformations, so that modules
needing to perform such operations can be written more quickly and with fewer
bugs than if they had to perform the transformations themselves.

\begin{figure}[tb]
  \centering
  \includegraphics[height=3in]{fig/whatevs_1}
%  \vspace{-6pt}
  \caption{A running KFS configuration. {\it/} is a soft updated
    file system on an IDE drive; {\it/loop} is an externally journalled
    file system on loop devices.}
  \label{fig:kfs-graph}
\end{figure}
%\vspace{-10pt}

\section {Future Work}
\label{sec:future}

There are several areas in which we would like to expand our work. The obvious
first area we would like to work on is the performance of the KFS system. We
have already improved the performance by literally several orders of magnitude,
since we only recently began examining performance in addition to correctness,
but the system is not as fast as it could be -- and not quite as fast as it
needs to be to be a viable option for most computer systems.

There are several ways in which the performance can be improved. First, we
create a very large number of change descriptors, and the sheer number of them
can cause problems for any of our algorithms which needs to traverse parts of
the change descriptor dependency graph. We can attack this problem from two
sides: we can reduce the number of change descriptors by intelligently merging
them when we determine that having separate change descriptors is not necessary,
or we can improve the efficiency of the traversal algorithms to need to examine
fewer change descriptors. We have worked on both of these approaches, especially
the second, but we believe further improvements can be made.

Another item we would like to add is an ext2 file system component. Currently,
we have written only a UFS component, aside from a component for a very simple
prototype file system called JOSFS. Once we have an ext2 component, we could use
our journal module to get ``ext3b''. Implementing an ext2 component would also
help us to identify more opportunities for sharing code between file system
implementations, and ways in which our interfaces might accomodate different
file systems more neutrally.

\begin{itemize}
\item Fast cycle checking
\item Better cache eviction algorithm
\item Transactional user-level change descriptors
\item More flexible rules for user-level change descriptors
\item Metadata-only journaling
\item More references
\end{itemize}


\bibliography{abstract}
\bibliographystyle{plain}

\end{document}


\section{Nuggets}

\begin{itemize}
\item soft updates related work
\item UHFS, LFS, CFS
\item chdesc patterns
\item write-back cache
\item RAID: low-level chdescs
\item cross-fs dependencies (cross-mount deps)
\item journal module
\item user-level chdesc iface (how we chose, guarantees)
\item UFS implementation
\item linux integration
\end{itemize}

\section{User-level Change Descriptors}
\begin{itemize}
\item Goal: Allow applications to specify ordering so that
  applications can implement custom consistency semantics and make
  custom performance requirements tradeoffs. It is desirable for the
  interface to fit in with existing calls naturally (eg write()) and
  the file system must be able to ensure userspace can not ``harm the
  system'' (eg non-satisfiable requirements).
\item interface
  \begin{itemize}
  \item Explain how we think applications might make use of opgroups
  \item We want to prevent apps from being able to screw the system up
    (grab huge amounts of memory that can't be released, stop other
    processes' data from going to disk, etc)
  \item Explain the ``core differences'' between chdescs and
    opgroups
  \item File systems operate at the block level whereas user-level
    programs (generally) operate at the file and file-offset level.
    So expose a higher level of change descritors. Also expose a
    higher level so that apps do not deal with file system
    implementation specifics; allows opgroups to better fit in with
    existing file system interactions (eg still use write(), don't
    create chdescs, create byte/bit differences, add deps). These are
    the reasons why why we chose to deal with groups of changes
    instead of individual changes.
  \item Opgroups are simpler than chdescs. Retaining not possible (app
    waits tomake write) and destruction not an issue (unless a
    transaction). Apply/rollback and dependency removal aren't seen as
    helpful since apps are not doing transformations, just specifying
    ordering requirements. (We should improve this reasoning.)
  \item We want to prevent cycles from complicating the standard fs
    calls; eg we don't want to apps to have to worry about write()
    causing a cycle and thus canceling the write (how would the app
    recover). Cycles from user data would are bad to have to worry
    about, cycles from internal-fs changes are horrible to have to
    worry about. Explain how we prevent cycles through cycle detection
    and allowed state transitions.
  \item Opgroups are designed to support different consistency
    requirements. ordering ``most fundamental'' (?). Also support
    atomic; more in the works.
  \item Explain why create and engage are distinct. Why disengage,
    release, and abandon are distinct. (Cycle prevention and for
    atomic/trancactional opgroups.) Give transitions for opgroups and
    atomic opgroups, or perhaps just a sampling, and the reasoning
    behind these.
  \end{itemize}
\item kernel-side implementation
  \begin{itemize}
  \item cycle detection
  \item chdesc implementation
  \item scopes and opgroups across forks (multithreaded todos?)
  \end{itemize}
\end{itemize}

\bibliography{paper}
\bibliographystyle{plain}

\end{document}
