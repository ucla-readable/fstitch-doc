\section {Change Descriptors}
\label{sec:chdescs}

Each in-memory modification to a cached disk block in the KFS system has an
associated change descriptor. Different change types correspond to different
forms of change descriptors; the change descriptor for a flipped bit -- such as
in a free-block bitmap -- contains an offset and mask, while larger changes
contain an offset, a length, and the new data. The change descriptor's
dependencies point to other change descriptors that must precede it to stable
storage. A change descriptor can be applied or reverted to switch the cached
block's state between old and new as necessary. Each change descriptor applies
to exactly one block.

Figure~\ref{fig:chdesc} gives a simplified version of the structure. The ability
to revert and re-apply change descriptors is inspired by the soft updates system
in BSD's FFS~\cite{ganger00soft}, but generalized so that it is not specific to
any particular file system.

\begin{figure}
\vskip-14pt
\begin{tabular}{@{\hskip0.58in}p{2in}@{}}
\begin{scriptsize}
\begin{verbatim}
struct chdesc {
    BD_t *device;
    bdesc_t *block;
    enum {BIT, BYTE, NOOP} type;
    union {
        struct {
            uint16_t offset;
            uint32_t xor;
        } bit;
        struct {
            uint16_t offset, length;
            uint8_t *data;
        } byte;
    };
    struct chdesc *dependencies[];
/* ... */ };
\end{verbatim}
\end{scriptsize}
\end{tabular}
\vspace{-10pt}
\caption{\label{fig:chdesc} Partial change descriptor structure.}
\end{figure}

When a KFS component first generates change descriptors to write to the disk, it
specifies write ordering requirements similar to those of soft updates. For
example, Figure~\ref{fig:softupdate} depicts change descriptors that allocate
and add a new block to an inode. The component passes these change descriptors
to another component closer to the disk. This second component can inspect,
delay, and even modify them before passing them on further. For instance, the
write-back cache component (\S\ref{sec:wbcache}) holds on to blocks and their
change descriptors instead of forwarding them immediately. When evicting a block
and associated change descriptors, the write-back cache enforces an order
consistent with the change descriptor dependency information.

Soft updates, journaling, and many application-specific consistency models all
correspond to different change descriptor arrangements, so these features can be
added to the system as components which appropriately connect or reconnect the
change descriptors. For example, the change descriptors in
Figure~\ref{fig:softupdate} can be transformed to provide journaling semantics.
The original four change descriptors are modified to depend on a journal commit
record, which depends on blocks journaling the changes. Once the actual changes
commit, the journal record is marked as completed. Figure~\ref{fig:journal}
shows these transformed change descriptors. This single journaling component
(see section \ref{sec:journal}) can attach to any file system component; it
performs transformations incrementally as change descriptors arrive.

Further, by changing our journal component to journal only change descriptors
that modify file system metadata -- and by adding additional dependencies to
prevent premature reuse of blocks -- we could even obtain metadata-only
journaling. The journal component can distinguish metadata change descriptors
because of the LFS interface described elsewhere. Other block device layering
systems, like GEOM~\cite{geom} or JBD in Linux, would or do need special hooks
into file system code to determine what disk changes represent metadata in order
to do metadata-only journaling. Change descriptors and the LFS interface allow
us to do this automatically.
