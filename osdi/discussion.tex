\section {Discussion}
\label{sec:discussion}

There are several areas in which we would like to expand our work. The obvious
first area we would like to work on is the performance of \Kudos. We have
already improved the performance by literally several orders of magnitude, since
we only recently began examining performance in addition to correctness, but the
system is not as fast as it could be -- and not quite as fast as it needs to be
to be a viable option for most computer systems.

There are several ways in which the performance can be improved. First, we
create a very large number of \chdescs, and the sheer number of them can cause
problems for any of our algorithms which needs to traverse parts of the \chdesc\
dependency graph. We can attack this problem from two sides: we can reduce the
number of \chdescs\ by intelligently merging them when we determine that having
separate \chdescs\ is not necessary, or we can improve the efficiency of the
traversal algorithms to need to examine fewer \chdescs. We have worked on both
of these approaches, especially the second, but we believe further improvements
can be made.

Another item we would like to add is an ext2 file system \module. Currently,
we have only written one production quality file system \module. The other
file system \module\ in \Kudos\ is for a very simple prototype file system
called JOSFS. Once we have an ext2 \module, we could use our journal module to
get ``ext3b''. Implementing an ext2 \module\ would also help us to identify
more opportunities for sharing code between file system implementations, and
ways in which our interfaces might accomodate different file systems more
neutrally.

To prevent the system from going into an infinite loop, it is important for
\chdescs\ to never form cycles. Although it is never our intent, cycles can
form as a result of implementation bugs. Currently we can do cycle checking,
but it is slow and not scalable as the \chdesc\ graph grows in size. We want to
delve into graph algorthms and find faster ways to check for cycles. Ideally,
the overhead of cycle checking will be low enough that we can afford to leave
it on all the time.

\begin{itemize}
\item Better cache eviction algorithm
\item Transactional user-level \chdescs
\item More flexible rules for user-level \chdescs
\item Metadata-only journaling
\item More references
\end{itemize}
