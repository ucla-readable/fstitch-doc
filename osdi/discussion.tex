\section {Discussion}
\label{sec:discussion}

There are several areas in which we would like to improve our work. The obvious
first area we would like to work on is the performance of \Kudos. We have
already improved the performance by literally several orders of magnitude, since
we only recently began examining performance in addition to correctness, but the
system is not as fast as it could be -- and not quite as fast as it needs to be
to be a viable option for most computer systems.

We would also like to identify places where \Kudos\ can be made more flexible,
by adding more features to the system and changing the existing parts that make
these features difficult to add. For instance, we would like to find more
applications which can take advantage of \opgroups, and see what changes might
need to be made to the \opgroup\ interface in order to facilitate adapting those
applications.

\subsection {Optimizations}
There are several ways in which the performance can be improved. For instance,
we create a very large number of \chdescs, and the sheer number of them can
cause problems for any algorithm which needs to traverse parts of the \chdesc\
dependency graph. We can attack this problem in two ways: first, we can reduce
the number of \chdescs\ by intelligently merging them when we determine that
having separate \chdescs\ is not necessary. For example, if the file system
creates a file and immediately deletes it, we can take the two opposing
\chdescs\ sets and coalesce them into a \noop.

As another strategy, we can improve the efficiency of the traversal algorithms
to examine fewer \chdescs. Many of the operations that \modules\ like the
write-back cache need to do involve quickly discovering the set of \chdescs\
that satisfy some property, and often even some ordering of those \chdescs. In
early prototypes, these sets were often calculated by doing large traversals of
the \chdesc\ dependency graph. However, these traversals are in practice
extremely expensive, and optimizating them has improved and will likely continue
to improve the performance of \Kudos.

We have worked on both of these approaches, especially the second, but we
believe further improvements can be made. One simple change which helped
enormously was hidden in the order that \chdescs\ were listed as dependencies of
other \chdescs. We had originally used a simple linked list for this structure,
and treated it like a stack for ease of insertion to the list. It turned out
that a much better way to use the linked list was to treat the list like a
queue, because almost always, elements being removed from the list were at the
end of the list. (Since they had been added longest ago, and often changes are
written in an order similar to the order in which they were made.) This
converted a linear time search into one which would usually execute in constant
time.

Another example is that when examining a block and deciding which \chdescs\ must
be rolled back in order for the block's data to be safe to write to the disk,
the dependency graph between the various changes on that block must be taken
into account so that changes which depend on one another are rolled back in
proper dependency order. Once again, we discovered that we were examining the
\chdescs\ in almost exactly the opposite order as the order they should be
rolled back in -- for the same reason as the previous example. By simply
changing the direction of the loop, we were able to improve the loop from
quadratic time to linear time.

We believe that there are still many places \chdesc\ operations can be optimized
to improve the performance of \Kudos. Many of these places likely will have the
same flavor as the examples here: information we already have can be used in
order to determine the answers we need, without needing to directly calculate
those answers. Even as it is, however, \Kudos\ does not run dramatically slower
than a standard Linux 2.6 kernel.

% FIXME: better cache eviction algorithm

\subsection {Features}
Currently, we have only written one production quality file system \module\
(UFS). The other file system \module\ in \Kudos\ is for a very simple prototype
file system. We would like to add an ext2 file system \module. Once we have such
a \module, we could use our journal \module\ to get ``ext3b''. Implementing an
ext2 \module\ would also help us to identify more opportunities for sharing code
between file system implementations, and ways in which our interfaces might
accomodate different file systems more neutrally.

For the \Kudos\ UFS module, we want to add more features to make the
implementation complete. Currently, we do not support symbolic links, triple
indirect blocks, or sparse files. The first two can easily be implemented
given time. Sparse file support will require the LFS interface to be aware of
files with holes. This also means the layer immediately above the LFS interface
may need to adjust some of the assumptions it currently makes about files and
block allocation.

We have not taken advantage of the UFS \module's infrastructure for supporting
more sophisticated allocation algorithms. There are auxiliary on-disk data
structures that assist resource allocation algorithms. We currently ignore many
of them, partially because our basic algorithms do not use them, and partially
due to the lack of documentation to explain the purpose of some data
structures. Over time, we hope to better understand UFS and improve our UFS
implementation.

To prevent the system from going into an infinite loop, it is important that
\chdescs\ never form cycles. Although it is never our intent, cycles can form
as a result of implementation bugs. Currently we can do cycle checking, but it
is slow and not scalable as the \chdesc\ graph grows in size. We want to delve
into graph algorthms and find faster ways to check for cycles. Ideally, the
overhead of cycle checking will be low enough that we can afford to leave it
enabled all the time.
