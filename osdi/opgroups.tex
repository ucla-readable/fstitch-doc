\section{User-level Change Descriptors}
\label{opgroup}

\begin{itemize}
\item Goal: Allow applications to specify ordering so that
  applications can implement custom consistency semantics and make
  custom performance requirements tradeoffs. It is desirable for the
  interface to fit in with existing calls naturally (eg write()) and
  the file system must be able to ensure userspace can not ``harm the
  system'' (eg non-satisfiable requirements).
\item interface
  \begin{itemize}
  \item Explain how we think applications might make use of opgroups
  \item We want to prevent apps from being able to screw the system up
    (grab huge amounts of memory that can't be released, stop other
    processes' data from going to disk, etc)
  \item Explain the ``core differences'' between chdescs and
    opgroups
  \item File systems operate at the block level whereas user-level
    programs (generally) operate at the file and file-offset level.
    So expose a higher level of change descritors. Also expose a
    higher level so that apps do not deal with file system
    implementation specifics; allows opgroups to better fit in with
    existing file system interactions (eg still use write(), don't
    create chdescs, create byte/bit differences, add deps). These are
    the reasons why why we chose to deal with groups of changes
    instead of individual changes.
  \item Opgroups are simpler than chdescs. Retaining not possible (app
    waits tomake write) and destruction not an issue (unless a
    transaction). Apply/rollback and dependency removal aren't seen as
    helpful since apps are not doing transformations, just specifying
    ordering requirements. (We should improve this reasoning.)
  \item We want to prevent cycles from complicating the standard fs
    calls; eg we don't want to apps to have to worry about write()
    causing a cycle and thus canceling the write (how would the app
    recover). Cycles from user data would are bad to have to worry
    about, cycles from internal-fs changes are horrible to have to
    worry about. Explain how we prevent cycles through cycle detection
    and allowed state transitions.
  \item Opgroups are designed to support different consistency
    requirements. ordering ``most fundamental'' (?). Also support
    atomic; more in the works.
  \item Explain why create and engage are distinct. Why disengage,
    release, and abandon are distinct. (Cycle prevention and for
    atomic/trancactional opgroups.) Give transitions for opgroups and
    atomic opgroups, or perhaps just a sampling, and the reasoning
    behind these.
  \end{itemize}
\item kernel-side implementation
  \begin{itemize}
  \item cycle detection
  \item chdesc implementation
  \item scopes and opgroups across forks (multithreaded todos?)
  \end{itemize}
\end{itemize}
