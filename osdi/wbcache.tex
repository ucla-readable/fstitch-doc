\subsection{Write-Back Cache and Block Revisioning}
\label{sec:modules:wbcache}

The write-back cache is in some sense the \module\ that does all the real work
in \Kudos. There can be many write-back caches in a configuration at once, but
each is responsible for holding on to the \chdescs\ sent to it by connected
\modules\ until they can be safely sent on towards the disk. To a write-back
cache, the complex consistency protocols that other \modules\ want to enforce
are nothing more than sets of dependencies among \chdescs\ -- it has no idea
what consistency protocol (or protocols) it is implementing, if any at all. Yet
it is the \module\ that ends up doing most of the work to make sure that
\chdescs\ are written in an acceptable order.

Even though a write-back cache has such an important and central role in the
system, there's not a lot to it. The current write-back cache is a simple LRU
cache, but with an important twist: blocks can't be evicted unless all the
\chdescs\ on them are ``ready'' to be sent to the next \module\ (for example,
the disk). So, when looking for a block to evict, the cache may not be able to
evict any block it chooses -- but it evicts the least recently used block that
it can.

\subsubsection{Block Cycles}
\label{sec:modules:wbcache:cycles}

Just as with soft updates~\cite{ganger00soft}, the dependencies among \chdescs\
(or just ``updates to the block'' in soft updates) can create cyclic
dependencies among blocks, even though the \chdescs\ themselves do not form a
cycle. To handle this case, some \chdescs\ may need to be ``held back'' in order
to write the others, allowing such cycles to be broken. To effect this behavior,
the write-back cache just holds on to the \chdescs\ that cannot yet be written,
but forwards the others on to the next \module\ as it writes the block. It
cannot evict the block yet, since it is still ``dirty,'' but progress has been
made that can make other blocks evictable.

\subsubsection{Block Revisioning}

Since many \modules\ may be stacked on top of one another in \Kudos,
and since many of them may want to refer to the same block at the same
time, only one copy of the data for each block is kept in memory at a
time. However, different \modules\ may ``know about'' different sets
of \chdescs. For example, in the case outlined in
\S\ref{sec:modules:wbcache:cycles}, the \module\ below the write-back
cache will know about some of the \chdescs\ on a block but not others
(the ones which are not ``ready'' to be sent to it yet). If this
\module\ is the disk, it will need to be able to write a version of
the block's data that does not include the \chdescs\ it does't know
about yet. \Kudos\ provides a revisioning system for blocks which can
automatically ``roll back'' the \chdescs\ which have not yet reached a
particular \module, and then roll them forward again after that
\module\ is done using the previous version of the block's data
(e.g. to write it to disk).

\subsubsection{Cross-device dependencies}

The write-back cache has one other property that makes it useful in \Kudos: it
respects dependencies between one cache and another, so that (for instance)
dependencies between the changes on a file system and its external journal are
properly respected. This is actually not a special case, nor does it require any
extra code -- it is a property that just falls out of the way \chdesc\ graphs
are processed in order to determine which \chdescs\ are ready to move towards
the disk. This property of write-back caches also extends to \opgroups, which
are explained in \S\ref{sec:opgroup}.
