\section{Write-Back Cache}

The write-back cache is in some sense the component that does all the real work
in a KFS system. There can be many write-back caches in a configuration at once,
but each is responsible for holding on to the change descriptors generated by
components connected to it until they can be safely sent towards the disk. To a
write-back cache, the complex consistency protocols that other modules want to
enforce are nothing more than sets of dependencies among change descriptors --
it has no idea what consistency protocol (or protocols) it is implementing, if
any at all. Yet it is what ends up doing most of the work to make sure that
change descriptors are written in an acceptable order.

Even though a write-back cache has such an important and central role in the
system, there's not a lot to it. (explain why not)...
