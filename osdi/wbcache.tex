\subsection{Write-Back Cache}
\label{sec:modules:wbcache}

Write-back cache modules act as the system's primary cache, and are
responsible for holding on to the \chdescs\ sent to them until they can be
safely sent on towards the disk.
%
%
There can be many write-back caches in a configuration at once (for
instance, one for each block device).
%
%
To a write-back cache, the complex consistency protocols that
other \modules\ want to enforce are nothing more than sets of dependencies among
\chdescs\ -- it has no idea what consistency protocols it is implementing, if
any at all.
%
Yet it is the \module\ that ends up doing most of the work to make
sure that \chdescs\ are written in an acceptable order.

Despite this, write-back caches are relatively simple
-- only about 550 lines of code, including
comments. The current write-back cache uses a simple LRU policy, except that
blocks can't be evicted unless all the \chdescs\ on them are
ready to be sent to the next \module\ (for example, the disk). A \chdesc\ is
ready if all of its dependencies are at least as close to the disk as it is. So,
when looking for a block to evict, the cache may not be able to evict any block
it chooses -- but it evicts the least recently used block that it can.

The write-back cache may need to write a non-evictable block anyway, however.
For instance, \chdescs\ may be arranged in a way that creates a dependency cycle
at the level of blocks, even though the \chdesc\ dependencies themselves are
acyclic. The same solution as soft updates uses is employed to break the cycle:
the write-back cache just holds on to the \chdescs\ that cannot yet be written,
and forwards the others on to the next \module\ as it writes the block. It
cannot evict the block yet, since it is still ``dirty,'' but progress has been
made that can make other blocks evictable.

The write-back cache has one other property that makes it useful in \Kudos: it
respects dependencies between one cache and another, so that (for instance)
dependencies between the changes on a file system and its external journal are
properly respected. This is actually not a special case, nor does it require any
extra code -- it is a property that just falls out of the way \chdesc\ graphs
are processed in order to determine which \chdescs\ are ready to move towards
the disk. This property also extends to \opgroups, which
are explained in \S\ref{sec:opgroup}.

\subsection{Elevator Scheduler}
\label{sec:modules:elevator}

% FIXME: Does the elevator scheduler actually improve our benchmarks?

The elevator scheduler is like a simpler version of the write-back cache. Unlike
the write-back cache, it will not hold on to \chdescs\ indefinitely until their
dependencies have been satisfied. Its purpose is instead to function like a
``peephole optimizer'' on the block writes done by some other \module, while
respecting any dependencies among the \chdescs\ on the blocks being written.
Effectively, a sliding window of writes to the disk (or some other \module) is
rearranged subject to the constraints mandated by the \chdesc\ dependencies.
Having a separate module to do this job, rather than the write-back cache
itself, allows us to separate the policy the elevator scheduler implements (in
this case, elevator scheduling) from the main write-back cache code.
