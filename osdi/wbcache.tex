\section{Write-Back Cache and Block Revisioning}

The write-back cache is in some sense the component that does all the real work
in a KFS system. There can be many write-back caches in a configuration at once,
but each is responsible for holding on to the change descriptors sent to it by
connected components until they can be safely sent on towards the disk. To a
write-back cache, the complex consistency protocols that other modules want to
enforce are nothing more than sets of dependencies among change descriptors --
it has no idea what consistency protocol (or protocols) it is implementing, if
any at all. Yet it is the component that ends up doing most of the work to make
sure that change descriptors are written in an acceptable order.

Even though a write-back cache has such an important and central role in the
system, there's not a lot to it. The current write-back cache is a simple LRU
cache, but with an important twist: blocks can't be evicted unless all the
change descriptors on them are ``ready'' to be sent to the next component (for
example, the disk). So, when looking for a block to evict, the cache may not be
able to evict any block it chooses -- but it evicts the least recently used
block that it can.

\subsection{Block Cycles}
\label{subsec:blockcycles}

Just as with soft updates~\cite{ganger00soft}, the dependencies among change
descriptors (or just ``updates to the block'' in soft updates) can create cyclic
dependencies among blocks, even though the change descriptors themselves do not
form a cycle. To handle this case, some change descriptors may need to be ``held
back'' in order to write the others, allowing such cycles to be broken.  To
effect this behavior, the write-back cache just holds on to the change
descriptors that cannot yet be written, but forwards the others on to the next
component as it writes the block. It cannot evict the block yet, since it is
still ``dirty,'' but progress has been made that can make other blocks
evictable.

\subsection{Block Revisioning}

Since many components may be stacked on top of one another in a KFS system, and
since many of them may want to refer to the same block at the same time, only
one copy of the data for each block is kept in memory at a time. However,
different components may ``know about'' different sets of change descriptors.
For example, in the case outlined in \S\ref{subsec:blockcycles}, the component
below the write-back cache will know about some of the change descriptors on a
block but not others (the ones which are not ``ready'' to be sent to it yet). If
this component is the disk, it will need to be able to write a version of the
block's data that does not include the change descriptors it does't know about
yet. KFS provides a revisioning system for blocks which can automatically ``roll
back'' the change descriptors which have not yet reached a particular component,
and then roll them forward again after that component is done using the previous
version of the block's data (e.g. to write it to disk).
