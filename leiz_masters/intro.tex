% -*- mode: latex; tex-main-file: "paper.tex" -*-

\section {Introduction}
\label{sec:intro}

Trends in file system development have led to improvements in reliability,
security, and usability. Reliability mechanisms like journaling and soft
updates are commonplace in modern file systems. Some file systems like ZFS
even have built-in checksumming. In terms of security and usability, an
increasing number of file systems have features like access control lists,
extended attributes, encryption, and compression. With growing demand for
additional capabilities from the file system, there comes a need to examine and
enhance methodologies for file system development.

Traditional file system implementations have been tightly coupled with a given
operating system.
While this yields benefits in term of performance and integrations, strong
reliance on specific operating system features like buffer cache behavior
makes file systems less portable and harder to develop. Currently,
sophisticated file system consistency features like soft updates and
journaling exist on FreeBSD and Linux, respectively. Despite the wide
availability of the source code to both operating systems, neither operating
system has been able to adopt the other's consistency mechanism.
The file system development process can be improved, with code that is easier
to comprehend, with more focus on the file system rather than other parts of
the operating system, while taking advantage of programming concepts such as
modules and well-defined interfaces. We strive to meet these goals in \Kudos,
our novel file system architecture.
\Kudos\ includes a file system module that implements the popular Unix File
System, to showcase the advances we made in extensibility and flexibility.
Our Unix File System component implements the functionalities necessary to
manipulate the on-disk format for a complex file system. It cooperates with
other \Kudos\ components to make file system consistency features easier to
understand. Our file system design also takes advantage of modularity to
support pluggable policies for tasks such as block allocation and directory
index traversal.

The Kudos File System Architecture is a joint project with Andrew de los Reyes,
Chris Frost, Mike Mammarella, and Professor Eddie Kohler. The work on
\Kudos\ is the result of our cumulative efforts. My main contribution to Kudos
is the development of the Unix File System component and infrastructure for
pluggable file system policies. This document highlight my individual
contributions, in the context of the \Kudos\ architecture.

\subsection{Related Work}

\paragraph{Stackable File Systems}

In the field of file system research, there have been many projects that
aim to improve file system functionality through modular
design~\cite{geom, rosenthal90evolving, heidemann91layered, skinner93stacking,
heidemann94filesystem, zadok99extending, zadok00fist, wright01ncryptfs,
wright06versatility}. Many of these projects concentrate their efforts on
specific portions of the system. For example, FiST~\cite{zadok00fist} and
related ``stackable'' file systems attach modules above some existing file
system to add new features. They do not deal with lower level issues like
caches, disks, and consistency.
GEOM~\cite{geom} works from the opposite direction, where it improves
modularity at the disk block level, but cannot assert any control over higher
level file operations. Neither of these approaches assert enough control over
the file system architecture to implement consistency mechanism like soft
updates and journaling.

\paragraph{Consistency}

Soft updates~\cite{ganger00soft} significantly lowers the overhead required to
provide file system consistency. By carefully ordering writes to disk, soft
updates avoids the need for synchronous writes to disk or duplicate writes to
a journal. Soft updates also guarantees a strong level of consistency after a
crash, enough so that the system can avoid time-consuming file system
consistency checks using a utility like \emph{fsck}.

Another method to protecting the integrity of the file system is to write
upcoming operations to a journal first. The content and the layout of the
journal varies in each implementation, but in all cases, the system can use
the journal to play out or roll back the operations that did not complete
as a result of a crash. Thus, \emph{fsck} can be avoided by consulting the
journal when recovering from a crash. 

Current soft update and journaling implementations are specifically designed
for some given file system. Without a formal way to express write ordering
and dependencies, many file systems had to make their own mechanisms to enforce
consistency.

\paragraph{File System Enhancements}

A variety of extensions to file systems have been proposed in recent work, like
the FS2 Free Space File System~\cite{huang05fs2} and encrypting file systems
like NCryptfs~\cite{wright01ncryptfs}. For UFS, Dowse and
Malone~\cite{dowse02recent} outline recent improvements in the FreeBSD
implementation, including soft updates, directory index hashing, and better
allocation algorithms.

\subsection{Our Approach}

The Kudos File System Architecture makes file system development easier
through a modular system, with a new abstraction for data dependencies.
By decomposing the entire disk I/O subsystem in pluggable modules connected via
clean interfaces, we improve the system's configurability, extensibility, and
readability. Having modules that span from system calls to block devices,
\Kudos\ is not limited to one layer of the system. To provide a universal
method for modules to specify writes to disk and their dependencies, we
created the concept of \emph{change descriptors}.

In \Kudos, modularity separates the system into components with well-defined
roles. Each component represents a small problem that is simple to understand
and relatively easy to solve. We recognized the core functionality of a file
system component is to correctly manipulate the on-disk format that defines
the file system. \Kudos\ file system developers concentrate their efforts on
this, and leave problems such as enforcing disk consistency to others.

The complete \Kudos\ \emph{architecture} consists of more than just the file
system. We have created our own modules to interface with userland, to
interface with the file system, to add journaling, and more. With control
over the entire system, we can move many responsibilities of a traditional
file system into separate components. For instance, many file systems
implementations have similar code for satisfying user requests. We
consolidated this code into its own module, so \Kudos\ file systems do not have
to reimplement it. Likewise, each file system should not do its own journaling
and write ordering enforcement, so those features are encapsulated into their
own modules as well.

A modular system like \Kudos\ needs good inter-module communication to function
effectively. For file system modules, there needs to be a way to convey 
write ordering information to satisfy file system consistency. To do this,
data structures called change descriptors represent writes to disk.
More so, change descriptors contain dependency information to indicate the
order writes should be applied to disk. This offers file system developers a
convenient way to describe consistency constraints for mechanisms such as soft
updates and journaling.

In Section \ref{sec:consistency} we discuss the file system consistency
protocols that motivated our work.
Section \ref{sec:kudos} gives an overview of the interfaces and modules in the
Kudos File System Architecture, as well as the change descriptor concept.
Section \ref{sec:ufs} explains the Unix File System, its data structures, and
properties that make it unique.
Next, we go into detail about our Unix File System implementation for
\Kudos\ in Section \ref{sec:implementation}. We describe the design and
implementation, and we share our experiences on using change descriptors
effectively.
Section \ref{sec:evaluation} compares the \Kudos\ Unix File System to file
systems on FreeBSD and Linux. 
Section \ref{sec:discussion} looks at future improvements for \Kudos\ file
system implementations.
Finally, we conclude in Section \ref{sec:conclusion}.

