\subsection{Kudos Interfaces}
\label{sec:interfaces}

Componentized systems need well-defined interfaces to connect modules together.
Previously, Heidemann~\cite{heidemann91layered} have tried to hook up modules
together like Unix pipes. For such a solution to work, there can only be one
interface, as the input and output interfaces must be symmetrical. It is
difficult to settle on a single interface for all modules. If the interface is
too narrow, then that limits the level of module interaction. Make the
interface too wide, on the other hand, and it becomes unnecessarily complex
and open for misuse. We recognized this as a design limitation that reduces
flexibility in extensible file systems. With this in mind, we designed several
interfaces, each for a specific purpose. For \Kudos, the three different
interfaces are CFS, LFS, and BD.

The interface that sits closest to userland is CFS, the
\emph{common file system} interface. Like a VFS interface, CFS provides a set
of high-level file operations. To hook into the operating system, a top-level
module receives system calls and make the corresponding CFS calls.
\Modules\ connected via the CFS interface can then translate and manipulate any
high-level file operation that passes through. On the other end, the interface
that sits directly above the storage devices is BD, the \emph{block device}
interface. BD deals with data in terms of blocks and facilitates the transfer
of data blocks to and from the underlying storage devices. Similar to CFS,
modules with BD interfaces can transform and control the data flow in a number
of ways.

The third, and more novel \Kudos\ \module\ interface is LFS, the
\emph{low-level file system} interface. Other systems have interfaces similar
to CFS and BD, but \Kudos\ introduces an intermediate interface that sits in
between CFS and BD. Here, the interface works at the file system level, and
consists of methods that deal with operations like the creation and deletion
of directory entries; the allocation, appending, and truncating of file blocks;
and management of file system metadata. LFS acts as a middle layer between CFS
and BD, since CFS has no knowledge of disk blocks, and BD has no concept of
files. Naturally, this is where the file system modules and related parts go.
They take high-level CFS calls, break them down into low-level LFS calls, and
issue BD calls in response. Figure \ref{fig:lfs} outlines the methods we
expose at the LFS interface.

\begin{figure}[thb]
\vskip-14pt
\begin{tabular}{@{\hskip0.25in}p{2in}@{}}
\begin{scriptsize}
\begin{alltt}
int (*\textbf{get_root})(inode_t *inode);
uint32_t (*\textbf{allocate_block})(
    fdesc_t *file, int purpose,
    chdesc_t **head);
int (*\textbf{free_block})(
    fdesc_t *file, uint32_t block,
    chdesc_t **head);
bdesc_t *(*\textbf{lookup_block})(uint32_t number);
int (*\textbf{write_block})(
    bdesc_t *block, chdesc_t **head);
int (*\textbf{lookup_name})(
    inode_t parent, const char *name,
    inode_t *inode);
fdesc_t *(*\textbf{lookup_inode})(inode_t inode);
void (*\textbf{free_fdesc})(fdesc_t *fdesc);
uint32_t (*\textbf{get_file_numblocks})(fdesc_t *file);
uint32_t (*\textbf{get_file_block})(
    fdesc_t *file, uint32_t offset);
int (*\textbf{get_dirent})(
    fdesc_t *file, struct dirent *entry,
    uint16_t size, uint32_t *basep);
int (*\textbf{append_file_block})(
    fdesc_t *file, uint32_t block,
    chdesc_t **head);
uint32_t (*\textbf{truncate_file_block})(
    fdesc_t *file, chdesc_t **head);
fdesc_t *(*\textbf{allocate_name})(
    inode_t parent, const char *name,
    uint8_t type, fdesc_t *link,
    inode_t *newinode, chdesc_t **head);
int (*\textbf{remove_name})(
    inode_t parent, const char *name,
    chdesc_t **head);
int (*\textbf{rename})(
    inode_t oldparent, const char *oldname,
    inode_t newparent, const char *newname,
    chdesc_t **head);
int (*\textbf{get_metadata})(
    inode_t inode, uint32_t id,
    size_t size, void *data);
int (*\textbf{set_metadata})(
    inode_t inode, uint32_t id,
    size_t size, const void *data,
    chdesc_t **head);
\end{alltt}
\end{scriptsize}
\end{tabular}
\vspace{-10pt}
\caption{\label{fig:lfs} Important functions in the LFS interface.}
\end{figure}
