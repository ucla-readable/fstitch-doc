\section{UFS}
\label{sec:ufs}

The Unix File System (UFS) is a popular file system used by the BSD family of
operating systems. It is the modern incarnation of the Fast File
System~\cite{mckusick84fast} used in 4.2 BSD, which in turn is derived from the
original file system used on Unix. Given its deeply rooted history in Unix,
many operating systems based on Unix either use a variant of UFS, like with
Solaris and HP-UX, or a file system strongly influenced by UFS, like with
Linux's ext2. Simplified versions of UFS, such as the Minix file system, are
commonly used in academia to teach basic file system concepts.

For the current version of UFS, there are two variants. The original, UFS1, is
supported by all operating systems in the BSD family. The newer variant, UFS2,
is only supported by FreeBSD and NetBSD. UFS2 switched to 64-bit pointers in
order to break existing size limitations. UFS2 also added support for features
like extended attributes and got rid of legacy data structure fields that are
no longer relevant.

\subsection{UFS Data Structures}
\label{sec:ufs:structs}

\begin{figure}[htb]
  \centering
  \includegraphics[width=200pt]{fig/figures_8}
  \caption{\label{fig:ufs_sb} Layout of the Unix File System.}
\end{figure}

The seven major data structures in UFS are: the superblock, cylinder groups,
resource allocation bitmaps, inodes, directories, data blocks, and summary
information.

The superblock contains general information about the file system and disk
geometry. Some of the essential fields include the size of fragments and
blocks, the size of the file system, the number of cylinder groups, and where
data structures are relative to the beginning of the cylinder groups. There
are also non-essential fields that provide information such as whether the
file system is clean and how much free space to reserve. For convenience, the
UFS superblock has many redundant fields that are not strictly necessary
because it is possible to calculate them using other fields. An example of this
is the number of fragments in a block, which is the block size divided by the
fragment size. Some fields in the superblock are no longer used but exist for
historical reasons. Many years ago, knowledge of the disk geometry enabled
storage optimizations. Although the data fields remain, they are useless for
modern disk drives because the logical disk geometry no longer reflects the
actual disk layout.

\begin{figure}[htb]
  \centering
  \includegraphics[width=200pt]{fig/figures_9}
  \caption{\label{fig:ufs_cg} Layout of the UFS cylinder group.}
\end{figure}

To achieve better disk locality and to distribute the data structures for
more robustness, UFS divides the file system into a number of cylinder groups.
Each cylinder group in UFS is a smaller, self contained file system, with its
own bitmaps, inode table, data blocks, and backup of the superblock. At the
beginning of each cylinder group, there exists a group descriptor with
information about that cylinder. In particular, cylinder group descriptors
contain lots of statistics and summary information regarding resource
allocations, so the file system driver can make better allocation decisions.

UFS uses bitmaps to keep track of three important resource: blocks, fragments,
and inodes. Each cylinder group contains a fixed number of these resources,
thus UFS uses contiguous, static bitmaps to keep track of them efficiently.
Since different file systems can have different sized cylinder groups, the
bitmaps' locations are given as relative offsets to the start of the group
descriptors. To indicate free blocks and fragments, UFS marks them with the
value 1, but for free inodes, UFS marks them with the value 0.

Inodes are internal objects that UFS uses to represent files. Rather than
identifying files directly with names, UFS refers to inodes by numbers. This
indirection makes it easier to have hard links, where multiple names refer to
the same file object. The first few inodes are special: inode 0 and 1 are
reserved, and inode 2 refers to the root directory. Inodes contain all the
vital information for files, including file type, link count, size, data blocks
used, and ownership information. To refer to the data blocks, UFS has 12 direct
block pointers, and three additional pointers for the single, double, and
triple indirect pointers. The direct block pointer values are the block numbers
for file data blocks. The indirect block pointer values are the block numbers
for blocks that contain block pointers with one less degree of indirectness.

To associate file names with inodes, directories contain entries that map
the two identifiers. Each directory entry contains an inode value, a
variable-sized file name, the length of the file name string, and the length
of the directory entry. All directory entry sizes are multiples of 4 bytes,
and they are chained together in a linked list, so the entry length field act
as the pointer to the next entry.

Besides areas reserved for metadata, UFS divides the rest of the disk into
data blocks. One problem many general purpose file systems face is the sizing
of the data block. Big data blocks are easier to manage, but they do not
utilize the disk efficiently for smaller files. Small data blocks use disk
space more efficiently, but they can cause external fragmentation and they
present more overhead. To solve this problem, UFS use relatively big data
blocks, but UFS further divides data blocks into fragments. Typically, each
block contains 8 equal-sized fragments. For small files, UFS will only
allocate fragments to conserve file space. As files grow in size, UFS will
switch to blocks for speed.

\begin{figure}[htb]
  \centering
  \includegraphics[width=200pt]{fig/figures_10}
  \caption{\label{fig:ufs_data} Directories point to inodes, which in turn,
	  point to data blocks.}
\end{figure}

Throughout the UFS data structures, there are fields that contain summary
information. For instance, every cylinder group and the superblock has
knowledge of the number of free blocks, inodes, and fragments. In principle,
UFS does not need summary information, since it is possible to read the bitmaps
and calculate the summary information. However, caching this information saves
UFS from having to recalculate it, at the risk of the summary information
being out of sync from the actual values.


\subsection{UFS Features and Requirements}
\label{sec:ufs:features}

As a complete and relatively complex file system, UFS has peculiarities that
makes it difficult to fully comprehend how some parts of it work. Specifically,
the UFS directory structure and the relationship between fragments and blocks
are two of the harder concepts to understand.

The basic UFS directory structure seems fairly simple, but the actual rules
governing its use have several quirks. Directories are just names that point
to inodes, like any other file object in UFS. The size of a directory is
always a multiple of 512 bytes, even if it does not need that much space, or
if more disk space is actually allocated. To see how these two situations can
occur, take an empty directory, with only the ``.'' and ``..'' entries.
This takes up 24 bytes in the directory, yet its size is 512 bytes.
At the same time, if the fragment size for this file system is 2048 bytes, at
the minimum, UFS has to allocate this directory 2048 bytes. Directory entries
can only increase in size, never decrease. They also never cross this 512 byte
boundary. Given a 512 byte directory with 480 bytes worth of entries, if the
next added entry is 36 bytes, UFS will not put it right after the existing
entries. Instead, it extend the directory size to 1024 bytes, and start the new
entry at the 512 byte offset, wasting the last 32 bytes in the first sector.
This behavior may be due to historical reasons, but it also seems to be more
resilient to corruption that breaks the linked list, since every 512 byte
region is in a way separated from the adjacent sectors. Should some content in
the directory become corrupt, UFS can still recover other parts of the
directory starting from any 512 byte offset within the directory. Being a
linked list, when items get deleted, the next pointer of the previous entry
needs to be modified to point to the next item of the deleted entry. This
presents a problem when the first entry in a sector, the head, gets deleted.
In UFS, when this scenario occurs, the inode number of that entry becomes 0.
UFS understands this entry has been intentionally blanked out and it is not a
valid file, but its next pointer is valid and should be used to find the next
file in the chain.

The other unusual property of UFS involves fragments and blocks, where a
constant set of aligned fragments make up a block. Most files have data stored
in blocks, with the trailing bits that do not need a full block stored in a
contiguous chain of fragments. For very small files, UFS conserves space and
stores them in just fragments. As the file grows, UFS extends the file using
fragments. For example, if the file uses fragment 60 initially, then UFS will
append to it fragment 61, 62, and so on. One issue that presents a problem is
when fragment 62 has been allocated to another file. Although most UFS
implementations will try to avoid this, it can occur either due to poor
allocation algorithms or lack of free space. To get around it, UFS will have
to relocate the entire fragment chain elsewhere. Likewise, a chain of fragments
cannot cross the block alignment boundaries, so if the example file tries to
go past fragment 64, the chain has to be relocated. Once a file gets big enough
to require the use of indirect blocks, then UFS no longer bothers with
fragments and just use blocks, as the amount of waste relative to the file size
is much smaller.

