\subsection{Journaling}
\label{sec:journaling}

Journaling improves file system consistency by keeping track of writes to
disk. Typically the file system reserves a small section of the disk and
designates it as the journal. Instead of directly writing a file system
operation to the disk, journaled file systems will first log the operation to
the journal area. After writing the journal record, the actual operation takes
place. To complete the operation, journaled file systems will write a commit
record to the journal to indicate the operation has been completed. By
ensuring writes to the journal occur before writes to the file system, a
journaled file system records enough information to recover from a crash.

With journaling, if the system experiences a failure that causes a restart,
the system can refer to the journal and either complete or rollback any
outstanding operations from the time of the crash. This knowledge of the
status of the file system means time-consuming file system checks can be
avoided. Depending on the amount of data recorded to the journal, the system
can make different levels of guarantees regarding the integrity of the file
system.
The robustness gained from journaling comes at the price of redundant writes.
For any data written to the journal, the file system will have to write it
again to the disk. Journaling also incurs extra disk seeks between the journal
and data area, reducing disk bandwidth even further.

%For instance, Linux's ext3 supports three journaling modes. In
%\emph{journal} mode, ext3 records everything to the journal to achieve the
%best level of integrity. In both \emph{ordered} and \emph{writeback} mode, the
%file system will only write metadata to the journal. The difference being
%\emph{ordered} mode will write data to disk before committing the associated
%metadata to the journal, whereas \emph{writeback} mode can write data and
%metadata in any order. Both metadata only modes ensures the file system will
%be consistent, but makes little or no guarantees for data integrity.

%Compared to a file system without journaling, metadata only journaling is
%slightly slower for write operations. Full data journaling takes a more
%significant performance hit, oftentimes halving the write throughput rate.

