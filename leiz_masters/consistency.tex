\section{Consistency}
\label{sec:consistency}

To better understand factors that influenced our design decisions, we examine
file system consistency protocols and shortcomings in their implementations.
Two mainstream methods for file system consistency are journaling and soft
updates. Journaling has been around longer and this is evident by the number of
file systems that use it: ext3, HFS+, JFS, NTFS, Reiserfs, and XFS, to name a
few. In contrast, soft updates is a relatively new concept and is only widely
used for UFS on the BSD family of operating systems.

\section{Journaling}

The proceedings from the 2000 USENIX conference features a comparison
between journaling and soft updates~\cite{seltzer00journaling}. This
paper discusses the two different techniques, variations among
different types of journals, and how well they maintain metadata
integrity. It then benchmarks the different systems against the
baseline filesystem to see how well they perform.

The main goal of a filesystem is to reliably store data. Filesystems
must keep track of everything using bookkeeping information called
metadata. The metadata must remain consistent or the filesystem will
get confused about where it stored data. Early filesystems like FFS
write data out in a specific order using synchronous writes to
guarantee consistency. However, synchronous writes are slow and the
filesystem becomes a performance bottleneck. Two solutions to this
problem are journaling and soft updates. Both techniques attempt to
maintain consistency as well as FFS with synchronous writes, while
providing better performance.

Soft updates also order writes to the disk such that metadata
consistency is maintained. Unlike synchronous FFS, soft updates allow
the writes to be delayed.  Soft updates keep track of dependencies
between different operations. As such, when writes occur, soft updates
will examine the dependencies and do appropriate rollbacks so that the
writes to the disk will never destroy filesystem consistency. If the
system crashes, the disk will still be consistent. However, soft
updates do not guarantee all free resources are accounted for. In this
way, soft updates has a looser definition of `consistency.' Soft
updates can recover instantly and work in the background to recount
the free resources.

Journaling maintains filesystem integrity by keeping a log of
operations that will occur, and checking them off after the operations
finish. This way, in the event of a system crash, the computer can
replay the journal on the next boot and figure out the events leading
up to the crash. The system can then either finish playing out the
journal or rollback to a consistent state. By writing data to the
journal synchronously, journaling can guarantee the same level of
consistency as synchronous FFS. Writing data to the journal
asynchronous will guarantee the same protections as soft updates.

The benchmarks compare asynchronous FFS, synchronous FFS, soft
updates, and a variety of journaling variations. For journaling, there
are asynchronous and synchronous versions of Logging-FFS that keeps
the log in a file and that which keeps the log on a separate
filesystem. The benchmarks themselves separate into two catagories:
microbenchmarks that tests specific operations, and macrobenchmarks
that simulates workloads.

With both soft updates and journaling, the system will have to do
additional writes. For soft updates, the ability to defer writes
offsets the additional work. In the case of journaling, the additional
writes can be done efficiently because the writes are sequential. For
the benchmarks, soft updates performed very well. In particular, the
ability to delay writes helped out greatly on the file deletion
benchmark. On the other hand, journaling filesystems all incurred a
penalty at the 96 KB mark because they had to read an indirect
block. Overall, both soft updates and asynchronous journaling achieved
performance close to that of asynchronous FFS.

From this paper, it is obvious that any modern operating system that
demands good filesystem performance will want either journaling or
soft updates. For KudOS, we would like the ability to support both
techniques. After much thought, we determined both journaling and soft
updates require dependencies to work. In the case of journaling, the
dependency states the journal has to be written before the actual
data.

To support dependencies, we created the concept of change
descriptors. Change descriptors describe changes to a specific block
device. The change descriptors link to each other to form the
dependency relationships. As filesystems perform operations, they will
have to generate change descriptors and chain them up in the proper
order depending. We will need an intelligent cache that understands
how to deal with change descriptors. Lastly, we will need have a
dependency manager to send change descriptors between components.


\section{Soft Updates}
\label{sec:softupdates}

Ganger, McKusick, Soules, and Patt proposed a system called ``Soft
Updates''~\cite{ganger00soft} for solving the problem of metadata
consistency in filesystems. Instead of requiring a long consistency
check upon recovery from a crash, Soft Updates uses disk write
ordering in order to make certain guarantees about the state of the
filesystem on disk that allow safe use of the filesystem immediately.

In the implementation of Soft Updates in 4.4BSD, metadata changes to
the filesystem are stored in a custom dependency graph which is
consulted by the disk cache when it wants to write a block to disk.
Any metadata changes in a disk block that have unmet dependencies
(that is, they must not be written until some other block has been
written, and that block has not yet been written) are rolled back
before writing the block, and then rolled forward again after the
block is written. These metadata changes are stored in terms of the
file operation that caused them, and in some cases they actually cause
the ``second half'' of the operation to happen upon being written (for
instance, the inode usage count is decreased only after a nullified
directory entry is written to disk when a file is deleted).

We wanted to support a system like soft updates in our design, but not
limit it to a single filesystem's metadata issues. Further, we wanted
to be able to implement other order-sensitive filesystem features like
journalling, and use a single order-aware cache to handle all such
cases. Our design therefore calls for a generalized ``change
descriptor'' structure that describes an arbitrary change to a disk
block, and which can have other change descriptors as dependencies.
Using this system, we can acheive the same effect as Soft Updates, and
even allow our change descriptors to cross block translation modules
like the partitioner or the loopback device so that the underlying
disk is updated in the order needed by the top-level filesystem.

We plan to implement Soft Updates using change descriptors for a
simple filesystem, and demonstrate that minimal changes are actually
required in the filesystem code to effect this. Most of the work will
be in the cache, and this will be completely reusable for different
filesystems, or different styles of dependencies (like journalling,
where change descriptors can optionally depend on blocks on a
different block device to allow external journals without requiring
synchronous writes).


\subsection{Current Implementations}
\label{sec:consistency:implementations}
While implementations of journaling and soft updates protect file systems
and provide good performance, they leave much to be desired in terms of
modularity, code reuse, and simplicity. Many file system implementations have
journaling code that is closely-coupled with the rest of the file system.
For instance, all the journaled file systems on Linux have their own code for
journaling. Although there have been attempts to provide journaling capability
as a separate \module, none of the attempts have been successful. On FreeBSD,
the Gjournal project for GEOM is incomplete and cannot replace a proper
journaled file system. Apple's OS X provides a generic journaling layer, but
thus far it has only been used with HFS+. Similarly, Linux's ext3
implementation is the sole user of Linux's generic journaling block device.
The source code for ext3 also has much in common with the code for ext2, its
non-journaled predecessor, yet there is little code sharing between the two
other than common header files. For soft updates, the fundamental idea of
sequencing writes is built into the implementation, thus making modularity
hard or impossible. The soft updates implementations can improve on
simplicity, however. For UFS on FreeBSD, soft updates uses 16 different types
of data structures to manage dependencies for various parts of the file
system. Though there are advantages to having specialized dependency
structures, it is more difficult to understand them and manage their
interactions with each other.

