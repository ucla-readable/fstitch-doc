\section{Consistency}
\label{sec:consistency}

To better understand factors that influenced our design decisions, we examine
file system consistency protocols and shortcomings in their implementations.
Two mainstream methods for file system consistency are journaling and soft
updates. Journaling has been around longer and this is evident by the number of
file systems that use it: ext3, HFS+, JFS, NTFS, Reiserfs, and XFS, to name a
few. In contrast, soft updates is a relatively new concept and is only widely
used for UFS on the BSD family of operating systems.

\section{Journaling}

The proceedings from the 2000 USENIX conference features a comparison
between journaling and soft updates~\cite{seltzer00journaling}. This
paper discusses the two different techniques, variations among
different types of journals, and how well they maintain metadata
integrity. It then benchmarks the different systems against the
baseline filesystem to see how well they perform.

The main goal of a filesystem is to reliably store data. Filesystems
must keep track of everything using bookkeeping information called
metadata. The metadata must remain consistent or the filesystem will
get confused about where it stored data. Early filesystems like FFS
writes data out in a specific order using synchronous writes to
guarantee consistency. However, synchronous writes are slow and the
filesystem becomes a performance bottleneck. Two solutions to this
problem are journaling and soft updates. Both techniques attempt to
maintain consistency as well as FFS with synchronous writes, while
providing better performance.

Soft updates also order writes to the disk such that metadata
consistency is maintained. Unlike synchronous FFS, soft updates allow
the writes to be delayed.  Soft updates keep track of dependencies
between different operations. As such, when writes occur, soft updates
will examine the dependencies and do appropriate rollbacks so that the
writes to the disk will never destroy filesystem consistency. If the
system crashes, the disk will still be consistent. However, soft
updates does not guarantee all free resources are accounted for. In
this way, soft updates has a looser definition of 'consistency'. Soft
update can recover instantly and work in the background to recount the
free resources.

Journaling maintains filesystem integrity by keeping a log of
operations that will occur, and checking them off after the operations
finish. This way, in the event of a system crash, the computer can
replay the journal on the next boot and figure out the events leading
up to the crash. The system can then either finish playing out the
journal or rollback to a consistent state. By writing data to the
journal synchronously, journaling can guarantee the same level of
consistency as synchronous FFS. Writing data to the journal
asynchronous will guarantee the same protections as soft updates.

The benchmarks compare asynchronous FFS, synchronous FFS, soft
updates, and a variety of journaling variations. For journaling, there
are asynchronous and synchronous versions of Logging-FFS that keeps
the log in a file and that which keeps the log on a separate
filesystem. The benchmarks themselves separate into two catagories:
microbenchmarks that tests specific operations, and macrobenchmarks
that simulates workloads.

With both soft updates and journaling, the system will have to do
additional writes. For soft updates, the ability to defer writes
offsets the additional work. In the case of journaling, the additional
writes can be done efficiently because the writes are sequential. For
the benchmarks, soft updates performed very well. In particular, the
ability to delay writes helped out greatly on the file deletion
benchmark. On the other hand, journaling filesystems all incurred a
penalty at the 96 KB mark because they had to read an indirect
block. Overall, both soft updates and asynchronous journaling achieved
performance close to that of asynchronous FFS.

From this paper, it is obvious that any modern operating system that
demands good filesystem performance will want either journaling or
soft updates. For KudOS, we would like the ability to support both
techniques. After much thought, we determined both journaling and soft
updates require dependencies to work. In the case of journaling, the
dependency states the journal has to be written before the actual
data.

To support dependencies, we created the concept of change
descriptors. Change descriptors describe changes to a specific block
device. The change descriptors link to each other to form the
dependency relationships. As filesystems perform operations, they will
have to generate change descriptors and chain them up in the proper
order depending. We will need an intelligent cache that understands
how to deal with change descriptors. Lastly, we will need have a
dependency manager to send change descriptors between components.


\subsection{Soft Updates}
\label{sec:softupdates}

Soft updates takes a different approach where the system carefully orders
writes to disk. By following a few safety principles, soft
updates' \emph{update dependencies} ensures the disk is consistent after every
write. This gives file system integrity guarantees without a journal and the
associated redundant writes. 

For soft updates, the three golden rules are: never point to a structure
before it has been initialized; never reuse a resource before nullifying
all previous pointers to it; and never reset the last pointer to a live
resource before a new pointer has been set. Although it seems easy to just
follow these rules and write changes out to disk in a proper order, the
reality of the situation is not that simple. Caches flush data out to disk
in the unit of sectors or blocks, not in terms of byte-level changes. This
leads to situations where the cache needs to write changes out to disk, but
the byte-level dependencies create cyclic dependencies at the block level.
To solve this problem, soft updates uses \emph{dependency structures} to
retain the state of data structures before and after changes occur. With the
ability to undo changes, soft updates can break block-level cyclic dependencies
by reverting changes that cannot be safely written to disk. Once the cycle has
been broken, soft updates restores the reverted changes and writes them out at
a later time.
%Compared to an unsafe file system, this presents a small
%overhead, as it does generate more writes. With respect to journaling, soft
%updates writes less to the disk, since it does its bookkeeping in memory.
%
%There are some drawbacks to soft updates for consistency and crash recovery.
%Whereas a journaled file system can refer to the journal, soft updates has no
%persistent information available after a crash. The soft updates rules protect
%against inconsistencies that leads to corrupt data, but it does not protect
%against resource leaks. As a result, unused disk blocks and inodes can be
%marked as used, and inodes may have higher link counts than actual link counts.
%To correct these problems, a file system using soft updates still requires
%a file system check, though it can be done in the background while the file
%system is online.



\subsection{Current Implementations}
\label{sec:consistency:implementations}
While implementations of journaling and soft updates protect file systems
and provide good performance, they leave much to be desired in terms of
modularity, code reuse, and simplicity. Many file system implementations have
journaling code that is closely-coupled with the rest of the file system.
For instance, all the journaled file systems on Linux have their own code for
journaling. Although there have been attempts to provide journaling capability
as a separate \module, none of the attempts have been successful. On FreeBSD,
the Gjournal project for GEOM is incomplete and cannot replace a proper
journaled file system. Apple's OS X provides a generic journaling layer, but
thus far it has only been used with HFS+. Similarly, Linux's ext3
implementation is the sole user of Linux's generic journaling block device.
The source code for ext3 also has much in common with the code for ext2, its
non-journaled predecessor, yet there is little code sharing between the two
other than common header files. For soft updates, the fundamental idea of
sequencing writes is built into the implementation, thus making modularity
hard or impossible. The soft updates implementations can improve on
simplicity, however. For UFS on FreeBSD, soft updates uses 16 different types
of data structures to manage dependencies for various parts of the file
system. Though there are advantages to having specialized dependency
structures, it is more difficult to understand them and manage their
interactions with each other.

