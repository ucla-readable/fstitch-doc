\subsection{Kudos Modules}
\label{sec:modules}

\Kudos\ \modules\ are designed to be easily understood and capable of being
connected together in an extensible manner. A complete \Kudos\ system contains
a graph of modules, working together to provide a functioning file server.
Some core modules in \Kudos\ are absolutely necessary, while others can be
plugged in to add functionality like journaling or RAID. By convention,
modules in \Kudos\ take requests from the input, and issue requests via the
output. Responses flow in the opposite direction, coming back from the
module connected to the output. In principle, modules can have any
combination of input and output interfaces, though so far only some have been
used. \Kudos\ \modules\ that need to accomplish special tasks can also have
multiple fan-ins or multiple fan-outs. Some of the common modules in
\Kudos\ are described below.

\subsubsection{Service Modules}
\label{sec:modules:service}
In order to serve file operations, there exists a family of modules that take
requests from userland and transform them into CFS calls. Each module in this
category connects to a different system interface. On Linux, the in-kernel
version of \Kudos\ uses a \module\ called \emph{kernel\_serve} to interface
with the Linux VFS layer. For the userland version of \Kudos,
\emph{fuse\_serve} talks to FUSE and generate the appropriate CFS calls.

\subsubsection{UHFS Modules}
\label{sec:modules:uhfs}
CFS calls eventually get translated into LFS calls by the Universal High-level
File System (UHFS) module. The UHFS module encompasses all logic for
decomposing higher level CFS calls into lower level LFS calls. By
consolidating the job of tying LFS micro-ops together to serve CFS requests
into this single module, file system module developers can instead concentrate
on implementing LFS functions that deal with the particulars of the file
system.
To see how this simplifies the development process, consider the VFS
\texttt{write()} call, which has the task of writing some amount of data to a
file at a given offset. In \Kudos, the logic to determine the correct offsets
within blocks and whether new blocks must be allocated is built into UHFS. A
file system module need only implement four LFS calls:
\texttt{get\_file\_block()}, \texttt{allocate\_block()},
\texttt{append\_block()}, and \texttt{write\_block()}.

\subsubsection{File System Modules}
\label{sec:modules:fs}
Below UHFS, file system modules take in LFS requests and generate BD requests.
A file system module understands the on-disk layout for the given file system,
and has knowledge of all the file system data structures needed for reading
and writing. Reads occur in the conventional manner, where the file system
reads a block and extracts the relevant information. For writes, instead of
directly modifying data blocks, file system modules make modifications by
generating \chdescs\ (See \S\ref{sec:chdescs}). Using change
descriptors, file system modules can specify write ordering dependencies,
to implement features such as soft updates.
Currently, there exist two file systems in \Kudos: a simple file system
designed for academia called JOSFS, and the more widely used UFS. More details
about UFS and our implementation can be found in section
\S\ref{sec:ufs} and \S\ref{sec:implementation}, respectively.

\subsubsection{Journal Modules}
\label{sec:modules:journal}
To enhance \Kudos with extra features like journaling, users simply need to
attach a journal module below the file system module. From the file system
module's perspective, the journal module is simply a block device. However,
the journal module will take the \chdescs\ generated by the file
system and create a duplicate set to send to the journal. By hooking up the
\chdescs\ properly, the journal module achieves full data journaling.
Unlike most modules, the journal module has two BD output interfaces. This
allows for the use of an external journal, one that is located on a different
block device, rather than an internal journal on the same device.
For an external journal, one BD interface connects with the file system's disk
device, while the other connects to another disk that contains the journal.
For an internal journal, the second BD interface connects to a loopback module,
which translates BD calls into LFS calls for a specific file on the journaled
file system.

\subsubsection{Cache Modules}
\label{sec:modules:cache}
Eventually, writes in the form of \chdescs\ have to go to disk. Along
the path down to the disk are cache modules. Not only do they retain disk
blocks in memory for faster reads, they can also influence when writes occur.
In particular, the \Kudos\ write-back cache enforces write orderings among
\chdescs\ by writing blocks out in an order that is within bounds
specified by the \chdescs. To solve circular dependencies among disk
blocks, the write-back cache uses the rollback technique inspired by soft
updates.

\subsubsection{Disk Modules}
\label{sec:modules:disk}
At the bottom of the modules graph, disk modules read and write data blocks
to some storage device. Like the CFS modules at the very top of the graph,
the disk modules are specific to some given implementation. For instance,
inside the Linux kernel, the disk module sends \emph{bio requests} to Linux
disk device drivers, whereas in our experimental OS, the disk module is an IDE
device driver.

