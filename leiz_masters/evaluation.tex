\section {Evaluation}
\label{sec:evaluation}

We evaluate our prototype implementation of \Kudos\ with macro-benchmarks that
perform nontrivial file operations, as well as micro-benchmarks that compare
specific operations.
We show that the overall performance is within reason, even though it is slower
than Linux or BSD.

\subsection {Overall Performance}

Our first benchmark test is to untar the Linux 2.6.15 source code from
\texttt{linux-2.6.15.tar}, which is already decompressed and cached in RAM. We
did this test on identical machines running Linux 2.6.12 (using ext3), FreeBSD
5.4 (using UFS, with and without soft updates), and \Kudos\ running as a Linux
2.6.12 kernel module (using UFS, with and without soft updates). As a second
test, we then delete the resulting source tree. The times below include time to
fully sync the changes to disk.

% make this into a table
\begin{figure}[htb]
\centering{
\begin{tabular}{|l|r|r|} \hline
System & Untar (sec) & Delete (sec) \\ \hline\hline
\Kudos\ (SU) & 24.50 & 11.69 \\ \hline
\Kudos\ (async) & 20.14 & 11.27 \\ \hline
FreeBSD 5.4 (SU) & 20.47 & 4.71 \\ \hline
FreeBSD 5.4 (async) & 9.63 & 3.72 \\ \hline\hline
% double line to separate linux
Linux 2.6.12 (journaling) & 12.90 & 4.90 \\ \hline
\end{tabular}
}
\caption{\label{fig:macro} Untar/delete times for \Kudos, FreeBSD, and Linux,
with soft updates (SU) and with async mode.}
\end{figure}

\Kudos\ with UFS is about 20\% slower than FreeBSD using soft updates for
writes, and a little over a factor of two slower for deletions. This shows
that, although it is slow, \Kudos\ is close to running in acceptable time.
While \Kudos\ has CPU usage and I/O delay problems, we believe they can be
fixed, given time and effort. Improvements to \chdesc\ arrangements and
\chdesc\ traversal algorithms will make a strong impact on performance.
The file system has a significant role in regards to optimizations, since it
is the primary source for the creation of
\chdescs\ (see \S\ref{sec:discussion}).

\subsection {Block Writes}

We have also looked at the number of block writes that \Kudos\ makes
relative to other systems for several operations. To do this, we
instrumented the \Kudos\ and FreeBSD disk drivers to report the number
of writes to disk. In one test, we create 100 small files in a directory and
measure the number of blocks written to disk. \Kudos\ writes 122 blocks, while
FreeBSD writes 135 blocks. Differing policies for summary information updates,
as well as incomplete support for auxiliary data structures, account for the
discrepancy.

Additionally, we ran the two small benchmarks with the journal module,
which does full data journaling. Creating 100 small files with the journal
resulted in 245 blocks written to the disk, but only 126 blocks
without the journal. Deleting and adding a file resulted in writing
31 with the journal module and 18 without it. As expected, the number of writes
with journaling are approximately twice as much as those for without
journaling.

To verify the correctness of our implementation, we begin with a valid file
system created by FreeBSD. Then we run our tests and copy the resulting disk
images to a FreeBSD machine. Using the \emph{fsck} tool on FreeBSD, we check
the modified file systems and verify their integrity. From this we gather that
our UFS, soft updates, and journal implementations are correct.

