\subsection{Change Descriptors}
\label{sec:chdescs}

In order to indicate modifications to disk blocks, \Kudos\ uses a data
structure called the \chdesc. Fundamentally, a \chdesc\ contains
information about the block it is on, what part of the block it is modifying,
the content of the block before and after the change, and ordering information
relative to other \chdescs. Whenever \Kudos\ \modules\ in the system
need to make changes to the disk, they generate \chdescs. Like
packets in a network, \chdescs\ flow through the \module\ graph until
they eventually reach the disk, where they are \emph{satisfied}. To place
ordering restraints on which \chdescs\ reach the disk first,
\modules\ can create dependencies between \chdescs. For any \chdesc,
it cannot be satisfied unless its dependencies are satisfied, and none of its
dependents will be satisfied until it is satisfied. \Kudos\ uses this ability
to control write ordering for consistency mechanisms like journaling and soft
updates.

For efficiency, \Kudos\ has three different types of \chdescs.
The byte \chdesc\ is the one most similar to the ideal change
descriptor mentioned above. It records changes for a byte range up to the size
of a block. The bit \chdesc\ is an optimized version used for
bitmap changes common to file systems. Rather than storing a byte range, the
bit \chdesc\ stores an xor value for 32-bits. The third type, the
\noop\ \chdesc, does not modify the disk at all. It exists as a
convenient way to refer to a set of \chdescs. When a
\noop\ \chdesc\ depends on a set of other \chdesc, or vice versa, the
entire set will inherit any dependencies or dependents added to the
\noop\ \chdesc. Additionally, at times it may come in handy to temporarily
stop the flow of \chdescs\ through the \module\ graph, such as in the
case of the journal \module. Using a \noop\ \chdesc, stopping it from
being satisfied will also prevent any of its dependents from being satisfied.

