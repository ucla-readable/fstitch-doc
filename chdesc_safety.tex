\documentclass{article}

\usepackage{fullpage}
\usepackage{amsthm}
\usepackage{amssymb}

\newtheorem{thm}{Theorem}[section]
\theoremstyle{definition}
\newtheorem{chrule}{Rule}[section]

\newcommand{\cd}[1]{\ensuremath{#1}}
\newcommand{\cdb}[2]{\ensuremath{\cd{#1}_#2}}

\newcommand{\dcd}[1]{\ensuremath{\cd{#1}^\mathrm{d}}}
\newcommand{\ncd}[1]{\ensuremath{\cd{#1}^\mathrm{n}}}
\newcommand{\rb}[1]{\ensuremath{\cd{#1}^\mathrm{rb}}}
\newcommand{\nrb}[1]{\ensuremath{\cd{#1}^\mathrm{nrb}}}

\newcommand{\drb}[1]{\ensuremath{\cd{#1}^\mathrm{d,rb}}}
\newcommand{\nooprb}[1]{\ensuremath{\cd{#1}^\mathrm{n,rb}}}
\newcommand{\dnrb}[1]{\ensuremath{\cd{#1}^\mathrm{d,nrb}}}

\newcommand{\depends}[2]{\ensuremath{#1\! \rightarrow\! #2}}
\newcommand{\ldepends}[2]{\ensuremath{#1\! \leftarrow\! #2}}
\newcommand{\indirdepends}[2]{\ensuremath{#1\! \leadsto\! #2}}
\newcommand{\cycledepends}[2]{\ensuremath{#1\! \leftrightarrow\! #2}}

\newcommand{\cdset}[1]{\ensuremath{\mathbf{#1}}}

\title{Change Descriptor Optimization Safety}
\author{}
\date{}

\begin{document}
\maketitle

This document's purpose is to prove the safety of kudos change
descriptor optimizations. Safety means not inducing dependency cycles
right now, but later on we may be interested in preserving other
properties (e.g. comitting changes to disk in reasonable time).
%
In this document \textbf{Rule}s describe all permitted change
descriptor operations and \textbf{Theorem}s state their
safety/preservation properties.
%
Don't be fooled that since these proofs are typeset that they are thus
holeless or unquestionable; they are merely enough to mostly convince
me that the theorems are true. Mention anything that seems
questionable to you.

Exciting notation:
\begin{itemize}
\item \cd{A} names an arbitrary change descriptor
\item \dcd{A} changes disk data, \ncd{A} makes no changes itself
\item \rb{A} can be rolled back, \nrb{A} cannot
\item \cdb{A}{k} is on block $k$
\item \depends{\cd{A}}{\cd{B}} shows that \cd{A} directly depends on
  \cd{B}, \indirdepends{\cd{A}}{\cd{B}} shows that \cd{A}, perhaps
  indirectly, depends on \cd{B}
\end{itemize}
\section{Basics}
\label{sec:basics}

Nonrollbackable change descriptors require a restriction of dependency
operations; this section introduces these restricted versions of
existing change descriptor operations.

\subsection{General}
\label{sec:basics:general}

\begin{chrule}[Create \rb{A}]\label{chrule:create}
  \mbox{}\\
  Precondition: \(\cdset{B} \subseteq \mbox{Existing change descriptors}\).\\
  Allowed: Create \rb{A} and \(\forall \cd{B}\! \in\! \cdset{B}\!: \mbox{create}\ \depends{\rb{A}}{\cd{B}}\).
\end{chrule}

\begin{thm}\label{thm:createsafe}
  No change descriptor dependency cycle can arise from
  Rule~\ref{chrule:create}.
\end{thm}
\begin{proof}
  By contradiction.
  %
  For there to be a cycle there must be some change descriptor(s)
  \cd{A} and \cd{B} such that \indirdepends{\cd{A}}{\cd{B}} and
  \indirdepends{\cd{B}}{\cd{A}}.
  %
  By Rule~\ref{chrule:create}, \indirdepends{\cd{X}}{\cd{Y}} implies
  \cd{X} was created after \cd{Y}. Thus \cd{A} existed prior to \cd{B}
  and \cd{B} existed prior to \cd{A}. But this existance ordering is
  invalid.  Therefore there is no way to create a cycle.
\end{proof}

\begin{chrule}[Destroy \cd{A}]\label{chrule:destroy}
  \mbox{}\\
  Precondition: \(\nexists \cd{B}\!: \depends{\cd{A}}{\cd{B}}\).\\
  Allowed: Destroy all \depends{\cd{B}}{\cd{A}} and destroy \cd{A}.
\end{chrule}

\begin{thm}\label{thm:destroysafe}
  No change descriptor dependency cycle can arise from
  Rules~\ref{chrule:create} and~\ref{chrule:destroy}.
\end{thm}
\begin{proof}
  Any application of Rule~\ref{chrule:destroy} reduces a change
  descriptor forest to one creatable using only
  Rule~\ref{chrule:create}. Theorem~\ref{thm:createsafe} proves such
  a forest cannot contain a cycle.
\end{proof}
\subsection{Noops}
\label{sec:basics:noops}

Section~\ref{sec:basics:general} is sufficient for all change descriptor
usage, except for the journal block device and opgroups, which currently
use less restrictive dependency additions.

\begin{chrule}[Create \depends{\ncd{A}}{\cd{B}}]\label{chrule:noop-nc}
  \mbox{}\\
  Precondition: \(\nexists \cd{C}\!: \indirdepends{\cd{C}}{\ncd{A}}\).\\
  Allowed: Create \depends{\ncd{A}}{\cd{B}}.
\end{chrule}

\begin{thm}\label{thm:noop-nc-unsafe}
  Rules~\ref{chrule:create} and~\ref{chrule:noop-nc}:
  \begin{enumerate}
  \item can induce cycles of noops
  \item can be used to make a non-noop depend on a cycle of noops
  \item cannot induce cycles that involve non-noops
  \end{enumerate}
\end{thm}
\begin{proof}
  \mbox{}
  \begin{enumerate}
  \item Example of noop cycle creation: Create \ncd{A}. Add
    \depends{\ncd{A}}{\ncd{A}}, allowed by Rule~\ref{chrule:noop-nc}.
    This creates a cycle of noops.
  \item Given the cycle \depends{\ncd{A}}{\ncd{A}}, create change
    descriptor \dcd{B} with the dependency set \(\{\ncd{A}\}\). This creates
    a \dcd{B} that depends on a noop cycle.
  \item If a \ncd{A} has no non-noop dependents, adding
    \depends{\ncd{A}}{\dcd{B}} cannot add a dependency on \cd{\dcd{B}}
    to any non-noop. This is the only operation allowed by
    Rule~\ref{chrule:noop-nc}, therefore the rule cannot be used to
    induce cycles that involve a non-noop.
  \end{enumerate}
\end{proof}

\begin{chrule}[Create \depends{\dcd{A}}{\ncd{B}}]\label{chrule:noop-cn}
  \mbox{}\\
  Precondition: \(\nexists \dcd{C}\!: \indirdepends{\ncd{B}}{\dcd{C}}\).\\
  Allowed: Create \depends{\dcd{A}}{\ncd{B}}.
\end{chrule}

\begin{thm}\label{thm:noop-cn-unsafe}
  Rules~\ref{chrule:create} and~\ref{chrule:noop-cn}:
  \begin{enumerate}
  \item can be used to make a non-noop depend on a cycle of noops
  \item cannot induce cycles
  \end{enumerate}
\end{thm}
\begin{proof}
  \mbox{}
  \begin{enumerate}
  \item Example: Given \dcd{A} and the noop cycle \depends{\ncd{B}}{\ncd{B}},
    Rule~\ref{chrule:noop-cn} allows the dependency addition
    \depends{\dcd{A}}{\ncd{B}}.
  \item By contradiction.
    %
    Suppose we use Rule~\ref{chrule:noop-cn} to add
    \depends{\dcd{A}}{\ncd{B}}. For this to induce a cycle, it must
    already be that \indirdepends{\ncd{B}}{\dcd{A}}. But this violates
    the application of Rule~\ref{chrule:noop-cn}.
  \end{enumerate}
\end{proof}

\begin{thm}\label{thm:noop-cnnc}
  Rules~\ref{chrule:create}, \ref{chrule:noop-nc},
  and~\ref{chrule:noop-cn} cannot be used to create a cycle involving
  non-noops.
\end{thm}
\begin{proof}
  By assumption.
\end{proof}

\section{Nonrollbackable Change Descriptors}

After bdescs, kudos's largest memory usage goes to byte change
descriptors' copy of the previous data. Many change descriptors are
never rolled back, however. This section introduces rules that detect
that a change descriptor will never need to be rolled back, allowing
its block data copy to be omitted (rendering the change descriptor
nonrollbackable).

\begin{chrule}[Create \cdb{\dnrb{A}}{k}]\label{chrule:nrb}
  \mbox{}\\
  Precondition: \(\cdset{B} \subseteq \mbox{Existing change descriptors}\).\\
  Precondition: \(\forall \cdb{B}{k}\!: \nexists \cdb{C}{l}, k \neq l\!: \depends{\cdb{C}{l}}{\cdb{B}{k}}\).\\
  Allowed: Create \cdb{\dnrb{A}}{k},
  \(\forall \cd{B}\! \in\! \cdset{B}\!: \mbox{create}\ \depends{\cdb{\dnrb{A}}{k}}{\cd{B}}\),
  and \(\forall \cdb{B}{k}\!: \mbox{create}\ \depends{\cdb{B}{k}}{\cdb{\dnrb{A}}{k}}\).\\
\end{chrule}

Rule~\ref{chrule:nrb} can induce cycles, e.g. two nonrollbackables on
a block must each be commited no later than the other. This is OK;
avoiding inter-block dependency cycles is what actually matters to
preserve dependencies. Kudos's nonrollbackable implementation does not
explicity state nonrollbackable cyclic dependencies, it is simpler to
change the revision slice code to not allow revisions that do not
include all nonrollbackables on a revision's block. Proving that
Rule~\ref{chrule:nrb} is safe is easiest if nonrollbackables
dependencies are explicity, however. This means that for each
nonrollbackable on a block, all other change descriptors on that block
depend on the nonrollbackable.

Note that this rule does not itself make change descriptors later
created on the same block depend on nonrollbackables. This is allowed
via the dependency set but its enforcement is not needed to consider
dependency cycles.

\begin{thm}\label{thm:nrbsafe}
  No inter-block change descriptor dependency cycle can arise from
  Rules~\ref{chrule:create} and~\ref{chrule:nrb}.
\end{thm}
\begin{proof}
  By induction on the number of change descriptors.
  %
  Notice that Rule~\ref{chrule:nrb} has no effect on dependency cycles
  unless there is a nonrollbackable on some block (else same as
  non-nonrollbackable case) and at least a second change descriptor on
  that same block (else nonrollbackables change nothing).
  %
  Further, there must be at least one change descriptor on another
  block for there to be an inter-block cycle. This is thus the base
  case.
  \begin{itemize}
  \item \textbf{Base case}\\
    We consider three change descriptors. WLOG from the above, these
    are \cdb{\dcd{A}}{1}, \cdb{\dcd{B}}{1}, and \cdb{\dcd{C}}{2}
    (TODO: consider noops?).
    %
    Notice that creating a change descriptor with dependencies on all
    existing change descriptors, as opposed to some subset, allows for
    any cycle that could exist among these change descriptors to
    exist. Therefore there are three cases to consider, one for each
    of the three unique creation orderings:
    \begin{enumerate}
    \item Case: create \cdb{\dcd{A}}{1}, \cdb{\dcd{C}}{2},
      then \cdb{\dcd{B}}{1}. This yields
      \ldepends{\cdb{\dnrb{A}}{1}}
               {\depends{\depends{\cdb{\drb{B}}{1}}
                                 {\cdb{\dnrb{C}}{2}}}
                        {\cdb{\dnrb{A}}{1}}}.
    \item Case: create \cdb{\dcd{C}}{2}, \cdb{\dcd{A}}{1},
      then \cdb{\dcd{B}}{1}. This yields
      \ldepends{\cdb{\dnrb{C}}{2}}
               {\depends{\cycledepends{\cdb{\dnrb{A}}{1}}
                                      {\cdb{\dnrb{B}}{1}}}
                        {\cdb{\dnrb{C}}{2}}}.
    \item Case: create \cdb{\dcd{A}}{1}, \cdb{\dcd{B}}{1},
      then \cdb{\dcd{C}}{2}. This yields
      \depends{\cdb{\dnrb{C}}{2}}
              {\cycledepends{\cdb{\dnrb{A}}{1}}
                            {\ldepends{\cdb{\dnrb{B}}{1}}
                                      {\cdb{\dnrb{C}}{2}}}}.
    \end{enumerate}
    This creates no inter-block dependency cycles.
    \item \textbf{$n$ change descriptors}\\
      Given: a change descriptor forest with $n-1$ change descriptors with no
      inter-block cycles.\\
      Show: creating a change descriptor \dcd{A} cannot induce an inter-block
      cycle.
      \begin{enumerate}
      \item Case: \cdb{\dnrb{A}}{k}, where there are no existing
        change descriptors on block $k$.\\
        \cdb{\dnrb{A}}{k} depends on the set of all existing change
        descriptors.  The existing change descriptors had no
        inter-block cycles and the only added dependencies are from
        the dependent-less \cdb{\dnrb{A}}{k} to the elements in this
        set, so no inter-block cycles are created.
      \item Case: \cdb{\dnrb{A}}{k}, where there are existing change
        descriptors on block $k$ and none have dependents.\\
        Block $k$ has no dependents, thus no inter-block cycles can be
        induced.
      \item Case: \cdb{\dcd{A}}{k}, where there are existing change
        descriptors on block $k$ and at least one of them has a dependent.\\
        \cdb{\dcd{A}}{k} has no dependents, thus no cycles can be induced.
      \end{enumerate}
  \end{itemize}
  By induction on the number of change descriptors, Rules~\ref{chrule:create}
  and~\ref{chrule:nrb} cannot be used to create an inter-block dependency
  cycle.
\end{proof}

\begin{thm}\label{thm::nrbnoopsafe}
  Rules~\ref{chrule:create}, \ref{chrule:noop-nc}, \ref{chrule:noop-cn},
  and~\ref{chrule:nrb} cannot be used to create an inter-block dependency
  cycle that involves non-noops.
\end{thm}
\begin{proof}
  Todo.
\end{proof}

\begin{thm}\label{thm::nrbimpsafe}
  The Kudos nonrollbackable implementation cannot create cycles at all.
\end{thm}
\begin{proof}
  Todo.
\end{proof}

\section{Change Descriptor Merging}
Todo.
\end{document}