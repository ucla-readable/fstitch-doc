\section{Soft Updates}
\label{sec:softupdates}

Ganger, McKusick, Soules, and Patt proposed a system called ``Soft
Updates''~\cite{ganger00soft} for solving the problem of metadata
consistency in filesystems. Instead of requiring a long consistency
check upon recovery from a crash, Soft Updates uses disk write
ordering in order to make certain guarantees about the state of the
filesystem on disk that allow safe use of the filesystem immediately.

In the implementation of Soft Updates in 4.4BSD, metadata changes to
the filesystem are stored in a custom dependency graph which is
consulted by the disk cache when it wants to write a block to disk.
Any metadata changes in a disk block that have unmet dependencies
(that is, they must not be written until some other block has been
written, and that block has not yet been written) are rolled back
before writing the block, and then rolled forward again after the
block is written. These metadata changes are stored in terms of the
file operation that caused them, and in some cases they actually cause
the ``second half'' of the operation to happen upon being written (for
instance, the inode usage count is decreased only after a nullified
directory entry is written to disk when a file is deleted).

We wanted to support a system like soft updates in our design, but not
limit it to a single filesystem's metadata issues. Further, we wanted
to be able to implement other order-sensitive filesystem features like
journalling, and use a single order-aware cache to handle all such
cases. Our design therefore calls for a generalized ``change
descriptor'' structure that describes an arbitrary change to a disk
block, and which can have other change descriptors as dependencies.
Using this system, we can achieve the same effect as Soft Updates, and
even allow our change descriptors to cross block translation modules
like the partitioner or the loopback device so that the underlying
disk is updated in the order needed by the top-level filesystem.

We plan to implement Soft Updates using change descriptors for a
simple filesystem, and demonstrate that minimal changes are actually
required in the filesystem code to effect this. Most of the work will
be in the cache, and this will be completely reusable for different
filesystems, or different styles of dependencies (like journalling,
where change descriptors can optionally depend on blocks on a
different block device to allow external journals without requiring
synchronous writes).
