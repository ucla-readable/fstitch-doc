\paragraph{Change Descriptors}
\label{sec:chdescs}

In contrast with traditional systems, where changes to filesystem data are
accomplished by changing in-memory copies of the disk blocks and then marking
the block as needing to be written to disk (thus losing all record of what was
changed), every change to filesystem data in the KudOS file server effects a
corresponding ``change descriptor'' that describes what was changed. Change
descriptors (``chdescs'' for short) allow a dependency graph to be created
describing which changes ``depend'' on which other changes (i.e. which changes
must be written to disk after which other changes), as well as allowing
individual changes to be undone and redone. These two capabilities allow any
module in the system to inspect and even modify the changes that other modules
are making.

The ability to revert and re-apply chdescs is inspired by the ``soft updates''
system in BSD's FFS~\cite{ganger00soft}, but it is generalized so that it is not
specific to any particular filesystem. A chdesc can describe a change as small
as a single bit, or as large as a whole disk block. Soft updates, journalling,
and many application-specific consistency models all correspond to different
change descriptor arrangements. Block data is almost secondary to the chdescs
from the point of view of the file system server -- chdescs are what really move
around in the system. This concept turns out to be very powerful, and it allows
some interesting configurations of modules as well as very simple
implementations of traditionally complicated features. For instance, journalling
is implemented as a single BD module, and it can automatically add journalling
-- even metadata-only journalling -- to {\it any} LFS filesystem. Other block
device layering systems, like GEOM~\cite{geom}, need special hooks into
filesystem code in order to get the necessary hints (i.e. what is metadata and
what is not) to do metadata-only journalling\footnote{Check this claim}. Change
descriptors and the LFS/CFS division allow us to do this automatically.

For many complex operations, like RAID or journalling, the chdesc graph is
changed in specific ways which can be broken down into sequences of simple graph
transformations. For example, in the (mirroring) RAID module it is necessary to
duplicate a change descriptor, including all its dependents and dependencies. We
expect that after implementing a relatively small number of more complex
modules, we will have seen similarities in many of the transformations and will
have built up a fairly complete library of common transformation functions.
