\section{Filesystem Module Interfaces}
\label{sec:interfaces}

Traditional filesystem software involves two interfaces, the
filesystem (eg VFS) and the block device. With current filesystem
implementations ranging from tens of thousands to millions of lines of
code and many operating systems containing support for multiple
filesystems, we have focused on modularizing the filesystem layer into
smaller components to increase reusability, flexibility, and
understandability. Our research studies dividing filesystem software
interfaces into three types:

\begin{itemize}
\item \itemvspace{} {\bf Common File System} (CFS): Resembles Unix's
  VFS or the stackable filesystem semantics supported by
  FiST~\cite{zadok00fist}. Provide methods to manipulate files and
  arrays of bytes within files.
\item \itemvspace{} {\bf Low-level File System} (LFS): Abstract the
  on-disk representations of blocks, provide methods to allocate
  blocks, append blocks to files, allocate file names, retrieve amount
  of free disk space, and similar operations.
\item \itemvspace{} {\bf Block Device} (BD): Resembles Unix's block
  layer. Provide methods to manipulate an array of blocks.
\end{itemize}
\postlistspacing{}

Using these interfaces one is able to create components such as the
Uniform High-level File System (UHFS), which converts CFS-level
operations to the LFS-level. UHFS implements common filesystem
functionality such as file write, read, append, and truncate in a
filesystem-agnostic way. Thus it is LFS that defines what is treated
as filesystem representation specific and allows the simplification of
filesystem implementation by splitting a filesystem into the
filesystem-layout specific module and the UHFS module.
