\section{Filesystem Software Interfaces}

Traditional filesystem software involves two interfaces, the
filesystem (VFS) and the block device. With current filesystem
implementations being in the hundreds to millions of lines of code and
many operating systems containing implementations of multiple
filesystems, we have focused on modularizing the filesystem layer into
smaller components to increase reusability, flexibility, and
understandability. Our research studies dividing filesystem software
interfaces into three types:

\begin{itemize}
\item \textbf{Common Filesystem} (CFS): Resembles Unix's VFS or the
  stackable filesystem semantics supported by FiST~\cite{fist}.
\item \textbf{Lowlevel Filesystem} (LFS): Abstract the on-disk
  representations of blocks, provide methods to allocate blocks,
  append blocks to files, allocate file names, retrieve free disk
  space, and similar operations.
\item \textbf{Block Device} (BD): Resembles Unix's block layer. BD
  provides methods to manipulate an array of blocks.
\end{itemize}

Using these interfaces one is able to create components such as the
Uniform High-level File System (UHFS), which translates CFS-level
operations to the LFS-layer. UHFS implements common filesystem
functionality such as file write, read, append, and truncate in a
filesystem-agnostic way.\footnote{Should we give more examples here?}
