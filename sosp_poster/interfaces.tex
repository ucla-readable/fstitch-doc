\section{Filesystem Module Interfaces}
\label{sec:interfaces}

Traditional filesystem software involves two interfaces, the abstract filesystem
and the block device. With current filesystem implementations ranging from tens
of thousands to millions of lines of code and many operating systems containing
support for multiple filesystems, one of our aims has been to modularize the
filesystem layer into smaller components to increase reusability, flexibility,
and understandability. We keep block devices (BD) and the high-level filesystem
(``Common File System'', CFS), but we add an additional interface between these
two that helps to divide filesystem implementations into common (i.e. reusable)
code and filesystem-specific code. We call this intermediate interface the
``Low-level File System'' (LFS).

The LFS interface has functions to allocate blocks, add blocks to files,
allocate file names, and other filesystem micro-ops. The idea is that a module
implementing the LFS interface should define how bits are laid out on the disk,
but not have to actually know how to combine the micro-ops into larger, more
familiar filesystem operations.

Using these interfaces one is able to create components such as the Uniform
High-level File System (UHFS), which converts CFS-level operations to the
LFS-level. UHFS implements common filesystem functionality such as file write,
read, append, and truncate in a filesystem-agnostic way. Thus it is LFS that
defines what is treated as filesystem representation specific and allows the
simplification of filesystem implementation by splitting a filesystem into the
filesystem-layout specific module and the UHFS module.

All three types of modules can be stacked to add functionality incrementally,
similar to the semantics provided by FiST~\cite{zadok00fist}.
