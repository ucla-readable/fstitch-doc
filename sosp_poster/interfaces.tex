\preparagraphspacing{}
\paragraph{Filesystem Module Interfaces}
\label{sec:interfaces}

Traditional filesystem software involves two interfaces: the ``abstract
filesystem'' (like VFS), and the block device. With current filesystem
implementations ranging from tens of thousands to millions of lines of code
and many operating systems containing support for multiple filesystems, one of
our aims has been to modularize the filesystem code into smaller, stackable
components to increase reusability. To this end, we add an additional
interface that helps to divide filesystem implementations into common (i.e.
reusable) code and filesystem-specific code. We call this intermediate
interface the ``Low-level File System'' (LFS).

The LFS interface has functions to allocate blocks, add blocks to files,
allocate file names, and other filesystem micro-ops. The idea is that a module
implementing the LFS interface should define how bits are laid out on the
disk, but not have to actually know how to combine the micro-ops into larger,
more familiar filesystem operations. We have implemented a generic VFS to LFS
module that decomposes the larger file write, read, append, and other standard
operations into LFS micro-ops.
