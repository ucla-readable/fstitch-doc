\section{Semantic File System}
\label{sec:sfs}

Gifford, Jouvelot, Sheldon, and O'Toole's Semantic File
System~\cite{jouvelot91semantic} provides field-value access to file
information through a file system interface. Their works explores the
thesis ``\textit{semantic file systems} present a more effective
storage abstraction than do traditional tree structured file systems
for information sharing and command level programming.''

An example of a field-value view, given that the semantic file system
is mounted at /sfs/, is /sfs/owner:/smith/ containing all files owned
by user smith. SFS provides this associative access to a system's
contents by automatically extracting associative attributes
(field-value pairs) using type specific \textit{transducers}. They
describe five example transducers, for New York Times articles (type:,
date:, subject:, etc), object files (exports: and imports:), the
programming languages C, Pascal, and Scheme (exports: and imports:),
mail (from:, to:, subject:, and text:), text files (text:), and
file system metadata (owner:, group:, name:, etc). Associate access is
integrated into the tree structured view of file systems using
\textit{virtual directories}, which are interpreted as queries
describing the desired attributes, such as /sfs/owner:/smith/ in the
above example. Rapid access to contents is implemented using a static
file index.

In relation to KFS, SFS can be seen as an example use of the classifier
concept, operating as a CFS-CFS block. In fact, operating as a classifier
block allows SFS three benefits:
\begin{enumerate}
\item SFS is no longer restricted to integrating with the file
  hierarchy through /sfs/. For example, given a mail file
  /home/smith/mail, /home/\-smith/\-mail/\-from/\-kohler/ could
  contain all mails from kohler. Semantic information could thus be
  seen as an extension of existing directories and files, perhaps
  helpful in both finding and using semantic information, and scaling
  the SFS interface to larger numbers of users and/or files by
  allowing users' file layouts to layout file semantic information.
\item As individual environments also use the CFS interface,
  environment, user, and group specific transducers are natural
  extensions. Further, such extensions come without a need to modify
  the machine or network-wide SFS server.
\item SFS' implementation and administration can be simplified. The
  authors of SFS found implementing the SFS server program as an NFS
  server to be the easiest way to integrate SFS with the existing file
  hierarchy. Implementing SFS as a classifier block means SFS need
  only implement the CFS interface, not a more complicated NFS server.
  This also easily allows exporting SFS interfaces using any existing
  file system interface, not just NFS.
\end{enumerate}
