\section{Stackable Filesystem Works}

The FiST project at Columbia describes a system for stackable file
systems. The goal of the project was to allow programmers the ability
to easily add one or many features to existing filesystems. When many
additions are added, they are stacked. For maximum utility, the
modifications are designed to be platform independent, but are still
able to take advantage of platform-specific features. Additionally,
the modifications are written in a special language that facilitates
ease of development, freeing the developer from wading through
thousands of lines of C code.

There are a few components to FiST that allow it to fulfill these
goals. First of all, glue code has been written that exposes a place
in the kernel where stackable, FiST produced filesystems can reside
and operate on the vnode layer. This glue was ported to three
operating systems (Solaris, FreeBSD, Linux).

The next component is a filesystem generator. This generator accepts
code written according to the FiST grammar. This code contains a few
sections: C declarations, FiST declarations, FiST rules, and
additional C code. C and FiST declarations allow the programmer to
define data structures. The heart of the filesystem is in the FiST
Rules, section, though. Each vnode operation has three rules. The
first rule is code that is executed in the filesystem before the call
to the underlying filesystem is made. The second rule is code that is
executed after the call to the underlying filesystem is made, and the
final rule is any code that can execute in place of the underlying
filesystem call. By default, the first two rules do nothing, and the
third simply calls the underlying filesystem. Thus, all aspects of a
call at the vnode level can be altered. The final section, additional
C code, contains extra C functions for the filesystem.

The stackable file system lives in the kernel and resides between the
VFS layer and the base filesystem layer, intercepting calls that the
VFS layer makes to the underlying filesystem driver.

The benefits of FiST are most apparent to developers who wish to make
small changes to an existing filesystem. An observation from the FiST
paper is that many programmers, when they need a custom filesystem,
develop their own derivative of an existing filesystem. They must then
worry about keeping up with software updates of original software they
modified, and the software around it. Since FiST presents a consistent
interface, the developer only needs to worry about being compatible
with FiST.

FiST then, is operating on a far different scale than our
filesystem. FiST sits between VFS and the underlying filesystem driver
on a normal UNIX system. This closely resembles the area around our
element that converts from CFS (a very high level interface) to LFS (a
very low level interface) in KudOS, since CFS is higher level than
standard VFS and LFS is lower level than standard VFS. The stackable
filesystems that FiST generates could be used in a KudOS-like
environment: two elements would replace the existing CFS to LFS
conversion node and expose a vnode-like interface in the middle of the
two new elements. FiST can then stack filesystems here.

The vnode interface, while a convenient location to allow many
programmers the ability to easily implement filesystem changes, also
has some drawbacks. One drawback is that the vnode layer can only
operate on fixed blocks of data in files. As such, it can modify file
data but not the size of the data, which makes a compression system
impossible. Also, it cannot affect the order of lower level
operations, making a soft updates layer impossible.

Another related work, a precursor to FiST, is File System Development
with Stackable Layers. It describes a similar vnode layer to the FiST
paper. One unique idea this paper gives is the ability to have a vnode
layer be split between two machines, and pass the calls over an RPC
interface. Again, work is operating on a very different scale from
ours, and could be implemented in our work in the same manner as FiST.
