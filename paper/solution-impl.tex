\subsection{Implementation}
\label{sec:solution:impl}

(talk about how we did some speed optimizations already somewhere in this file)

intro paragraph, we implemented these things... blah blah

\subsubsection{KPL}

must integrate user level file descriptor interface. (why this is
warranted). provides capabilities page.

\subsubsection{UHFS}
\label{sec:solution:impl:uhfs}

If the CFS layer is akin to C and the LFS layer to assembly, the UHFS
(Uniform High level File System) module is a CFS to LFS interpreter. UHFS
is intended to be the common CFS-LFS module for all file systems with an LFS
interface, this includes josfs\_base and does not include devfs\_cfs.
UHFS's CFS method implementations take care of non-block aligned accesses,
lower level operations such as block allocation during write, and 
orders inter-LFS method call chdesc dependencies.

uhfs\_write() provides an example of a typical UHFS method. UHFS's
client passes an open file's fid, data to write, and where in the file
this data belongs. uhfs\_write() looks up the fid's associated LFS
fdesc and if the below LFS supports file sizes, looks up the file's
current size in bytes.  uhfs\_write() then writes the data a block at
a time to the below LFS, allocating and then appending blocks as the
file needs enlargement. Once all data is written and if the below LFS
supports the file size feature, uhfs\_write() updates the file size
metadata.

\subsubsection{josfs\_base}

in principle, only module in the system which needs to know about hwo
the bytes are arranged on the disk. in fact, that's all it knows. also
must order writes, generate change descriptors, possible chain them
up. this module implements the low level operations that UHFS requests
(e.g. allocate a block).

\subsubsection{josfs\_cfs Legacy Module}

In the beginning of the KudOS file server's development we wanted to
begin testing CFS modules, CFS RPC, and KPL before the LFS and BD
layers were ready for use. The josfs\_cfs module allowed this testing
by providing a full file system, using JOS's existing file server
daemon as a back-end. josfs\_cfs's simplicity also served as a first
validation that the CFS interface is flexible enough to host new and
helpful uses.

\subsubsection{Reliable File System Features}

soft updates is more pervasive. because of this, we cna't just have a
moduel to do it. we need clues (a must happen beofre b). josfs base
and uhfs cooperate to give us the hints/clues (actually dependencies)
to hookup the chdecs into a graph that would comply w/ soft
updates. (wb cache make sure this is honored).

diagram that shows journal format. full data only. uses Q to intercept
things. unlike ext3 we have a bandwidth limitation which we could (?)
work around. very similar to ext3. fs doesn't know ti's being
journaled. it's a xaction layer on top of BD. guaranatess sets fo
writes are atomic. each high level cfs op is atomic now. maybe sets
clumped together, but a cfs call is never broken apart.

\subsubsection{wb\_cache}

makes sure that dep graph is satisifed

direct mapped cache. talk about algo used to evict blocks. the algo is
not optimal. in practice it is optimal, but in practice we don't have
very complex graphs. produces ooutput that's serialized and safe for
writing to stable storage.

design philosophy: can't stack the wb cache. it has to know that
writes go to disk in the correct order. sync() to make sure stuff goes
to disk at the correct order.

sure, stacking wb cache seems cool, but why not just put a wb cache
above a disk and call it a day? floating deps down gives no benefit.

you could probably do a wb cache that doesn't honor deps but depends
on this wb cache to honor them. but aht's a problem sucking blocks
down.

\subsubsection{RAID}
\label{sec:solution:impl:raid}

lets you strip or mirror reads/writes to 1 or 2 disks. mirroring
driver can handle disk failure and go into degraded mode where it's a
pass thru. also mirroring can do what geom did where you hot swap in a
new disk, syn, ditch the old disk (w/ help of modman and userland
utilities).

\subsubsection{Loop Device}
\label{sec:solution:impl:loop}

take a file and makes it look like a block device. depds go in and
right out. forwarded by depman.

\subsubsection{Network Block Device}

The network block device is extremely simple. It uses a straightforward
serialization of the BD interface over a TCP connection. During initialization,
it receives the block size and number of blocks from the server. For each read
request, it sends a read command and a block number to the server, and waits for
the block to be returned. For a write request, it sends a write command, block
number, and the block data. Both the client and server (which can run on either
a POSIX system or KudOS itself) required only a few hours to develop and test,
and they fit right into the rest of the system like any other block device.

\subsubsection{Online Configuration}
\label{sec:solution:impl:online}

The modman component is central to most online configuration and
introspection, providing existence, usage, configuration, and status
information for CFS, LFS, and BD module instances. Each module instance
registers/unregisters itself with modman at creation/destruction time
and registers/unregisters module instance usage, modman stores this
information to respond to others' queries.

KFS RPC is implemented in a similar manner as CFS RPC, a serialized
KFS is used to communicate between KFS server and clients via KudOS
IPC message passing. Each kfs method is re-implemented as an rpc stub,
allowing an object file to be linked to either the KFS IPC Client or
Server library. Client programs are thus able to reconfigure the kfsd
environment using the same methods that would be used within kfsd.

The program kfsgraph uses KFS RPC to construct a module usage graph,
showing module instance existence, which instances use which others,
and instances' configuration and status. kfsgraph can output this
graph to text or to AT\&T's graphviz dot
format. Section~\ref{sec:eval} shows examples of kfsgraph's output.

The programs mount and umount, named for their similarity to the
Unix tools of the same names, provide a general case, easy to use
configuration interface. Given a BD type of ide, nbd, loop, or an existing
BD and a mount point, mount constructs the given base BD, caches and block
resizers as appropriate, a josfs\_base or wholedisk as appropriate, uhfs,
and connects these to the table classifier at the given mount point. mount
optionally allows specification of whether journaling is to be used or not
(and where to journal to), whether to fsck the filesystem, how large a cache
to use, and whether to use a write back or write through cache. umount takes
a mount point and destructs all modules down the chain that are no longer
in use.
