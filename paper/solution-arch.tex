\subsection{Architecture}
\label{sec:solution:arch}

The KudOS file server is composed of many modules, each of which may use the
services of other modules and provides a service of its own for other modules.
There are three interface types between modules, called BD (Block Device), LFS
(Low level File System), and CFS (Common File System).

In addition to the modules, there are also a few server-wide components that are
not modules. They provide management functions for the modules themselves and
the data structures that are used to store disk blocks and other file system
data, like change dependency information and client file IDs.

\subsubsection{Interface Layers}
\label{sec:solution:arch:layers}

The modules that make up the file system server are generally divided into four
major layers in the simple case. At the lowest layer, there are modules that
provide a BD interface. They can be disks, partitions on other block devices,
caches, network block devices, RAM disks, and so on. The BD interface contains
functions for reading and writing blocks, as well as querying the block size and
number of blocks. Although the interface is simple, there is hidden complexity
in some types of block device modules when dealing with change descriptors (see
section \ref{sec:solution:arch:structs}).

The next layer starts with modules that use BD modules and provide LFS
interfaces. It includes any other modules which are LFS-LFS modules, that is,
modules which use an LFS interface and provide another LFS interface. The LFS
interface provides extremely low-level file system operations, sort of like the
``assembly language'' of file system operations. There are functions to allocate
and free disk blocks and file names, to append and remove disk blocks from file
names, and to query for metadata associated with files in an extensible way. It
is possible, using the LFS interface, to effect changes to the underlying
file system that are not ordinarily allowed -- for instance, assigning the same
disk block to multiple files.

The third layer starts with modules that use an LFS module and provide CFS
interfaces. It includes any other modules which are CFS-CFS modules, as well as
the IPC mechanism which allows client environments to access the top-level CFS
modules in the file server. The CFS interface is similar to the ``VFS''
interface found in many Unix systems, but a little bit lower-level: instead of
providing a specific set of supported metadata, it provides an extensible
mechanism for using file metadata. In comparison to LFS, which provides only
block-level operations for the file system, and which can be used incorrectly
with undesired results, CFS provides safe byte-level operations on files.

The last layer is in the client environment, and provides whatever native
interface the user code expects. In KudOS, which is originally based on a simple
exokernel system called JOS, the original file server provides an interface
which we emulate in the ``KudOS Presentation Layer'' or KPL. The KPL provides
high-level functionality like the stat() call by using the lower-level CFS
metadata query functions, and is nominally 100\% reverse-compatible with the old
file server.

\subsubsection{Structures}
\label{sec:solution:arch:structs}

There are three major data structures in the KudOS file server. They are block
descriptors (bdesc), data descriptors (ddesc), and change descriptors (chdesc).
Block descriptors represent a disk block. They store a pointer to the block
device they came from, the block number, a pointer to the block data, and a
reference count. Data descriptors represent the actual data in a disk block,
independent of any particular bdesc. This allows the data to be shared between
several bdescs, which is an optimization to avoid unnecessary copying in many
situations. Finally, change descriptors represent changes to disk data. They
store the old and new data, which block they changed, and lists of other chdescs
which depend upon them or which they depend on.

\subsubsection{Dependency Management}
\label{sec:solution:arch:depman}

When a change descriptor is created, it gets registered with the dependency
manager. The dependency manager keeps track of all outstanding change
descriptors, and what blocks (or block descriptors, really) they change. As the
block descriptors are written from one block device to another (for instance, as
they pass through caches, partitions, or RAID), the dependency manager tracks
their associated change descriptors and automatically forwards them from block
descriptors at one level of the graph to the next. A few types of block
operations require explicit hints to the dependency manager for it to track the
change descriptors correctly, like block resizer changes, but most of the
operations that the dependency manager does are automatically invoked by the
block descriptor reference counting code. When a change descriptor is finally
satisfied by being written to disk, the dependency manager updates all the other
change descriptors that depended upon it, so that they will know that their
dependency has been satisfied. Then the change descriptor is freed.

\subsubsection{Online Configuration}
\label{sec:solution:arch:online}

% This paragraph is set to 78 columns because it looks much nicer in the TeX that way

The KudOS file server provides services allowing for online configuration and
introspection through three means. The modman component (module manager)
tracks the existence of module instances and which module instances use
another module instance. Many modules provide ``out of band'' functions for
changing a module instance's configuration; for example, the mirror block
device (\S~\ref{sec:solution:impl:raid}) has functions to add and remove block
devices from the mirror. The KudOS file server also provides remote procedure
calls for all such out of band and modman functions, allowing client programs
to mount new file systems or inspect the KudOS file server's current state,
for example.

\subsubsection{Client Tracking}
\label{sec:solution:arch:clients}

Each file opened by a client has an associated FID (File ID) that the client
then uses to perform operations on the open file. CFS modules use FIDs to
associate internal information with client requests. For example, the table
classifier (a CFS-CFS module similar to a Unix mount table) associates a FID
with its corresponding CFS module instance. These FIDs need to be unique across
all modules, and across time to make bugs easier to detect. The fidman component
(FID manager) generates unique FIDs. The two ``fidfairies,'' fidcloser and
fidprotector, are CFS-CFS modules. Fidcloser determines when a FID can be
closed, taking into account client dup()s, fork()s, spawn()s, and death, while
fidprotector prevents unauthorized access to FIDs by clients.
