\section{Related Work}
\label{sec:related}

\subsection{Linux JFS and ext3}

In Linux, journalling in the ext3 filesystem is implemented as two distinct
modules that cooperate: JFS and ext3~\cite{tweedie98journaling}. The JFS module
does not know anything about filesystems, and the ext3 module does not know
anything about journalling. What the JFS module does is to provide transaction
support for block devices. That is, it provides the facility to tell the block
device what writes will be part of what transaction, and when transactions are
to be committed, and it makes sure that either the whole transaction is
committed, or none of it. It does this by keeping a journal, but this is not
part of the interface specification.

The ext3 module uses transactions to accomplish its safety properties. It groups
changes to the filesystem into transactions that guarantee the disk will always
be in a consistent state, and lets the JFS module do the hard work. It keeps a
flag in the filesystem superblock that indicates whether it has been cleanly
unmounted, and upon mounting a dirty filesystem, it signals the JFS module to do
any necessary recovery before it proceeds. In this way, the complexity of
journalling is broken into two separate pieces.

Full journalling is actually not very difficult to implement. It is what we have
implemented in the KudOS file server. In full journalling, everything that will
be written to the disk is written to the journal first, as part of some
transaction. Obviously this is much less efficient than we'd like, which is why
there is another technique for doing journalling: metadata-only journalling. In
metadata-only journalling, only changes to the filesystem metadata (like
directory structure, file names, sizes, and link counts) are grouped into
transactions and written to the journal. File data changes are written directly,
since they do not contribute to the validity of the filesystem and do not need
to be made ``safe'' in order to avoid needing fsck. Although it seems simple
initially, metadata-only journalling introduces a large number of special cases
involving resource reuse which greatly increase its complexity and the necessary
implementation time.

\subsection{BSD Soft Updates}

The BSD implementation of soft updates \cite{ganger00soft} is specific to BSD's
Fast File System (FFS), and uses dependency structures that know what filesystem
structures they are representing alterations to. As a result, it can do a few
tricks to optimize its behavior. For instance, when two links to an inode are
deleted, each gets an unlink structure that decrements the usage count of the
inode when the change is written to disk. Either unlink may be written to disk
first, because the change to the inode is actually calculated and made when the
directory entry change is written, not when it is requested. In the KudOS
implementation of soft updates, changes are created when requested, and written
passively at a later time. The second change to the inode's usage count would
have to depend on the first, as although each logically does the same thing
(decrement the usage count), they actually do different things (one writes $n -
1$, the other $n - 2$ as the usage count). The changes would then
correspondingly have to be written in the same order as they were requested.

To do something like the BSD implementation, we would have to allow change
descriptors, which are currently passive data structures, contain enough
information to describe some action that will be taken when it is written to
disk, or that must happen before it is written to disk. This could be done, but
we decided to take the more general approach that we did in order to minimize
the complexity of the already rather complex write-back cache, and to make the
system more efficient when dealing with other uses of dependency information
like journalling.

\subsection{Stackable File Systems}

Stackable File Systems \cite{heidemann93stack} attempts to make file system
development easier by providing an extensible file system interface. With such
an interface, developers can add features to existing file systems. These new
features can then be used together by stacking them on top of each other.
Existing stackable file systems implementations, such as FiST
\cite{zadok00fist}, start with a ``basefs'' module that interfaces with the
pre-existing file system and exposes an extensible interface for use with
stacking modules. This base file system needs to be ported to every operating
system it runs on. For FiST, they included a description language for writing
stackable modules, so that modules written in the FiST language can easily
generate code for different platforms.

While Stackable File Systems promise faster development time with smaller code
size, there seem to be limitations to what a Stackable File System can do.
Due to the fact that modules can only be added in the space between basefs and
the VFS layer, Stackable File Systems does not have the same level of
flexibility as a fully componentized file system. From the examples in FiST, it
appears trivial to add encryption or ACL to an existing file system, but it is
not clear how a system such as FiST can implement more sophisticated features
like journaling or soft updates.

With the KudOS File System, we applied the idea of stacking to all interface
layers. By having modules that use the same interface on both ends, it is easy
to attach modules together. Many KudOS modules can be arbitrarily arrange, and
have little dependencies on other modules. The lack of complex dependencies also
make individual components easy to write. In comparison, KudOS has a much larger
scope that ranges from the userspace interface to the block devices, compared to
Stackable File Systems which are confined to a small area between VFS and the
underlying file system.

\subsection{BSD GEOM}
\label{sec:related:geom}

they did the hot-swap that we can do w/ modules. we can't hot-swap
modules w/ state. problem for cfs moduels and some lfs ones. even bd
modules have some state (point to other bd modules). we have what we
think is important (raid thingy).

\subsection{Flux OSKit}

The Flux OSKit \cite{ford97oskit} provides a framework for constructing
operating systems. It breaks up the operating system into modules, and provides
a library of pre-made modules. This way researchers can concentrate on the more
interesting components in their OS, like the scheduler or the VM, and use
pre-existing modules for the more mundane components, like the boot loader or
device drivers.

One interesting aspect of Flux is the use of components from other operating
systems. Rather than writing drivers and networking code from scratch, the
Flux OSKit provides glue code to make it possible for the FreeBSD TCP/IP stack,
Linux network device drivers, and the NetBSD file system code to work as parts
of the same kernel. In KudOS, we did something similar in josfs\_cfs, which is
the glue code to tie the JOS File Server with our CFS layer.

With both Flux and KudOS, the philosophy is to componentize the system. Of
course, the two projects have different level of granularity. The two projects
also have different goals. Flux tries to facilitate faster OS development
through reuse of existing component. KudOS uses components to make it easier to
come up with new and interesting configurations.
