\section{Future Work}
\label{sec:future}

\subsection{Increased Efficiency}
speed problems, async I/O (now not clear when you'll clear the
chdesc), memory mapping

interleve stuff w/ writing the journal out.

\subsection{Increased Flexibility}

One goal of a modular system is to give the module user complete
freedom to arrange modules in any way she pleases. We have taken great
care to allow almost complete freedom with the system, except in some
clear well defined places. For example, mixing of interfaces is not
allowed; one cannot connect a block device directly to a table
classifier. For example, we currently disallow stacking write back
caches. This is explained at the end of
section~\ref{sec:solution:impl:wbcache}. It seems that stacking caches
might be desirable, though. For example, a user may prefer to have a
small write back cache very high in the BD layer (similar to an L1
cache), and another larger one very low in the layer, just before the
disk (similar to an L2 cache). This could give optimal performance,
but currently this is disallowed.

Perhaps the problem is not in the write back cache, but in our model
of change descriptors. Perhaps we should have change descriptors keep
track of their distance from stable storage, or their distance from
the place they were generated.

\subsection{More Features}
journal size limitations, metadata only journalling
