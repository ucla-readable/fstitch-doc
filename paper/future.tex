\section{Future Work}
\label{sec:future}

\subsection{Increased Efficiency}
\label{sec:future:efficiency}

We would like to increase both the internal efficiency of the file server and
the efficiency of programs that use the file server.

One big problem in the current implementation of the KudOS file server is slow
speed. This is likely due to many factors, including context switches because of
IPC, duplicate and/or unnecessary disk I/O, the lack of DMA disk access, and the
long round trip times involved in accessing a network block device. We have
considered ways to speed all of these up:

To speed up IPC, we could batch requests and/or send messages using shared
memory instead of needing to use an IPC message (and therefore a context switch)
for each page of data. This same technique sped up the userspace pipe code in
JOS enormously, and we expect it would have a similar performance boost for the
file server.

The problem of duplicate and unnecessary disk I/O in our modules has been fixed
to a certain degree. For example, the old implementation of truncate\_block()
would have to call read\_block() to return a block descriptor. This meant that
deleting a file would have to read the entire file. Fixing this massively
improved the performance of deleting files. There are probably several more
opportunities for optimization in this area.

Adding IDE DMA support will probably speed up all disk accesses.

We have noticed that a typical read request takes much longer than a write
request to a network block device. This is likely due to the long round trip
time of the network. To help speed this up, we have considered batching reads to
do automatic prefetching.

Journaling is also a performance problem. Right now, while a journal is being
written out, the file server is blocking and thus hangs when requests come in.
We would like to interleave requests with writing of the journal.

Also, user programs may be able to be more efficient if we supported memory
mapped files. The current implementation provides no support for memory mapping.

\subsection{Increased Flexibility}
\label{sec:future:flexibility}

One goal of a modular system is to give the module user complete freedom to
arrange modules in any way she pleases. We have taken great care to allow almost
complete freedom with the system, except in some clear and well defined places.
For example, mixing of interfaces is not allowed; one cannot connect a block
device directly to a table classifier.

There are some restrictions that seem somewhat unnecessary, though. For example,
we currently disallow stacking write back caches (discussed at the end of
section~\ref{sec:solution:impl:wbcache}). It seems that stacking caches might,
however, be desirable. For example, a user may prefer to have a small write back
cache very high in the BD layer (similar to an L1 cache), and another larger one
very low in the layer, just before the disk (similar to an L2 cache). This might
give better performance, but currently this is not possible.

In principle, the problem is not in the write back cache, but in our model of
change descriptors. Perhaps we should have change descriptors keep track of
their distance from stable storage, or their distance from the place they were
generated.

\subsection{More Features}
\label{sec:future:features}

The KudOS file server could also benefit from more features. Specifically there
are two journaling limitations that would be nice to change.

Firstly, the journal currently has a fixed length. If the length of the journal
is exceeded before it can be flushed, the system temporarily goes into non-journaled mode.
Perhaps a better solution would be to grow the journal, or detect when the
journal needs flushing.

Secondly, the journal currently can only do full data journaling. We would like
to be able to do metadata only journaling, but at the moment we don't have the
facilities for the journal to know which writes are for data and which for
metadata.
