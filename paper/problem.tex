\section{The Problem}
\label{sec:problem}

File systems today are an integral part of any computer system, and for many
applications like databases or web servers, the choice of file system can have a
dramatic impact on the resulting performance and stability of the application.
Often these performance and stability differences come from relatively small
feature differences in the file system, like supporting large directory
traversals efficiently at the expense of disk space, being able to store many
small files in the same disk block, or having some sort of reliable recovery
mechanism for system crashes.

Even though this is the case, existing file systems are generally implemented as
monolithic blocks of code that conform to the disk infrastructure used in the OS
on one side, and to the userspace file system interface on the other. These
modules can be tens of thousands of lines long, and are difficult to change
without understanding large amounts of code that in principle does not have
anything to do with file systems. Furthermore, there is a lot of duplication
between different file systems, especially in the higher-level parts of the file
system code that interfaces with userspace.
