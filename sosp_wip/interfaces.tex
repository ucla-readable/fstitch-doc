\preparagraphspacing{}
\section*{Filesystem Module Interfaces}
\label{sec:interfaces}

A complete KudOS configuration is composed of many modules (an example module
graph is shown in Figure~\ref{fig:kfs-graph}). By breaking filesystem code into
small, stackable modules, we are able to significantly increase code reuse. To
further this end, we add an additional interface that helps to divide filesystem
implementations into common (i.e. reusable) code and filesystem-specific code.
We call this intermediate interface the ``Low-level File System'' (LFS). This
new interface is a substantial departure from other stackable module systems,
like FiST~\cite{zadok00fist}, which stack higher-level operations.

The LFS interface has functions to allocate blocks, add blocks to files,
allocate file names, and other filesystem micro-ops. The idea is that a module
implementing the LFS interface should define how bits are laid out on the disk,
but not have to actually know how to combine the micro-ops into larger, more
familiar filesystem operations. We have implemented a generic VFS to LFS module
that decomposes the larger file write, read, append, and other standard
operations into LFS micro-ops. This one module can be used with many different
LFS modules implementing different filesystems.
