\preparagraphspacing{}
\section*{Filesystem Module Interfaces}
\label{sec:interfaces}

The KudOS file system is composed of various modules. By breaking
filesystem code into small, stackable modules, we are able to
significantly increase code reuse. To this end, we add an additional
interface that helps to divide filesystem implementations into common
(i.e.  reusable) code and filesystem-specific code. We call this
intermediate interface the ``Low-level File System'' (LFS).

The LFS interface has functions to allocate blocks, add blocks to files,
allocate file names, and other filesystem micro-ops. The idea is that a module
implementing the LFS interface should define how bits are laid out on the
disk, but not have to actually know how to combine the micro-ops into larger,
more familiar filesystem operations. We have implemented a generic VFS to LFS
module that decomposes the larger file write, read, append, and other standard
operations into LFS micro-ops. This one module can be used with many different
LFS modules implementing different filesystems.
