\preparagraphspacing{}
\section*{Filesystem Module Interfaces}
\label{sec:interfaces}

A complete KudOS configuration is composed of many modules.
By breaking file system code into
small, stackable modules, we are able to significantly increase code reuse. To
further this end, we add an additional interface that helps to divide file system
implementations into common (i.e. reusable) code and file system-specific code.
We call this intermediate interface the ``Low-level File System'' (LFS). This
new interface is a substantial departure from other stackable module systems,
like FiST~\cite{zadok00fist}, which stack higher-level operations.

The LFS interface has functions to allocate blocks, add blocks to
files, allocate file names, and other file system micro-ops. A module
implementing the LFS interface should define how bits are laid out on
the disk, but not have to actually know how to combine the micro-ops
into larger, more familiar file system operations. We have implemented
a generic VFS to LFS module that decomposes the larger file write,
read, append, and other standard operations into LFS micro-ops. This
one module can be used with many different LFS modules implementing
different file systems.

Figure~\ref{fig:kfs-graph} shows a contrived example taking advantage of the LFS
interface and change descriptors. In this example a file system image is mounted
with an external journal, both of which are loop devices on the root file system
which uses soft updates. The journalled file system's ordering requirements are
sent through the loop device as change descriptors. Despite being transformed
by the loop device, all ordering requirements are still respected.
%
BSD, without change descriptors and the ability to forward change
descriptors through loop devices, is not able to express soft updates'
consistency requirements through loop back filesystems.
