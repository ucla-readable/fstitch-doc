% -*- mode: latex; tex-main-file: "abstract.tex" -*-
\preparagraphspacing{}
\section*{Change Descriptors}
\label{sec:chdescs}

Each in-memory modification to a cached disk block in KudOS has an associated
change descriptor.
%
Different change types correspond to different forms of change descriptors;
the change descriptor for a flipped bit---such as in a free-block
bitmap---contains an offset and mask, while larger changes contain an
offset, a length, and the new data.
%
The change descriptor's dependencies point to other change descriptors that
must precede it to stable storage.
%
A change descriptor can be applied or reverted to switch the cached block's
state between old and new as necessary.
%
Each change descriptor applies to exactly one block.
%
Figure~\ref{fig:chdesc} gives
a simplified version of the structure. The ability to revert and
re-apply change descriptors is inspired by the soft updates system in
BSD's FFS~\cite{ganger00soft}, but generalized so that it is not
specific to any particular file system.

\begin{figure}
\vskip-14pt
\begin{tabular}{@{\hskip0.58in}p{2in}@{}}
\begin{scriptsize}
\begin{verbatim}
struct chdesc {
    BD_t *device;
    bdesc_t *block;
    enum {BIT, BYTE, NOOP} type;
    union {
        struct {
            uint16_t offset;
            uint32_t xor;
        } bit;
        struct {
            uint16_t offset, length;
            uint8_t *data;
        } byte;
    };
    struct chdesc *dependencies[];
/* ... */ };
\end{verbatim}
\end{scriptsize}
\end{tabular}
\vspace{-10pt}
\caption{\label{fig:chdesc} Partial change descriptor structure.}
\end{figure}

When a KudOS module first generates change descriptors to write to the disk, it
specifies write ordering requirements similar to those of soft updates. For
example, Figure~\ref{fig:softupdates} depicts change descriptors that allocate
and add a new block to an inode.
%
The module passes these change descriptors to another module closer to
the disk.  This second module can inspect, delay, and even modify them before
passing them on further.
%
For instance, the write-back cache module holds on to blocks and their change
descriptors instead of forwarding them immediately.
%
When evicting a block and associated change descriptors, the write-back
cache enforces an order consistent with the change descriptor dependency
information.

\begin{figure}[b]
  \centering
  \includegraphics[width=92pt]{fig/whatevs_3}%
  \caption{\label{fig:softupdates} Soft updates change descriptor graph,
  including the dependencies for adding a newly allocated block to an
  inode. Writing the new block pointer to an inode (Attach) depends on
  initializing the block (Clear) and updating the free block map (Alloc).
  Updating the size of the inode (Size) depends on writing the block
  pointer.}
\end{figure}

Soft updates, journalling, and many application-specific consistency models all
correspond to different change descriptor arrangements, so these features can be
added to the system as modules which appropriately connect or reconnect the
change descriptors.
%
For example, the change descriptors in Figure~\ref{fig:softupdates} can be
transformed to provide journalling semantics. The original four change
descriptors are modified to depend on a journal commit record, which
depends on blocks journalling the changes. Once the actual changes are
committed, the journal record is marked as completed.
Figure~\ref{fig:journal} shows these transformed change descriptors.
%
This single journalling module can attach to any file system module;
it performs transformations incrementally as the change descriptors
are generated.

\begin{figure}
  \centering
  \includegraphics[width=\hsize]{fig/whatevs_2}%
  \caption{\label{fig:journal} Journal change descriptor graph for the
    change in Figure~\ref{fig:softupdates}. Empty circles are
    ``NOOP'' change descriptors with no associated block data.}
\end{figure}

Further, by changing our journal module to include in the journal only
change descriptors which modify file system metadata and to add
additional dependencies to prevent premature reuse of blocks, we could
even obtain metadata-only journalling. The journal module can
distinguish metadata change descriptrs because of the LFS interface
division (described in the next section).
%
Other block device layering
systems, like GEOM~\cite{geom} or JBD in Linux, would or do need special hooks
into file system code in order to get the necessary hints (i.e.  what is
metadata and what is not) to do metadata-only journalling. Change descriptors
and the LFS interface allow us to do this automatically.
