% -*- mode: latex; tex-main-file: "abstract.tex" -*-

\section*{Introduction}
\label{sec:intro}

For robustness, stability, and reboot speed, file system implementations
must ensure that the file system's stored image is kept consistent
or easy to return to consistency.
%
Advanced consistency mechanisms such as soft updates~\cite{ganger00soft}
and journalling make this possible; unfortunately,
%
they are generally tied to a particular file system, and
can't be ported or adapted without significant engineering
effort.
%
Furthermore, interfaces like \verb+fsync()+ give user code only coarse
control over consistency.
%
Applications with custom consistency and performance requirements get
little help from conventional file systems, which either impose high
overhead (data journalling) or don't guarantee data consistency (soft
updates, for example, ensures metadata consistency only).



We propose a new file system implementation architecture, called
\emph{KudOS},
%
where \emph{change descriptor} structures represent any and all changes to
stable storage.
%
File systems generate change descriptors for all writes, then
send them to block devices for eventual commit.
%
Each change descriptor stores the old state of the block and the change's
\emph{dependencies}---other change descriptors that must be committed before
it is safe to commit this change.
%
Explicit dependencies let KudOS modules preserve necessary file system
invariants without understanding the file system itself; the old state
lets KudOS roll back changes when necessary to break cyclic block dependencies.
%
Change descriptors can implement many
consistency mechanisms, including soft updates and journalling.

KudOS is decomposed into fine-grained modules which generate, consume,
forward, and manipulate change descriptors. A particular innovation of the
module design is the separation of the low-level specification of on-disk layout
from higher-level file system-independent code, which operates on abstract disk
structures.

We have implemented a prototype of the KudOS
architecture as part of a new operating system.
%
Although results are premature and performance has not been measured,
change descriptors have helped us construct consistent file
system structures.
%
Our journalling module should automatically add journalling to any file
system, and combinations of simple modules can support, for example,
correct consistency on RAID over loop-back devices.
%
Eventually, we plan to support user-defined dependencies,
allowing applications to define consistency protocols for the file system
to follow.


%
%% They also support
%% the implementation of many other possible definitions of file system
%% consistency, which in the future we hope will allow for userspace
%% specification. 
%
%% Besides eventually supporting user-defined dependency structures, the
%% current module system already allows interesting new interactions, such as
%% correct consistency on RAID over loop-back devices.
