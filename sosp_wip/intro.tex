\section*{Introduction}
\label{sec:intro}

Despite the recent proliferation of advanced filesystem consistency features,
such as journalling and soft updates~\cite{ganger00soft}, the software which
implements these features remains largely inflexible and difficult to reason
about. It is generally tied to a particular filesystem, and can neither be
reused in another filesystem nor adapted to allow even slightly different
semantics without significant engineering effort. To address this problem, we
propose a new system called the KudOS File Server Architecture which introduces
two significant changes to traditional designs. First, we introduce a mechanism
for explicitly representing changes to disk blocks, and the dependencies among
those changes necessary to implement various consistency semantics. Second, we
decompose the filesystem software into small modules and add a new interface
type between the block device interface and the abstract filesystem interface.
This interface separates the low-level specification of a filesystem's on-disk
layout from higher-level filesystem-independent code which operates only
abstractly on disk structures.

We have implemented a mostly-complete, functional prototype of the KFS
Architecture. In our prototype, these two changes greatly increase the
flexibility of standard consistency features, e.g.~allowing our journalling
module to automatically add journalling to any filesystem. They also support
the implementation of many other possible definitions of filesystem
consistency, which in the future we hope will allow for userspace
specification. Finally, the module system combined with the dependency
tracking mechanism allows interesting new interactions, such as correct
consistency on RAID over loopback devices.
