% -*- mode: latex; tex-main-file: "paper.tex" -*-

\section {Introduction}
\label{sec:intro}


Ongoing massive growth in both disk capacities and user and application
 storage requirements, combined with the increasing
 disparity between disk and processor speeds, have led to ongoing
 innovation in file system design.
%
File systems are expected to do more things: they can automatically encrypt
 stored data~\cite{wright03ncryptfs}, support extensible metadata, such as
 content indexes~\cite{jouvelot91semantic}, automatically
 version~\cite{soules03metadata,fast04versionfs,quinlan02venti,cornell04wayback},
 and even detect viruses~\cite{miretskiy04avfs}.
%
Improved disk interfaces can make use of file system usage patterns to
 improve
 performance~\cite{sivathanu03semantically-smart,sivathanu05database-aware}.



As file system functionality increases, file system \emph{correctness} is
 also increasingly a focus of
 research~\cite{sivathanuetal05-logic,denehyetal05-journal-guided}.
%
File system extensions and disk-layer interventions risk violating
 the dependency relationships between written blocks on which file system
 correctness relies~\cite{ganger00soft}.
%
Stackable file
 systems~\cite{zadok00fist,zadok99extending,heidemann94filesystem,rosenthal90evolving}
 avoid some of these issues by placing extensions over an existing
 file system, which is responsible for maintaining consistency.
%
Unfortunately, not all extensions are easily implemented entirely above an
 unmodified file system.
%
Furthermore, a stackable file system may implement a single system call
 with multiple operations; for instance, imagine creating a file whose
 additional metadata information is stored in another file with a related
 name.
%
Since current VFS (virtual file system) implementations can't
 express dependencies among operations, the stackable
 file system must must either impose an ordering using \texttt{fsync},
 which is expensive, or accept and account for the possibility of
 inconsistency.
%
This problem exists for applications as well.
%
The expensive \texttt{fsync} and \texttt{sync} system calls are the only
 tools available for enforcing file consistency.
%
Robust applications, including databases, mail servers, and source code
 management tools, either accept the performance penalty of these system
 calls or rely on specialized raw-disk interfaces.



Proposed systems for improving file system integrity and consistency differ
 mainly in the kind of consistency they aim to impose, ranging from
 metadata consistency to full data journaling~\cite{tweedie98journaling} and even full ACID
 transactions~\cite{gal05transactional,liskov04transactional}.
%
However, different extensions within a file system, or different
 applications over the file system, may require different types of
 consistency semantics, and performance suffers when lower layers are
 unnecessarily denied the opportunity to reorder
 writes~\cite{ganger00soft}.
%
A better choice might be an abstraction that could express many consistency
 models.



This work develops a modular file system architecture called \Kudos\ that
 facilitates both file system extension and consistency
 maintenance.
%
We decompose the entire file system layer, from system calls to block
 devices, into pluggable \modules, hopefully making the system as a whole
 more configurable, extensible, and easier to understand.
%
\Kudos\ \modules\ are built around a new fundamental data type called a
 \emph{\chdesc}.
%
A \chdesc\ represents both a change to disk data and any
 \emph{dependencies} between that change and other changes.
%
\Chdescs\ were inspired by the dependency abstraction from BSD's
 soft updates~\cite{ganger00soft}, but whereas the soft updates implementation
 is specific to the UFS file system, \chdescs\ are fully general.
%
Any file system can generate \chdescs, and those \chdescs\
are obeyed by all other file system layers.
%
\Chdescs\ may be examined and modified by other modules
 as well, and even passed through layers such as loop-back file systems.
%
As a result, the loosely-coupled \modules\ that make up a file system
 implementation can cooperate to implement strong and often complex
 consistency guarantees, even though each individual \module\ does a
 relatively small part of the work.

%% We propose a new file system implementation architecture, called \emph{\Kudos},
%% where \emph{\chdesc} structures represent any and all changes to
%% stable storage. File systems generate \chdescs\ for all writes, then
%% send them to block devices for eventual commit. Explicit dependencies between
%% \chdescs\ let \Kudos\ \modules\ preserve necessary file system
%% invariants without understanding the file system itself. 

\Chdescs\ can implement many consistency mechanisms, including
 soft updates and journaling.
%
Our file system modules impose soft updates-style \chdesc\
 requirements by default.
%
A single module placed downstream of any file system can analyze these
 \chdesc\ layouts and transform them into dependencies that
 implement a journal.
%
Additionally, we give user-level applications controlled access to \chdescs,
allowing applications to impose their own limited consistency
 policies.
%
Modifying an IMAP mail server to use this interface requires only localized
 changes; the resulting server follows IMAP's consistency
 requirements while writing fewer blocks to disk.
%
Although implementation experience is preliminary, and \chdescs\
 do not yet support all consistency policies (for instance, not ACID
 transactions: transactions should be independent, but any file system
 client can observe all active \chdescs), we believe the \Kudos\
 architecture will allow the construction of consistent, modular,
 extensible file systems that are much easier to understand.



Our contributions include the generalized \chdesc\ design,
 \Kudos's module interfaces (a particular innovation is the separation of
 the low-level specification of on-disk layout from higher-level
 file-system-independent code), particular modules including the journal
 module and the write-back cache, and the changegroup interface that
 exports \chdescs\ to userspace.


\begin{comment}
 The rest of this 
 A particular innovation of the
 \module\ design is the separation of the low-level specification of on-disk
 layout from higher-level file system-independent code, which operates on
 abstract disk structures.
 Our journaling \module\ can automatically add
 journaling to any file system, and combinations of simple \modules\ can support,
 for example, correct consistency on RAID over loop-back devices.
\end{comment}


%% They can also be extended, in a carefully controlled way, into userspace --
%% enabling applications with custom consistency and performance requirements to
%% specify explicit write ordering restrictions to be honored by the file system.
%% We will show that this can give such applications several benefits over existing
%% interfaces like \texttt{fsync()} which provide only coarse control over
%% consistency, or which either impose high overhead (data journaling) or don't
%% guarantee data consistency (soft updates, for example, ensures metadata
%% consistency only).


%
%% All \Kudos\ \modules\ obey the ordering restrictions implied 
%% Thus, arrangements of \chdescs, by enforcing 
%% \Kudos's lower level layers 


\begin{comment}
Once upon a time, our research began as an attempt to decompose file system
software into small \modules, which would make the system as a whole more
configurable, extensible, and easier to understand -- as has been done in other
domains, like packet forwarding~\cite{kohler00click}. Upon examining existing
systems that accomplish similar tasks, and after some initial design for the new
system, it became clear that a key part of modern file system design is not
addressed by \modules\ alone: consistency. For robustness, stability, and
recovery speed, modern file system implementations must ensure that the file
system's stored image is kept consistent or easy to return to consistency.
Advanced consistency mechanisms such as soft updates~\cite{ganger00soft} and
journaling make this possible; unfortunately, they are generally tied to a
particular file system and can't be ported or adapted without significant
engineering effort. Furthermore, existing module systems did not provide a way
to actually \emph{implement} (or even just \emph{change}) such consistency
mechanisms; the stacking is done either above or below where that part of the
system would need to be implemented.

These consistency mechanisms all hinge on one critical ability in the file
system software: the ability to place specific ordering requirements on writes
to the disk. What was needed, then, was a way to formalize these ordering
requirements in a file system-independent way, so that all the \modules\ in our
system would be able to work with them. In this way, the loosely-coupled
\modules\ that make up the complete file system implementation would be able to
cooperate to implement strong and often complex consistency guarantees, even
though each individual \module\ does only a small and simple part of the work.

We propose a new file system implementation architecture, called \emph{\Kudos},
where \emph{\chdesc} structures represent any and all changes to
stable storage. File systems generate \chdescs\ for all writes, then
send them to block devices for eventual commit. Explicit dependencies between
\chdescs\ let \Kudos\ \modules\ preserve necessary file system
invariants without understanding the file system itself. \Chdescs\ can
implement many consistency mechanisms, including soft updates and journaling.
They can also be extended, in a carefully controlled way, into userspace --
enabling applications with custom consistency and performance requirements to
specify explicit write ordering restrictions to be honored by the file system.
We will show that this can give such applications several benefits over existing
interfaces like \texttt{fsync()} which provide only coarse control over
consistency, or which either impose high overhead (data journaling) or don't
guarantee data consistency (soft updates, for example, ensures metadata
consistency only).

\Kudos\ is decomposed into fine-grained \modules\ which generate, consume,
forward, and manipulate \chdescs. A particular innovation of the
\module\ design is the separation of the low-level specification of on-disk
layout from higher-level file system-independent code, which operates on
abstract disk structures. Our journaling \module\ can automatically add
journaling to any file system, and combinations of simple \modules\ can support,
for example, correct consistency on RAID over loop-back devices.
\end{comment}
