% -*- mode: latex; tex-main-file: "paper.tex" -*-

\section {Introduction}
\label{sec:intro}

\begin{comment}
This paper aims to evaluate whether a simple, unified abstraction that
represents all modifications to stable storage, including
\emph{dependencies} among modifications, can be used to efficiently
implement a complete file system layer, where modifications are common
and cache sizes are large.
%
The answer is a qualified yes.
\end{comment}


Write-before relationships, which require that some changes be committed
 to stable storage before others, underlie every mechanism for ensuring file system
 consistency and reliability from journaling to synchronous writes.
%
\Featherstitch\ is a complete storage system
 %% called \emph{\Featherstitch} that uses
 built on a concrete form of these relationships,
 a simple, uniform, and file system agnostic data type called the \emph{patch}.
%
\Kudos's API design and performance optimizations make patches
 a promising implementation strategy as well as a useful abstraction.


\begin{comment}
As file system functionality increases, maintaining file system
 correctness in the presence of failures is increasingly a focus of
 research~\cite{sivathanuetal05-logic,denehyetal05-journal-guided}.
%
File systems today deal with many challenges that make implementing this
 property difficult: power losses, software failures, and even user
 intervention all pose significant threats.
%
To meet this challenge, file systems use a variety of techniques, like
 journaling and soft updates.
%
These mechanisms are each based on imposing some write-before
 relationship among buffered changes to the data in stable storage.
%
The answer is a qualified yes.
\end{comment}


A patch represents both a change to disk data and any \emph{dependencies}
 between that change and other changes. 
%
\Patches\ were initially inspired by BSD's soft updates dependencies%
~\cite{ganger00soft}, but whereas soft updates
 implement a particular type of consistency
 and involve many structures specific to the UFS file
 system~\cite{mckusick99soft}, \patches\ are fully general,
 specifying only how a range of bytes should be changed.
%
This lets file system implementations specify a
 write-before relationship between changes without dictating
% (or worrying about)
 a write order that honors that relationship.
%
It lets storage system components examine and
 modify dependency structures independent of the file system's layout,
 possibly even changing one type of consistency into another.
%
It also lets applications modify \patch\ dependency structures,
%% : a
%% user-level interface called \emph{\patchgroups} shows that applications can
 thus defining consistency policies for the underlying
 storage system to follow.


A uniform and pervasive \patch\ abstraction may simplify
 implementation, extension, and experimentation for
 file system consistency and reliability mechanisms.
%
File system implementers currently find it difficult to provide
 consistency guarantees~\cite{tweedie98journaling,mckusick99soft}
 and implementations are often buggy~\cite{yang04using,yang06explode},
 a situation further complicated by file system extensions and
 special disk interfaces~\cite{soules03metadata,fast04versionfs,quinlan02venti,cornell04wayback,wright03ncryptfs,sivathanu03semantically-smart,sivathanu05database-aware}.
%
File system extension techniques such as stackable file
 systems~\cite{zadok00fist,zadok99extending,heidemann94filesystem,rosenthal90evolving}
 leave consistency up to the underlying file system; any extension-specific
 ordering requirements are difficult to express at the VFS layer.
%
Although maintaining file system
 correctness in the presence of failures is increasingly a focus of
 research~\cite{sivathanuetal05-logic,denehyetal05-journal-guided},
%
other proposed systems for improving file system integrity
 differ mainly in the kind of consistency they aim to impose, ranging from
 metadata consistency to full data journaling and full ACID
 transactions~\cite{gal05transactional,liskov04transactional}.
%
Some users, however, implement their own end-to-end reliability for some data
 and prefer to avoid \emph{any} consistency
 slowdowns in the file system layer~\cite{googleext2}.
%
\Patches\ can represent all these choices, and since they provide a common
 language for file systems and extensions to discuss consistency
 requirements, even combinations of consistency mechanisms can
 comfortably coexist.


\begin{comment}
But different extensions within a file system, or different
 applications over the file system, may require different types of
 consistency semantics, and performance suffers when lower layers are
 unnecessarily denied the opportunity to reorder writes;
 %
\Patches\ can implement many consistency mechanisms, including
 soft updates and journaling, and can allow combinations of different
 consistency protocols to exist at the same time.
 %
\Patches\ provide a simple and effective way for such extensions to
 express their requirements of the storage system, adding to the
 requirements already expressed by the file system itself.
\end{comment}


Applications likewise have few mechanisms
 for controlling buffer cache behavior in today's systems, and
%
robust applications, including databases, mail servers, and source code
 management tools, must choose between several mediocre options.
%
They can accept the performance penalty of expensive system calls like
 \texttt{fsync} and \texttt{sync} or use tedious and fragile sequences
 of operations that assume particular file system consistency semantics.
%
\emph{\Patchgroups}, our example user-level \patch\ interface,
 export to applications some of \patches' benefits
 for kernel file system implementations and extensions.
%
Modifying an IMAP mail server to use \patchgroups\ required only localized
 changes.  The result both meets IMAP's consistency requirements on any reasonable
 patch-based file system and avoids the performance hit of full
 synchronization.


\begin{comment}
Our file system modules impose soft updates-style \patch\ requirements by
 default, since doing so requires some knowledge of the file system's
 structures; we have also written a journal module that can change
 existing dependencies to express either full or metadata-only journaling.
%
A file system module not interested in supporting soft updates support
 could instead impose no \patch\ requirements, and count on the journal
 module to provide a consistency guarantee.


The \Kudos\ storage system implementation is decomposed entirely into
 pluggable \modules\ that manipulate \patches, hopefully making the system
 as a whole more configurable, extensible, and easier to understand.
%
Any storage system \module\ can generate \patches; other modules can examine
 them and modify them when required.
%
\Patch\ dependencies are obeyed by all other storage system layers, allowing
 them to be passed through layers such as loopback block devices.
%
As a result, the loosely-coupled \modules\ that implement a file system
 can cooperate to enforce strong and often complex consistency guarantees,
 even though each \module\ only does a small part of the work.
\end{comment}


Production file systems use system-specific optimizations to achieve
 consistency without sacrificing performance; we had to improve
 performance in a general way.
%
A naive \patch-based storage system scaled terribly,
 spending far more space and time on dependency manipulation than
 conventional systems.
%
However, optimizations reduced \patch\ memory and
 CPU overheads significantly.
%
A PostMark test that writes approximately 3.2~GB of data
 allocates 75~MB of memory throughout the test to store \patches\ and
 soft updates-like dependencies, less than 3\% of the memory used for file
 system data and about 1\% of that required by unoptimized
 \Featherstitch.
%
Room for improvement remains, particularly in system time, but
 \Featherstitch\ outperforms equivalent Linux configurations on
 many of our benchmarks; it is at most 30\% slower on others.


Our contributions include the \patch\ model and design, our
 optimizations for making \patches\ more efficient,
 the \patchgroup\ mechanism that exports
 \patches\ to applications, and several individual \Kudos\ modules, such as
 the journal.


In this paper, we describe \patches\ abstractly, state their behavior and safety
 properties, give examples of their use, and reason about the
 correctness of our optimizations.
%
We then describe the \Kudos\ implementation, which is decomposed
 into pluggable modules, hopefully making it configurable,
 extensible, and relatively easy to understand.
%
Finally, our evaluation compares \Kudos\ and Linux-native file
 system implementations, and we conclude.




\begin{comment}
%
Our benchmarks show that our optimizations can reduce the number of
 \patches\ \Kudos\ creates by \patchoptcount\ and the amount of undo data
 memory it allocates by \patchoptundo.
%
Our prototype is not yet as fast as we would like, but it is competitive
 with Linux on many of our benchmarks.
\end{comment}
