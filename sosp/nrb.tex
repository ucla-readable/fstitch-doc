% -*- mode: latex; tex-main-file: "paper.tex" -*-

\subsection{\Nrb\ \ChDescs}
\label{sec:patch:nrb}

The first optimization reduces the space overhead of rollback data by
detecting when it will never be necessary.
%
When a patch $p$ is created, \Kudos\ conservatively detects whether $p$
will ever need to be rolled back.
%
If not, $p$ is a \emph{hard patch}: its data can be applied
directly to the in-memory copy of the block without saving rollback data
for the previous version.


Only patches in a block-level dependency cycle might need to be rolled
back.
%
However, some of the patches in a cycle may be hard; only the crucial
patch that creates the cycle must be soft, for that is the patch that
must be rolled back to create a valid write order.
%
Thus, patch $p$ may be hard if it will \emph{never} be the head of a
block-level cycle: that is, if there will never exist true patches $q$
and $p'$ with $p \PDepend q \PDepend p'$ and $\PBlock{p} = \PBlock{p'}
\neq \PBlock{q}$.
%
In other words, if the only cycle between $\PBlock{p}$ and $\PBlock{q}$ is $p
\PDDepend q \PDDepend p'$, then $q$ and $p'$ may be hard, but $p$ may
not.
%
As a result, the set of all hard patches in memory will never contain
a block-level cycle, and the system can always find a valid order to
write hard patches to disk.
%
This assumes that the dependency structure correctly represents
dependencies among patches on the same block.
%
Specifically, all hard patches on a block depend on each other (since
they can't be written independently), and any soft patch on a block
depends on that block's hard patches, if any (since it cannot be written
without them).


When patch $p$ is created, \Kudos\ must decide whether any future
patch or dependency could create a block-level cycle with $p$ at the
head.
%
This decision is simplified by an API property: \emph{All of a patch's
direct dependencies are specified at creation time};
%
no new dependencies $p \PDDepend q$ may be introduced after $p$ is
created.\footnote{The actual rule is somewhat more flexible: modules may add
  new direct dependencies if they guarantee that those dependencies
  don't introduce any new block-level cycles.  As one example, if no
  patch depends on some empty patch $p$, then adding a new $p
  \PDDepend q$ dependency can't introduce a cycle.  Additionally,
  modules can specify at creation time that a patch is not allowed to
  become hard, because dependencies might be added to it later.}
%
Since every patch follows this rule, all possible block-level cycles $p
\PDepend q \PDepend p'$ are visible at creation time.


Traversing the dependency graph to check for a block-level cycle
proved to be quite expensive.
%
\Kudos\ thus implements a conservative approximation: patch $p$ is
created as hard if \emph{no} other block depends on the in-memory
version of $\PBlock{p}$.
%
That is, for all dependencies $q \PDepend p'$ with $p' \in
\PMem[\PBlock{p}]$, either $\PBlock{q} = \PBlock{p}$ or $q$ is an empty
patch.
%
If no other block depends on $p$'s block, then $p$ can never initiate a
block-level cycle no matter its dependencies.
%
Determining this property is less expensive than a full graph
traversal, and works well in practice.


\Kudos\ further ensures that the dependency structure correctly
represents dependencies on the same block through overlap
dependencies: since hard patches are considered to cover the entire
block, every succeeding patch will overlap at least one hard patch,
and \Kudos\ will automatically add a dependency.
%
(Some cases are handled by other optimizations.)


The buffer cache's ``write block'' behavior must account for hard
patches, as it \emph{must} write any hard patches that exist on a
block.
%
Let $\PHard[b]$ be the set of hard patches on block $b$.
%
Then to write block $b$, the buffer cache must choose some $P \subseteq
\PMem[b]$ with
%
\[ \PDepset{P} \subseteq P \cup \PDisk \text{ and } \PHard[b] \cap \PMem
\subseteq P. \]
%
If no such $P$ exists, then the cache must write a different block.


Applying hard patch rules to our example makes 14 of the 23 patches hard
(Figure~\ref{fig:opt}b),
%
reducing the rollback data required by slightly more than half.


\begin{comment}
%
To avoid this overhead, \Kudos\ identifies \chdescs\ that will never
need to be rolled back and omits their previous data copies. We call
these \emph{\nrb} \chdescs. (The opposite naturally being a \emph{\rb}
\chdesc, when necessary to differentiate them.)
%
Since a \nrb\ \chdesc\ cannot be rolled back, a write of any \chdescs\
on block $B$ must include all \nrb\ \chdescs\ on $B$. To accordingly
update our formal model we define a new set of \chdescs, \PHard, which
contains all \nrb\ \chdescs. We write \PHard[B] to restrict the set
to block $B$\todo{Introduce \PSoft\ and \PSoft[B].}:

\begin{tabbing}
\textbf{Write block.} \\
\quad Pick some block $b$ with $\PMem[b] \neq \emptyset$. \\
\quad Pick some $P \subseteq \PMem[b]$ with $\PDepset{P} \subseteq P \cup
\PDisk$ and $\PHard[B] \subseteq P$. \\
\quad For each $p \in P$, set $\PState{p} \gets \PInfst$. \\
\quad For each $p \in \PMem[b]-P$, set $\PDDepset{p} \gets \PDDepset{p}
\cup P$.
\end{tabbing}

\paragraph{}
To avoid (expensive) dependency traversals to determine whether a new
\chdesc\ will need to be rolled back,
%
\Kudos\ conservatively identifies \nrb\ \chdescs\ using only local
dependency information.
%
\Kudos\ detects that a new \chdesc\ on block $b$ may need to be rolled back if:
\todo{Which form is easier to read? Can we write \(\PMem - \PMem[b] - \PEmpty\) more concisely?}
%
\todo{Actually, our implementation also uses in flight \chdescs. Can we make
it not?}
%
\[ \PRDepset{\PMem[b]} \cap (\PMem - \PMem[b] - \PEmpty) \ne \emptyset \]
\[ \exists \inset{p}{\PMem[b]}\!:\
   \exists c\!:\ \exists \inset{q}{\PMem[c]}\!:\
   \indirdepends{q}{p} \]
%
This is both a safe and useful indicator because
%
the presence of an external \after\ is a necessary condition for a new
\chdesc's \before\ to induce a block-level cycle
%
and many blocks have no \chdescs\ with external \afters\ (e.g. most
file data blocks).

While this algorithm detects whether a \chdesc\ may need to be rolled
back, \Kudos\ must also be sure that no future dependency manipulation
will cause the \chdesc\ to require a rollback.
%
We introduce Invariant~\ref{cdinvar:add-before} to support such reasoning:
%
\cdinvar{add-before}{All block-level cycles induced through
\chdesc\ \p{p}'s \befores\ exist when \p{p} is
created\todo{Change this phrasing? ``Once created, a \chdesc\ will not
gain any \befores\ that induce block-level cycles.''}.}
%
\noindent \Kudos\ ensures this invariant by restricting \before\
additions to \chdesc\ creation, \noop\ \chdescs\ with no \afters, or
when the invariant is statically proven to hold for the affected
\chdescs.
\end{comment}
