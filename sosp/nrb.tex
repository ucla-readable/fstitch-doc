% -*- mode: latex; tex-main-file: "paper.tex" -*-

\subsection{\Nrb\ \Patches}
\label{sec:patch:nrb}

The first optimization reduces space overhead by
eliminating undo data.
%
When a \patch\ $p$ is created, \Kudos\ conservatively detects whether $p$
 might require reversion:
%
that is, whether any possible future patches and dependencies could force
 the buffer cache to undo $p$ before making further progress.
%
If no future patches and dependencies could force
 $p$'s reversion, then $p$ does not need undo data, and \Kudos\ does not
 allocate any.
%
This makes $p$ a \textbf{hard patch}: a patch without undo data.
%
The system aims to reduce memory usage by making most patches hard.
%
The challenge is to detect such patches without
 an oracle for future dependencies.
%
%% We solve this challenge by restricting the creation of dependencies.


%
(Since a hard patch $h$ cannot be rolled back, any other patch on its block
 effectively depends on it.
%
We represent this explicitly using, for example, overlap dependencies, and
%
as a result, the buffer cache will write all of a block's hard patches
 whenever it writes the block.)


We now characterize one type of patch that can be made hard.
%
Define a \emph{block-level cycle} as a dependency chain of uncommitted
 patches $p_n \PDepend \cdots \PDepend p_1$ where the ends are on the same
 block $\PBlock{p_n} = \PBlock{p_1}$, and at least one patch in the middle
 is on a different block $\PBlock{p_i} \neq \PBlock{p_1}$.
%
The patch $p_n$ is called the \emph{head} of the block-level cycle.
%
Now assume that a patch $p \in \PMem$ is not the head of any block-level
 cycle.
%
One can then show that the buffer cache can write at least one patch
 without rolling back $p$.
%
This is trivially possible if $p$ itself is ready to write.
%
If it is not, then $p$ must depend on some uncommitted patch $x$ on a different
 block.
%
However, we know that $x$'s uncommitted dependencies, if any, are all on
 blocks other than $p$'s; otherwise there would be a block-level cycle.
%
Since \Featherstitch\ disallows circular dependencies, every
 chain of dependencies starting at $x$ has finite length, and therefore
 contains an uncommitted patch $y$, all of whose dependencies have
 been committed.
%
(If $y$ has in-flight dependencies, simply wait
 for the disk controller to commit them.)
%
Since $y$ is not on $p$'s block, the buffer cache can write $y$ without
 rolling back $p$.


\Featherstitch\ may thus make a patch hard when it can prove that patch
 will \emph{never} be the head of a block-level cycle.
%
Its proof strategy has two parts.
%
First, the \Kudos\ API restricts the creation of block-level cycles by
 restricting the creation of dependencies:
%
\emph{a \patch's direct dependencies are all supplied at creation time}.
%
Once $p$ is created, the system can add new dependencies $q \PDDepend p$,
 but will never add new dependencies $p \PDDepend q$.\footnote{The actual
 rule is somewhat more flexible: modules may add new direct dependencies if
 they guarantee that those dependencies don't produce any new block-level
 cycles.  As one example, if no \patch\ depends on some \noop\ \patch\ $e$,
 then adding a new $e \PDDepend q$ dependency can't produce a cycle.}
%
Since every \patch\ follows this rule, all possible block-level cycles with
 head $p$ are present in the dependency graph when $p$ is created.
%
\Featherstitch\ must still check for these cycles, of course, and
%
actual graph traversals proved expensive.
%
We thus implemented a conservative approximation.
%
\Patch\ $p$ is
created as \nrb\ if \emph{no} patches on other blocks depend on uncommitted
 patches on $\PBlock{p}$---that is, if
%
for all $y \PDepend x$ with $x$ an uncommitted patch on $p$'s block,
 $y$ is also on $p$'s block.
%
If no other block depends on $p$'s, then clearly $p$ can't head up
 a current block-level cycle no matter its dependencies.
%
This heuristic works well in practice and, given some bookkeeping, 
 takes $O(1)$ time to check.


\begin{comment}
\Kudos\ further ensures that the dependency structure correctly
represents dependencies on the same block through overlap
dependencies: since \nrb\ \patches\ are considered to cover the entire
block, every succeeding \patch\ will overlap at least one \nrb\ \patch,
and \Kudos\ will automatically add a dependency.
%
(Some cases are handled by other optimizations.)


The buffer cache's ``write block'' behavior must account for \nrb\
\patches, as it \emph{must} write any \nrb\ \patches\ that exist on a
block.
%
Let $\PHard[b]$ be the set of \nrb\ \patches\ on block $b$.
%
Then to write block $b$, the buffer cache must choose some $P \subseteq
\PMem[b]$ with
%
\[ \PDepset{P} \subseteq P \cup \PDisk \text{ and } \PHard[b] \cap \PMem
\subseteq P. \]
%
If no such $P$ exists, then the cache must write a different block.
\end{comment}


Applying \nrb\ \patch\ rules to our example makes 16 of the 23 \patches\ \nrb\
(Figure~\ref{fig:opt}b),
%
reducing the undo data required by slightly more than half.


\begin{comment}
%
To avoid this overhead, \Kudos\ identifies \patches\ that will never
need to be reverted and omits their undo data. We call these \emph{\nrb}
\patches. (The opposite naturally being a \emph{\rb} \patch, when
necessary to differentiate them.)
%
Since a \nrb\ \patch\ cannot be reverted, a write of any \patches\
on block $\PB$ must include all \nrb\ \patches\ on $\PB$. To accordingly
update our formal model we define a new set of \patches, \PHard, which
contains all \nrb\ \patches. We write \PHard[\PB] to restrict the set
to block $\PB$\todo{Introduce \PSoft\ and \PSoft[\PB].}:

\begin{tabbing}
\textbf{Write block.} \\
\quad Pick some block $b$ with $\PMem[b] \neq \emptyset$. \\
\quad Pick some $P \subseteq \PMem[b]$ with $\PDepset{P} \subseteq P \cup
\PDisk$ and $\PHard[\PB] \subseteq P$. \\
\quad Move each $p \in P$ to $\PInf$ (in-flight). \\
\quad For each $p \in \PMem[\PB]-P$, set $\PDDepset{p} \gets \PDDepset{p}
\cup P$.
\end{tabbing}

\paragraph{}
To avoid (expensive) dependency traversals to determine whether a new
\patch\ will need to be reverted,
%
\Kudos\ conservatively identifies \nrb\ \patches\ using only local
dependency information.
%
\Kudos\ detects that a new \patch\ on block $\PB$ may need to be reverted if:
\todo{Which form is easier to read? Can we write \(\PMem - \PMem[\PB] - \PEmpty\) more concisely?}
%
\todo{Actually, our implementation also uses in flight \patches. Can we make
it not?}
%
\[ \PRDepset{\PMem[b]} \cap (\PMem - \PMem[b] - \PEmpty) \ne \emptyset \]
\[ \exists \inset{p}{\PMem[b]}\!:\
   \exists c\!:\ \exists \inset{q}{\PMem[c]}\!:\
   \indirdepends{q}{p} \]
%
This is both a safe and useful indicator because
%
the presence of an external \after\ is a necessary condition for a new
\patch's \before\ to induce a block-level cycle
%
and many blocks have no \patches\ with external \afters\ (e.g. most
file data blocks).

While this algorithm detects whether a \patch\ may need to be reverted,
\Kudos\ must also be sure that no future dependency manipulation
will cause the \patch\ to require being reverted.
%
We introduce Invariant~\ref{cdinvar:add-before} to support such reasoning:
%
\cdinvar{add-before}{All block-level cycles induced through
\patch\ \p{p}'s \befores\ exist when \p{p} is
created\todo{Change this phrasing? ``Once created, a \patch\ will not
gain any \befores\ that induce block-level cycles.''}.}
%
\noindent \Kudos\ ensures this invariant by restricting \before\
additions to \patch\ creation, \noop\ \patches\ with no \afters, or
when the invariant is statically proven to hold for the affected
\patches.
\end{comment}
