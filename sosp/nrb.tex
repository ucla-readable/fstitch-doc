\subsection{\Nrb\ \ChDescs}
\label{sec:chdescs:nrb}
To enable \chdesc\ rollback, each \chdesc\ contains a copy of the
previous block data.\footnote{Actually, \Kudos\ supports a specialized
type of \chdesc\ for efficiently flipping individual bits using an
inline exclusive-or mask instead of a copy of the previous data, but
most \chdescs\ are not of this type.}
%
While these copies are needed to roll back \chdescs, many
\chdescs\ are never actually rolled back (e.g. file data blocks) and
previous data copies nearly double the memory usage of \chdescs\ and
cached blocks.
%
\Kudos\ addresses this issue by identifying those \chdescs\ which will never
need to be rolled back, and omitting the previous data copies for them. We call
these \emph{\nrb} \chdescs. (The opposite naturally being a \emph{\rb} \chdesc,
when necessary to differentiate them.) Since \nrb\ \chdescs\ cannot be rolled
back, whenever a block containing one is written, it must be written with the
block. If there is more than one, all of them must be written.

To identify potential \nrb\ \chdescs, \Kudos\ uses a conservative algorithm
based on a \chdesc\ graph invariant. Merely detecting that the current
\chdesc\ graph can be written without rolling back a particular \chdesc\ is
not sufficient, as the graph may change in the future. Thus we need an
invariant which will guarantee that future changes to the graph will not
require \nrb\ \chdescs\ to be rolled back, where \nrb\ \chdescs\ are
selected based on some property of the current graph.

\cdinvar{add-before}{All block-level cycles induced through
\chdesc\ $C$'s \befores\ exist when $C$ is
created\todo{Change this phrasing? ``Once created, a \chdesc\ will not
gain any \befores\ that induce block-level cycles.''}.}
%
\Kudos\ ensures this invariant by restricting \before\ additions to
\chdesc\ creation, \noop\ \chdescs\ with no \afters, or when the invariant
is statically proven to hold.

To avoid a costly dependency traversal to determine whether a new
\chdesc\ will need to be rolled back, \Kudos\ conservatively identifies
a new \chdesc\ as \nrb\ only if the block has no \chdescs\
with external (on a different block\todo{Descriptive enough? Mention
\noop\ \chdesc\ \after\ recursion?}) \afters. Each block tracks the number of
external \afters\ through incremental updates as \afters\ are added and
removed.
