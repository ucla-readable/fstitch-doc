\subsection {\Module\ Interfaces}
\label{sec:design:interfaces}

A complete \Kudos\ configuration is composed of many \modules. 
%% By breaking file
%% system code into small, stackable \modules, we are able to significantly
%% increase code reuse. More importantly, 
New \modules\ are simple to write, and
%%can implement a variety of new features for the file system. Finally, 
by
changing the \module\ arrangement, a broad range of behaviors can be
implemented.
%
It's also easy to tell what behavior a given arrangement will
give just by looking at the connections between the \modules.

\Kudos\ has three major types of \modules.
%
Closest to the disk are block device
(BD) \modules, which have a fairly conventional block device
interface with interfaces such as ``read block'' and ``flush''. 
%
Closest to the system call interface are
\emph{common file system} (CFS) \modules, which have an interface similar to
VFS~\cite{kleiman86vnodes}. 
%
In traditional systems, a CFS-like \module\ would be
connected directly to a BD-like \module\ in order to set up a file system for
mounting.
%
In \Kudos, however, there is an intermediate interface which interposes
between block devices and the high-level CFS interface.
%
This \emph{low-level file system} (LFS) interface helps divide
file system implementations into common (reusable) code and file system
specific code. A \Kudos\ file system designer combines modules with all
three interfaces in many ways -- a departure from stackable file systems,
which act only at the VFS/CFS layer. \Kudos\ \modules\ are implemented in C
using structures of function pointers to achieve object oriented behavior.

\begin{figure}[thb]
\vskip-14pt
\begin{tabular}{@{\hskip0.25in}p{2in}@{}}
\begin{scriptsize}
\begin{alltt}
int (*\textbf{get_root})(inode_t *inode);
uint32_t (*\textbf{allocate_block})(
    fdesc_t *file, int purpose,
    chdesc_t **head);
int (*\textbf{free_block})(
    fdesc_t *file, uint32_t block,
    chdesc_t **head);
bdesc_t *(*\textbf{lookup_block})(uint32_t number);
int (*\textbf{write_block})(
    bdesc_t *block, chdesc_t **head);
int (*\textbf{lookup_name})(
    inode_t parent, const char *name,
    inode_t *inode);
fdesc_t *(*\textbf{lookup_inode})(inode_t inode);
void (*\textbf{free_fdesc})(fdesc_t *fdesc);
uint32_t (*\textbf{get_file_numblocks})(fdesc_t *file);
uint32_t (*\textbf{get_file_block})(
    fdesc_t *file, uint32_t offset);
int (*\textbf{get_dirent})(
    fdesc_t *file, struct dirent *entry,
    uint16_t size, uint32_t *basep);
int (*\textbf{append_file_block})(
    fdesc_t *file, uint32_t block,
    chdesc_t **head);
uint32_t (*\textbf{truncate_file_block})(
    fdesc_t *file, chdesc_t **head);
fdesc_t *(*\textbf{allocate_name})(
    inode_t parent, const char *name,
    uint8_t type, fdesc_t *link,
    inode_t *newinode, chdesc_t **head);
int (*\textbf{remove_name})(
    inode_t parent, const char *name,
    chdesc_t **head);
int (*\textbf{rename})(
    inode_t oldparent, const char *oldname,
    inode_t newparent, const char *newname,
    chdesc_t **head);
int (*\textbf{get_metadata})(
    inode_t inode, uint32_t id,
    size_t size, void *data);
int (*\textbf{set_metadata})(
    inode_t inode, uint32_t id,
    size_t size, const void *data,
    chdesc_t **head);
\end{alltt}
\end{scriptsize}
\end{tabular}
\vspace{-10pt}
\caption{\label{fig:lfs} Important functions in the LFS interface.}
\end{figure}

The LFS interface (Figure~\ref{fig:lfs}) has functions to allocate blocks, add
blocks to files, allocate file names, and other file system micro-ops. A
\module\ implementing the LFS interface should define how bits are laid out on
the disk, but doesn't have to know how to combine the micro-ops into larger,
more familiar file system operations. A generic CFS-to-LFS \module\ decomposes
the larger file write, read, append, and other standard operations into LFS
micro-ops. This one \module, called the ``universal high-level file system'' or
UHFS, can be used with many different LFS \modules\ implementing different file
systems. (There is also a generic VFS-to-CFS layer, which has two
implementations: one for the Linux kernel module, and one for the
FUSE~\cite{fuse} version. See \S\ref{sec:implementation} for details.)

The UHFS \module\ encompasses all logic for decomposing higher level CFS calls
into lower level LFS calls, and for connecting the resulting change descriptors
together. The finer granularity of LFS calls divides the problem space into
smaller chunks. Since the issue of how to tie the LFS micro-ops together has
already been solved, file system \module\ developers can give more attention to
the particulars of the file system, such as how to allocate a new filename or
how to look up the Nth data block for a file. To see how this simplifies the
development process, consider the VFS \texttt{write()} call, which has the task
of writing some amount of data to a file at a given offset. In \Kudos, the logic
to determine the correct offsets within blocks and whether new blocks must be
allocated is built into UHFS. A file system \module\ need only implement four
LFS calls: \texttt{get\_file\_block()}, \texttt{allocate\_block()},
\texttt{append\_block()}, and \texttt{write\_block()}. Additionally, the
granularity of calls at the LFS layer makes it an appropriate layer for
inserting test harnesses and developing file system unit tests.

\begin{figure}[tb]
  \centering
  \includegraphics[height=3in]{fig/figures_1}
  \caption{A running \Kudos\ configuration. {\it/} is a soft updated
    file system on an IDE drive; {\it/loop} is an externally journaled
    file system on loop devices.}
  \label{fig:kfs-graph}
\end{figure}

% FIXME: show multiple WB caches?
Figure~\ref{fig:kfs-graph} shows a somewhat contrived example taking advantage
of the LFS interface and \chdescs. A file system image is mounted with an
external journal, both of which are loop devices on the root file system, which
uses soft updates. The journaled file system's ordering requirements are sent
through the loop device as \chdescs, allowing dependency information to be
maintained across boundaries that might otherwise lose that information. In
contrast, without \chdescs\ and the ability to forward \chdescs\ through loop
devices, BSD cannot express soft updates' consistency requirements through
loop-back file systems. The \modules\ in Figure~\ref{fig:kfs-graph} are a
complete and working \Kudos\ configuration, and although the use of a loop
device is somewhat contrived in the example, they are increasingly being used in
conventional operating systems. For instance, Mac OS X uses them in order to
allow users to encrypt their home directories.
