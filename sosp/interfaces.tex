% -*- mode: latex; tex-main-file: "paper.tex" -*-

%% \subsection {Interfaces}
\label{sec:modules:interfaces}

\begin{comment}
New \modules\ are
simple to write, and by changing the \module\ arrangement, a broad range of
behaviors can be implemented. It's also easy to tell what behavior a given
arrangement will give just by looking at the connections between the \modules.
\end{comment}

A complete \Kudos\ configuration is composed of many \modules, making it
a finer-grained variant of a stackable file system.
%
There are has three major types of \modules.
%
Closest to the disk are block device (BD) \modules, which have a fairly
conventional block device interface with interfaces such as ``read block'' and
``flush''. 
%
Closest to the system call interface are \emph{common file system} (CFS)
\modules, which have an interface similar to VFS~\cite{kleiman86vnodes}. 
%
\Kudos\ also supports an intermediate interface between BD and CFS.
%
This \emph{low-level file system} (\LFS) interface helps divide file system
implementations into code common across block-structured file systems and
file system-specific code.
%
\begin{comment}
A
\Kudos\ file system designer combines modules with all three interfaces in many
ways -- a departure from stackable file systems, which act only at the VFS/CFS
layer. \Kudos\ \modules\ are implemented in C using structures of function
pointers to achieve object oriented behavior, very much like the rest of the
Linux kernel.
\end{comment}
%
The \LFS\ interface has functions to allocate blocks, add blocks to files,
allocate file names, and other file system micro-operations. A \module\ implementing
the \LFS\ interface defines how bits are laid out on the disk, but doesn't
have to know how to combine the micro-operations into larger, more familiar file system
operations. A generic CFS-to-\LFS\ \module\ decomposes the larger file write,
read, append, and other standard operations into \LFS\ micro-operations. 
%
File system extensions like those often implemented by stackable file
systems would generally use the CFS interface; for example, we wrote a
simple CFS module that provides case-insensitive access to a case-sensitive
file system.
%
File system implementations, such as our ext2 and UFS implementations, use
the \LFS\ interface.


\Patches\ are explicitly part of the \LFS\ interface.
%
Every \LFS\ function that might modify the file system takes a
\texttt{\textit{patch\char`\_t **p}} argument.
%
When the function is called, the patch \texttt{*p} is set to the patch, if
any, on which the modification should depend;
%
when the function returns, the patch \texttt{*p} must be set to depend on
the modification itself.
%
(Empty patches allow this interface to generalize to multiple
dependencies.)
%
For example, this function is called to append a block to an \LFS\ inode
(which is called ``\verb+fdesc_t+''):

\begin{small}
\begin{alltt}
int (*append_file_block)(LFS_t *module, 
   fdesc_t *file, uint32_t block, patch_t **p);
\end{alltt}
\end{small}

\noindent%
This design lets \LFS\ modules examine and modify the dependency structure.


Each \chdesc\ on a cached block may or may not be visible to a given \module.
For example, \modules\ that respond to user requests generally view the most
current state of every block -- the block with all \chdescs\ applied. However, a
write-back cache may choose to write some \chdescs\ on a block while reverting
others, since those others currently have outstanding dependencies. In this
case, \modules\ below the write-back cache (i.e. closer to the disk) should view
those \chdescs\ in the reverted state. \Kudos\ provides a block revisioning
library function that automatically rolls back those \chdescs\ that should not
be visible at a particular \module, and then rolls them forward again after that
\module\ is done with the block.
