\section{Implementation}
\label{sec:implementation}

\Kudos\ was originally implemented as a stand-alone file system server daemon
for a small operating system (called ``KudOS''), then later as a
FUSE~\cite{fuse} file system server under Linux or BSD. It currently runs as a
Linux kernel module. The FUSE implementation was particularly useful for
debugging, but we have found (unsurprisingly) that the Linux kernel module
version is best for real-world performance testing and actual use. Due to the
complexity involved in maintaining both versions, we eventually dropped the
FUSE version.

The \Kudos\ Linux kernel module looks like a VFS file system to Linux, but
unlike traditional file system modules,
 %% which are connected directly to block devices, 
it is merely a front end to a \Kudos\ \module\ graph. The \Kudos\
server starts up a kernel thread which performs all the initialization to get
the \module\ graph set up, and \modules\ then can make file systems available to
mount though the VFS front end. Each file system can be mounted using a
command like \mbox{\texttt{mount -t kfs kfs:\textit{name} /mnt/point}}.

The Linux kernel must be modified in order to
provide support for \opgroups.  As a kernel module, there is no way to
get notifications about when processes are created, or when they terminate; this
is necessary to clone the \opgroup\ scope 
from the parent process when the process is created, and to \abandon\ all
the \opgroups\ when it terminates. Ordinarily, this sort of state would be kept
as part of the Linux \texttt{task\_struct} (i.e. process) structure, but this is
not a viable option for a dynamically loadable kernel module. Instead, the
\Kudos\ module keeps this state internally and uses new kernel hooks
to be notified when processes fork or exit. 
%% The patch to add these hooks
%% is based on the ``process events connector'' which was introduced in Linux
%% 2.6.15, and is only about 250 lines long. Without it, \Kudos\ can still operate,
%% but it does not allow \opgroups.

At the other end of the \Kudos\ \module\ graph, the terminal block device
\modules\ are wrappers for Linux's normal block devices. By using Linux's
generic block layer, \Kudos\ can interface with any existing block device (e.g.
RAM disk, IDE disk, etc.). When the \module\ receives a read or write call, it
passes the request to Linux via \texttt{generic\_make\_request()}. The block is
of course not written immediately upon making this call, so \Kudos\ does not
satisfy the \chdescs\ on the block that are being written until it receives
notification that the block has made it to the disk controller (or, with
sufficient hardware support like SCSI TCQ or ATA NCQ, to the physical disk,
though a small patch to the Linux kernel is necessary to allow this). As a
result of this mechanism, block writes coming out of \Kudos\ may safely be
reordered and/or merged at will by Linux (or the hardware), since \Kudos\ will
not consider the \chdescs\ which depend on these \emph{in flight} blocks to be
ready to be written themselves.

The \module\ also implements rudimentary read-ahead. When a read call needs to
be made to the physical device, the \module\ checks to see if any of the
following several blocks are in memory already. If any of them are not, they are
also requested at the same time. After all blocks are read in, the read call is
complete.

Our interfaces bypass all Linux caches, because \Kudos\ provides its own
\chdesc-aware caches (\S\ref{sec:modules:wbcache}) in order to provide its
write-ordering guarantees.

Our current implementation uses Linux's queuing structures for queuing I/O
requests for block devices. Although this is working, there are a couple subtle
problems that we would like to solve in a future version. For instance, we
would like to be able to insert read and write requests to different sides of
the I/O queue. Right now a read request will be satisfied only after all the
pending writes are completed, but a better queue design would allow reads to
``cut the line'' since they are more urgent. \todo{is this still true?}
