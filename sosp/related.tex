\section{Related Work}
\label{sec:related}

\paragraph{Consistency}

Soft updates~\cite{ganger00soft} significantly lowers the overhead required to
provide file system consistency. By carefully ordering writes to disk, soft
updates avoids the need for synchronous writes to disk or duplicate writes to
a journal. Soft updates also guarantees a strong level of consistency after a
crash, enough so that the system can avoid time-consuming file system
consistency checks using a utility like \emph{fsck}. 

Another approach to protecting the integrity of the file system is to write
upcoming operations to a journal first. The content and the layout of the
journal vary in each implementation, but in all cases, the system can use the
journal to play out (or roll back) the operations that did not complete as a
result of a crash. Thus, \emph{fsck} can be avoided by consulting the journal
when recovering from a crash. Section \ref{sec:using:journal} explains
journaling with \chdescs. \cite{seltzer00journaling} compares journaling and
soft updates in practice.

External synchrony~\cite{nightingale06rethink} builds on journaling to
automatically provide strict file system operation ordering for applications,
without requiring them to block on each write. It combines operations into a
journal, but tracks the activity of the calling processes after returning
control to them from the file system. If a process later performs some
\emph{user-visible} operation like printing text to the screen or sending
network traffic, the journal transaction containing the changes is forced to
commit before the process can continue.

\cite{sivathanu05ensuring}

Customizable application-level consistency protocols have previously been
considered in the context of distributed, parallel file systems by
CAPFS~\cite{vilayannur05providing} and Echo~\cite{mann94coherent}.
%
CAPFS allows application writers to design plug-ins for the parallel file store,
which define what actions to take before and after each client-side system call
to enforce additional consistency policies.
%
Echo maintains a partial order on the locally cached updates to the remote file
system, and guarantees that the server will store the updates accordingly. It
also provides a mechanism for applications to extend the partial order, thus
reducing the server's flexibility to choose how to write the data or make it
available to other clients.
%
Both systems are based on the principle that not providing the right
consistency protocol can cause unpredictable failures, yet enforcing
unnecessary consistency protocols can be extremely expensive.
%
However, this is also true with a local file system -- and as a result,
applications must use expensive interfaces like \texttt{fsync()} when they
require specific consistency guarantees.
%
\Kudos\ brings this sort of customizable consistency to all applications, not
just those using specialized distributed file systems.

% "allow the kernel to safely and efficiently handle any metadata layout without understanding the layout itself"
% \cite{kaashoek97application}

\paragraph{Stackable File Systems}

% \cite{webber93portable}

Stackable \module\ software for file systems continues to attract active
research~\cite{rosenthal90evolving, heidemann91layered, skinner93stacking,
heidemann94filesystem,zadok99extending,
zadok00fist,wright03ncryptfs,wright06versatility}. Previous
systems like FiST~\cite{zadok00fist} or GEOM~\cite{geom} generally focus on
an individual portion of the system and thus restrict both what a \module\
can do and how \modules\ can be arranged. FiST, for instance, does not
provide a way to deal with structures on the disk directly -- it provides
only ``wrapper'' functionality around existing file
systems. %% (Wrapfs~\cite{zadok99stackable, zadok99extending} is similar.)
GEOM, on the other hand, deals only with the block device layer, and has no
way to work with the file systems stored on those block devices. Neither
has a formal way of specifying or honoring complex write-ordering
information, which is what \chdescs\ in \Kudos\ provide. We imagine that
systems like these could be adapted to work with \chdescs, giving the
benefits of both ideas.

\paragraph{Applications}

A variety of extensions to file systems and disk interfaces have been proposed
in recent work, like the FS2 Free Space File System~\cite{huang05fs2},
encrypting file systems like NCryptfs~\cite{wright03ncryptfs}, and type-safe
disks~\cite{sivathanu06typesafe}. Although we have not implemented \Kudos\
\modules\ for these extensions, we believe that the \Kudos\ \module\ system is
flexible enough to allow straightforward implementations of these ideas as
system \modules.
%% NCryptfs has been written in a
%% stackable way already, allowing it to be easily adapted for new underlying file
%% systems -- but FS2 is currently specific to ext2, since it deals directly with
%% low-level disk structures. The \Kudos\ \module\ interface should allow such an
%% extension to be written in a portable way, giving it the same benefit.
