\section{Consistency Models}
\label{sec:using}

Soft updates, journaling, and other consistency models
correspond to different \chdesc\ arrangements.
%% , so these features can be added to
%% the system as \modules\ which appropriately connect or reconnect the \chdescs.
%
\Kudos's current file system \modules\ generate dependencies corresponding
to soft updates orderings by default~\cite{ganger00soft}.
%
Adding an independent journal \module\ rearranges the \chdescs\ to provide
journaling semantics.
%
%% for maximum speed and minimum durability, a \module\ that disconnects
%% \chdesc\ dependencies can be added.
%
%% Many other semantics could be supported by the addition of similar modules. In
%% this section, we will discuss how \chdescs\ are used to implement different
%% consistency semantics in \Kudos.

% -*- mode: latex; tex-main-file: "paper.tex" -*-

\paragraph{Soft updates}
\label{sec:using:softupdate}

\begin{comment}
% This paragraph is basically replicated in sec:patch:examples
%% Generally, a file system image is consistent if a program like \emph{fsck}
%% would report no errors -- that is, all the structures are in a completely
%% correct organization.  
%
Soft updates enforces orderings between disk writes that maintain
a ``relatively consistent'' file system state at all times, speeding
reboot by avoiding \emph{fsck}.
%
Not all invariants can be preserved, so soft updates relaxes the least
critical: a crash or sudden reboot can leak blocks or other disk
structures, but will never create inconsistencies that could lead to
metadata corruption.
%
%% It is not generally possible to maintain this invariant,
%% so in the ``soft updates''~\cite{ganger00soft} system this definition is
%% relaxed slightly by allowing some structures (like inodes or disk blocks) to be
%% marked as allocated even though there are no references to them. The key
%% observation is that this will not cause any harm beyond making the structures
%% unusable until they are discovered and reclaimed, which can be done safely in
%% the background while the file system is in use.
%
%% This relaxed consistency invariant can be implemented by carefully ordering the
%% writes to the disk. A simple set of rules governing the order for writes,
%% mostly consisting of the idea that a structure should never be marked free
%% while a reference to a structure still exists on the disk, accomplishes this
%% easily. 
%
\end{comment}
In the BSD UFS implementation of soft updates, each UFS operation is
represented in memory by a structure encapsulating the disk changes
necessary to implement that operation. As a result, many
specialized data structures represent the different possible file system
operations. These structures, their relationships to one another, and their uses
for tracking and enforcing dependencies are quite
complex~\cite{mckusick99soft}.

\Kudos, in contrast, represents all dependencies by \patches\ and relies on
general optimizations to reduce memory usage.
%
The resulting system is not as optimized as the BSD implementation, but
much simpler to specify and easier for other modules to manipulate.
%
The \patches\ that make up a file system operation are connected to specify
the order in which the changes must be written to disk.
%
For instance, most file systems require at least two \patches\ to remove
a block from a file, one ($p$) to clear the block reference from the file's
block list and one ($q$) to mark the block as free; adding the $q \PDDepend
p$ dependency straightforwardly implements soft updates semantics.
%
Another example is given in Section~\ref{sec:patch:examples}.

\begin{comment}
Figure~\ref{fig:softupdate} shows another example, the \patches\ generated in
UFS when a file is extended by one block.\footnote{The dependencies differ
from the ext2 dependencies for a similar operation in Figure~\ref{f:opt}
because of a difference in file system semantics.}

\begin{figure}[t]
  \centering
  \includegraphics[width=92pt]{fig/figures_3}
  \caption{\label{fig:softupdate} Soft updates \patches\
  for extending a UFS inode by one block.
  Attaching the block pointer to the inode depends on
  initializing the block (Clear) and updating the free block map (Alloc).
  Updating the inode's size depends on writing the block
  pointer.}
\end{figure}
\end{comment}

%% The difference between soft updates and \Kudos\ is that soft updates tracks
%% these updates to the disk image at the level of file system operations, which
%% are specific to the file system in use (which, in practice, is FFS), while
%% \Kudos\ represents the changes in a file system independent way.

\begin{comment}
The \Kudos\ approach can take more memory than the BSD soft updates approach,
which limits the state required by any individual file system operation; we do
not yet address the issue of memory exhaustion.
%
However, \Kudos\ separates dependency enforcement from dependency
specification. This makes the actual implementation easier to read, and allows
the dependency structure to be examined and modified by other \modules\ of the
system that may not have any idea what the changes are actually doing.
\end{comment}


% -*- mode: latex; tex-main-file: "paper.tex" -*-

\subsection{Journal}
\label{sec:modules:journal}

The journal module sits below a regular file system, such as ext2, and transforms
incoming \patches\ into patches implementing journal transactions.
%
File system blocks are copied into the journal; a commit record depends on the
journal \patches; and the original file system \patches\ depend in turn on the
commit record.
%
Any soft updates-like dependencies among the original \patches\ are removed,
since they are not needed when the journal handles consistency; however, the
journal does care to ensure that user-specified dependencies, such as
\patchgroups, are not violated.
%
%% itself provides consistency for each high-level file system operation by
%% replaying outstanding transactions on recovery.
%% \footnote{However, it does take care to ensure that the user-specified
%% dependencies described in \S\ref{sec:patchgroup} are not violated.}
%
The journal format is similar to ext3's~\cite{tweedie98journaling}: a
transaction contains a list of block numbers, the data to be written to
those blocks, and finally a single commit record.
%
Although the journal modifies existing \patches' direct dependencies, it
ensures that the new dependencies will never introduce a block-level
cycle.

Like ext3, transactions are required to commit in sequence. Therefore the
journal \module\ sets each commit record to depend on the previous commit record, and each
completion record to depend on the previous completion record. This allows
multiple outstanding transactions in the journal, which benefits performance,
but ensures that in the event of a crash, the journal will contain only
contiguous sequential transactions.

Since the commit record is created at the end of the transaction, the journal
\module\ uses a special \noop\ \patch\ explicitly held in memory to prevent
file system changes from being written to the disk until the transaction is
complete. This \noop\ \patch\ is set to depend on the previous transaction's
completion record, which prevents merging between transactions while allowing
merging within a transaction. This temporary dependency is removed when the
real commit record is created and its dependencies are set up as described
above.

%Due to this design, the journal \module\ is completely independent of any
%specific file system. It is a block device \module\ that automatically journals
%whatever file system is stored on it. In fact, the incoming \patches\ need not
%be arranged for soft updates, or for that matter in any particular configuration
%at all.

% Is it important to specify how we figure out where transaction boundaries
% are? It seemed confusing to one reviewer due to this section preceeding the
% modules section.

Our journal module prototype can run in full data journal mode, where every
updated block is written to the journal, or in metadata-only mode, where only
blocks containing file system metadata are written to the journal. It can
tell which blocks are which by looking for a special flag on each \patch\ set
by the UHFS module.

We provide several other modules that modify dependencies, including an
``asynchronous mode'' module that removes all dependencies, allowing the
buffer cache to write blocks in any order.
%
This implements similar semantics as existing file systems like ext2 in
asynchronous write mode.


\paragraph{Asynchronous writes}
\label{sec:modules:unlink}

Finally, we also wrote a trivial module that removes all dependencies from
incoming patches, allowing the buffer cache to write blocks in any order.
%
This implements similar semantics to existing file systems like ext2 in
asynchronous write mode.

%% In the event that no consistency semantics are desired, for instance to compare
%% a prototype implementation of a system for working with explicit dependency
%% information to an existing asynchronous file system like ext2, \Kudos\ provides
%% a \module\ which unlinks the dependencies from all \chdescs\ passing through it.
%% By using this \module, the \chdesc\ graph will not enforce any ordering at all,
%% and the buffer cache will be free to write blocks in any order. \Chdescs\ are
%% not really necessary to implement this ``consistency'' protocol, but they
%% nevertheless support it as a degenerate case.
