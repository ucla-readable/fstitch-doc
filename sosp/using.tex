\section{Consistency Models}
\label{sec:using}

Soft updates, journaling, and other consistency models
correspond to different \chdesc\ arrangements.
%% , so these features can be added to
%% the system as \modules\ which appropriately connect or reconnect the \chdescs.
%
\Kudos's current file system \modules\ generate dependencies corresponding
to soft updates orderings by default~\cite{ganger00soft}.
%
Adding an independent journal \module\ rearranges the \chdescs\ to provide
journaling semantics.
%
%% for maximum speed and minimum durability, a \module\ that disconnects
%% \chdesc\ dependencies can be added.
%
%% Many other semantics could be supported by the addition of similar modules. In
%% this section, we will discuss how \chdescs\ are used to implement different
%% consistency semantics in \Kudos.

% -*- mode: latex; tex-main-file: "paper.tex" -*-

\paragraph{Soft updates}
\label{sec:using:softupdate}

%% Generally, a file system image is consistent if a program like \emph{fsck}
%% would report no errors -- that is, all the structures are in a completely
%% correct organization.  
%
Soft updates enforces orderings between disk writes that maintain
a relatively consistent file system state at all times, speeding reboot
by avoiding \emph{fsck}.
%
Not all invariants can be preserved, so soft updates relaxes the least
critical: a crash or sudden reboot can leak blocks or other disk
structures, but will never create more serious inconsistencies.
%
%% It is not generally possible to maintain this invariant,
%% so in the ``soft updates''~\cite{ganger00soft} system this definition is
%% relaxed slightly by allowing some structures (like inodes or disk blocks) to be
%% marked as allocated even though there are no references to them. The key
%% observation is that this will not cause any harm beyond making the structures
%% unusable until they are discovered and reclaimed, which can be done safely in
%% the background while the file system is in use.
%
%% This relaxed consistency invariant can be implemented by carefully ordering the
%% writes to the disk. A simple set of rules governing the order for writes,
%% mostly consisting of the idea that a structure should never be marked free
%% while a reference to a structure still exists on the disk, accomplishes this
%% easily. 
%
In the BSD UFS implementation of soft updates, each UFS operation is
represented in memory by a structure encapsulating the disk changes
necessary to implement that operation. As a result, many
specialized data structures represent the different possible file system
operations. These structures, their relationships to one another, and their uses
for tracking and enforcing dependencies are quite
complex~\cite{mckusick99soft}.

\Kudos, in contrast, represents all dependencies by patches and relies on
general optimizations to reduce memory usage.
%
The resulting system is not as optimized as the BSD implementation, but
much simpler to specify and easier for other modules to manipulate.
%
The \chdescs\ that make up a file system operation are connected to specify
the order in which the changes must be written to disk.
%
For instance, most file systems require at least two \patches\ to remove
block from a file, one ($p$) to clear the block reference from the file's
block list and one ($q$) to mark the block as free; adding the $q \PDDepend
p$ dependency straightforwardly implements soft updates semantics.
%
Figure~\ref{fig:softupdate} shows another example, the patches generated in
UFS when a file is extended by one block.\footnote{The dependencies differ
from the ext2 dependencies for a similar operation in Figure~\ref{f:opt}
because of a difference in file system semantics.}

\begin{comment}
appr to the
block out when a block is removed from a file, we create (at least) two
\chdescs\ in most file systems: one that clears out the reference to that
block number in the file's list of blocks, and one that marks the block as
free. By hooking up the second \chdesc\ to depend upon the first, we can
implement the soft updates semantic straightforwardly. Another example is
depicted in Figure~\ref{fig:softupdate}.
\end{comment}

\begin{figure}[t]
  \centering
  \includegraphics[width=92pt]{fig/figures_3}
  \caption{\label{fig:softupdate} Soft updates \patches\
  for extending a UFS inode by one block.
  Attaching the block pointer to the inode depends on
  initializing the block (Clear) and updating the free block map (Alloc).
  Updating the inode's size depends on writing the block
  pointer.}
\end{figure}

%% The difference between soft updates and \Kudos\ is that soft updates tracks
%% these updates to the disk image at the level of file system operations, which
%% are specific to the file system in use (which, in practice, is FFS), while
%% \Kudos\ represents the changes in a file system independent way.

\begin{comment}
The \Kudos\ approach can take more memory than the BSD soft updates approach,
which limits the state required by any individual file system operation; we do
not yet address the issue of memory exhaustion.
%
However, \Kudos\ separates dependency enforcement from dependency
specification. This makes the actual implementation easier to read, and allows
the dependency structure to be examined and modified by other \modules\ of the
system that may not have any idea what the changes are actually doing.
\end{comment}


\subsection{Journaling}
\label{sec:consistency:journal}

Although \chdescs\ might initially seem to be specifically designed to implement
soft updates-like consistency semantics, they are in fact much more flexible and
can be used to implement journaling as well. In a journaling file system,
changes to disk structures are written to a journal before being written to the
main file system area on the disk, and a single disk block (the \emph{commit
record}) is written after the journal is written. Once the commit record has
been written, the changes (which collectively are called a \emph{transaction})
are considered to have been made to the file system: if the system crashes, the
data from the journal will be copied into the main file system as part of
recovery. After the commit record has been written, the original changes may be
written in any order desired, and once they have been written, the commit record
may be erased and the portion of the journal storing the data it referenced can
be reused.

Almost all of this description of journaling translates directly into \chdesc\
dependencies. The incoming \chdescs\ must be rearranged to implement the new
structure, but for the most part this transformation is straightforward. There
are two special situations which the journal \module\ must handle, however.
First, with soft updates, \chdescs\ can always be written to the disk in order
to empty the cache, while the journal must be able to ``lock'' \chdescs\ into
the cache while transactions are in progress. Second, the commit record is
created at the very end of the transaction, but the file system changes created
during the transaction (and thus before it) must be made to depend on it.
Ordinarily this is not allowed, in order to prevent cycles.

To accomplish the first of these tasks, the journal \module\ advertises a
managed \noop\ \chdesc\ (\S\ref{sec:design:chdescs:noop}) to \modules\ above it,
which they must make all changes they create depend on. This special \chdesc\ is
not considered satisfied until the journal \module\ explicitly satisfies it, so
no changes which depend on it will be written from the cache.  At the end of the
transaction, the journal \module\ satisfies this \chdesc, allowing all the
changes to be written to disk.

To accomplish the second task, a special flag is used to override the normal
rule prohibiting the addition of new dependencies to a \chdesc\ after it is
created. The condition for using this flag is a static proof that no cycle can
result from its use (immediately or in the future); we have determined this to
be the case for the journal \module\ by hand.

\begin{figure}
  \centering
  \includegraphics[width=\hsize]{fig/figures_2}
  \caption{\label{fig:journal} Journal \chdesc\ graph for the
    change in Figure~\ref{fig:softupdate}. Empty circles are
    ``\noop'' \chdescs\ with no associated block data.}
\end{figure}

Figure~\ref{fig:journal} shows the \chdesc\ configuration which is created by
applying this transformation to the \chdescs\ in Figure~\ref{fig:softupdate}.
The original four \chdescs\ have been modified to depend on a journal commit
record, and no longer have explicit dependencies on each other. The commit
record depends on blocks journal blocks containing copies of the changes.
Finally, a the commit record can be marked as completed once the original four
\chdescs\ have been written. This transformation is performed incrementally as
\chdescs\ arrive. The resulting journal on disk is similar in format to those
generated by ext3~\cite{tweedie98journaling} -- it has a list of block numbers,
followed by the data which should be in those blocks. Finally, there is a commit
record which applies to the whole set.

A particularly nice property of this arrangement is that the journal \module\ is
completely independent of any specific file system. It is a block device
\module\ that automatically journals whatever file system is stored on it.
Further, by changing our journal \module\ to journal only \chdescs\ that modify
file system metadata -- and by adding additional dependencies to prevent
premature reuse of blocks -- we could even obtain metadata-only journaling (as
opposed to the full data journaling described here). The extra \chdesc\
dependencies would serve the same purpose as the special hooks and corner cases
surrounding reuse of blocks discussed in \cite{tweedie00ext3}. The journal
\module\ can automatically identify metadata \chdescs\ because of the \LFS\
interface described in \S\ref{sec:design:interfaces}, and the UHFS module which
is responsible for writing to all non-metadata blocks. Other block device
layering systems, like GEOM~\cite{geom} or JBD in Linux, would or do need
special hooks into file system code to determine what disk changes represent
metadata in order to do metadata-only journaling. \Chdescs\ and the \LFS\
interface allow us to do this automatically.


\paragraph{Asynchronous writes}
\label{sec:modules:unlink}

Finally, we also wrote a trivial module that removes all dependencies from
incoming patches, allowing the buffer cache to write blocks in any order.
%
This implements similar semantics to existing file systems like ext2 in
asynchronous write mode.

%% In the event that no consistency semantics are desired, for instance to compare
%% a prototype implementation of a system for working with explicit dependency
%% information to an existing asynchronous file system like ext2, \Kudos\ provides
%% a \module\ which unlinks the dependencies from all \chdescs\ passing through it.
%% By using this \module, the \chdesc\ graph will not enforce any ordering at all,
%% and the buffer cache will be free to write blocks in any order. \Chdescs\ are
%% not really necessary to implement this ``consistency'' protocol, but they
%% nevertheless support it as a degenerate case.
