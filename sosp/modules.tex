\section{\Modules}
\label{sec:modules}

This section describes several \Kudos\ \modules\ which contribute significantly
to the overall functionality of \Kudos, or which demonstrate unique or
important capabilities of \chdescs\ and the \module\ system. Some \modules,
like the journal \module\ and the write-back cache, have already been described
(see \S\ref{sec:using:journal} and \S\ref{sec:using:wbcache}), while others
like the write-through cache, partitioner, and memory block device, have
functions which are fairly obvious from their names and need no further
description.

% -*- mode: latex; tex-main-file: "paper.tex" -*-

%% \subsection {Interfaces}
\label{sec:modules:interfaces}

\begin{comment}
New \modules\ are
simple to write, and by changing the \module\ arrangement, a broad range of
behaviors can be implemented. It's also easy to tell what behavior a given
arrangement will give just by looking at the connections between the \modules.
\end{comment}

A complete \Kudos\ configuration is composed of many \modules, making it
a finer-grained variant of a stackable file system.
%
There are three major types of \modules.
%
Closest to the disk are block device (BD) \modules, which have a fairly
conventional block device interface with interfaces such as ``read block'' and
``flush''. 
%
Closest to the system call interface are \emph{common file system} (CFS)
\modules, which have an interface similar to VFS~\cite{kleiman86vnodes}. 
%
\Kudos\ also supports an intermediate interface between BD and CFS.
%
This \emph{low-level file system} (\LFS) interface helps divide file system
implementations into code common across block-structured file systems and
code specific to a given file system layout.
%
\begin{comment}
A
\Kudos\ file system designer combines modules with all three interfaces in many
ways -- a departure from stackable file systems, which act only at the VFS/CFS
layer. \Kudos\ \modules\ are implemented in C using structures of function
pointers to achieve object oriented behavior, very much like the rest of the
Linux kernel.
\end{comment}
%
The \LFS\ interface has functions to allocate blocks, add blocks to files,
allocate file names, and other file system micro-operations. A \module\
implementing the \LFS\ interface defines how bits are laid out on the disk, but
doesn't have to know how to combine the micro-operations into larger, more
familiar file system operations. A generic CFS-to-\LFS\ \module\ called UHFS
(``universal high-level file system'') decomposes the larger file write, read,
append, and other standard operations into \LFS\ micro-operations. 
%
File system extensions like those often implemented by stackable file
systems would generally use the CFS interface; for example, we wrote a
simple CFS module that provides case-insensitive access to a case-sensitive
file system.
%
File system implementations, such as our ext2 and UFS implementations, use
the \LFS\ interface.


\Patches\ are explicitly part of the \LFS\ interface.
%
Every \LFS\ function that might modify the file system takes a
\texttt{\textit{patch\char`\_t **p}} argument.
%
Before the function is called, \texttt{*p} should be set to the \patch,
if any, on which the modification should depend;
%
when the function returns, \texttt{*p} will be set to some \patch\
corresponding to the modification itself.
%
(\Noop\ \patches\ allow this interface to generalize to multiple
dependencies.)
%
For example, this function is called to append a block to an \LFS\ inode
(which is called ``\verb+fdesc_t+''):

\begin{small}
\begin{alltt}
int (*append_file_block)(LFS_t *module, 
   fdesc_t *file, uint32_t block, patch_t **p);
\end{alltt}
\end{small}

\noindent%
This design lets \LFS\ modules examine and modify the dependency structure.


\subsection{UHFS}
\label{sec:modules:uhfs}

This one \module, called the ``universal high-level file system'' or
UHFS, can be used with many different \LFS\ \modules\ implementing different file
systems. (There is also a generic VFS-to-CFS layer, to interface with Linux. See
\S\ref{sec:implementation} for details.)
\todo{fix this intro paragraph}

The UHFS \module\ encompasses all logic for decomposing higher level CFS calls
into lower level \LFS\ calls, and for connecting the resulting \chdescs\
together. The finer granularity of \LFS\ calls divides the problem space into
smaller chunks. Since the issue of how to tie the \LFS\ micro-ops together has
already been solved, file system \module\ developers can give more attention to
the particulars of the file system, such as how to allocate a new filename or
how to look up the Nth data block for a file. To see how this simplifies the
development process, consider the VFS \texttt{write()} call, which has the task
of writing some amount of data to a file at a given offset. In \Kudos, the logic
to determine the correct offsets within blocks and whether new blocks must be
allocated is built into UHFS. A file system \module\ need only implement four
\LFS\ calls: \texttt{get\_file\_block()}, \texttt{allocate\_block()},
\texttt{append\_block()}, and \texttt{write\_block()}. Additionally, the
granularity of calls at the \LFS\ layer makes it an appropriate layer for
inserting test harnesses and developing file system unit tests.


\subsection{Loopback Block Device}
\label{sec:modules:loop}

The loopback block device is a BD module which uses a file in an LFS module as
its underlying data store. It is very similar to the device of the same name in
Linux, however it has one critical difference. The \Kudos\ loopback device is
aware of \chdescs, and therefore when a filesystem stored on a loopback device
sets up \chdesc\ dependencies, they will be correctly forwarded to and honored
by the rest of the system.

\begin{figure}[htb]
  \centering
  \includegraphics[height=3in]{fig/figures_1}
  \caption{A running \Kudos\ configuration. {\it/} is a soft updated
    file system on an IDE drive; {\it/loop} is an externally journaled
    file system on loop devices.}
  \label{fig:kfs-graph}
\end{figure}

Figure~\ref{fig:kfs-graph} shows an example configuration taking advantage of
this ability. A file system image is mounted with an external journal, both of
which are loopback block devices stored on the root file system, (which uses
soft updates). The journaled file system's ordering requirements are sent
through the loopback device as \chdescs, allowing dependency information to be
maintained across boundaries that might otherwise lose that information. In
contrast, without \chdescs\ and the ability to forward \chdescs\ through
loopback devices, BSD cannot express soft updates' consistency requirements
through loopback devices. The \modules\ in Figure~\ref{fig:kfs-graph} are a
complete and working \Kudos\ configuration, and although the use of a loopback
device is somewhat contrived in the example, they are increasingly being used in
conventional operating systems. For instance, Mac OS X uses them in order to
allow users to encrypt their home directories.

\subsection{Case Insensitivity}
\label{sec:modules:icase}

\todo{case insensitivity}
\todo{patch unhook}
\todo{sync directory}
\todo{file hiding}
\todo{umsdos?}
