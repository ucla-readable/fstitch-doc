\section{\Modules}
\label{sec:modules}

This section describes several \Kudos\ \modules\ which contribute significantly
to the overall functionality of \Kudos, or which demonstrate unique or
important capabilities of \chdescs\ and the \module\ system. Some \modules,
like the journal \module\ and the write-back cache, have already been described
(see \S\ref{sec:using:journal} and \S\ref{sec:using:wbcache}), while others
like the write-through cache, partitioner, and memory block device, have
functions which are fairly obvious from their names and need no further
description.

% -*- mode: latex; tex-main-file: "paper.tex" -*-

%% \subsection {Interfaces}
\label{sec:modules:interfaces}

\begin{comment}
New \modules\ are
simple to write, and by changing the \module\ arrangement, a broad range of
behaviors can be implemented. It's also easy to tell what behavior a given
arrangement will give just by looking at the connections between the \modules.
\end{comment}

A complete \Kudos\ configuration is composed of many \modules, making it
a finer-grained variant of a stackable file system.
%
There are has three major types of \modules.
%
Closest to the disk are block device (BD) \modules, which have a fairly
conventional block device interface with interfaces such as ``read block'' and
``flush''. 
%
Closest to the system call interface are \emph{common file system} (CFS)
\modules, which have an interface similar to VFS~\cite{kleiman86vnodes}. 
%
\Kudos\ also supports an intermediate interface between BD and CFS.
%
This \emph{low-level file system} (\LFS) interface helps divide file system
implementations into common (reusable) code and file system specific code. 
%
\begin{comment}
A
\Kudos\ file system designer combines modules with all three interfaces in many
ways -- a departure from stackable file systems, which act only at the VFS/CFS
layer. \Kudos\ \modules\ are implemented in C using structures of function
pointers to achieve object oriented behavior, very much like the rest of the
Linux kernel.
\end{comment}
%
The \LFS\ interface has functions to allocate blocks, add blocks to files,
allocate file names, and other file system micro-operations. A \module\ implementing
the \LFS\ interface should define how bits are laid out on the disk, but doesn't
have to know how to combine the micro-operations into larger, more familiar file system
operations. A generic CFS-to-\LFS\ \module\ decomposes the larger file write,
read, append, and other standard operations into \LFS\ micro-operations. This module,
called UHFS, is described in the next section.

Each \chdesc\ on a cached block may or may not be visible to a given \module.
For example, \modules\ that respond to user requests generally view the most
current state of every block -- the block with all \chdescs\ applied. However, a
write-back cache may choose to write some \chdescs\ on a block while reverting
others, since those others currently have outstanding dependencies. In this
case, \modules\ below the write-back cache (i.e. closer to the disk) should view
those \chdescs\ in the reverted state. \Kudos\ provides a block revisioning
library function that automatically rolls back those \chdescs\ that should not
be visible at a particular \module, and then rolls them forward again after that
\module\ is done with the block.


\subsection{UHFS}
\label{sec:modules:uhfs}

The ``universal high-level file system'' \module, or UHFS, implements a generic
translation from high-level CFS calls (essentially the familiar file system API)
to the \LFS\ interface. It can be used with many different \LFS\ \modules\
implementing different file systems. (There is also a generic VFS-to-CFS layer,
to interface with Linux. See \S\ref{sec:implementation} for details.)

The UHFS \module\ encompasses all logic for decomposing higher level CFS calls
into lower level \LFS\ calls, and for connecting the resulting \chdescs\
together. The finer granularity of \LFS\ calls divides the problem space into
smaller chunks. Since the issue of how to tie the \LFS\ micro-ops together has
already been solved, file system \module\ developers can give more attention to
the particulars of the file system, such as how to allocate a new filename or
how to look up the Nth data block for a file. To see how this simplifies the
development process, consider the VFS \texttt{write()} call, which has the task
of writing some amount of data to a file at a given offset. In \Kudos, the logic
to determine the correct offsets within blocks and whether new blocks must be
allocated is built into UHFS. A file system \module\ need only implement four
\LFS\ calls: \texttt{get\_file\_block()}, \texttt{allocate\_block()},
\texttt{append\_block()}, and \texttt{write\_block()}. Additionally, the
granularity of calls at the \LFS\ layer makes it an appropriate layer for
inserting test harnesses and developing file system unit tests.


Perhaps the most unique property of our journal \module, however, is that it can
automatically selectively journal only \chdescs\ that modify file system
metadata -- thus achieving metadata-only journaling \footnote{There are at least
two major variants of metadata-only journaling, depending on when the
non-journaled data is written relative to the commit record. Our journal module
writes it after the commit record; writing it before the commit record requires
very careful checks to avoid premature reuse of blocks~\cite{tweedie00ext3}.}
(as opposed to the full data journaling implied above), without any knowledge of
the file system.
%
The journal \module\ can automatically identify metadata \chdescs\ because of
the \LFS\ interface described in Section~\ref{sec:modules:interfaces}, and the
UHFS module~(\S\ref{sec:modules:uhfs}) which is responsible for writing to all
non-metadata blocks. Other block device layering systems, like GEOM~\cite{geom}
or JBD in Linux, would or do need special hooks into file system code to
determine what disk changes represent metadata in order to do metadata-only
journaling. \Chdescs\ and the \LFS\ interface allow us to do this automatically.


\subsection{Loopback Block Device}
\label{sec:modules:loop}

The loopback block device is a BD module which uses a file in an LFS module as
its underlying data store. It is very similar to the device of the same name in
Linux, however it has one critical difference. The \Kudos\ loopback device is
aware of \chdescs, and therefore when a filesystem stored on a loopback device
sets up \chdesc\ dependencies, they will be correctly forwarded to and honored
by the rest of the system.

\begin{figure}[htb]
  \centering
  \includegraphics[height=3in]{fig/figures_1}
  \caption{A running \Kudos\ configuration. {\it/} is a soft updated
    file system on an IDE drive; {\it/loop} is an externally journaled
    file system on loop devices.}
  \label{fig:kfs-graph}
\end{figure}

Figure~\ref{fig:kfs-graph} shows an example configuration taking advantage of
this ability. A file system image is mounted with an external journal, both of
which are loopback block devices stored on the root file system, (which uses
soft updates). The journaled file system's ordering requirements are sent
through the loopback device as \chdescs, allowing dependency information to be
maintained across boundaries that might otherwise lose that information. In
contrast, without \chdescs\ and the ability to forward \chdescs\ through
loopback devices, BSD cannot express soft updates' consistency requirements
through loopback devices. The \modules\ in Figure~\ref{fig:kfs-graph} are a
complete and working \Kudos\ configuration, and although the use of a loopback
device is somewhat contrived in the example, they are increasingly being used in
conventional operating systems. For instance, Mac OS X uses them in order to
allow users to encrypt their home directories.

\section{UFS implementation}
\label{sec:ufs}

\begin{itemize}
\item Motivation: Show Kudos is capable of accommodating soft updates
  sematics by writing an implementation of UFS.
\item What is UFS / soft updates.
  \begin{itemize}
  \item UFS is the modern incarnation of the classic FFS used in BSD.
    The general concepts for UFS are well understood and implementations
    exist for many UNIX-like operating systems.
  \item Soft updates for UFS is one of the biggest innovations in file
    systems in terms of lowering the overhead required to provide file
    system consistency. By carefully ordering writes to disk, soft updates
    avoid the need for synchronous writes or duplicate writes to a journal.
  \end{itemize}
\item Kudos UFS Implementation
  \begin{itemize}
  \item The LFS interface naturally divides the implementation into a set
    of simpler tasks. With an understanding of the UFS on-disk format and
    soft updates requirements, we added functionality to our UFS module
    in a progressive manner.
  \item Using change descriptors, we can control the ordering for writes
    to disk. This allow us to express the dependencies between different
    writes and achieve soft updates consistency.
  \item Applying our modularity philosophy to UFS, we refactored the file
    system module and separated many functions into sub-modules internal
    to UFS. Modularity makes it easy to exchange one piece of code for
    another. For example, we can change the inode allocation policy or
    the behavior when updating the superblock by swapping in different
    sub-module.
  \end{itemize}
\item Shortcomings
  \begin{itemize}
  \item Our UFS code is mostly complete, but some features are missing:
    i.e. handling triple indirect blocks and sparse files.
  \item Modularity means there exists places where we are less efficient
    than a more integrated implementation. These trouble spots presents
    an opportunity for improvement.
  \end{itemize}
\end{itemize}


\subsection{Case Insensitivity}
\label{sec:modules:icase}

\todo{case insensitivity}
\todo{patch unhook}
\todo{sync directory}
\todo{file hiding}
\todo{umsdos?}
