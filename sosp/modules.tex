% -*- mode: latex; tex-main-file: "paper.tex" -*-

\section{\Modules}
\label{sec:modules}

\begin{comment}
This section describes several \Kudos\ \modules\ which contribute significantly
to the overall functionality of \Kudos, or which demonstrate unique or
important capabilities of \patches\ and the \module\ system. The journal
\module\ has already been described (in \S\ref{sec:using:journal}), while
others like the write-through cache, partitioner, and memory block device, have
functions which are fairly obvious from their names and need no further
description.
\end{comment}

%% \subsection {Interfaces}
\label{sec:modules:interfaces}

\begin{comment}
New \modules\ are
simple to write, and by changing the \module\ arrangement, a broad range of
behaviors can be implemented. It's also easy to tell what behavior a given
arrangement will give just by looking at the connections between the \modules.
\end{comment}

A complete \Kudos\ configuration is composed of many \modules, making it
in some sense a finer-grained variant of a stackable file system.
%
There are three major types of \modules.
%
Closest to the disk are block device (BD) \modules, which have a fairly
conventional block device interface with interfaces such as ``read block'' and
``flush''. 
%
Closest to the system call interface are \emph{common file system} (CFS)
\modules, which have an interface similar to VFS~\cite{kleiman86vnodes}. 
%
\Kudos\ also supports an intermediate interface between BD and CFS.
%
This \emph{low-level file system} (\LFS) interface helps divide file system
implementations into code common across block-structured file systems and
code specific to a given file system layout.
%
\begin{comment}
A
\Kudos\ file system designer combines modules with all three interfaces in many
ways -- a departure from stackable file systems, which act only at the VFS/CFS
layer. \Kudos\ \modules\ are implemented in C using structures of function
pointers to achieve object oriented behavior, very much like the rest of the
Linux kernel.
\end{comment}
%
The \LFS\ interface has functions to allocate blocks, add blocks to files,
allocate file names, and other file system micro-operations. A \module\
implementing the \LFS\ interface defines how bits are laid out on the disk, but
doesn't have to know how to combine the micro-operations into larger, more
familiar file system operations. A generic CFS-to-\LFS\ \module\ called UHFS
(``universal high-level file system'') decomposes the larger file write, read,
append, and other standard operations into \LFS\ micro-operations. 
%
File system extensions like those often implemented by stackable file
systems would generally use the CFS interface; for example, we wrote a
simple CFS module that provides case-insensitive access to a case-sensitive
file system.
%
File system implementations, such as our ext2 and UFS implementations, use
the \LFS\ interface.


\Patches\ are explicitly part of the \LFS\ interface.
%
Every \LFS\ function that might modify the file system takes a
\texttt{\textit{patch\char`\_t **p}} argument.
%
Before the function is called, \texttt{*p} should be set to the \patch,
if any, on which the modification should depend;
%
when the function returns, \texttt{*p} will be set to some \patch\
corresponding to the modification itself.
%
(\Noop\ \patches\ allow this interface to generalize to multiple
dependencies.)
%
For example, this function is called to append a block to an \LFS\ inode
(which is called ``\verb+fdesc_t+''):

\begin{small}
\begin{alltt}
int (*append_file_block)(LFS_t *module, 
   fdesc_t *file, uint32_t block, patch_t **p);
\end{alltt}
\end{small}

\noindent%
This design lets \LFS\ modules examine and modify the dependency structure.

%% \subsection{UHFS}
\label{sec:modules:uhfs}

The ``universal high-level file system'' \module, or UHFS, implements a generic
translation from high-level CFS calls (essentially the familiar file system API)
to the \LFS\ interface. It can be used with many different \LFS\ \modules\
implementing different file systems. (There is also a generic VFS-to-CFS layer,
to interface with Linux. See \S\ref{sec:implementation} for details.)

The UHFS \module\ encompasses all logic for decomposing higher level CFS calls
into lower level \LFS\ calls, and for connecting the resulting \chdescs\
together. The finer granularity of \LFS\ calls divides the problem space into
smaller chunks. Since the issue of how to tie the \LFS\ micro-ops together has
already been solved, file system \module\ developers can give more attention to
the particulars of the file system, such as how to allocate a new filename or
how to look up the Nth data block for a file. To see how this simplifies the
development process, consider the VFS \texttt{write()} call, which has the task
of writing some amount of data to a file at a given offset. In \Kudos, the logic
to determine the correct offsets within blocks and whether new blocks must be
allocated is built into UHFS. A file system \module\ need only implement four
\LFS\ calls: \texttt{get\_file\_block()}, \texttt{allocate\_block()},
\texttt{append\_block()}, and \texttt{write\_block()}. Additionally, the
granularity of calls at the \LFS\ layer makes it an appropriate layer for
inserting test harnesses and developing file system unit tests.


Perhaps the most unique property of our journal \module, however, is that it can
automatically selectively journal only \chdescs\ that modify file system
metadata -- thus achieving metadata-only journaling \footnote{There are at least
two major variants of metadata-only journaling, depending on when the
non-journaled data is written relative to the commit record. Our journal module
writes it after the commit record; writing it before the commit record requires
very careful checks to avoid premature reuse of blocks~\cite{tweedie00ext3}.}
(as opposed to the full data journaling implied above), without any knowledge of
the file system.
%
The journal \module\ can automatically identify metadata \chdescs\ because of
the \LFS\ interface described in Section~\ref{sec:modules:interfaces}, and the
UHFS module~(\S\ref{sec:modules:uhfs}) which is responsible for writing to all
non-metadata blocks. Other block device layering systems, like GEOM~\cite{geom}
or JBD in Linux, would or do need special hooks into file system code to
determine what disk changes represent metadata in order to do metadata-only
journaling. \Chdescs\ and the \LFS\ interface allow us to do this automatically.


\subsection{Write-Back Cache and Block Revisioning}
\label{sec:modules:wbcache}

The write-back cache is in some sense the \module\ that does all the real work
in \Kudos. There can be many write-back caches in a configuration at once, but
each is responsible for holding on to the \chdescs\ sent to it by connected
\modules\ until they can be safely sent on towards the disk. To a write-back
cache, the complex consistency protocols that other \modules\ want to enforce
are nothing more than sets of dependencies among \chdescs\ -- it has no idea
what consistency protocol (or protocols) it is implementing, if any at all. Yet
it is the \module\ that ends up doing most of the work to make sure that
\chdescs\ are written in an acceptable order.

Even though a write-back cache has such an important and central role in the
system, there's not a lot to it. The current write-back cache is a simple LRU
cache, but with an important twist: blocks can't be evicted unless all the
\chdescs\ on them are ``ready'' to be sent to the next \module\ (for example,
the disk). So, when looking for a block to evict, the cache may not be able to
evict any block it chooses -- but it evicts the least recently used block that
it can.

\subsubsection{Block Cycles}
\label{sec:modules:wbcache:cycles}

Just as with soft updates~\cite{ganger00soft}, the dependencies among \chdescs\
(or just ``updates to the block'' in soft updates) can create cyclic
dependencies among blocks, even though the \chdescs\ themselves do not form a
cycle. To handle this case, some \chdescs\ may need to be ``held back'' in order
to write the others, allowing such cycles to be broken. To effect this behavior,
the write-back cache just holds on to the \chdescs\ that cannot yet be written,
but forwards the others on to the next \module\ as it writes the block. It
cannot evict the block yet, since it is still ``dirty,'' but progress has been
made that can make other blocks evictable.

\subsubsection{Block Revisioning}

Since many \modules\ may be stacked on top of one another in \Kudos,
and since many of them may want to refer to the same block at the same
time, only one copy of the data for each block is kept in memory at a
time. However, different \modules\ may ``know about'' different sets
of \chdescs. For example, in the case outlined in
\S\ref{sec:modules:wbcache:cycles}, the \module\ below the write-back
cache will know about some of the \chdescs\ on a block but not others
(the ones which are not ``ready'' to be sent to it yet). If this
\module\ is the disk, it will need to be able to write a version of
the block's data that does not include the \chdescs\ it does't know
about yet. \Kudos\ provides a revisioning system for blocks which can
automatically ``roll back'' the \chdescs\ which have not yet reached a
particular \module, and then roll them forward again after that
\module\ is done using the previous version of the block's data
(e.g. to write it to disk).

\subsubsection{Cross-device dependencies}

The write-back cache has one other property that makes it useful in \Kudos: it
respects dependencies between one cache and another, so that (for instance)
dependencies between the changes on a file system and its external journal are
properly respected. This is actually not a special case, nor does it require any
extra code -- it is a property that just falls out of the way \chdesc\ graphs
are processed in order to determine which \chdescs\ are ready to move towards
the disk. This property of write-back caches also extends to \opgroups, which
are explained in \S\ref{sec:opgroup}.


\subsection{Loopback Module}
\label{sec:modules:loop}

The loopback module is a BD module that uses a file in an \LFS\ module as
its underlying data store. It is very similar to the device of the same
name in Linux, but with one critical difference: it is aware of \patches.
%
The loopback module preserves its file system's dependencies and forwards
them to its underlying data store.
%
As a result, the data store will honor those dependencies and preserve the
loopback file system's consistency, even if the data store would normally
provide no guarantees for consistency of file system data (e.g., it used
metadata-only journaling).

\begin{figure}[t]
  \centering
  \includegraphics[height=2.5in]{fig/figures_1}
  \caption{A running \Kudos\ configuration. {\it/} is a soft updated
    file system on an IDE drive; {\it/loop} is an externally journaled
    file system on loop devices.}
  \label{fig:kfs-graph}
\end{figure}

Figure~\ref{fig:kfs-graph} shows an example configuration using the
loopback module.
%
A file system image is mounted with an external journal, both of
which are loopback block devices stored on the root file system (which uses
soft updates). The journaled file system's ordering requirements are sent
through the loopback module as \patches, allowing dependency information to be
maintained across boundaries that might otherwise lose that information. In
contrast, without \patches\ and the ability to forward \patches\ through
loopback devices, BSD cannot express soft updates' consistency requirements
through loopback devices. The \modules\ in Figure~\ref{fig:kfs-graph} are a
complete and working \Kudos\ configuration, and although the use of a loopback
device is somewhat contrived in the example, they are increasingly being used in
conventional operating systems. For instance, Mac OS X uses them in order to
allow users to encrypt their home directories.

\begin{comment}
An example configuration taking advantage of this ability could be a file
system image mounted with an external journal, both of which are loopback block
devices stored on the root file system (which could use soft updates). The
journaled file system's ordering requirements are sent through the loopback
device as \patches, allowing dependency information to be maintained across
boundaries that might otherwise lose that information. In contrast, without
\patches\ and the ability to forward \patches\ through loopback devices, BSD
cannot express soft updates' consistency requirements through loopback devices.
Although the use of a loopback device is somewhat contrived in the example,
they are increasingly being used in conventional operating systems. For
instance, Mac OS X uses them in order to allow users to encrypt their home
directories.
\end{comment}

\subsection{ext2 and UFS}

\Kudos\ currently has \modules\ that implement two file system types, Linux
ext2 and 4.2 BSD UFS (Unix File System, the modern incarnation of the Fast File
System~\cite{mckusick84fast}).
%
Both of these \modules\ initially generate dependencies arranged according to the
soft updates rules; other dependency arrangements are achieved by transforming these.
%
% We briefly describe these \modules\ as examples of complete and relatively
% complex file systems.
%
To the best of our knowledge, our implementation of ext2 is the first to provide
soft updates consistency guarantees.
%
%% UFS is of particular interest because it is the only file system that has been
%% extended with both soft updates and journaling.~\cite{seltzer00journaling}
%
% We chose UFS1 over UFS2, as UFS1 is well established and more widely
% supported.
%
We verified that file systems generated by our modules are considered
correct by their reference implementations on FreeBSD and Linux by mounting
and running \emph{fsck} on \Kudos-generated disk images.

Both \modules\ are implemented at the \LFS\ interface. 
%
%% This keeps properties
%% specific to the file system (such as the on-disk format and rules governing
%% block allocation) hidden within the \module. 
%
The \modules\ create \patches\ for all their changes to the disk and
connect them to form subgraphs that enforce the soft updates
rules~\cite{ganger00soft} as applied to each file system. 
%
The UHFS \module\ is also aware of soft updates order when necessary; when
it implements a single operation using multiple \LFS\ calls, it hooks the
resulting \patches\ up in the correct order.
%
Unlike FreeBSD's soft updates implementation, once the dependencies have
been hooked up, the ext2 and UFS \modules\ no longer need to concern
themselves with their \patches, as the block device subsystem tracks and
enforces the dependency orderings.

\begin{comment}
The UFS \module\ provides a good demonstration of some of the flexibility of the
\module\ interfaces in \Kudos. For instance, UFS uses 2KB \emph{fragments} to
store small files efficiently. Once a file gets big enough to require the use of
indirect blocks, UFS changes its allocation policy and starts allocating 16KB
\emph{blocks}, where a block is made up of 8 aligned and contiguous fragments.
Our UFS \module\ implements this by using fragments as the basic block size. For
large files, the UFS \module\ internally allocates a block, but returns only the
first fragment in that block at the \LFS\ level. The next 7 allocation calls
will simply return the subsequent fragments in the already allocated block.
% In this way, the UFS \module\ can stay consistent internally without special
% support from other \Kudos\ modules.
\end{comment}

%\subsection{Ext2}
\label{sec:modules:ext2}

\Kudos\ currently has implementations of three base file system types: a simple
file system without inodes called josfs, Linux ext2, and 4.2 BSD UFS. File
system \modules\ in \Kudos\ are supposed to generate soft updates dependencies
by default; other dependency arrangements are achieved by transforming these.
We briefly describe ext2 as an example of a complete and relatively complex file
system.

The ext2 \module\ is implemented at the \LFS\ interface. This keeps properties
specific to ext2 (such as the ext2 on-disk format and rules governing block
allocation) hidden within the file system \module. The ext2 \module\ creates
\chdescs\ for all its changes to the disk, and connects them to form
configurations that achieve soft updates consistency. To the best of our
knowledge, this is the first implementation of ext2 to provide soft updates
consistency guarantees. Unlike FreeBSD's soft updates implementation, once the
dependencies have been hooked up, the ext2 \module\ no longer needs to concern
itself with the \chdescs\ it has created. The block device subsystem will track
and enforce the \chdesc\ orderings.

In order to implement soft updates ordering, each \LFS\ call to the ext2
\module\ must connect the \chdescs\ it creates according to the rules for soft
updates~\cite{ganger00soft}. The UHFS \module\ (\S\ref{sec:modules:interfaces})
will connect the resulting \chdesc\ subgraphs together in soft updates order,
completing the implementation of soft updates using \chdescs.

%\section{UFS implementation}
\label{sec:ufs}

\begin{itemize}
\item Motivation: Show Kudos is capable of accommodating soft updates
  sematics by writing an implementation of UFS.
\item What is UFS / soft updates.
  \begin{itemize}
  \item UFS is the modern incarnation of the classic FFS used in BSD.
    The general concepts for UFS are well understood and implementations
    exist for many UNIX-like operating systems.
  \item Soft updates for UFS is one of the biggest innovations in file
    systems in terms of lowering the overhead required to provide file
    system consistency. By carefully ordering writes to disk, soft updates
    avoid the need for synchronous writes or duplicate writes to a journal.
  \end{itemize}
\item Kudos UFS Implementation
  \begin{itemize}
  \item The LFS interface naturally divides the implementation into a set
    of simpler tasks. With an understanding of the UFS on-disk format and
    soft updates requirements, we added functionality to our UFS module
    in a progressive manner.
  \item Using change descriptors, we can control the ordering for writes
    to disk. This allow us to express the dependencies between different
    writes and achieve soft updates consistency.
  \item Applying our modularity philosophy to UFS, we refactored the file
    system module and separated many functions into sub-modules internal
    to UFS. Modularity makes it easy to exchange one piece of code for
    another. For example, we can change the inode allocation policy or
    the behavior when updating the superblock by swapping in different
    sub-module.
  \end{itemize}
\item Shortcomings
  \begin{itemize}
  \item Our UFS code is mostly complete, but some features are missing:
    i.e. handling triple indirect blocks and sparse files.
  \item Modularity means there exists places where we are less efficient
    than a more integrated implementation. These trouble spots presents
    an opportunity for improvement.
  \end{itemize}
\end{itemize}


\begin{comment}
\subsection{Case Insensitivity}
\label{sec:modules:icase}

\Kudos\ provides a case insensitivity \module\ at the CFS layer. It allows file
systems such as UFS and ext2, which inherently handle filenames case
sensitively, to become case insensitive. If a user needs to run an application
that expects the underlying filenames to be case insensitive on top of a case
sensitive file system, they can simply add this \module\ to the \Kudos\
\module\ graph. By allowing a \module\ to intercept actions between the CFS and
\LFS\ layers, filename transformations can be made transparently for the user
regardless of the on-disk storage format of the actual filename.
\end{comment}
