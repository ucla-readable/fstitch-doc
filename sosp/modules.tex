\section{\Modules}
\label{sec:modules}

This section describes several \Kudos\ \modules\ which contribute significantly
to the overall functionality of \Kudos, or which demonstrate unique or important
capabilities of \chdescs\ and the \module\ system. Some \modules, like the
journal \module\ and the write-back cache, have already been described (in
\S\ref{sec:using:journal} and \S\ref{sec:using:wbcache}), while others like the
write-through cache, partitioner, and memory block device, have functions which
are fairly obvious from their names and need no further description.

% -*- mode: latex; tex-main-file: "paper.tex" -*-

%% \subsection {Interfaces}
\label{sec:modules:interfaces}

\begin{comment}
New \modules\ are
simple to write, and by changing the \module\ arrangement, a broad range of
behaviors can be implemented. It's also easy to tell what behavior a given
arrangement will give just by looking at the connections between the \modules.
\end{comment}

A complete \Kudos\ configuration is composed of many \modules, making it
a finer-grained variant of a stackable file system.
%
There are three major types of \modules.
%
Closest to the disk are block device (BD) \modules, which have a fairly
conventional block device interface with interfaces such as ``read block'' and
``flush''. 
%
Closest to the system call interface are \emph{common file system} (CFS)
\modules, which have an interface similar to VFS~\cite{kleiman86vnodes}. 
%
\Kudos\ also supports an intermediate interface between BD and CFS.
%
This \emph{low-level file system} (\LFS) interface helps divide file system
implementations into code common across block-structured file systems and
code specific to a given file system layout.
%
\begin{comment}
A
\Kudos\ file system designer combines modules with all three interfaces in many
ways -- a departure from stackable file systems, which act only at the VFS/CFS
layer. \Kudos\ \modules\ are implemented in C using structures of function
pointers to achieve object oriented behavior, very much like the rest of the
Linux kernel.
\end{comment}
%
The \LFS\ interface has functions to allocate blocks, add blocks to files,
allocate file names, and other file system micro-operations. A \module\
implementing the \LFS\ interface defines how bits are laid out on the disk, but
doesn't have to know how to combine the micro-operations into larger, more
familiar file system operations. A generic CFS-to-\LFS\ \module\ called UHFS
(``universal high-level file system'') decomposes the larger file write, read,
append, and other standard operations into \LFS\ micro-operations. 
%
File system extensions like those often implemented by stackable file
systems would generally use the CFS interface; for example, we wrote a
simple CFS module that provides case-insensitive access to a case-sensitive
file system.
%
File system implementations, such as our ext2 and UFS implementations, use
the \LFS\ interface.


\Patches\ are explicitly part of the \LFS\ interface.
%
Every \LFS\ function that might modify the file system takes a
\texttt{\textit{patch\char`\_t **p}} argument.
%
Before the function is called, \texttt{*p} should be set to the \patch,
if any, on which the modification should depend;
%
when the function returns, \texttt{*p} will be set to some \patch\
corresponding to the modification itself.
%
(\Noop\ \patches\ allow this interface to generalize to multiple
dependencies.)
%
For example, this function is called to append a block to an \LFS\ inode
(which is called ``\verb+fdesc_t+''):

\begin{small}
\begin{alltt}
int (*append_file_block)(LFS_t *module, 
   fdesc_t *file, uint32_t block, patch_t **p);
\end{alltt}
\end{small}

\noindent%
This design lets \LFS\ modules examine and modify the dependency structure.


\subsection{UHFS}
\label{sec:modules:uhfs}

This one \module, called the ``universal high-level file system'' or
UHFS, can be used with many different \LFS\ \modules\ implementing different file
systems. (There is also a generic VFS-to-CFS layer, to interface with Linux. See
\S\ref{sec:implementation} for details.)
\todo{fix this intro paragraph}

The UHFS \module\ encompasses all logic for decomposing higher level CFS calls
into lower level \LFS\ calls, and for connecting the resulting \chdescs\
together. The finer granularity of \LFS\ calls divides the problem space into
smaller chunks. Since the issue of how to tie the \LFS\ micro-ops together has
already been solved, file system \module\ developers can give more attention to
the particulars of the file system, such as how to allocate a new filename or
how to look up the Nth data block for a file. To see how this simplifies the
development process, consider the VFS \texttt{write()} call, which has the task
of writing some amount of data to a file at a given offset. In \Kudos, the logic
to determine the correct offsets within blocks and whether new blocks must be
allocated is built into UHFS. A file system \module\ need only implement four
\LFS\ calls: \texttt{get\_file\_block()}, \texttt{allocate\_block()},
\texttt{append\_block()}, and \texttt{write\_block()}. Additionally, the
granularity of calls at the \LFS\ layer makes it an appropriate layer for
inserting test harnesses and developing file system unit tests.


\subsection{Loopback Block Device}
\label{sec:modules:loop}

The loopback block device is a BD module which uses a file in an \LFS\ module as
its underlying data store. It is very similar to the device of the same name in
Linux, however it has one critical difference. The \Kudos\ loopback device is
aware of \chdescs, and therefore when a filesystem stored on a loopback device
sets up \chdesc\ dependencies, they will be correctly forwarded to and honored
by the rest of the system.

\begin{figure}[htb]
  \centering
  \includegraphics[height=3in]{fig/figures_1}
  \caption{A running \Kudos\ configuration. {\it/} is a soft updated
    file system on an IDE drive; {\it/loop} is an externally journaled
    file system on loop devices.}
  \label{fig:kfs-graph}
\end{figure}

Figure~\ref{fig:kfs-graph} shows an example configuration taking advantage of
this ability. A file system image is mounted with an external journal, both of
which are loopback block devices stored on the root file system, (which uses
soft updates). The journaled file system's ordering requirements are sent
through the loopback device as \chdescs, allowing dependency information to be
maintained across boundaries that might otherwise lose that information. In
contrast, without \chdescs\ and the ability to forward \chdescs\ through
loopback devices, BSD cannot express soft updates' consistency requirements
through loopback devices. The \modules\ in Figure~\ref{fig:kfs-graph} are a
complete and working \Kudos\ configuration, and although the use of a loopback
device is somewhat contrived in the example, they are increasingly being used in
conventional operating systems. For instance, Mac OS X uses them in order to
allow users to encrypt their home directories.

\subsection{ext2}
\label{sec:using:ext2}

\Kudos\ currently has implementations of three base file system types: a simple
file system without inodes called JOSFS, Linux ext2, and 4.2 BSD UFS. File
system \modules\ in \Kudos\ are supposed to generate soft updates dependencies
by default; other dependency arrangements can be achieved by transforming these.
We briefly describe ext2 as an example of a complete and relatively complex file
system.

The ext2 \module\ is implemented at the \LFS\ interface. This keeps properties
specific to ext2 (such as the ext2 on-disk format and rules governing block
allocation) hidden within the file system \module. The ext2 \module\ creates
\chdescs\ for all its changes to the disk, and connects them to form
configurations that achieve soft updates consistency. To the best of our
knowledge, this is the first implementation of ext2 to provide soft updates
consistency guarantees. Unlike FreeBSD's soft updates implementation, once the
dependencies have been hooked up, the ext2 \module\ no longer needs to concern
itself with the \chdescs\ it has created. The block device subsystem will track
and enforce the \chdesc\ orderings.

In order to implement soft updates ordering, each \LFS\ call to the ext2
\module\ must connect the \chdescs\ it creates according to the rules for soft
updates~\cite{ganger00soft}. The UHFS \module\ (\S\ref{sec:modules:uhfs})
connects the resulting \chdesc\ subgraphs together in soft updates order,
completing the implementation of soft updates using \chdescs.

\subsection{UFS implementation}
\label{sec:modules:ufs}

\subsubsection {Overview}
UFS (UNIX File System) is the modern incarnation of the classic Berkeley Fast
File System~\cite{mckusick84fast} used in 4.2 BSD. The general concepts for
UFS, such as inodes, cylinder groups, and indirect blocks, are well understood.
UFS implementations exist for many popular UNIX-derived operating systems,
including FreeBSD, Solaris, and Mac OS X. Many vendors have added extentions to
UFS. Examples include UFS with journaling on Solaris and UFS with soft updates
on FreeBSD. Linux's ext2 file system has much in common with UFS, since it
borrowed many ideas from FFS. In \Kudos, we implemented the UFS1 file system as
an LFS \module. We chose UFS because it has been extended in many ways. In
particular, it is the only file system that has been extended with both soft
updates and journaling. We chose UFS1 over UFS2, as UFS1 is well established
and more widely supported.

\subsubsection {Design Philosophy}
As part of the overall \Kudos\ design, the \emph{UHFS} module encompasses all
logic for decomposing higher level CFS calls into lower level LFS calls. The
finer granularity of LFS calls divides the problem space into smaller chunks.
Since the issue of how to tie the LFS micro-ops together has already been
solved, file system \module\ developers can give more attention to the
particulars of the file system, such as how to allocate a new filename and
how to lookup the Nth data block for a file. To show how this simplifies the
developmment process, consider the VFS write() call, which has the task of
writing N bytes of data to a file at a given offset. In \Kudos, the problem is
well defined and requires the implementation of these four LFS calls:
get\_file\_block(), allocate\_block(), append\_block(), and write\_block().
Additionally, the granularity of calls at the LFS layer makes it an appropriate
layer for inserting test harnesses and developing file system unit tests.

In accordance with our ideas on the division of responsibilities between the
file system \module\ and higher \modules, we chose to keep properties specific
to UFS (such as the UFS on-disk format and rules governing block allocation)
hidden within the file system \module. Although file systems can have uncommon
requirements, LFS is flexible enough to accommodate many of them. One problem
we ran into concerns UFS and its use of fragments and blocks. In UFS, typically
a block is divided into 8 equal sized fragments. For space efficiency, UFS
allocates fragments to small files. Once a file gets big enough to require the
use of indirect blocks, then UFS changes its allocation policy and starts
allocating blocks for speed. To implement this property of UFS, we used
fragments as the basic block size. For large files, our UFS \module\ internally
allocates a block, but returns only the first fragment in that block at the LFS
level. The next 7 allocation calls will be no-ops that simply return the
subsequent fragments in the allocated block. Overall, we believe most
conventional file systems fit within the bounds set by the LFS interface.

In order to provide the rest of the system with write-ordering requirements for
soft updates, the UFS \module\ creates \chdescs\ for all its changes to the
disk, and connects them to form configurations that achieve soft updates
consistency. Unlike FreeBSD's soft updates implementation, once the dependencies
have been created, UFS no longer needs to concern itself with the \chdescs\ it
has created. The block device subsystem will track and enforce the \chdesc\
orderings.

\begin{figure}[htb]
  \centering
  \includegraphics[width=108pt]{fig/figures_4}
  \caption{\label{fig:ufsmodules} The UFS \module\ is itself composed of
  sub-\modules. The UFS base \module\ holds the core code that does not
  vary, while plug-in components that should be flexible, such as those that
  relate to block/inode allocation, cylinder groups, directory entries, and the
  superblock, can be easily exchanged.}
\end{figure}

\subsubsection {Modularity}
Previously, \Kudos\ used a minimal file system called JOSFS. Even though JOSFS
is extremely simple, at the time it was the largest \module\ in \Kudos\ in
terms of lines of code. This stems from the fact that much of the actual work
occurs in the file system \modules. For instance, most \chdescs\ in the system
are created when a file system \module\ needs to write to disk. With a real
file system like UFS, a monolithic version will easily be 3 to 4 times larger
than JOSFS. In earlier versions of the \Kudos\ UFS implementation, having one
monolithic \module\ made it difficult to change the algorithms and policies
for particular parts of UFS. In addition, the large amount of code for one
\module\ made it more difficult to manage.

From our experiences with JOSFS, we realized that UFS would become harder to
maintain as it grew in size. The problem of how to generalize the subdivision
within a file system is a difficult one, since not all file systems share the
same characteristics. In light of this issue, we decided to have specialized
interfaces specific to each file system \module. As we implement more file
system \modules, we will discover commonalities between different file systems,
and use that knowledge to better modularize \Kudos.

Applying our modularity philosophy to UFS, we refactored the file system
\module\ and separated many parts into sub-\modules\ internal to UFS, as shown
in Figure~\ref{fig:ufsmodules}. Within the UFS \module, the \emph{base} module
contains the core code for the file system. Using object-orientation
techniques, we encapsulated the code for data structures in the file system,
like the superblock and the cylinder groups. We also recognized two locations
where we can apply different search algorithms. One is with respect to the
allocation functions for blocks, fragments, and inodes. The other is for
searching and writing directory entries (see Figure~\ref{fig:moduleinterface}).
In addition to making the code easier to manage, having a modular file system
means it is easy to try out new algorithms. Currently we have two allocator
modules as a proof of concept for modularity.

\begin{figure}[htb]
Cylinder Group Interface:
\vspace{-0.5\baselineskip}
\begin{scriptsize}
\begin{alltt}
const UFS_cg_t * \textbf{read}(int num);
int \textbf{write_time}(int num, int value, chdesc_t ** head);
int \textbf{write_rotor}(int num, int value, chdesc_t ** head);
int \textbf{write}...
int \textbf{sync}(int num, chdesc_t ** head);
\end{alltt}
\end{scriptsize}

Allocator Interface:
\vspace{-0.5\baselineskip}
\begin{scriptsize}
\begin{alltt}
int \textbf{find_free_block}(fdesc_t * file, int purpose);
int \textbf{find_free_frag}(fdesc_t * file, int purpose);
int \textbf{find_free_inode}(fdesc_t * file, int purpose);
\end{alltt}
\end{scriptsize}
\vspace{-0.5\baselineskip}
\caption{\label{fig:moduleinterface} UFS Internal Module Interfaces}
\end{figure}

The UFS cylinder group module interface regulates access to the cylinder
groups. While there is unrestricted read access to cylinder groups, the
interface limits write access to certain fields within the \emph{UFS\_cg}
struct. This is because, under normal operations, fields like the number of
data blocks per cylinder group remain constant. The cylinder group module can
also define the policy for when changes to the cylinder group are written to
disk. It can, for example, make the policy ``write-through'' and have all
changes immediately go to disk. However, choosing this policy means that on
every block allocation, UFS needs to write to disk \emph{cg\_rotor}, the
position of the last used block. To avoid performance hits like this, we
implemented the ``write-back'' policy instead. To support flushing dirty data
to disk, the cylinder group module interface also has a sync call.

The UFS allocator module interface has similar methods for allocating blocks,
fragments, and inodes. All three methods take in a \emph{file descriptor} and a
\emph{purpose} variable. The UFS \emph{file descriptor} contains all relevant
information pertaining to a given file, including an in-memory copy of the
file's inode. The intent of the \emph{purpose} variable is to provide hints to
the allocator about how the newly allocated resource will be used.  Given these
two pieces of information, as well as cylinder group information from the
previously mentioned module, UFS allocator modules can make informed decisions
to allocate resources in an efficient manner.

\begin{figure}[htb]
  \centering
  \includegraphics[width=192pt]{fig/figures_6}
  \caption{\label{fig:chdescarrange} Change descriptor dependencies, when
  not strictly needed, restrict the possible choices for write ordering.
  This results in suboptimal write ordering and more scans through the
  \chdescs\ for \Kudos. On the left, \chdescs\ C1, C2, and C3 can be written
  in any order. Only one ordering is possible on the right.}
\end{figure}

\subsubsection {Change Descriptor Arrangements}
While optimizing our UFS implementation, we gained some insight and found a
couple practices critical to good performance. First, do not create unnecessary
dependencies between \chdescs. Doing so artificially limits the commit order
for \chdescs, which results in bad performance for several reasons. Not only
will unneeded dependencies force the disk to do more writes and seeks, but
\Kudos\ will have to scan through the \chdesc\ graph multiple times, since the
dependencies prevent the \chdescs\ from being flushed out to disk at once.

In Figure~\ref{fig:chdescarrange}, we have two possible arrangements for three
byte \chdescs. The noop \chdesc\ represents a root node that can reach all
other \chdescs. In the parallel arrangement on the left, \Kudos\ has the
freedom to write \chdescs\ $C_1$, $C_2$, and $C_3$ to disk in any order. All
three \chdescs\ can be marked writable with one graph traversal. In the serial
arrangement on the right, there exists only one valid write ordering. For
\Kudos\ to write this arrangement out to disk, it will have to scan through
the graph three times, since \chdesc\ $C_n$ cannot be marked as writable until
$C_{n-1}$ has been written.
An instance like this came up because our UFS implementation frequently writes
the \emph{cylinder group summaries} out to disk. By simply changing the
arrangement between three \chdescs\ created in a single function, UFS got a
33\% speed increase for several common file operations.

A corollary of this observation is to create \chdescs\ of the minimal size to
avoid accidental overlaps, which in turn, create unnecessary dependencies.
\Chdescs\ can represent changes to regions as small as one byte, and as large
as an entire block. Many times, it can be tedious for developers to calculate
exactly what parts in a large data structure have been modified and need to be
written to disk. As such, laziness will result in the creation of \chdescs\
for the entire data structure. Doing this can be detrimental to performance,
as shown in Figure~\ref{fig:overlap}.

An occurrance of this problem came up for inodes, where we make several
independent modifications to different fields within an inode. In principle,
the \chdescs\ created are conflict-free. Previously, because we did not take
the time to calculate the exact offsets and lengths for the fields that
changed, we just created a \chdesc\ for the entire inode every time we modified
any part of the inode. Thus all changes to any particular inode would always
overlap, causing unnecessary dependencies. Our solution to this problem is a
utility function called \texttt{chdesc\_create\_diff()}, which compares a
modified copy of a data structure to the original, and creates a minimal set of
\chdescs\ accordingly. Due to the frequent use of inodes, one simple use of
\texttt{chdesc\_create\_diff()} in the UFS inode functions reduced \chdesc\
graph traversal time significantly.

\begin{figure}[htb]
  \centering
  \includegraphics[width=64pt]{fig/figures_5}
  \caption{\label{fig:overlap} On inode 17, the gray regions represent
  modified fields that do not overlap. If \chdesc\ A and \chdesc\ B are
  exactly the size of the gray regions, then there is no implicit dependency.
  Making \chdescs\ for the entire inode data structure will, in turn, make
  one \chdesc\ depend on the other because they overlap.}
\end{figure}



\subsection{Case Insensitivity}
\label{sec:modules:icase}

\Kudos\ provides a case insensitivity \module\ at the CFS layer. It allows file
systems such as UFS and ext2, which inherently handle filenames case
sensitively, to become case insensitive. If a user needs to run an application
that expects the underlying filenames to be case insensitive on top of a case
sensitive file system, they can simply add this \module\ to the \Kudos\
\module\ graph. By allowing a \module\ to intercept actions between the CFS and
\LFS\ layers, filename transformations can be made transparently for the user
regardless of the on-disk storage format of the actual filename.

\subsection{\ChDesc\ Unlinking}
\label{sec:modules:unlink}

In the event that no consistency semantics are desired, for instance to compare
a prototype implementation of a system for working with explicit dependency
information to an existing asynchronous file system like ext2, \Kudos\ provides
a \module\ which unlinks the dependencies from all \chdescs\ passing through it.
By using this \module, the \chdesc\ graph will not enforce any ordering at all,
and the buffer cache will be free to write blocks in any order. \Chdescs\ are
not really necessary to implement this ``consistency'' protocol, but they
nevertheless support it as a degenerate case.
