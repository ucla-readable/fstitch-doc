% -*- mode: latex; tex-main-file: "paper.tex" -*-

\section{\Modules}
\label{sec:modules}

A \Kudos\ configuration is composed of \modules\ that cooperate to
 implement file system functionality.
%
\Modules\ fall into three major categories.
%
\emph{Block device} (BD) \modules\ are closest to the disk; they have a fairly
conventional block device interface with interfaces such as ``read block'' and
``flush.''
%
\emph{Common file system} (CFS) \modules\ are closest to the system call
interface, and have an interface similar to VFS~\cite{kleiman86vnodes}. 
%
In between these interfaces are modules implementing a  \emph{low-level file
system} (\LFS) interface, which helps divide file system implementations
into code common across block-structured file systems and code specific to
a given file system layout.
%
The \LFS\ interface has functions to allocate blocks, add blocks to files,
allocate file names, and other file system micro-operations. 
%% A \module\
%% implementing the \LFS\ interface defines how bits are laid out on the disk, but
%% doesn't have to know how to combine the micro-operations into larger, more
%% familiar file system operations. 
A generic CFS-to-\LFS\ \module\ called UHFS
(``universal high-level file system'') decomposes familiar VFS operations
like write, read, and append into \LFS\ micro-operations. 
%
%% File system extensions like those often implemented by stackable file
%% systems would generally use the CFS interface; for example, we wrote a
%% simple CFS module that provides case-insensitive access to a case-sensitive
%% file system.
%
Our ext2 and UFS file system implementations
use the \LFS\ interface.


Modules to examine and modify dependencies via patches passed to them as
arguments.
%
For instance, every \LFS\ function that might modify the file system takes a
\texttt{\textit{patch\char`\_t **p}} argument.
%
Before the function is called, \texttt{*p} is set to the \patch,
if any, on which the modification should depend;
%
when the function returns, \texttt{*p} is set to some \patch\
corresponding to the modification itself.
%
\begin{comment}
(\Noop\ \patches\ allow this interface to generalize to multiple
dependencies.)
\end{comment}
%
For example, this function is called to append a block to an \LFS\ inode
\verb+f+:

\vspace{-0.5\baselineskip}
\begin{small}
\begin{alltt}
int (*append_file_block)(LFS_t *module, 
   fdesc_t *f, uint32_t block, patch_t **p);
\end{alltt}
\end{small}
\vspace{-0.5\baselineskip}

\begin{comment}
\noindent
This design lets \LFS\ modules examine and modify the dependency structure.
\end{comment}



\subsection{ext2 and UFS}

\Kudos\ currently has \modules\ that implement two file system types, Linux
ext2 and 4.2 BSD UFS (Unix File System, the modern incarnation of the Fast File
System~\cite{mckusick84fast}).
%
Both of these \modules\ initially generate dependencies arranged according to the
soft updates rules; other dependency arrangements are achieved by transforming these.
To the best of our knowledge, our implementation of ext2 is the first to provide
soft updates consistency guarantees.
%
%We verified that file systems generated by our modules are considered
%correct by their reference implementations on FreeBSD and Linux by mounting
%and running \emph{fsck} on \Kudos-generated disk images.

Both \modules\ are implemented at the \LFS\ interface. 
%
%% This keeps properties
%% specific to the file system (such as the on-disk format and rules governing
%% block allocation) hidden within the \module. 
%
%The \modules\ create \patches\ for all their changes to the disk and
%connect them to form subgraphs that enforce the soft updates
%rules~\cite{ganger00soft} as applied to each file system. 
%
%The UHFS \module\ is also aware of soft updates order when necessary; when
%it implements a single operation using multiple \LFS\ calls, it hooks the
%resulting \patches\ up in the correct order.
%
Unlike FreeBSD's soft updates implementation, once these modules set up 
dependencies, they no longer need to concern themselves with file system
consistency; the block device subsystem will track
and enforce the dependencies.


\begin{figure}[t]
  \centering
  \includegraphics[height=2.5in]{fig/figures_1}
  \caption{A running \Kudos\ configuration. {\it/} is a soft updated
    file system on an IDE drive; {\it/loop} is an externally journaled
    file system on loop devices.}
  \label{fig:kfs-graph}
\end{figure}


\subsection{Journaling}
\label{sec:consistency:journal}

Although \chdescs\ might initially seem to be specifically designed to implement
soft updates-like consistency semantics, they are in fact much more flexible and
can be used to implement journaling as well. In a journaling file system,
changes to disk structures are written to a journal before being written to the
main file system area on the disk, and a single disk block (the \emph{commit
record}) is written after the journal is written. Once the commit record has
been written, the changes (which collectively are called a \emph{transaction})
are considered to have been made to the file system: if the system crashes, the
data from the journal will be copied into the main file system as part of
recovery. After the commit record has been written, the original changes may be
written in any order desired, and once they have been written, the commit record
may be erased and the portion of the journal storing the data it referenced can
be reused.

Almost all of this description of journaling translates directly into \chdesc\
dependencies. The incoming \chdescs\ must be rearranged to implement the new
structure, but for the most part this transformation is straightforward. There
are two special situations which the journal \module\ must handle, however.
First, with soft updates, \chdescs\ can always be written to the disk in order
to empty the cache, while the journal must be able to ``lock'' \chdescs\ into
the cache while transactions are in progress. Second, the commit record is
created at the very end of the transaction, but the file system changes created
during the transaction (and thus before it) must be made to depend on it.
Ordinarily this is not allowed, in order to prevent cycles.

To accomplish the first of these tasks, the journal \module\ advertises a
managed \noop\ \chdesc\ (\S\ref{sec:design:chdescs:noop}) to \modules\ above it,
which they must make all changes they create depend on. This special \chdesc\ is
not considered satisfied until the journal \module\ explicitly satisfies it, so
no changes which depend on it will be written from the cache.  At the end of the
transaction, the journal \module\ satisfies this \chdesc, allowing all the
changes to be written to disk.

To accomplish the second task, a special flag is used to override the normal
rule prohibiting the addition of new dependencies to a \chdesc\ after it is
created. The condition for using this flag is a static proof that no cycle can
result from its use (immediately or in the future); we have determined this to
be the case for the journal \module\ by hand.

\begin{figure}
  \centering
  \includegraphics[width=\hsize]{fig/figures_2}
  \caption{\label{fig:journal} Journal \chdesc\ graph for the
    change in Figure~\ref{fig:softupdate}. Empty circles are
    ``\noop'' \chdescs\ with no associated block data.}
\end{figure}

Figure~\ref{fig:journal} shows the \chdesc\ configuration which is created by
applying this transformation to the \chdescs\ in Figure~\ref{fig:softupdate}.
The original four \chdescs\ have been modified to depend on a journal commit
record, and no longer have explicit dependencies on each other. The commit
record depends on blocks journal blocks containing copies of the changes.
Finally, a the commit record can be marked as completed once the original four
\chdescs\ have been written. This transformation is performed incrementally as
\chdescs\ arrive. The resulting journal on disk is similar in format to those
generated by ext3~\cite{tweedie98journaling} -- it has a list of block numbers,
followed by the data which should be in those blocks. Finally, there is a commit
record which applies to the whole set.

A particularly nice property of this arrangement is that the journal \module\ is
completely independent of any specific file system. It is a block device
\module\ that automatically journals whatever file system is stored on it.
Further, by changing our journal \module\ to journal only \chdescs\ that modify
file system metadata -- and by adding additional dependencies to prevent
premature reuse of blocks -- we could even obtain metadata-only journaling (as
opposed to the full data journaling described here). The extra \chdesc\
dependencies would serve the same purpose as the special hooks and corner cases
surrounding reuse of blocks discussed in \cite{tweedie00ext3}. The journal
\module\ can automatically identify metadata \chdescs\ because of the \LFS\
interface described in \S\ref{sec:design:interfaces}, and the UHFS module which
is responsible for writing to all non-metadata blocks. Other block device
layering systems, like GEOM~\cite{geom} or JBD in Linux, would or do need
special hooks into file system code to determine what disk changes represent
metadata in order to do metadata-only journaling. \Chdescs\ and the \LFS\
interface allow us to do this automatically.


% -*- mode: latex; tex-main-file: "paper.tex" -*-

\subsection{Buffer Cache}
\label{sec:modules:wbcache}

The \Kudos\ buffer cache both
%
caches blocks in memory and
ensures that modifications are written to stable storage in a safe order.
%
Modules ``below'' the buffer cache---that is, between its output interface
and the disk---are considered part of the ``disk controller''; they can
reorder block writes at will without violating dependencies, since those block
writes will contain only in-flight patches.
%
%% There can be many write-back caches in a configuration at once (for
%% instance, one for each block device). 
%
The buffer cache sees the complex
consistency mechanisms that other \modules\ define as nothing more
than sets of dependencies among \patches; it has no idea what consistency
mechanisms it is implementing, if any.
% Yet it is the \module\ that ends up doing most of the work to make sure that
% \patches\ are written in an acceptable order.

% Despite this, write-back caches are relatively simple; they are only slightly
% more complex than the naive implementation already suggested.
Our prototype buffer cache \module\ 
%% is little more than a front end to
%% the automatically-maintained ready \patch\ lists described in
%% Section~\ref{sec:patch:readylist}.
%
uses a modified FIFO policy to write dirty blocks and an LRU policy to
evict clean blocks.  (Upon being written, a dirty block becomes clean and
may then be evicted.)
%
The FIFO policy used to write blocks is modified only to preserve the
in-flight safety property: a block will not be written if none of its
\patches\ are ready to write.
%
Once the cache finds a block with ready \patches, it extracts all ready
patches $P$ from the block, reverts any remaining \patches\ on that block,
and sends the resulting data to the disk driver.  The ready \patches\ are
marked in-flight and will be committed when the disk driver acknowledges
the write.
%% \patches\ on that block are reverted, the resulting block data is copied
%% and sent to the disk driver, the ready \patches\ are marked in-flight, and
%% the reverted \patches\ are re-applied.
%
The block itself is also marked in flight until the current version
commits, ensuring that the cache will wait until then to write the block
again.


As a performance heuristic, when the cache finds a writable block $n$, it
then checks to see if block $n+1$ can be written as
well.
%
It continues writing increasing block numbers until some block is either
unwritable or not in the cache.
%
\begin{comment}
The block itself is also marked \PInfst, so that only
one version of its data will be in flight at a time. (This whole procedure is
basically the buffer cache \textit{Write block} action.)
\end{comment}
%
This simple optimization greatly improves I/O wait time, since the I/O
requests are merged and reordered in Linux's elevator scheduler.
%
Nevertheless, there may still be important opportunities for further
optimization: for example, since the cache will write a block even if only
one of its \patches\ is ready, it can choose to revert \patches\
unnecessarily when a different order would have required fewer writes.


\begin{comment}
Each \patch\ on a cached block may or may not be visible to a given \module.
For example, \modules\ that respond to user requests generally view the most
current state of every block -- the block with all \patches\ applied. However, a
write-back cache may choose to write some \patches\ on a block while reverting
others, since those others currently have outstanding dependencies. In this
case, \modules\ below the write-back cache (i.e. closer to the disk) should view
those \patches\ in the reverted state. \Kudos\ provides a block revisioning
library function that automatically reverts those \patches\ that should not
be visible at a particular \module, and then re-applies them after that
\module\ is done with the block.
\end{comment}


\subsection{Loopback}
\label{sec:modules:loop}

The \Featherstitch\ loopback module demonstrates how pervasive
support for patches can implement previously unfamiliar dependency
semantics.
%
Like Linux's loopback device, it provides a block device interface that
uses a file in some other file system as its data store; unlike Linux's
block device, consistency requirements on this block device are obeyed by
the underlying file system.
%% interface provides consistency  for a block device.
%% demonstrates how is a BD module that uses a file in an \LFS\ module as
%% its underlying data store. It is very similar to the device of the same
%% name in Linux, but with one critical difference: it is aware of \patches.
%
The loopback module preserves incoming dependencies and forwards
them to the underlying data store.
%
As a result, the data store will honor those dependencies and preserve the
loopback file system's consistency, even if the data store would normally
provide no guarantees for consistency of file system data (e.g., it used
metadata-only journaling).

Figure~\ref{fig:kfs-graph} shows a complete, albeit contrived, example
configuration using the loopback module.
%
A file system image is mounted with an external journal, both of
which are loopback block devices stored on the root file system (which uses
soft updates). The journaled file system's ordering requirements are sent
through the loopback module as \patches, maintaining dependency information
across boundaries that might otherwise lose ordering relationships. 
Most systems cannot enforce consistency requirements through loopback
devices this way.
%
The \Featherstitch\ journaling configurations in our evaluation use
loopback modules in a similar way to store the file system's journal as a
file on the file system itself.

%% The \modules\ in Figure~\ref{fig:kfs-graph} are a complete \Kudos\ configuration.

%%  and although the use of a loopback
%% device is somewhat contrived in the example, they are increasingly being used in
%% conventional operating systems. For instance, Mac OS X uses them in order to
%% allow users to encrypt their home directories.

\begin{comment}
\subsection{Asynchronous writes}
\label{sec:modules:unlink}

Finally, we also wrote a trivial module that removes all dependencies from
incoming \patches, allowing the buffer cache to write blocks in any order.
%
This implements similar semantics to existing file systems like ext2 in
asynchronous write mode.
\end{comment}
