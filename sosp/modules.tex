% -*- mode: latex; tex-main-file: "paper.tex" -*-

\section{\Modules}
\label{sec:modules}

A \Kudos\ configuration is composed of many \modules\ that cooperate to
 implement file system functionality.
%
There are three major types of \modules.
%
Closest to the disk are block device (BD) \modules, which have a fairly
conventional block device interface with interfaces such as ``read block'' and
``flush''. 
%
Closest to the system call interface are \emph{common file system} (CFS)
\modules, which have an interface similar to VFS~\cite{kleiman86vnodes}. 
%
In between these interfaces are modules implementing a  \emph{low-level file
system} (\LFS) interface, which helps divide file system implementations
into code common across block-structured file systems and code specific to
a given file system layout.
%
The \LFS\ interface has functions to allocate blocks, add blocks to files,
allocate file names, and other file system micro-operations. 
%% A \module\
%% implementing the \LFS\ interface defines how bits are laid out on the disk, but
%% doesn't have to know how to combine the micro-operations into larger, more
%% familiar file system operations. 
A generic CFS-to-\LFS\ \module\ called UHFS
(``universal high-level file system'') decomposes familiar VFS operations
like write, read, and append into \LFS\ micro-operations. 
%
%% File system extensions like those often implemented by stackable file
%% systems would generally use the CFS interface; for example, we wrote a
%% simple CFS module that provides case-insensitive access to a case-sensitive
%% file system.
%
File system implementations, such as our ext2 and UFS implementations,
generally use the \LFS\ interface.


Module interfaces include patches explicitly, allowing modules to examine
and modify dependencies.
%
For instance, every \LFS\ function that might modify the file system takes a
\texttt{\textit{patch\char`\_t **p}} argument.
%
Before the function is called, \texttt{*p} should be set to the \patch,
if any, on which the modification should depend;
%
when the function returns, \texttt{*p} is set to some \patch\
corresponding to the modification itself.
%
\begin{comment}
(\Noop\ \patches\ allow this interface to generalize to multiple
dependencies.)
\end{comment}
%
For example, this function is called to append a block to an \LFS\ inode
\verb+f+:

\vspace{-0.5\baselineskip}
\begin{small}
\begin{alltt}
int (*append_file_block)(LFS_t *module, 
   fdesc_t *f, uint32_t block, patch_t **p);
\end{alltt}
\end{small}
\vspace{-0.5\baselineskip}

\begin{comment}
\noindent
This design lets \LFS\ modules examine and modify the dependency structure.
\end{comment}



\subsection{ext2 and UFS}

\Kudos\ currently has \modules\ that implement two file system types, Linux
ext2 and 4.2 BSD UFS (Unix File System, the modern incarnation of the Fast File
System~\cite{mckusick84fast}).
%
Both of these \modules\ initially generate dependencies arranged according to the
soft updates rules; other dependency arrangements are achieved by transforming these.
To the best of our knowledge, our implementation of ext2 is the first to provide
soft updates consistency guarantees.
%
%We verified that file systems generated by our modules are considered
%correct by their reference implementations on FreeBSD and Linux by mounting
%and running \emph{fsck} on \Kudos-generated disk images.

Both \modules\ are implemented at the \LFS\ interface. 
%
%% This keeps properties
%% specific to the file system (such as the on-disk format and rules governing
%% block allocation) hidden within the \module. 
%
%The \modules\ create \patches\ for all their changes to the disk and
%connect them to form subgraphs that enforce the soft updates
%rules~\cite{ganger00soft} as applied to each file system. 
%
%The UHFS \module\ is also aware of soft updates order when necessary; when
%it implements a single operation using multiple \LFS\ calls, it hooks the
%resulting \patches\ up in the correct order.
%
Unlike FreeBSD's soft updates implementation, once these modules set up the
desired dependencies for the \patches\ they create, they no longer need to
concern themselves with those \patches---the block device subsystem will track
and enforce the dependencies.


\begin{figure}[t]
  \centering
  \includegraphics[height=2.5in]{fig/figures_1}
  \caption{A running \Kudos\ configuration. {\it/} is a soft updated
    file system on an IDE drive; {\it/loop} is an externally journaled
    file system on loop devices.}
  \label{fig:kfs-graph}
\end{figure}


% -*- mode: latex; tex-main-file: "paper.tex" -*-

\subsection{Journal}
\label{sec:modules:journal}

The journal module sits below a regular file system, such as ext2, and transforms
incoming \patches\ into patches implementing journal transactions.
%
File system blocks are copied into the journal; a commit record depends on the
journal \patches; and the original file system \patches\ depend in turn on the
commit record.
%
Any soft updates-like dependencies among the original \patches\ are removed,
since they are not needed when the journal handles consistency; however, the
journal does care to ensure that user-specified dependencies, such as
\patchgroups, are not violated.
%
%% itself provides consistency for each high-level file system operation by
%% replaying outstanding transactions on recovery.
%% \footnote{However, it does take care to ensure that the user-specified
%% dependencies described in \S\ref{sec:patchgroup} are not violated.}
%
The journal format is similar to ext3's~\cite{tweedie98journaling}: a
transaction contains a list of block numbers, the data to be written to
those blocks, and finally a single commit record.
%
Although the journal modifies existing \patches' direct dependencies, it
ensures that the new dependencies will never introduce a block-level
cycle.

Like ext3, transactions are required to commit in sequence. Therefore the
journal \module\ sets each commit record to depend on the previous commit record, and each
completion record to depend on the previous completion record. This allows
multiple outstanding transactions in the journal, which benefits performance,
but ensures that in the event of a crash, the journal will contain only
contiguous sequential transactions.

Since the commit record is created at the end of the transaction, the journal
\module\ uses a special \noop\ \patch\ explicitly held in memory to prevent
file system changes from being written to the disk until the transaction is
complete. This \noop\ \patch\ is set to depend on the previous transaction's
completion record, which prevents merging between transactions while allowing
merging within a transaction. This temporary dependency is removed when the
real commit record is created and its dependencies are set up as described
above.

%Due to this design, the journal \module\ is completely independent of any
%specific file system. It is a block device \module\ that automatically journals
%whatever file system is stored on it. In fact, the incoming \patches\ need not
%be arranged for soft updates, or for that matter in any particular configuration
%at all.

% Is it important to specify how we figure out where transaction boundaries
% are? It seemed confusing to one reviewer due to this section preceeding the
% modules section.

Our journal module prototype can run in full data journal mode, where every
updated block is written to the journal, or in metadata-only mode, where only
blocks containing file system metadata are written to the journal. It can
tell which blocks are which by looking for a special flag on each \patch\ set
by the UHFS module.

We provide several other modules that modify dependencies, including an
``asynchronous mode'' module that removes all dependencies, allowing the
buffer cache to write blocks in any order.
%
This implements similar semantics as existing file systems like ext2 in
asynchronous write mode.


\subsection{Write-Back Cache and Block Revisioning}
\label{sec:modules:wbcache}

The write-back cache is in some sense the \module\ that does all the real work
in \Kudos. There can be many write-back caches in a configuration at once, but
each is responsible for holding on to the \chdescs\ sent to it by connected
\modules\ until they can be safely sent on towards the disk. To a write-back
cache, the complex consistency protocols that other \modules\ want to enforce
are nothing more than sets of dependencies among \chdescs\ -- it has no idea
what consistency protocol (or protocols) it is implementing, if any at all. Yet
it is the \module\ that ends up doing most of the work to make sure that
\chdescs\ are written in an acceptable order.

Even though a write-back cache has such an important and central role in the
system, there's not a lot to it. The current write-back cache is a simple LRU
cache, but with an important twist: blocks can't be evicted unless all the
\chdescs\ on them are ``ready'' to be sent to the next \module\ (for example,
the disk). So, when looking for a block to evict, the cache may not be able to
evict any block it chooses -- but it evicts the least recently used block that
it can.

\subsubsection{Block Cycles}
\label{sec:modules:wbcache:cycles}

Just as with soft updates~\cite{ganger00soft}, the dependencies among \chdescs\
(or just ``updates to the block'' in soft updates) can create cyclic
dependencies among blocks, even though the \chdescs\ themselves do not form a
cycle. To handle this case, some \chdescs\ may need to be ``held back'' in order
to write the others, allowing such cycles to be broken. To effect this behavior,
the write-back cache just holds on to the \chdescs\ that cannot yet be written,
but forwards the others on to the next \module\ as it writes the block. It
cannot evict the block yet, since it is still ``dirty,'' but progress has been
made that can make other blocks evictable.

\subsubsection{Block Revisioning}

Since many \modules\ may be stacked on top of one another in \Kudos,
and since many of them may want to refer to the same block at the same
time, only one copy of the data for each block is kept in memory at a
time. However, different \modules\ may ``know about'' different sets
of \chdescs. For example, in the case outlined in
\S\ref{sec:modules:wbcache:cycles}, the \module\ below the write-back
cache will know about some of the \chdescs\ on a block but not others
(the ones which are not ``ready'' to be sent to it yet). If this
\module\ is the disk, it will need to be able to write a version of
the block's data that does not include the \chdescs\ it does't know
about yet. \Kudos\ provides a revisioning system for blocks which can
automatically ``roll back'' the \chdescs\ which have not yet reached a
particular \module, and then roll them forward again after that
\module\ is done using the previous version of the block's data
(e.g. to write it to disk).

\subsubsection{Cross-device dependencies}

The write-back cache has one other property that makes it useful in \Kudos: it
respects dependencies between one cache and another, so that (for instance)
dependencies between the changes on a file system and its external journal are
properly respected. This is actually not a special case, nor does it require any
extra code -- it is a property that just falls out of the way \chdesc\ graphs
are processed in order to determine which \chdescs\ are ready to move towards
the disk. This property of write-back caches also extends to \opgroups, which
are explained in \S\ref{sec:opgroup}.


\subsection{Loopback}
\label{sec:modules:loop}

The \Featherstitch\ loopback module demonstrates how pervasive
support for patches can implement previously unfamiliar dependency
semantics.
%
Like Linux's loopback device, it provides a block device interface that
uses a file in some other file system as its data store; unlike Linux's
block device, consistency requirements on this block device are obeyed by
the underlying file system.
%% interface provides consistency  for a block device.
%% demonstrates how is a BD module that uses a file in an \LFS\ module as
%% its underlying data store. It is very similar to the device of the same
%% name in Linux, but with one critical difference: it is aware of \patches.
%
The loopback module preserves incoming dependencies and forwards
them to the underlying data store.
%
As a result, the data store will honor those dependencies and preserve the
loopback file system's consistency, even if the data store would normally
provide no guarantees for consistency of file system data (e.g., it used
metadata-only journaling).

Figure~\ref{fig:kfs-graph} shows an albeit contrived example
configuration using the loopback module.
%
A file system image is mounted with an external journal, both of
which are loopback block devices stored on the root file system (which uses
soft updates). The journaled file system's ordering requirements are sent
through the loopback module as \patches, allowing dependency information to be
maintained across boundaries that might otherwise lose that information. In
contrast, without \patches\ and the ability to forward \patches\ through
loopback devices, BSD cannot express soft updates' consistency requirements
through loopback devices. The \modules\ in Figure~\ref{fig:kfs-graph} are a
complete \Kudos\ configuration.

%%  and although the use of a loopback
%% device is somewhat contrived in the example, they are increasingly being used in
%% conventional operating systems. For instance, Mac OS X uses them in order to
%% allow users to encrypt their home directories.

\begin{comment}
\subsection{Asynchronous writes}
\label{sec:modules:unlink}

Finally, we also wrote a trivial module that removes all dependencies from
incoming \patches, allowing the buffer cache to write blocks in any order.
%
This implements similar semantics to existing file systems like ext2 in
asynchronous write mode.
\end{comment}
