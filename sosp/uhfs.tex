\subsection{UHFS}
\label{sec:modules:uhfs}

This one \module, called the ``universal high-level file system'' or
UHFS, can be used with many different \LFS\ \modules\ implementing different file
systems. (There is also a generic VFS-to-CFS layer, to interface with Linux. See
\S\ref{sec:implementation} for details.)
\todo{fix this intro paragraph}

The UHFS \module\ encompasses all logic for decomposing higher level CFS calls
into lower level \LFS\ calls, and for connecting the resulting \chdescs\
together. The finer granularity of \LFS\ calls divides the problem space into
smaller chunks. Since the issue of how to tie the \LFS\ micro-ops together has
already been solved, file system \module\ developers can give more attention to
the particulars of the file system, such as how to allocate a new filename or
how to look up the Nth data block for a file. To see how this simplifies the
development process, consider the VFS \texttt{write()} call, which has the task
of writing some amount of data to a file at a given offset. In \Kudos, the logic
to determine the correct offsets within blocks and whether new blocks must be
allocated is built into UHFS. A file system \module\ need only implement four
\LFS\ calls: \texttt{get\_file\_block()}, \texttt{allocate\_block()},
\texttt{append\_block()}, and \texttt{write\_block()}. Additionally, the
granularity of calls at the \LFS\ layer makes it an appropriate layer for
inserting test harnesses and developing file system unit tests.
