\subsection{UHFS}
\label{sec:modules:uhfs}

The ``universal high-level file system'' \module, or UHFS, implements a generic
translation from high-level CFS calls (essentially the familiar file system API)
to the \LFS\ interface. It can be used with many different \LFS\ \modules\
implementing different file systems. (There is also a generic VFS-to-CFS layer,
to interface with Linux. See \S\ref{sec:implementation} for details.)

The UHFS \module\ encompasses all logic for decomposing higher level CFS calls
into lower level \LFS\ calls, and for connecting the resulting \chdescs\
together. The finer granularity of \LFS\ calls divides the problem space into
smaller chunks. Since the issue of how to tie the \LFS\ micro-ops together has
already been solved, file system \module\ developers can give more attention to
the particulars of the file system, such as how to allocate a new filename or
how to look up the Nth data block for a file. To see how this simplifies the
development process, consider the VFS \texttt{write()} call, which has the task
of writing some amount of data to a file at a given offset. In \Kudos, the logic
to determine the correct offsets within blocks and whether new blocks must be
allocated is built into UHFS. A file system \module\ need only implement four
\LFS\ calls: \texttt{get\_file\_block()}, \texttt{allocate\_block()},
\texttt{append\_block()}, and \texttt{write\_block()}. Additionally, the
granularity of calls at the \LFS\ layer makes it an appropriate layer for
inserting test harnesses and developing file system unit tests.


Perhaps the most unique property of our journal \module, however, is that it can
automatically selectively journal only \chdescs\ that modify file system
metadata -- thus achieving metadata-only journaling \footnote{There are at least
two major variants of metadata-only journaling, depending on when the
non-journaled data is written relative to the commit record. Our journal module
writes it after the commit record; writing it before the commit record requires
very careful checks to avoid premature reuse of blocks~\cite{tweedie00ext3}.}
(as opposed to the full data journaling implied above), without any knowledge of
the file system.
%
The journal \module\ can automatically identify metadata \chdescs\ because of
the \LFS\ interface described in Section~\ref{sec:modules:interfaces}, and the
UHFS module~(\S\ref{sec:modules:uhfs}) which is responsible for writing to all
non-metadata blocks. Other block device layering systems, like GEOM~\cite{geom}
or JBD in Linux, would or do need special hooks into file system code to
determine what disk changes represent metadata in order to do metadata-only
journaling. \Chdescs\ and the \LFS\ interface allow us to do this automatically.
