% -*- mode: latex; tex-main-file: "paper.tex" -*-

\subsection{Ready \Patch\ Lists}
\label{sec:patch:readylist}
\label{readylist}

\newcommand{\PReady}[1]{\ensuremath{#1.\textit{ready}}}

A different class of optimization addresses CPU time spent in the
\Kudos\ buffer cache.
%
The buffer cache's main task is to choose sets of \patches\ $P$ that
satisfy the in-flight safety property $\PDepset{P} \subseteq P \cup
\PDisk$.
%
A naive implementation would guess a set $P$ and then traverse the dependency graph starting
at $P$, looking for problematic dependencies.
%
Patch merging limits the size of these traversals by reducing the number of
patches.
%
Unfortunately, even modest traversals become painfully slow when executed
on every block in a large buffer cache, and in our initial implementation
these traversals were a bottleneck for cache
sizes above 128 blocks (!).
 
Luckily, much of the information
required for the buffer cache to choose a set $P$ can be precomputed.
%
\Kudos\ explicitly tracks, for each \patch, how many of its
direct dependencies remain uncommitted or in flight.
%
These counts are incremented as \patches\ are added to the system and
decremented as the system receives commit notifications from the disk.
%
When both counts reach zero, the \patch\ is safe to write, and it is moved
into a \emph{ready list} on its containing block.
%
\begin{comment}
(\Noop\ \patches\ automatically commit when all their dependencies commit.)
\end{comment}
%
The buffer cache, then, can immediately tell whether a block has writable
patches.
%
To write a block $\PB$, the buffer cache initially populates the set $P$ with the
contents of $\PB$'s ready list.
%
While moving a patch $p$ into $P$, \Kudos\ checks whether there exist
dependencies $q \PDDepend p$ where $q$ is also on block $\PB$.
%
The system can write $q$ at the same time as $p$, so $q$'s
counts are updated as if $p$ has already committed.
%
This may make $q$ ready, after which it in turn is added to $P$.
%
(This premature accounting is safe because the system won't try to write
$\PB$ again until $p$ and $q$ actually commit.)


While the basic principle of this optimization is simple, its efficient
implementation depends on several other optimizations, such as soft-to-hard
patch merging, that preserve important dependency invariants.
%
Although ready count maintenance makes some
\patch\ manipulations more expensive, ready lists save enough duplicate work in
the buffer cache that the system as a whole is more efficient by multiple orders of
magnitude.


\begin{comment}
For a \module\ like the write-back cache to forward \patches\ in a
dependency-preserving order, the \module\ must find \patches\ whose \befores\
are all ``closer to the disk'' (or are also being forwarded as part of the same
block write). We say that such \patches\ are \emph{ready}. 


Each \patch\ has a count of the number of \befores\ it has at block device
modules just as close to the disk as it currently is, and a count of the number
of \befores\ it has which are in flight. When these counts are both zero, it is
ready. A \patch's \before\ counts are incrementally updated as \befores\ are
added and removed and as \beforing\ \patches\ are moved closer to the disk.

Because \Kudos\ makes sure that the \befores\ of a \patch\ are at least as
close to the disk as it is, only directly reachable \beforing\ \patches\ need to
be included in a \patch's \before\ counts. \Noop\ \patches, with the exception
of managed \noop\ \patches\ (which have an explicit owning block device), add a
wrinkle to this simplifying rule, however. They are considered to be as close to
the disk as their \before\ which is the farthest from the disk, in effect,
propagating the distance to the disk metric through them.

When a \before\ count update changes whether a \patch\ is ready to write, the
\patch's inclusion in its block's ready list is updated. To write a block, a
\module\ thus iterates through the block's ready list, sending \patches\ to the
target block device, until the list is empty. Thus instead of having to
repeatedly traverse \patch\ graphs to determine readiness on demand, we have
this information maintained automatically as it changes. This automatic
maintenance adds some cost to forwarding \patches\ and changing the graph
structure, but since it saves so much duplicate work\footnote{The amount of
duplicate work saved is actually superlinear in the size of the write-back
cache.} it is much more efficient.
\end{comment}
