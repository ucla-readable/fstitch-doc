% -*- mode: latex; tex-main-file: "paper.tex" -*-

\subsection{Ready \ChDesc\ Lists}
\label{sec:patch:readylist}

\newcommand{\PReady}[1]{\ensuremath{#1.\textit{ready}}}

Our first optimization precomputes much of the information required for the
buffer cache to choose a set of patches to write.
%
For each patch $p$ we maintain a bit $\PReady{p}$ that specifies whether or
not $p$ is ready to write to disk.
%
Thus, $\PReady{p}$ is true iff $\PDepset{p} \subseteq \{p\} \cup \PDisk$.
%



For a \module\ like the write-back cache to forward \chdescs\ in a
dependency-preserving order, the \module\ must find \chdescs\ whose \befores\
are all ``closer to the disk'' (or are also being forwarded as part of the same
block write). We say that such \chdescs\ are \emph{ready}. Because the
write-back cache frequently searches for and writes many ready \chdescs,
redundant \chdesc\ graph traversals to calculate \chdesc\ readiness would
severly limit cache size scalabity. \Kudos\ therefore explictly tracks a
\chdesc's \before\ counts through incremental dependency updates, and uses these
counts to maintain a \chdesc\ ready list for each block.

Each \chdesc\ has a count of the number of \befores\ it has at block device
modules just as close to the disk as it currently is, and a count of the number
of \befores\ it has which are in flight. When these counts are both zero, it is
ready. A \chdesc's \before\ counts are incrementally updated as \befores\ are
added and removed and as \beforing\ \chdescs\ are moved closer to the disk.

Because \Kudos\ makes sure that the \befores\ of a \chdesc\ are at least as
close to the disk as it is, only directly reachable \beforing\ \chdescs\ need to
be included in a \chdesc's \before\ counts. \Noop\ \chdescs, with the exception
of managed \noop\ \chdescs\ (which have an explicit owning block device), add a
wrinkle to this simplifying rule, however. They are considered to be as close to
the disk as their \before\ which is the farthest from the disk, in effect,
propagating the distance to the disk metric through them.

When a \before\ count update changes whether a \chdesc\ is ready to write, the
\chdesc's inclusion in its block's ready list is updated. To write a block, a
\module\ thus iterates through the block's ready list, sending \chdescs\ to the
target block device, until the list is empty. Thus instead of having to
repeatedly traverse \chdesc\ graphs to determine readiness on demand, we have
this information maintained automatically as it changes. This automatic
maintenance adds some cost to forwarding \chdescs\ and changing the graph
structure, but since it saves so much duplicate work\footnote{The amount of
duplicate work saved is actually superlinear in the size of the write-back
cache.} it is much more efficient.
