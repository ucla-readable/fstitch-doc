\section{\Patchgroups}
\label{sec:patchgroup}

% notation removed to notation.tex


Currently,
robust applications
%% often manipulate valuable data, e.g. a mail server's mail
%% or a version control system's local checkout.
%
can enforce necessary write-before relationships, and thus ensure the
consistency of on-disk data even after system crash, in only limited
ways:
%% requiring their changes be committed in
%% a specified order.
%% %
%% Existing systems can ensure such orders in two ways:
%
they can force synchronous writes using
\texttt{sync}, \texttt{fsync}, or \texttt{sync\_file\_range}, or
%
they can assume particular file system implementation
semantics, such as journaling.
%
With the \patch\ abstraction, however, a process might specify
just dependencies; the storage system could use those dependencies to implement
an appropriate ordering.
%
This approach assumes little about file system implementation semantics,
but unlike synchronous writes, the storage system can still
buffer, combine, and reorder disk operations.
%
This section describes \emph{\patchgroups}, an example API for extending
\patches\ to user space.
%
Applications \emph{engage} \patchgroups\ to associate them with file system
changes; dependencies are
defined among groups.  A parent process can set up a dependency structure
that its child process will obey unknowingly.  \Patchgroups\ apply to
any file system, including raw block device writes.  

In this section we describe the \patchgroup\ abstraction
%
and apply it to three robust applications.

%%  it usefulness on three and show how it allows gzip, Subversion, and UW IMAP to easily specify
%% their application-level consistency protocols.

\subsection{Interface and Implementation}
\label{sec:patchgroup:interface}

\Patchgroups\ encapsulate sets of file system operations into units among
which dependencies can be applied.
%
%% applications can specify ordering requirements among \patchgroups.
%
The \patchgroup\ interface is as follows:

\vspace{-0.5\baselineskip}
\begin{scriptsize}
\begin{alltt}
  typedef int pg_t;          pg_t \textbf{pg_create}(void);
  int \textbf{pg_depend}(pg_t \pgQg, pg_t \pgPg);  /* \textrm{adds \(\pgQgNf\PDepend\pgPgNf\)} */
  int \textbf{pg_engage}(pg_t \pgPg);     int  \textbf{pg_disengage}(pg_t \pgPg);
  int \textbf{pg_sync}(pg_t \pgPg);       int  \textbf{pg_close}(pg_t \pgPg);
\end{alltt}
\end{scriptsize}
\vspace{-0.5\baselineskip}

Each process has its own set of \patchgroups, which are currently shared
among all threads.
%
The call \texttt{\pgDepend(\pgQg, \pgPg)} makes \patchgroup\ \pgQg\ depend on
\patchgroup\ \pgPg: all \patches\ associated with \pgPg\ will
commit prior to any of those associated with \pgQg.
%
\emph{Engaging} a patchgroup with \pgEngage\ associates subsequent file
system operations with that patchgroup.
%
\pgSync\ forces an immediate write of a \patchgroup\ to disk.
%
\pgCreate\ creates a new \patchgroup\ and returns its ID
%
and \pgClose\ disassociates a \patchgroup\ ID from the underlying
\patches\ which implement it.

\begin{figure}[t]
\centering
\includegraphics[width=\hsize]{fig/pg_1}
\caption{\label{fig:patchgroup-transitions} \Patchgroup\ lifespan.}
\end{figure}
%
Whereas \Kudos\ \modules\ are presumed to not create cyclic
dependencies, the kernel cannot safely trust user applications to be
so well behaved, so
%
the \patchgroup\ API makes cycles
inconstructible.
%
Figure~\ref{fig:patchgroup-transitions} shows when different
\patchgroup\ dependency operations are valid.
%
As with patches themselves, all a patchgroup's direct dependencies are
added first.  After this, a patchgroup becomes engaged zero or more times;
however, once a patchgroup \pgPg\ gains a dependency via \texttt{\pgDepend(*,
\pgPg)}, it is sealed and can never be engaged again.  
%
This prevents applications from using patchgroups to hold dirty blocks in
memory: \pgQg\ can depend on \pgPg\ only once the system has seen the
complete set of \pgPg's changes.

%% but the system will refuse to engage any patc   example, a dependency to an engaged (or previously-engaged)
%% patchgroup cannot For example, \texttt{\pgDepend(\pgQg, \pgPg)} returns an
%% error if \pgQg\ has \emph{ever} been engaged; if it succeeds, a subsequent
%% \texttt{\pgEngage(\pgPg)} will return an error.\footnote{These rules are
%% really just a strictly enforced version of the requirement from
%% \S\ref{sec:patch:nrb} that all dependencies must be specified up-front.}

\Patchgroups\ and file descriptors are managed similarly---they are copied
across \texttt{fork}, preserved across \texttt{exec}, and closed on
\texttt{exit}.
%
This allows existing, unaware programs to interact with \patchgroups,
in the same way that the shell can connect pipe-oblivious programs
into a pipeline.
%
For example, a \texttt{depend} program could apply \patchgroups\ to
unmodified applications by setting up the \patchgroups\ before calling
\texttt{exec}.  The following command line would ensure that \texttt{in} is
not removed until all changes in the preceding \texttt{sort} have committed
to disk:

\vspace{-0.5\baselineskip}
\begin{center}
\begin{small}
\verb+depend "sort < in > out" "rm in"+
\end{small}
\end{center}
\vspace{-0.5\baselineskip}

% FIXME: Mention simplicity using patchgroups vs fsync?

Inside the kernel, each \patchgroup\ corresponds to a pair of containing
\noop\ \patches,
and each inter-\patchgroup\ dependency corresponds to a dependency between
the \noop\ \patches.
%
All file system changes are inserted
between all engaged \patchgroups' \noop\ \patches.
%
Figure~\ref{fig:patchgroup-patches} shows an example \patch\ arrangement for
two \patchgroups.
%
(The actual implementation uses additional empty patches for bookkeeping.)

\begin{figure}[t]
\centering
%% \includegraphics[width=0.7\hsize]{fig/figures_5}
%% \caption{\label{fig:patchgroup-patches} \Patches\ corresponding to two
%%   \patchgroups, $p$ and $q$.  The $h$ and $t$ \patches\ are created by the 
%%   \patchgroup\ module; the heavy dependency between $t_p$ and $h_q$ was added
%%   by \texttt{pg~depend(p, q)}.  Each of $a_i$, $b$, and $c$ corresponds to
%%   a different file system change.}
\includegraphics[width=\hsize]{fig/figures_7}
\caption{\Patchgroup\ implementation (simplified).  Empty
patches $h_{\pgPgNf}$ and $t_{\pgPgNf}$ bracket file system patches created while
\patchgroup\ \pgPg\ is engaged.  \pgDepend\ connects one
\patchgroup's $t$ patch to another's $h$.}
\label{fig:patchgroup-patches} 
\end{figure}

These dependencies suffice to enforce patchgroups when using soft
updates-like dependencies, but for journaling, some extra work is required.
%
Since writing the commit record atomically commits a
transaction, additional \patchgroup-specified dependencies for the data in each
transaction must be shifted to the commit record itself.
%
These dependencies then collapse into harmless dependencies from the commit
record to itself or to previous commit records.
%
Also, a metadata-only journal, which by default does not journal data
blocks at all, pulls \patchgroup-related data blocks into its journal,
making it act like a full journal for those data blocks.

Patchgroups currently \emph{augment} the underlying file system's
consistency semantics, although a fuller implementation might let a
patchgroup declare \emph{reduced} consistency requirements as well.

\subsection{Case Studies}
\label{sec:patchgroup:casestudies}

%% \todo{Introduce and contrast gzip, Subversion, and UW IMAP}
%% %
%% UW IMAP case study (faster, same safety, server) vs
%% Subversion case study (about same speed, additional safety on non-ext3
%% ordered, client).

We studied the \patchgroup\ interface by adding \patchgroup\ support to three
applications: the gzip compression utility, the Subversion version control
client, and the UW IMAP mail server daemon.
%
These applications were chosen for their relatively simple and explicit
consistency requirements; we intended to test how well patchgroups
implement existing consistency mechanisms, not to create new mechanisms.
One effect of this choice is that versions of
these applications could attain similar consistency guarantees
by running on a
fully-journaled file system with a conventional API, although at least IMAP
would require modification to do so.  Patchgroups, however,
make the required guarantees explicit, can be implemented on other types of
file system, and introduce no additional cost on fully-journaled systems.

% discussion potential gains (in addition to actual gains)?

\paragraph{Gzip}
\label{sec:patchgroup:gzip}

%% Our first test showed that simple consistency requirements were simple to
%% add.  
Our modified gzip uses \patchgroups\ to make the input file's
removal depend on the output file's data being written; thus,
a crash cannot lose both files. The update adds 10 lines of code to gzip
v1.3.9, showing that simple consistency requirements are simple to
implement with patchgroups.

\paragraph{Subversion}
\label{sec:patchgroup:svn}

% describe why pgs are easier/better than journaling

The Subversion version control system's client~\cite{svn} manipulates a
local working copy of a repository.
%% In this case study we take an interest in how the Subversion version
%% control system client~\cite{svn} manipulates a local checkout (a
%% Subversion ``working copy'').
%
The working copy library is designed to avoid data corruption or loss
should the process exit prematurely from a working copy operation.
%
This safety is achieved using application-level write ahead journaling,
where each entry in Subversion's journal is either idempotent or
atomic.
%
Depending on the file system, however, even this precaution may not protect
a working copy operation against a crash.
%
For example, the journal file is marked as complete by moving it from
a temporary location to its live location.
%
Should the file system completely commit the file rename before
the file data, and crash before completing the file data commit, then
a subsequent journal replay could corrupt the working copy.

The working copy library could ensure a safe commit ordering by
syncing files as necessary, and the Subversion server (repository) library
takes this approach, but
%
developers deemed this approach too slow to be worthwhile at the
client~\cite{svntradeoff}.
%
Instead, the working copy library assumes that
%
first, all preceding writes to a file's data are committed before the file
is renamed,
%
and second, metadata updates are effectively committed in their system call
order.
%
%% The combination of these properties ensures the creation of file $i$
%% precedes the creation of file $(i+1)$.
%
This does not hold on
many systems; for example, neither NTFS with journaling nor BSD UFS with
soft updates provide the required properties.  The Subversion developers
essentially specialized their consistency mechanism for one file system,
ext3 in either ``ordered'' or full journaling mode.
%
%% Additionally, non-journaled systems such as BSD UFS with soft updates
%% do not provide the metadata ordering property.  On such systems the
%% journal file's installation may precede earlier file installations,
%% leaving the working copy in an inconsistent state.

We updated the Subversion working copy library to
express commit ordering requirements directly using \patchgroups.
%% \todo{Use consistency protocol phrase here and throughout?} 
%% without relying on properties of the
%% underlying file system implementation.
%
The file rename property was replaced in two ways.
%
Files created in a temporary location and then moved into their
live location, such as directory status and journal files, now
make the rename depend on the file data writes; but
%
files only referenced by live files, such as updated file
copies used by journal file entries, can live with a weaker ordering:
the installation of referencing files is made to depend on the
file data writes.
%
The use of linearly ordered metadata updates was also replaced by
\patchgroup\ dependencies, and
%
making the dependencies explicit let us reason about Subversion's actual
order requirements, which are much less strict than linear ordering.
%
%% Although the original operations assumed a linear ordering of metadata
%% updates, the actual order requirements are considerably less strict.
%
For example, the updated file copies used by the journal may be
committed in any order, and most journal playback operations
may commit in any order.
%
Only interacting operations, such as a
file read and subsequent rename, require ordering.

Once we understood Subversion v1.4.3's requirements, it took a day to add
the 220 lines of code that enforce safety for conflicted updates (out of
25,000 in the working copy library).

\paragraph{UW IMAP}
\label{sec:patchgroup:uwimap}

We updated the University of Washington's IMAP mail server
(v2004g)~\cite{uwimap} to ensure mail updates are safely committed to disk.
%
The Internet Message Access Protocol (IMAP)~\cite{rfc3501} provides
remote access to a mail server's email message store.
%
The most relevant IMAP commands synchronize changes to the server's
disk (\imapCheck), copy a message from the selected mailbox to another
mailbox (\imapCopy), and delete messages marked for deletion (\imapExpunge).

We updated the imapd and mbox mail storage drivers to use
\patchgroups, ensuring that all disk writes occur in a safe ordering
without enforcing a specific block write order.
%
The original server conservatively preserved command ordering by
syncing the mailbox file after each \imapCheck\ on it or \imapCopy\ into it.
%
For example, Figure~\ref{fig:imap}a illustrates moving messages from
one mailbox to another.
%
With \patchgroups, each command's file system updates are executed under a
distinct \patchgroup\ and, through the \patchgroup, made to depend on the
previous command's updates. This is necessary, for example, so that
moving a message to another folder (accomplished by copying to the
destination file and then removing from the source file) cannot lose
the copied message should the server crash part way through the disk
updates.
%
The updated \imapCheck\ and \imapExpunge\ commands use \pgSync\ to sync all preceding disk
updates. This removes the requirement that \imapCopy\
sync its destination mailbox: the client's \imapCheck\ or \imapExpunge\ request will ensure
changes are committed to disk, and the \patchgroup\ dependencies ensure
changes are committed in a safe ordering.
%
Figure~\ref{fig:imap}b illustrates using patches to move messages.

\begin{figure}[tb]
\centering
\begin{tabular}{@{}cc@{}}
\includegraphics[scale=0.6]{fig/pg_2} & 
\includegraphics[scale=0.6]{fig/pg_3}\\
\textbf{a)} Unmodified, \texttt{fsync} & 
\textbf{b)} \Patchgroups
\end{tabular}
\caption{UW IMAP server, without and with \patchgroups, moving three
messages from \texttt{mailbox.src} to \texttt{mailbox.dst}.}
\label{fig:imap}
\end{figure}

These changes improve UW IMAP by
%
ensuring disk write ordering correctness
%
and by performing disk writes more efficiently than synchronous writes.
%
As each command's changes now depend on the preceding command's
changes, it is no longer required that all code
specifically ensure its changes are committed before any later, dependent
command's changes. Without \patchgroups, modules like the mbox driver
forced a conservative disk sync protocol because ensuring safety more
efficiently required additional state information, adding further
complexity. The Dovecot IMAP server's source code notes this exact
difficulty~\cite[maildir-save.c]{dovecot}:

\vspace{-0.5\baselineskip}
\begin{scriptsize}
\begin{verbatim}
/* FIXME: when saving multiple messages, we could get
   better performance if we left the fd open and
   fsync()ed it later */
\end{verbatim}
\end{scriptsize}
\vspace{-0.5\baselineskip}

The performance of the \patchgroup{}-enabled UW IMAP mail server is
evaluated in Section~\ref{sec:evaluation:uwimap}.
