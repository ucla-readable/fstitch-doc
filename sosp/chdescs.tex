\section {\ChDescs}
\label{sec:chdescs}

\newcommand{\ChAll}{\ensuremath{\textit{All}}}
\newcommand{\ChAllB}[1]{\ensuremath{\textit{All}[#1]}}
\newcommand{\ChMem}{\ensuremath{\textit{Mem}}}
\newcommand{\ChMemB}[1]{\ensuremath{\textit{Mem}[#1]}}
\newcommand{\ChDisk}{\ensuremath{\textit{Disk}}}
\newcommand{\ChDiskB}[1]{\ensuremath{\textit{Disk}[#1]}}
\newcommand{\ChInf}{\ensuremath{\textit{Inf}}}
\newcommand{\ChInfB}[1]{\ensuremath{\textit{Inf\/}[#1]}}
\newcommand{\ChNrb}{\ensuremath{\textit{\Nrb}}}
\newcommand{\ChNrbB}[1]{\ensuremath{\textit{\Nrb}[#1]}}

\newcommand{\Before}[1]{\ensuremath{\textit{Pre}[#1]}}
\newcommand{\BeforeS}[1]{\ensuremath{\textit{Pre}^*[#1]}}
\newcommand{\After}[1]{\ensuremath{\textit{Post}[#1]}}
\newcommand{\AfterS}[1]{\ensuremath{\textit{Post}^*[#1]}}

\newcommand{\statenone}{\ensuremath{\textit{cached}}}
\newcommand{\stateinf}{\ensuremath{\textit{inflight}}}
\newcommand{\statedisk}{\ensuremath{\textit{ondisk}}}

% {{{ fig:chdesc
\begin{figure}[t]
\vskip-14pt
\begin{tabular}{@{\hskip0.58in}p{2in}@{}}
\begin{scriptsize}
\begin{verbatim}
struct patch {
    bdev_t *device;
    bdesc_t *block;
    enum {BIT, BYTE, NOOP} type;
    union {
        struct {
            uint16_t offset;
            uint32_t xor;
        } bit;
        struct {
            uint16_t offset, length;
            uint8_t *data;
        } byte;
    };
    struct patch_queue *leaders;
    uint32_t nleaders[NBDLEVEL];
    patch_t * ddesc_ready_next;
    patch_t ** ddesc_ready_prev;
/* ... */ };
\end{verbatim}
\end{scriptsize}
\end{tabular}
\vspace{-10pt}
\caption{\label{fig:chdesc} Partial \chdesc\ structure.}
\end{figure}
% }}}

% {{{ fig:chdapi
\begin{figure}[t]
\vskip-14pt
\begin{tabular}{@{\hskip0.25in}p{2in}@{}}
\begin{scriptsize}
\begin{alltt}
int \textbf{patch_create_byte}(
    bdesc_t *block, bdev_t *owner,
    uint16_t offset, uint16_t length,
    const void *data, patch_t **head);
int \textbf{patch_create_bit}(
    bdesc_t *block, bdev_t *owner,
    uint16_t offset, uint32_t xor,
    patch_t **head);
int \textbf{patch_create_empty}(
    bdev_t *owner, patch_t **tail,
    size_t nheads, patch_t * heads[]);
int \textbf{patch_create_diff}(
    bdesc_t *block, bdev_t *owner,
    uint16_t offset, uint16_t length,
    const void *data, patch_t **head);
int \textbf{patch_add_depend}(
    patch_t *\after, patch_t *\before);
void \textbf{patch_remove_depend}(
    patch_t *\after, patch_t *\before);
int \textbf{patch_apply}(patch_t *patch);
int \textbf{patch_rollback}(patch_t *patch);
int \textbf{patch_satisfy}(patch_t *patch);
int \textbf{patch_push_down}(
    bdev_t *current_bd, bdesc_t *current_block,
    bdev_t *target_bd, bdesc_t *target_block);
\end{alltt}
\end{scriptsize}
\end{tabular}
\vspace{-10pt}
\caption{\label{fig:chdapi} Partial \chdesc\ API.}
\end{figure}
% }}}

The fundamental new abstraction in \Kudos\ is called a \chdesc. Each in-memory
modification to a cached disk block has an associated \chdesc. A \chdesc's
\emph{\befores} point to other \chdescs\ that must precede it to stable storage;
the other \chdescs\ which point to it as a \before\ are its \emph{\afters}. A
\chdesc\ can be applied or reverted to switch the cached block's state between
old and new.

Figure~\ref{fig:chdesc} gives a simplified version of the structure, and
Figure~\ref{fig:chdapi} shows much of the API for working with them. The ability
to revert and re-apply \chdescs\ is inspired by soft updates dependencies, but
generalized so that it is not specific to any particular file system. \Chdesc\
dependencies can create cyclic dependencies among blocks. \Chdescs\ themselves
are not allowed to form cycles to ensure that they can always be written to disk
in an order which does not violate the dependency graph. To break block-level
cycles, some \chdescs\ may need to be reverted, or ``rolled back,'' in order to
first write other \chdescs\ on the same block, just as in soft updates.

\Kudos\ \modules\ change blocks by attaching \chdescs\ to them, using functions
such as \texttt{patch\_create\_byte}. Most file system \modules\ initially
generate \chdescs\ whose dependencies impose soft-update-like ordering
requirements (see \S\ref{sec:using:softupdate}). These \chdescs\ are then passed
down, through other \modules, in the general direction of the disk. The
intervening \modules\ can inspect, delay, and even modify them before passing
them on further. For instance, the write-back cache \module\
(\S\ref{sec:using:wbcache}), essentially a buffer cache, holds on to blocks and
their \chdescs\ instead of forwarding them immediately. Finally, \chdescs\ are
\emph{satisfied} when their associated data reaches stable storage.

\subsection{Notation}
\label{sec:chdescs:notation}

It will be convenient to introduce some notation for dealing with \chdescs.
First we will define a set \ChAll\ of all \chdescs. Although in the actual
implementation we destroy \chdescs\ when they are no longer necessary (i.e. when
they have successfully been written to the disk), \ChAll\ contains all \chdescs,
written to disk or not. We also define sets \ChMem\ and \ChDisk, which contain
all \chdescs\ only in memory or already on the disk, respectively. Next, we
define a set \ChInf\ which contains all \chdescs\ that are currently ``in
flight'' to the disk: block data reflecting their contents has been sent to the
disk controller but has not yet been written to the disk media itself. Finally,
we will also write \ChAllB{B}, \ChMemB{B}, \ChDiskB{B}, and \ChInfB{B} to
succinctly represent the subsets of these four sets specifically associated with
disk block $B$.

If we have some \chdesc\ \p{p}, we can also refer to its block or state using a
``member'' notation as in many programming languages: \p{p}.block is the block
modified by \p{p}, and \p{p}.state is its state (one of \statenone, \stateinf,
or \statedisk). We write \depends{\p{p}}{\p{q}} if \p{q} is listed as a \before\
of \p{p} (that is, if \p{p} directly depends on \p{q}), and
\indirdepends{\p{p}}{\p{q}} if there is some path of \befores\ from \p{p} to
\p{q} (that is, if \p{p} indirectly depends on \p{q}).

The last bit of notation we'll need is for representing sets of \chdescs\
related by the dependency graph. We'll write \Before{p} to represent the set of
all \chdescs\ \p{q} such that \depends{p}{q}, and \BeforeS{p} for the set of all
\chdescs\ \p{q} such that \indirdepends{p}{q}. Likewise, \After{p} is the set of
all \chdescs\ \p{q} such that \depends{q}{p}, and \AfterS{p} is the set of all
\chdescs\ \p{q} such that \indirdepends{q}{p}. These four operators can also be
applied to sets of \chdescs: in that case, the resulting set is the union of the
results of applying the same operator to each element of the operand set.

\subsection{Formal Model}
\label{sec:chdescs:model}

We can model the interaction between the disk (and its controller) and the
software maintaining the \chdesc\ graph by a set of randomly occurring events
that change the state of \chdescs\ according to simple rules. By stipulating
that each event may occur at any time when its preconditions are met, we can
reason about all possible hardware behavior and all possible software
algorithms. Here are the three possible events:

\paragraph{Create new \chdesc}
For some \chdesc\ \p{p} $\not\in$ \ChAll\: \\
Let \p{p}.state $:=$ \statenone \\
Let \depends{\p{p}}{\p{q}} for all \p{q} $\in Q$ where $Q \subseteq$ \ChAll %\\
% Let \ChAll\ $:=$ \ChAll\ $\cup$ \{\p{p}\}

\paragraph{Write block to disk controller}
For some block $B$: \\
Let $P \subseteq$ \ChMemB{B} such that \BeforeS{P} $\subseteq$ \ChDisk\ $\cup\ P$ \\
Let \p{p}.state $:=$ \stateinf\ for all \p{p} $\in$ $P$

\paragraph{Disk writes block to media}
For some block $B$: \\
Let \p{p}.state $:=$ \statedisk\ for all \p{p} $\in$ \ChInfB{B}

\paragraph{} It is not difficult to prove based on the events above that a
fundamentally important safety property is maintained, namely that
\BeforeS{\ChDisk} $\subseteq$ \ChDisk. In essence, the system honors the
dependency information correctly. A second, and related, safety property is that
\BeforeS{\ChInfB{B_1}} $\cap$ \ChInfB{B_2} $= \emptyset$ for all $B_1 \neq B_2$.
This means that none of the currently in-flight blocks depend on any of the
other currently in-flight blocks, so the disk, disk controller, and disk driver
are all free to reorder them any way they see fit.

A third property of the events above is that if at some point we never see the
first event again, eventually we can get to a state where \ChDisk\ $=$ \ChAll.
This is a liveness property; namely, that eventually all the \chdescs\ will make
it to disk.

\Kudos\ controls when the first two events occur, but is notified of the third
event by the disk driver when it occurs. Depending on the hardware
configuration, the third event may actually represent one of two real-world
situations: first, the data may merely be in the volatile disk cache, so that a
power outage may cause it to be lost. This can be a catastrophic disaster, as
shown in~\cite{nightingale06rethink}. The second possible situation is that the
data is truly on the disk media, which can be arranged by disabling the drive's
write-back cache or by using hardware features like SCSI tagged command queueing
(TCQ) or SATA native command queueing (NCQ) together with their ``force unit
access'' (FUA) setting. In these configurations, \Kudos\ is notified of command
success only when the data has actually been written to the disk media. \Kudos\
can be run in any of these four configurations, including the first, though
obviously the safety of the system is compromised in that case.

% -*- mode: latex; tex-main-file: "paper.tex" -*-

\subsection{Implementation and \noop\ \chdescs}
\label{sec:chdescs:noop}

\Kudos\ file system implementations create \chdescs\ with one of the
following functions:

\begin{scriptsize}
\begin{alltt}
int \textbf{patch_create_byte}(
    bdesc_t *block, bdev_t *owner,
    uint16_t offset, uint16_t length,
    const void *data, patch_t **head);
int \textbf{patch_create_bit}(
    bdesc_t *block, bdev_t *owner,
    uint16_t offset, uint32_t xor,
    patch_t **head);
int \textbf{patch_create_\noop}(
    bdev_t *owner, patch_t **tail,
    size_t nheads, patch_t * heads[]);
int \textbf{patch_create_diff}(
    bdesc_t *block, bdev_t *owner,
    uint16_t offset, uint16_t length,
    const void *data, patch_t **head);
\end{alltt}
\end{scriptsize}


We first address dependency convenience and memory usage with \emph{\noop\
\chdescs}, which have no associated data or block.
%
This makes it just a means for tracking dependencies.


For example, imagine writing two files \texttt{before.txt}
The dependencies 
\Chdescs\ as so far described can be tedious and inefficient to manage when
dealing with large sets of them corresponding to file system operations. For
instance, if writing some file \texttt{\after.txt} is to depend on writing some
other file \texttt{\before.txt}, it will be inconvenient to keep arrays of all
the \chdescs\ corresponding to the two operations and inefficient to store the
potentially quadratic number of edges in the \chdesc\ graph.

To solve this problem, we introduce an additional type of \chdesc. The
prototypical \chdesc\ corresponds to some change on disk, but \Kudos\ also
supports \aemphnoop\ \chdesc\ type, which doesn't change the disk at all.
\Noop\ \chdescs\ can have \befores, like other \chdescs, but they don't need to
be written to disk: they are trivially satisfied when all of their \befores\ are
satisfied. Thus, they can be used to ``stand for'' entire sets of other changes.
%
This capability is extremely useful, and is used by most operations on disk
structures so that a single \chdesc\ can be returned that depends on the whole
change. Likewise, \anoop\ \chdesc\ can be passed in as a parameter to a disk
operation to make the whole operation depend on a set of other changes. \Noop\
\chdescs\ allow dependencies between sets with only a linear number of
dependency edges in the \chdesc\ graph, and without having to pass around arrays
of \chdescs.
%
The cost is that some functions may have to traverse trees of \noop\ \chdescs\
to determine true dependencies.

Modules can also use \noop\ \chdescs\ to \emph{prevent} changes from being
written. A \emph{managed} \noop\ \chdesc\ must be explicitly satisfied; any
changes that depend on that \noop\ \chdesc\ are delayed until the owning \module\
explicitly satisfies it. This is used, for instance, by the journal \module\
(\S\ref{sec:using:journal}) to prevent a transaction's \chdescs\ from
being written before the journal commits.

\Noop\ \chdescs\ are not included in our formal model of \chdescs\ for simplicity;
they add some additional complexity but do not change the basic ideas.
\todo{Well, they are now... the rules need updating since they have no blocks.}

\subsection{Ready \ChDesc\ Lists}

For a \module\ like the write-back cache to forward \chdescs\ in a
dependency-preserving order, the \module\ must find \chdescs\ whose \befores\
are all ``closer to the disk'' (or are also being forwarded as part of the same
block write). We say that such \chdescs\ are \emph{ready}. Because the
write-back cache frequently searches for and writes many ready \chdescs,
redundant \chdesc\ graph traversals to calculate \chdesc\ readiness would
severly limit cache size scalabity. \Kudos\ therefore explictly tracks a
\chdesc's \before\ counts through incremental dependency updates, and uses these
counts to maintain a \chdesc\ ready list for each block.

Each \chdesc\ has a count of the number of \befores\ it has at block device
modules just as close to the disk as it currently is, and a count of the number
of \befores\ it has which are in flight. When these counts are both zero, it is
ready. A \chdesc's \before\ counts are incrementally updated as \befores\ are
added and removed and as \beforing\ \chdescs\ are moved closer to the disk.

Because \Kudos\ makes sure that the \befores\ of a \chdesc\ are at least as
close to the disk as it is, only directly reachable \beforing\ \chdescs\ need to
be included in a \chdesc's \before\ counts. \Noop\ \chdescs, with the exception
of managed \noop\ \chdescs\ (which have an explicit owning block device), add a
wrinkle to this simplifying rule, however. They are considered to be as close to
the disk as their \before\ which is the farthest from the disk, in effect,
propagating the distance to the disk metric through them.

When a \before\ count update changes whether a \chdesc\ is ready to write, the
\chdesc's inclusion in its block's ready list is updated. To write a block, a
\module\ thus iterates through the block's ready list, sending \chdescs\ to the
target block device, until the list is empty. Thus instead of having to
repeatedly traverse \chdesc\ graphs to determine readiness on demand, we have
this information maintained automatically as it changes. This automatic
maintenance adds some cost to forwarding \chdescs\ and changing the graph
structure, but since it saves so much duplicate work\footnote{The amount of
duplicate work saved is actually superlinear in the size of the write-back
cache.} it is much more efficient.

\subsection{\Nrb\ \ChDescs}
\label{sec:chdescs:nrb}

Each data\todo{Name?} \chdesc\ contains a copy of its block's previous data to
allow rollback\footnote{Actually, \Kudos\ supports a specialized type of \chdesc\ for
efficiently flipping individual bits using an inline exclusive-or mask instead
of a copy of the previous data, but most \chdescs\ are not of this type.}.
%
In practice, many \chdesc\ are never actually rolled back (e.g. file
data blocks)
%
and the previous data copies nearly double the combined memory usage of
\chdescs\ and cached blocks.
%
To avoid this overhead, \Kudos\ identifies \chdescs\ that will never
need to be rolled back and omits their previous data copies. We call
these \emph{\nrb} \chdescs. (The opposite naturally being a \emph{\rb}
\chdesc, when necessary to differentiate them.)
%
Since a \nrb\ \chdesc\ cannot be rolled back, a write of any \chdescs\
on block $B$ must include all \nrb\ \chdescs\ on $B$. To accordingly
update our formal model we define a new set of \chdescs, \ChNrb, which
contains all \nrb\ \chdescs. We write \ChNrbB{B} to restrict the set
to block $B$\todo{Introduce \ChRb\ and \ChRbB{B}.}:

\paragraph{Write block to disk controller}
For some block $B$: \\
Let \(P \subseteq \ChMemB{B}\) s.t.
\(\BeforeS{P} \subseteq \ChDisk \cup P\) and \(\ChNrbB{B} \subseteq P\) \\
Set \p{p}.state $:=$ \stateinf\ for all \inset{p}{P}

\paragraph{}
To avoid (expensive) dependency traversals to determine whether a new
\chdesc\ will need to be rolled back to write \ChAll,
%
\Kudos\ conservatively identifies \nrb\ \chdescs\ using only local
dependency information.
%
\Kudos\ detects that a new \chdesc\ may need to be rolled back if any
\chdescs\ already on the block have external (on a different
block\todo{Descriptive enough? Mention \noop\ \chdesc\ \after\ recursion?})
\afters.
%
This is both a safe and useful indicator because
%
the presence of an external \after\ is a necessary condition for a new
\chdesc's \before\ to induce a block-level cycle
%
and many blocks have no \chdescs\ with external \afters\ (e.g. most
file data blocks).

While this algorithm detects whether a \chdesc\ may need to be rolled
back to write \ChAll, \Kudos\ must also be sure that no future
dependency manipulation will cause the \chdesc\ to require a rollback.
%
We introduce Invariant~\ref{cdinvar:add-before} to support such reasoning:
%
\cdinvar{add-before}{All block-level cycles induced through
\chdesc\ \p{p}'s \befores\ exist when \p{p} is
created\todo{Change this phrasing? ``Once created, a \chdesc\ will not
gain any \befores\ that induce block-level cycles.''}.}
%
\noindent \Kudos\ ensures this invariant by restricting \before\
additions to \chdesc\ creation, \noop\ \chdescs\ with no \afters, or
when the invariant is statically proven to hold for the affected
\chdescs.

\subsection{Overlap \ChDesc\ Merging}
\label{sec:chdescs:overlap-merge}
\Kudos\ drastically reduces the number of \chdescs\ created to
describe a system of changes by (conservatively) identifying when a
second \chdesc\ may safely be written at the same time as an existing
\chdesc\ and, instead of creating the second \chdesc, updating the
existing \chdesc's disk change and dependencies; in effect, merging
the two \chdescs.
%
In this section we describe overlapping \chdesc\ merging; after
introducing \nrb\ \chdescs\ in the next section we describe
two additional merge opportunties.
%
All three merge rules build on invariant~\ref{cdinvar:add-before}
and use fast, conservative checks during \chdesc\ creation to identify
\chdescs\ that will be possible to write together;
%
the merge rules differ in their conservatively applicable conditions.

Bitmap blocks and inode size fields accumulate many nearby and
overlapping mergeable \chdescs\ as data is appended to a file.
%
Many of these and similar \chdescs\ are mergeable and have
dependencies that allow simple (and fast) reasoning to identify many
of the mergeable pairs: two \chdescs\ that overlap no other in-memory\todo{Have we introduced this terminology?} \chdescs\
and which have no dependency path from the new to the existing \chdesc\
will not induce a block-level cycle and so are writeable together.
We know that \textit{later} changes will not cause them to induce a block-level cycle due to
invariant~\ref{cdinvar:add-before} and by not merging if the new \chdesc{}
has a before and the before is marked as allowed to violate
invariant~\ref{cdinvar:add-before}.
%
While path existance testing is expensive, a conservative path test
of only a depth of two identifies most mergeable \chdescs. If the new
\chdesc{} has an explicit \before\ that is not the existing \chdesc\ and
this \before\ has a \before, then there may be a path to the existing
\chdesc.
%
To merge two such overlapping \chdescs, add the new \chdesc{}'s explicit
before to the existing \chdesc\ (if any and if not the existing \chdesc).

\subsection{\Nrb\ \ChDesc\ Merging}
\label{sec:chdescs:nrb-merge}
Recall from FOO\todo{Name FOO} that to write any \chdesc\ on a given block,
all \nrb\ \chdescs\ on the block must also be written. 
%
These implicit dependencies complicate determining whether a block can
be written, requiring a search for non-ready \nrb\ \chdescs.
%
By merging \nrb\ \chdescs\ together and by merging existing \rb\
\chdescs\ into the block's \nrb\ \chdesc\ when the \nrb\ \chdesc\ is
created,
%
\Kudos\ ensures that there is at most one \nrb\ \chdesc\ per block and
that all \chdescs\ on the same block depend on the \nrb\
\chdesc\,
%
and so avoids dependency complication and further reduces \chdesc\
memory overhead.

While any two \nrb\ \chdescs\ on the same block will be written
together, to merge the \chdescs\ the dependencies involving the merged
\chdescs\ must be transformed to preserve the non-merged system's
semantics.  Sections~\ref{sec:chdescs:nrb-merge:hard-hard}
and~\ref{sec:chdescs:nrb-merge:hard-soft} discuss these
transformations.

Mention?:
%
\Nrb\ \chdescs, \rb\ \chdescs, and ready lists:
%
all post-NRB created \chdescs\ depend on NRB because of overlaps.
%
no NRB-prior created \chdescs\ exist (RB->NRB).
%
single NRB.

\subsubsection{\Nrb{}-\Nrb{} \ChDesc\ Merging}
\label{sec:chdescs:nrb-merge:hard-hard}
% Although merging two \chdescs\ will not induce block-level dependency
% cycles, without sufficient care merging could induce \chdesc-level
% dependency cycles.  A trivial example is merging \p{q} into \p{p} when
% \p{q} has an explicit dependency on \p{p}; the combined \p{(p+q)}
% should not and need not depend on itself.
To preserve the non-merged dependency semantics when merging a new
\nrb\ \chdesc\ \p{q} into an existing \nrb\ \chdesc\ \p{p}, the merged
\p{(p+q)} must depend on the union of \p{p} and \p{q}'s transitive
\befores.
%
From \chdesc\ invariant~\ref{cdinvar:add-before} and the \nrb\
\chdesc\ creation rule, the only possible dependencies involving \p{p}
and \p{q} are those show in Figure~\ref{fig:nrb-merge}\todo{Should we
  give these deductions or a flavor?}.
%
An algorithm to transform dependencies for \nrb\ \chdesc\ merges
follows from these possible dependencies.
%
Notice, for example, that the dependency paths that could become a
cycle upon the merge of \p{p} of \p{q} involve only \noop\ \chdescs\
and that \chdesc\ invariant~\ref{cdinvar:add-before} ensures these
\noop\ \chdesc\ will not gain data \chdesc\ \befores.
%
TODO: explain further?

\begin{figure}[htb]
  \centering
  \includegraphics[width=\columnwidth]{nrb_merge}
  \caption{Possible dependencies when merging \nrb\ \chdesc\ \p{q}
    into existing \nrb\ \chdesc\ \p{p}.}
  \label{fig:nrb-merge}
\end{figure}

\noindent Algorithm called on \p{q} and \p{p}:\\
Input: \chdesc\ \p{a} and existing \nrb\ \chdesc\ \p{p}.
\begin{itemize}
\item If \p{a} is external, return ``no path to \p{p}.''
\item If \p{a} equals \p{p}, return ``path to \p{p}.''
\item Call self on \p{a} and \p{p}.
\item If \p{a} has no path to \p{p}, return ``no path to \p{p}.''
\item For each \p{a} \befores\ \p{b}:
  \begin{itemize}
    \item If \p{b} has no path to \p{p}:
      \begin{itemize}
      \item Move \p{b} from a \before\ of \p{a} to a \before\ of \p{p}.
      \end{itemize}
  \end{itemize}
\end{itemize}

\subsubsection{\Nrb{}-\Rb{} \ChDesc\ Merging}
\label{sec:chdescs:nrb-merge:hard-soft}
TODO: talk about why \nrb\ \chdescs\ slow down revision slice creation or talk
about how we detect RB$\rightarrow$NRB changes and then do \nrb\ \chdesc\ merges
to avoid this slowdown.

Say something along the lines:
%
The dynamic optimizations facilitated through \nrb\
\chdescs\ implement the efficiency in systems using soft updates or
journaling\todo{Actually do this for journaling} while expressing
changes modularly through structural descriptions rather than through
internal and semantic file system descriptions.

\todo{Should we include memory/performance improvement numbers?}
\todo{Should we talk about why we allow NRBs to be disabled? (Debugging
simplicity and depend add to \noop\ \chdescs\ with \afters\ bug catching.)}

\cdinvar{extern-after}{Each block tracks the total number of
external \afters\ for \chdescs\ on that block.}

\cdinvar{one-nrb}{Each block has at most one \nrb\ \chdesc.}

\subsection{Discussion}
\label{sec:chdescs:discussion}

\subsubsection{Unnecessary \ChDesc\ Dependencies}
Even with \Kudos\ dynamically optimizing \chdesc\ graphs, there is only so much
that can be done while preserving the semantics of the dependencies specified by
the \module\ which created the \chdescs. It is obvious that having too few
dependencies compromises the correctness of the system; it is perhaps less
obvious but no less true that having too many dependencies can nontrivially
degrade the performance of the system.

In Figure~\ref{fig:chdescarrange}, we have two possible arrangements for three
\chdescs. The \noop\ \chdesc\ represents a root node that can reach all
other \chdescs. In the parallel arrangement on the right, \Kudos\ has the
freedom to write \chdescs\ $C_1$, $C_2$, and $C_3$ to disk in any order. In the
serial arrangement on the left, there exists only one valid write ordering.
Depending on the arrangement of other \chdescs, \Kudos\ may have to perform
additional rollbacks and write some blocks more times than would otherwise be
necessary in order to write these \chdescs\ to disk. Even if that is not
necessary, \Kudos\ will still have less flexibility in choosing an order to
write the blocks, potentially increasing disk seek times due to suboptimal
ordering.

% {{{ fig:chdescarrange
\begin{figure}[htb]
  \centering
  \includegraphics[width=192pt]{fig/figures_6}
  \caption{\label{fig:chdescarrange} \Chdesc\ dependencies, when
  not strictly needed, restrict the possible choices for write ordering.
  This results in suboptimal write ordering and more scans through the
  \chdescs\ for \Kudos. On the right, \chdescs\ C1, C2, and C3 can be written
  in any order. Only one ordering is possible on the left.}
\end{figure}
% }}}

Even when no unnecessary dependencies are explicitly created, they can still be
created implicitly and care must be taken to avoid them. When \chdescs\ overlap,
the later \chdesc\ is made to depend on the first so that it cannot precede the
change it is changing further to disk. As a result, however, unnecessary
dependencies may result if the \chdescs\ are larger than the data they are
actually changing. In Figure~\ref{fig:overlap}, to update a field in an inode
structure on disk, a \chdesc\ spanning the entire inode could be created even
though only a single field changed. A later change to a different field would
appear to overlap the first, creating an unnecessary dependency. Creating
\chdescs\ which correspond more precisely to the changes being made avoids this
problem, so a utility function is provided by \Kudos\ to make this operation as
convenient as creating a single \chdesc\ for an entire large structure.

% {{{ fig:overlap
\begin{figure}[htb]
  \centering
  \includegraphics[width=64pt]{fig/figures_5}
  \caption{\label{fig:overlap} On inode 17, the gray regions represent
  modified fields that do not overlap. If \chdesc\ A and \chdesc\ B are
  exactly the size of the gray regions, then there is no implicit dependency.
  However, making \chdescs\ for the entire inode data structure will make
  one \chdesc\ depend on the other because they overlap.}
\end{figure}
% }}}

\subsubsection{\ChDesc\ List Ordering}
Several functions in \Kudos\ iterate over lists of \chdescs\ looking for either
a single \chdesc\ or set of \chdescs\ satisfying some property, or trying to
process all the \chdescs\ in the list in some order determined by the dependency
graph. It is generally the case that the \chdescs\ satisfying the property or
the order in which the \chdescs\ should be processed can be determined very
quickly by keeping the lists sorted. For instance, the library function which
rolls \chdescs\ back needs to perform the rollbacks essentially in inverse
creation order, so that rolling back a \chdesc\ which has since been overwritten
by a later \chdesc\ does the right thing. Keeping the list of all \chdescs\ on a
block sorted in creation order (which is very easy) makes this an efficient
operation, while it might otherwise take $O(n^2)$ time to execute. Similarly,
many \chdesc\ merging functions need to find for a given block some \chdesc\
which has no \befores\ on the same block, and the oldest \chdesc\ on a block
always satisfies this requirement.

