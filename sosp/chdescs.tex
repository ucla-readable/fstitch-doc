% -*- mode: latex; tex-main-file: "paper.tex" -*-

\subsection{Block Descriptors}

Disk blocks are represented by objects called \emph{block descriptors}.
%
\Kudos\ maintains at least one block descriptor object for each cached
 block; in certain cases, such as encrypted file systems, where the on-disk
 representation of a block differs fundamentally from the memory
 representation, a block may have more than one block descriptor.


\Kudos\ expects to be able to write blocks atomically.
%
Thus, the modules that interface with the disk return blocks that are 512
 bytes long (the sector size).
%
A block resizer module joins sector-sized blocks into a more convenient
 size for the relevant file system, such as 2048 or 4096 bytes, and splits
 these large blocks into sector-sized chunks as they are written.


\begin{figure}[t]
\vskip-14pt
\begin{tabular}{@{\hskip0.58in}p{2in}@{}}
\begin{scriptsize}
\begin{verbatim}
struct chdesc {
    BD_t *device;
    bdesc_t *block;
    enum {BIT, BYTE, NOOP} type;
    union {
        struct {
            uint16_t offset;
            uint32_t xor;
        } bit;
        struct {
            uint16_t offset, length;
            uint8_t *data;
        } byte;
    };
    struct chdesc_queue *dependencies;
/* ... */ };
\end{verbatim}
\end{scriptsize}
\end{tabular}
\vspace{-10pt}
\caption{\label{fig:chdesc} Partial \chdesc\ structure.}
\end{figure}


\subsection {Change Descriptors}
\label{sec:design:chdescs}

%%\subsubsection {Overview}


% The first use of "change descriptor" is written out. All others, if not the
% same, should be defined in parentheses here.
Each in-memory modification to a cached disk block has an associated
change descriptor. 
%
Each change descriptor applies to data on exactly one block, and each block links
 to all associated change descriptors.
%
Different change types correspond to different forms of
\chdescs; the \chdesc\ for a flipped bit -- such as in a free-block bitmap --
contains an offset and mask, while larger changes contain an offset, a length,
and the new data. The \chdesc's \emph{dependencies} point to other \chdescs\ that must
precede it to stable storage. A \chdesc\ can be applied or reverted to switch
the cached block's state between old and new.

Figure~\ref{fig:chdesc} gives a simplified version of the structure, and
Figure~\ref{fig:chdapi} shows most of the API for working with them. The ability
to revert and re-apply \chdescs\ is inspired by soft updates dependencies, but
generalized so that it is not specific to any particular file system. 
\Chdesc\ dependencies can create cyclic dependencies among
blocks, although the \chdescs\ themselves must never form a cycle. To handle this
case, some \chdescs\ may need to be rolled back, or reverted, in order to write the
others, allowing such cycles to be broken.

\begin{figure}[t]
\vskip-14pt
\begin{tabular}{@{\hskip0.25in}p{2in}@{}}
\begin{scriptsize}
\begin{alltt}
chdesc_t *\textbf{chdesc_create_noop}(
    bdesc_t *block, BD_t *owner);
chdesc_t *\textbf{chdesc_create_bit}(
    bdesc_t *block, BD_t *owner,
    uint16_t offset, uint32_t xor);
int \textbf{chdesc_create_byte}(
    bdesc_t *block, BD_t *owner,
    uint16_t offset, uint16_t length,
    const void *data, chdesc_t **head);
int \textbf{chdesc_create_diff}(
    bdesc_t *block, BD_t *owner,
    uint16_t offset, uint16_t length,
    const void *newdata, chdesc_t **head);
int \textbf{chdesc_add_depend}(
    chdesc_t *depender, chdesc_t *dependee);
void \textbf{chdesc_remove_depend}(
    chdesc_t *depender, chdesc_t *dependee);
int \textbf{chdesc_apply}(chdesc_t *chdesc);
int \textbf{chdesc_rollback}(chdesc_t *chdesc);
int \textbf{chdesc_satisfy}(chdesc_t *chdesc);
int \textbf{chdesc_push_down}(
    BD_t *current_bd, bdesc_t *current_block,
    BD_t *target_bd, bdesc_t *target_block);
\end{alltt}
\end{scriptsize}
\end{tabular}
\vspace{-10pt}
\caption{\label{fig:chdapi} Partial \chdesc\ API.}
\end{figure}

\Kudos\ \modules\ change blocks by attaching change descriptors to them,
using functions such as \texttt{chdesc\_create\_bit}.
%
Most file system modules initially generate \chdescs\ whose
dependencies impose soft-update-like ordering requirements (see
\S\ref{sec:consistency:softupdate}).  These \chdescs\ are then passed down,
through other modules, in the general direction of the disk.  The
intervening \modules\ can
inspect, delay, and even modify them before passing them on further. For
instance, the write-back cache \module\ (\S\ref{sec:modules:wbcache}),
which is essentially a buffer cache, holds
on to blocks and their \chdescs\ instead of forwarding them
immediately. When evicting a block and associated \chdescs, the write-back
cache enforces an order consistent with the \chdesc\ dependency
information.

A change descriptor is \emph{satisfied} when the change to which it
 corresponds has been resolved---for instance, written to disk.
%
If one change descriptor depends on another, then the depender (the change
 descriptor depended on) must be satisfied before the dependee (the change
 descriptor that depends) can be written to disk itself.



%% Since many \modules\ may be stacked on top of one another in \Kudos, and since
%% many of them may want to refer to the same block at the same time, 
Each \chdesc\ on a block may or may not be visible to a given \module.
%
For example, modules that respond to user requests generally view the most
 current state of every block -- the block with all \chdescs\ applied.
%
However, a write-back cache may choose to write some \chdescs\ on a block
 while reverting others, since those others currently have unsatisfiable
 dependencies.
%
In this case, \modules\ below the write-back cache should view the
 unsatisfied \chdescs\ in the reverted state.
%
%% If this \module\ is the disk, it will need to be able to write a version of
%%  the block's data that does not include the \chdescs\ that have not yet
%%  been sent to it.
%
\Kudos\ provides a block revisioning library function that automatically rolls back those
 \chdescs\ that should not be visible at a particular \module, and then
 rolls them forward again after that \module\ is done with the block.


%% \paragraph{Manipulations and Transformations}
% FIXME: add concrete details here

%% \Chdescs\ play a central role in \Kudos, and therefore many parts of the system
%% need to generate, consume, forward, and manipulate them. Obviously there are

A collection of simple operations, such as shifting a \chdesc\ from one block to another,
 checking whether a \chdesc\ has unsatisfiable dependencies, and copying a
 \chdesc\ (useful for modules such as RAID),
%%  -- for instance, many
%% simple block devices like partitioners will need to simply forward \chdescs\ on
%% to the next block device after adjusting the block number being written. The
%% write-back cache (\S\ref{sec:modules:wbcache}) and elevator scheduler
%% (\S\ref{sec:modules:elevator}) both need to determine whether \chdescs\ are
%% ready to send to the next block device or not. In implementing the \modules\ we
are handled by common library functions available to any module.

%% Additionally, there are some more complex transformations which are useful for
%% devices like the mirroring block device (\S\ref{sec:modules:raid}), or the
%% journaling \module (\S\ref{sec:consistency:journal}). The mirroring block device
%% must duplicate the \chdescs\ passing through it, so that each of the mirrors has
%% its own copy. The journal must copy entire blocks into a journal area, and
%% overwrite that journal area efficiently when the blocks are changed again during
%% the same transaction. Both of these \chdesc\ transformations are handled by
%% library functions, so that if other \modules\ need to do similar things, the
%% functionality will already be available.



\paragraph{\Noop\ Change Descriptors}
\label{sec:design:chdescs:noop}

The prototypical \chdesc\ corresponds to some change on disk, but \Kudos\
 also supports a \emph{\noop} \chdesc\ type, which doesn't change the disk
 at all.
%
%% An important type of \chdesc\ is actually one that doesn't change the disk at
%% all: the \noop\ \chdesc. 
%
\Noop\ \chdescs\ can have dependencies, like any other
\chdesc, but they don't need to be written to disk:  they are trivially satisfied when all of their dependencies are
satisfied.
%
Thus, they can be used to ``stand for'' entire sets of other changes.
%
This capability is extremely useful, and is used by most operations on disk
structures so that a single \chdesc\ can be returned that depends on the whole
change. Likewise, a \noop\ \chdesc\ can be passed in as a parameter to a disk
operation to make the whole operation depend on a set of other changes. \Noop\
\chdescs\ allow dependencies between sets without a quadratic number
of dependency edges in the \chdesc\ graph, and without having to pass around
arrays of \chdescs.
%
The cost is that some functions may have to traverse trees of \noop\ change
 descriptors to determine true dependencies.

Modules can also use \noop\ \chdescs\ to \emph{prevent} changes from being
 written.
%
A \emph{managed} \noop\ must be explicitly satisfied; any changes that
 depend on that \noop\ are delayed until the owning module explicitly
 satisfies it.
%
This is used, for instance, by the journal \module\
 (\S\ref{sec:consistency:journal}) to prevent a transaction's \chdescs\
 from being written before the journal commits.

%% \subsubsection{Block Revisions}
% FIXME maybe make an analogy to packets flowing through a network? -LZ

% Should end with an example -- a little diagram, perhaps
% borrowed from soft updates -- and then perhaps discussion, with things like
% the last paragraph of optimizations above
