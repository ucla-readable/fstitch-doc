\section {\ChDescs}
\label{sec:chdescs}

\newcommand{\ChAll}{\ensuremath{\textit{All}}}
\newcommand{\ChAllB}[1]{\ensuremath{\textit{All}[#1]}}
\newcommand{\ChMem}{\ensuremath{\textit{Mem}}}
\newcommand{\ChMemB}[1]{\ensuremath{\textit{Mem}[#1]}}
\newcommand{\ChDisk}{\ensuremath{\textit{Disk}}}
\newcommand{\ChDiskB}[1]{\ensuremath{\textit{Disk}[#1]}}
\newcommand{\ChInf}{\ensuremath{\textit{Inf}}}
\newcommand{\ChInfB}[1]{\ensuremath{\textit{Inf\/}[#1]}}
\newcommand{\ChNrb}{\ensuremath{\textit{\Nrb}}}
\newcommand{\ChNrbB}[1]{\ensuremath{\textit{\Nrb}[#1]}}

\newcommand{\Before}[1]{\ensuremath{\textit{Pre}[#1]}}
\newcommand{\BeforeS}[1]{\ensuremath{\textit{Pre}^*[#1]}}
\newcommand{\After}[1]{\ensuremath{\textit{Post}[#1]}}
\newcommand{\AfterS}[1]{\ensuremath{\textit{Post}^*[#1]}}

\newcommand{\statenone}{\ensuremath{\textit{cached}}}
\newcommand{\stateinf}{\ensuremath{\textit{inflight}}}
\newcommand{\statedisk}{\ensuremath{\textit{ondisk}}}

% {{{ fig:chdesc
\begin{figure}[t]
\vskip-14pt
\begin{tabular}{@{\hskip0.58in}p{2in}@{}}
\begin{scriptsize}
\begin{verbatim}
struct patch {
    bdev_t *device;
    bdesc_t *block;
    enum {BIT, BYTE, NOOP} type;
    union {
        struct {
            uint16_t offset;
            uint32_t xor;
        } bit;
        struct {
            uint16_t offset, length;
            uint8_t *data;
        } byte;
    };
    struct patch_queue *leaders;
    uint32_t nleaders[NBDLEVEL];
    patch_t * ddesc_ready_next;
    patch_t ** ddesc_ready_prev;
/* ... */ };
\end{verbatim}
\end{scriptsize}
\end{tabular}
\vspace{-10pt}
\caption{\label{fig:chdesc} Partial \chdesc\ structure.}
\end{figure}
% }}}

% {{{ fig:chdapi
\begin{figure}[t]
\vskip-14pt
\begin{tabular}{@{\hskip0.25in}p{2in}@{}}
\begin{scriptsize}
\begin{alltt}
int \textbf{patch_create_byte}(
    bdesc_t *block, bdev_t *owner,
    uint16_t offset, uint16_t length,
    const void *data, patch_t **head);
int \textbf{patch_create_bit}(
    bdesc_t *block, bdev_t *owner,
    uint16_t offset, uint32_t xor,
    patch_t **head);
int \textbf{patch_create_empty}(
    bdev_t *owner, patch_t **tail,
    size_t nheads, patch_t * heads[]);
int \textbf{patch_create_diff}(
    bdesc_t *block, bdev_t *owner,
    uint16_t offset, uint16_t length,
    const void *data, patch_t **head);
int \textbf{patch_add_depend}(
    patch_t *\after, patch_t *\before);
void \textbf{patch_remove_depend}(
    patch_t *\after, patch_t *\before);
int \textbf{patch_apply}(patch_t *patch);
int \textbf{patch_rollback}(patch_t *patch);
int \textbf{patch_satisfy}(patch_t *patch);
int \textbf{patch_push_down}(
    bdev_t *current_bd, bdesc_t *current_block,
    bdev_t *target_bd, bdesc_t *target_block);
\end{alltt}
\end{scriptsize}
\end{tabular}
\vspace{-10pt}
\caption{\label{fig:chdapi} Partial \chdesc\ API.}
\end{figure}
% }}}

The fundamental new abstraction in \Kudos\ is called a \chdesc. Each in-memory
modification to a cached disk block has an associated \chdesc. A \chdesc's
\emph{\befores} point to other \chdescs\ that must precede it to stable storage;
the other \chdescs\ which point to it as a \before\ are its \emph{\afters}. A
\chdesc\ can be applied or reverted to switch the cached block's state between
old and new.

Figure~\ref{fig:chdesc} gives a simplified version of the structure, and
Figure~\ref{fig:chdapi} shows much of the API for working with them. The ability
to revert and re-apply \chdescs\ is inspired by soft updates dependencies, but
generalized so that it is not specific to any particular file system. \Chdesc\
dependencies can create cyclic dependencies among blocks. \Chdescs\ themselves
are not allowed to form cycles to ensure that they can always be written to disk
in an order which does not violate the dependency graph. To break block-level
cycles, some \chdescs\ may need to be reverted, or ``rolled back,'' in order to
first write other \chdescs\ on the same block, just as in soft updates.

\Kudos\ \modules\ change blocks by attaching \chdescs\ to them, using functions
such as \texttt{patch\_create\_byte}. Most file system \modules\ initially
generate \chdescs\ whose dependencies impose soft-update-like ordering
requirements (see \S\ref{sec:using:softupdate}). These \chdescs\ are then passed
down, through other \modules, in the general direction of the disk. The
intervening \modules\ can inspect, delay, and even modify them before passing
them on further. For instance, the write-back cache \module\
(\S\ref{sec:using:wbcache}), essentially a buffer cache, holds on to blocks and
their \chdescs\ instead of forwarding them immediately. Finally, \chdescs\ are
\emph{satisfied} when their associated data reaches stable storage.

\subsection{Notation and Model}
\label{sec:chdescs:notation}

It will be convenient to introduce some notation for dealing with \chdescs.
First we will define a set \ChAll\ of all \chdescs. Although in the actual
implementation we destroy \chdescs\ when they are no longer necessary (i.e. when
they have successfully been written to the disk), \ChAll\ contains all \chdescs,
written to disk or not. We also define sets \ChMem\ and \ChDisk, which contain
all \chdescs\ only in memory or already on the disk, respectively. Next, we
define a set \ChInf\ which contains all \chdescs\ that are currently ``in
flight'' to the disk: block data reflecting their contents has been sent to the
disk controller but has not yet been written to the disk media itself. Finally,
we will also write \ChAllB{b}, \ChMemB{b}, \ChDiskB{b}, and \ChInfB{b} to
succinctly represent the subsets of these four sets on some block $b$.

If we have some \chdesc\ \p{p}, we can also refer to its block or state using a
``member'' notation as in many programming languages: \p{p}.block is the block
modified by \p{p}, and \p{p}.state is its state (one of \statenone, \stateinf,
or \statedisk). We write \depends{\p{p}}{\p{q}} if \p{q} is listed as a \before\
of \p{p} (that is, if \p{p} directly depends on \p{q}), and
\indirdepends{\p{p}}{\p{q}} if there is some path of \befores\ from \p{p} to
\p{q} (that is, if \p{p} indirectly depends on \p{q}).

We can model the interaction between the disk (and its controller) and the
software maintaining the \chdesc\ graph by a set of randomly occurring events
that change the state of \chdescs\ according to simple rules. By stipulating
that each event may occur at any time when its preconditions are met, we can
reason about all possible hardware behavior and all possible software
algorithms. The first event corresponds to creating a new \chdesc:

% FIXME

\noindent At any time $t$, one of the following transformations may be applied:
\begin{enumerate}
\item For some \block{w}:\\
Let \(\pset{B} = \{a\ |\ \inset{a}{\psetinflight{t}} \land \blockof{a} = \block{w} \}\) in\\
\(\psetsat{t+1} := \psetsat{t} \cup \pset{B}\)\\
\(\psetinflight{t+1} := \psetinflight{t} \setminus \pset{B}\)
\label{action:write}

\item \(\deps{t+1} := \deps{t} \cup\ \pset{X}\) where
\(\notinset{a}{\psetall{t}} \land \pset{X} \subset \{\depends{a}{b}\}\)\\
\(\psetall{t+1} := \psetall{t} \cup\ \{a\}\)
\label{action:create}

\item \(\psetinflight{t+1} := \psetinflight{t} \cup \{a\}\) for some
\notinset{a}{\psetsat{t}} s.t. \(\forall b: \indirdepends{a}{b},\
\inset{b}{\psetsat{t}} \lor (\inset{b}{\psetinflight{t}} \land
\blockof{a} = \blockof{b}\)
\label{action:fly}
\end{enumerate}

We can prove that invariants~\ref{cdinvar:a} and~\ref{cdinvar:b}
with actions~\ref{action:write}, \ref{action:create}, and~\ref{action:fly}
imply that invariant~\ref{cdinvar:c} holds.

\noindent Action~\ref{action:create} replacements for hard patches:
\begin{enumerate}
\item \(\deps{t+1}\ := \deps{t} \cup\ \pset{X}\) for
\(\notinset{a}{\psetall{t}} \land \pset{X}\! \subset\!
\{\depends{a}{b}\ |\ \inset{b}{\psetall{t}}\}\)\\
\(\psetall{t+1} := \psetall{t} \cup\ \{a\}\)

\item \(\psethard{t+1}\ := \psethard{t} \cup \{a\}\) where \(\forall b\!:
\indirdepends{b}{a},\ \blockof{b} = \blockof{a}\)
\end{enumerate}

\noindent Action~\ref{action:fly} replacements for hard patches:
\begin{enumerate}
\item \(\psetinflight{t+1} := \psetinflight{t} \cup \{a\}\) for
\notinset{a}{\psetall{t} \cup \psethard{t}}\\
\(\forall b\!: \indirdepends{a}{b},\ \inset{b}{\psetsat{t}} \lor (\inset{b}{\psetinflight{t}} \land \blockof{a} = \blockof{b})\)\\
\(\not\exists\inset{b}{\psethard{t}}\ \mbox{s.t.}\ \blockof{b} = \blockof{a}\)

\item For some \block{w}:\\
\(\psetinflight{t+1} := \psetinflight{t} \cup \pset{X}\) for
\(\pset{X} = \{\inset{a}{\psethard{t}}\ |\ \blockof{a} = \block{w}\}\)\\
\(\forall \notinset{b}{\pset{X}}\!: \forall \inset{a}{\pset{X}}\!:
\indirdepends{a}{b},\ \inset{b}{\psetsat{t}} \lor
(\inset{b}{\psetinflight{t}} \land \blockof{a} = \blockof{b})\)\\
\(\psethard{t+1} := \psethard{t} \setminus \pset{X}\)
\end{enumerate}

\subsection{Crap}
When a \module\ asynchronously writes the data associated with a set of
\chdescs\ to stable storage it marks the \chdescs\ as \emph{in flight} to
indicate that the \chdescs\ are considered to still have their previous,
higher, level. Their \afters\ will therefore remain at that level (or higher),
thus correctly implementing the dependency structure. A \chdesc\ is
\emph{satisfied} when its associated data reaches stable storage.
\todo{mention NCQ and FUA/wtcache, safe reordering, remove levels}

Each \chdesc\ on a block may or may not be visible to a given \module. For
example, \modules\ that respond to user requests generally view the most current
state of every block -- the block with all \chdescs\ applied. However, a
write-back cache may choose to write some \chdescs\ on a block while reverting
others, since those others currently have unsatisfied \befores. In this case,
\modules\ below the write-back cache should view the unsatisfied \chdescs\ in
the reverted state.
%
\Kudos\ provides a block revisioning library function that automatically rolls
back those \chdescs\ that should not be visible at a particular \module, and
then rolls them forward again after that \module\ is done with the block.

% -*- mode: latex; tex-main-file: "paper.tex" -*-

\subsection{Implementation and \noop\ \chdescs}
\label{sec:patch:noop}

\Kudos\ file system implementations create \chdescs\ with one of the
following functions:

\begin{scriptsize}
\begin{alltt}
int \textbf{patch_create_byte}(
    bdesc_t *block, bdev_t *owner,
    uint16_t offset, uint16_t length,
    const void *data, patch_t **head);
int \textbf{patch_create_bit}(
    bdesc_t *block, bdev_t *owner,
    uint16_t offset, uint32_t xor,
    patch_t **head);
int \textbf{patch_create_\noop}(
    bdev_t *owner, patch_t **tail,
    size_t nheads, patch_t * heads[]);
int \textbf{patch_create_diff}(
    bdesc_t *block, bdev_t *owner,
    uint16_t offset, uint16_t length,
    const void *data, patch_t **head);
\end{alltt}
\end{scriptsize}


We first address dependency convenience and memory usage with \emph{\noop\
\chdescs}, which have no associated data or block.
%
This makes it just a means for tracking dependencies.


For example, imagine writing two files \texttt{before.txt}
The dependencies 
\Chdescs\ as so far described can be tedious and inefficient to manage when
dealing with large sets of them corresponding to file system operations. For
instance, if writing some file \texttt{\after.txt} is to depend on writing some
other file \texttt{\before.txt}, it will be inconvenient to keep arrays of all
the \chdescs\ corresponding to the two operations and inefficient to store the
potentially quadratic number of edges in the \chdesc\ graph.

To solve this problem, we introduce an additional type of \chdesc. The
prototypical \chdesc\ corresponds to some change on disk, but \Kudos\ also
supports \aemphnoop\ \chdesc\ type, which doesn't change the disk at all.
\Noop\ \chdescs\ can have \befores, like other \chdescs, but they don't need to
be written to disk: they are trivially satisfied when all of their \befores\ are
satisfied. Thus, they can be used to ``stand for'' entire sets of other changes.
%
This capability is extremely useful, and is used by most operations on disk
structures so that a single \chdesc\ can be returned that depends on the whole
change. Likewise, \anoop\ \chdesc\ can be passed in as a parameter to a disk
operation to make the whole operation depend on a set of other changes. \Noop\
\chdescs\ allow dependencies between sets with only a linear number of
dependency edges in the \chdesc\ graph, and without having to pass around arrays
of \chdescs.
%
The cost is that some functions may have to traverse trees of \noop\ \chdescs\
to determine true dependencies.

Modules can also use \noop\ \chdescs\ to \emph{prevent} changes from being
written. A \emph{managed} \noop\ \chdesc\ must be explicitly satisfied; any
changes that depend on that \noop\ \chdesc\ are delayed until the owning \module\
explicitly satisfies it. This is used, for instance, by the journal \module\
(\S\ref{sec:using:journal}) to prevent a transaction's \chdescs\ from
being written before the journal commits.

\Noop\ \chdescs\ are not included in our formal model of \chdescs\ for simplicity;
they add some additional complexity but do not change the basic ideas.
\todo{Well, they are now... the rules need updating since they have no blocks.}

% -*- mode: latex; tex-main-file: "paper.tex" -*-

\subsection{Ready \Patch\ Lists}
\label{sec:patch:readylist}

\newcommand{\PReady}[1]{\ensuremath{#1.\textit{ready}}}

Another important optimization greatly reduces CPU time spent in the
\Kudos\ buffer cache.
%
The buffer cache's main task is to choose sets of \patches\ $P$ that
satisfy the in-flight safety property $\PDepset{P} \subseteq P \cup
\PDisk$.
%
A naive implementation would simply traverse the dependency graph starting
at these patches, looking for problematic dependencies.
%
Patch merging can reduce the size of these traversals by combining patches
together.
%
Unfortunately, even modest traversals become painfully slow when executed
on every block in a large buffer cache, and in our initial implementation
these traversals were a performance bottleneck for even modest cache
sizes.
% 
\Featherstitch\ therefore precomputes much of the information
required for the buffer cache to choose a set of \patches\ to write.

\Kudos\ explicitly tracks, for each \patch, how many of its
direct dependencies remain uncommitted or in flight.
%
These counts are incremented as \patches\ are added to the system, and
decremented as the system receives commit notifications from the disk.
%
When both counts reach zero, the \patch\ is safe to write, and it is moved
into a \emph{ready list} on its containing block.
%
\begin{comment}
(\Noop\ \patches\ automatically commit when all their dependencies commit.)
\end{comment}
%
The buffer cache, then, can immediately tell whether any of a block's
patches are writable by examining its ready list.

To write a block $\PB$, the buffer cache initially populates the set $P$ with the
contents of the ready list.
%
While moving a patch $p$ into $P$, \Kudos\ checks whether there exist
dependencies $q \PDDepend p$ where $q$ is also on block $\PB$.
%
The system can potentially write $q$ at the same time as $p$, so $q$'s
counts are updated as if $p$ has already committed.
%
This may make $q$ ready, after which it in turn is added to $P$.
%
(This premature accounting is safe because the system won't try to write
$\PB$ again until $p$ actually commits.)


On-line maintenance of the ready counts adds some cost to several \patch\
manipulations, but since it saves so much duplicate work in the buffer
cache the resulting system is more efficient by multiple orders of
magnitude---and in particular, CPU time no longer scales superlinearly with
the size of the cache.


\begin{comment}
For a \module\ like the write-back cache to forward \patches\ in a
dependency-preserving order, the \module\ must find \patches\ whose \befores\
are all ``closer to the disk'' (or are also being forwarded as part of the same
block write). We say that such \patches\ are \emph{ready}. 


Each \patch\ has a count of the number of \befores\ it has at block device
modules just as close to the disk as it currently is, and a count of the number
of \befores\ it has which are in flight. When these counts are both zero, it is
ready. A \patch's \before\ counts are incrementally updated as \befores\ are
added and removed and as \beforing\ \patches\ are moved closer to the disk.

Because \Kudos\ makes sure that the \befores\ of a \patch\ are at least as
close to the disk as it is, only directly reachable \beforing\ \patches\ need to
be included in a \patch's \before\ counts. \Noop\ \patches, with the exception
of managed \noop\ \patches\ (which have an explicit owning block device), add a
wrinkle to this simplifying rule, however. They are considered to be as close to
the disk as their \before\ which is the farthest from the disk, in effect,
propagating the distance to the disk metric through them.

When a \before\ count update changes whether a \patch\ is ready to write, the
\patch's inclusion in its block's ready list is updated. To write a block, a
\module\ thus iterates through the block's ready list, sending \patches\ to the
target block device, until the list is empty. Thus instead of having to
repeatedly traverse \patch\ graphs to determine readiness on demand, we have
this information maintained automatically as it changes. This automatic
maintenance adds some cost to forwarding \patches\ and changing the graph
structure, but since it saves so much duplicate work\footnote{The amount of
duplicate work saved is actually superlinear in the size of the write-back
cache.} it is much more efficient.
\end{comment}

% -*- mode: latex; tex-main-file: "paper.tex" -*-

\subsection{\Nrb\ \Patches}
\label{sec:patch:nrb}

The first optimization reduces space overhead by
eliminating undo data.
%
When a \patch\ $p$ is created, \Kudos\ conservatively detects whether $p$
 might require reversion:
%
that is, whether any possible future patches and dependencies could force
 the buffer cache to undo $p$ before making further progress.
%
If no future patches and dependencies could force
 $p$'s reversion, then $p$ does not need undo data.
%
The challenge is to detect this condition without predicting the future.
%
We solve this challenge by restricting dependency creation.


We implemented support for \textbf{hard patches}, which are simply patches
 that lack undo data.
%
Since a hard patch $h$ cannot be reverted, any other patch on its block
 must depend on $h$ (the other patches can't be written without $h$).
%
We enforce this requirement, for example using overlap
 dependencies, and
%
as a result, the buffer cache will write a block's hard patches (if any)
 whenever it writes the block.
%
The system aims to reduce memory usage by making most patches hard.


But which patches can be made hard?
%
Define a \emph{block-level cycle} as a dependency chain of uncommitted
 patches $p_n \PDepend \cdots \PDepend p_1$ where the ends have the same
 block $\PBlock{p_n} = \PBlock{p_1}$, and at least one patch in the middle
 has a different block $\PBlock{p_i} \neq \PBlock{p_1}$.
%
The patch $p_n$ is called the \emph{head} of the block-level cycle.
%
Now assume that a patch $p \in \PMem$ is not the head of any block-level
 cycle.
%
One can then show that the buffer cache can always write a block without
 rolling back $p$.
%
The hard case is where $\PBlock{p}$ cannot itself be written without
 rolling back $p$, which occurs when $p$ has an uncommitted dependency $q$
 on a different block.
%
However, we know that $q$'s uncommitted dependencies, if any, are all on
 blocks other than $p$'s; otherwise there would be a block-level cycle.
%
Since \Featherstitch\ disallows circular dependencies, every
 chain of dependencies starting at $q$ has finite length, and therefore
 contains an uncommitted patch $x$ all of whose dependencies have
 been committed.
%
(If $x$ has in-flight dependencies, simply wait
 for the disk controller to commit them.)
%
Since $x$ is not on $p$'s block, the buffer cache can write $x$ without
 rolling back $p$.


\Featherstitch\ may thus make a patch hard when it can prove that patch
 will never be the head of a block-level cycle.
%
This requires two tricks.
%
First, an API restriction allows us to search for \emph{existing},
 rather than future, block-level cycles:
%
\emph{A \patch's direct dependencies are all supplied at creation time}.
%
After $p$ is created, the system can add new dependencies $q \PDDepend p$,
 but will never add new dependencies $p \PDDepend q$.\footnote{The actual
 rule is somewhat more flexible: modules may add new direct dependencies if
 they guarantee that those dependencies don't produce any new block-level
 cycles.  As one example, if no \patch\ depends on some \noop\ \patch\ $e$,
 then adding a new $e \PDDepend q$ dependency can't produce a cycle.}
%
Since every \patch\ follows this rule, all possible block-level cycles with
 head $p$ are present in the dependency graph when $p$ is created.
%
\Featherstitch\ must still check for these cycles, of course, and
%
actual graph traversals proved expensive.
%
We thus implemented a conservative approximation: \patch\ $p$ is
created as \nrb\ if \emph{no} patches on other blocks depend on uncommitted
 patches on $\PBlock{p}$.
%
That is, given any dependency between uncommitted patches $y \PDepend x$,
 either $\PBlock{x}$ isn't on $p$'s block or \emph{both} $x$ and $y$ are on
 $p$'s block.
%
If this holds, then $p$ cannot head a block-level cycle no matter its
 dependencies.
%
This heuristic works well in practice and, given some bookkeeping, 
 takes $O(1)$ time to check.


\begin{comment}
\Kudos\ further ensures that the dependency structure correctly
represents dependencies on the same block through overlap
dependencies: since \nrb\ \patches\ are considered to cover the entire
block, every succeeding \patch\ will overlap at least one \nrb\ \patch,
and \Kudos\ will automatically add a dependency.
%
(Some cases are handled by other optimizations.)


The buffer cache's ``write block'' behavior must account for \nrb\
\patches, as it \emph{must} write any \nrb\ \patches\ that exist on a
block.
%
Let $\PHard[b]$ be the set of \nrb\ \patches\ on block $b$.
%
Then to write block $b$, the buffer cache must choose some $P \subseteq
\PMem[b]$ with
%
\[ \PDepset{P} \subseteq P \cup \PDisk \text{ and } \PHard[b] \cap \PMem
\subseteq P. \]
%
If no such $P$ exists, then the cache must write a different block.
\end{comment}


Applying \nrb\ \patch\ rules to our example makes 14 of the 23 \patches\ \nrb\
(Figure~\ref{fig:opt}b),
%
reducing the undo data required by slightly more than half.


\begin{comment}
%
To avoid this overhead, \Kudos\ identifies \patches\ that will never
need to be reverted and omits their undo data. We call these \emph{\nrb}
\patches. (The opposite naturally being a \emph{\rb} \patch, when
necessary to differentiate them.)
%
Since a \nrb\ \patch\ cannot be reverted, a write of any \patches\
on block $\PB$ must include all \nrb\ \patches\ on $\PB$. To accordingly
update our formal model we define a new set of \patches, \PHard, which
contains all \nrb\ \patches. We write \PHard[\PB] to restrict the set
to block $\PB$\todo{Introduce \PSoft\ and \PSoft[\PB].}:

\begin{tabbing}
\textbf{Write block.} \\
\quad Pick some block $b$ with $\PMem[b] \neq \emptyset$. \\
\quad Pick some $P \subseteq \PMem[b]$ with $\PDepset{P} \subseteq P \cup
\PDisk$ and $\PHard[\PB] \subseteq P$. \\
\quad Move each $p \in P$ to $\PInf$ (in-flight). \\
\quad For each $p \in \PMem[\PB]-P$, set $\PDDepset{p} \gets \PDDepset{p}
\cup P$.
\end{tabbing}

\paragraph{}
To avoid (expensive) dependency traversals to determine whether a new
\patch\ will need to be reverted,
%
\Kudos\ conservatively identifies \nrb\ \patches\ using only local
dependency information.
%
\Kudos\ detects that a new \patch\ on block $\PB$ may need to be reverted if:
\todo{Which form is easier to read? Can we write \(\PMem - \PMem[\PB] - \PEmpty\) more concisely?}
%
\todo{Actually, our implementation also uses in flight \patches. Can we make
it not?}
%
\[ \PRDepset{\PMem[b]} \cap (\PMem - \PMem[b] - \PEmpty) \ne \emptyset \]
\[ \exists \inset{p}{\PMem[b]}\!:\
   \exists c\!:\ \exists \inset{q}{\PMem[c]}\!:\
   \indirdepends{q}{p} \]
%
This is both a safe and useful indicator because
%
the presence of an external \after\ is a necessary condition for a new
\patch's \before\ to induce a block-level cycle
%
and many blocks have no \patches\ with external \afters\ (e.g. most
file data blocks).

While this algorithm detects whether a \patch\ may need to be reverted,
\Kudos\ must also be sure that no future dependency manipulation
will cause the \patch\ to require being reverted.
%
We introduce Invariant~\ref{cdinvar:add-before} to support such reasoning:
%
\cdinvar{add-before}{All block-level cycles induced through
\patch\ \p{p}'s \befores\ exist when \p{p} is
created\todo{Change this phrasing? ``Once created, a \patch\ will not
gain any \befores\ that induce block-level cycles.''}.}
%
\noindent \Kudos\ ensures this invariant by restricting \before\
additions to \patch\ creation, \noop\ \patches\ with no \afters, or
when the invariant is statically proven to hold for the affected
\patches.
\end{comment}

\subsection{\ChDesc\ Merging}
\label{sec:patch:merge}

File operations such as block allocations, inode updates, and directory updates
create a large number of small, distinct \chdescs. Keeping track of many small
\chdescs\ and their dependencies requires significant amounts of memory.
%
\Kudos\ uses \emph{merges} to drastically reduces the number of \chdescs, and
thus \chdesc\ memory usage, by conservatively identifying when a new and an
existing \chdesc\ pair can be written to disk together. Rather than creating
a new \chdesc, Kudos updates the disk change and dependencies to merge the
changes into the existing \chdesc.
%
In this section we present three \chdesc\ merge algorithms. All three
use Invariant~\ref{cdinvar:add-before} to reason about future graph
changes and use fast, conservative checks during \chdesc\ creation; they
differ in their applicable conditions.

\subsubsection{\Nrb\ \ChDesc\ Merging}
\label{sec:chdescs:merge:nrb}

Recall from \S\ref{sec:chdescs:nrb} that a write of any \chdescs\ on block
$b$ must include all \nrb\ \chdescs\ on $b$.
%
This additional requirement is in fact an exquisite optimization
opportunity; it implies that all \nrb\ \chdescs\ on a given block can
be merged.
%
Further, merging can remove the need for the \nrb\ \chdesc\ implicit
dependency rules by ensuring that
%
there is at most one \nrb\ \chdesc\ per block (\nrb-\nrb\ merging)
%
and that all \rb\ \chdescs\ on a given block depend on the \nrb\ \chdesc\
(\nrb-\rb\ merging).
%
We describe these two \chdesc\ merging algorithms and how they
preserve dependency semantics in this section.

\paragraph{\Nrb-\Nrb\ \ChDesc\ Merging}
\label{sec:chdescs:merge:nrb:hard-hard}

\emph{\Nrb-\nrb\ \chdesc\ merging} merges a new \nrb\ \chdesc\ \p{q}
into an existing \nrb\ \chdesc\ \p{p} instead of creating \p{q}.
%
Any two \nrb\ \chdescs\ on the same block may be (and are) merged.
%
Merging all \nrb\ \chdescs\ ensures:
%
\cdinvar{one-nrb}{\(\forall\! b\!: |\PHard[b]| \leq 1\)}
%
\noindent
%
Invariant~\ref{cdinvar:one-nrb} simplifies \nrb\ \chdesc\ handling by
%
removing the implicit dependencies that ensure all \nrb\ \chdescs\
are written together
%
and by removing the need to scan for an existing \nrb\ \chdesc\ when
\nrb-\nrb\ \chdesc\ merging.
%
% Although merging two \chdescs\ will not induce block-level dependency
% cycles, without sufficient care merging could induce \chdesc-level
% dependency cycles.  A trivial example is merging \p{q} into \p{p} when
% \p{q} has an explicit dependency on \p{p}; the combined \p{(p+q)}
% should not and need not depend on itself.
%
To preserve dependency semantics, the merged \p{(p+q)} must depend on
the union of \p{p} and \p{q}'s transitive \befores. Additionally, while the
\chdescs\ can be merged without forming a \chdesc-level dependency cycle,
the merge must ensure that it does not introduce a needless cycle, e.g.
through \anoop\ \chdesc\ \p{e} in \depends{q}{\depends{e}{p}}
\todo{Is cycle avoidance worth mentioning? Is this a good way to mention it?}.

From Invariant~\ref{cdinvar:add-before} and the \nrb\ \chdesc\
creation condition (no external \afters), the only possible
dependencies involving \p{p} and \p{q} are those shown in
Figure~\ref{fig:nrb-merge}\todo{Should we give these deductions or a
  flavor?}.
%
Notice, for example, that any \p{r} such that
\indirdepends{q}{\indirdepends{r}{p}} is \anoop\ \chdesc\todo{This is
  a strong statement. Expand on its implications?}.
%
Our algorithm to transform dependencies for \nrb-\nrb\ \chdesc\ merges
(Figure~\ref{algo:merge:hard-hard}) follows from these possible
dependencies.
%
It updates \p{p}'s transitive \befores\ to ensure
\(\PDepset{q}\todo{Incorrect! Only the non-\noop{}s.} \subseteq
\PDepset{p}\)\todo{Note that Invariant~\ref{cdinvar:add-before}
  ensures that \noop\ \chdescs\ reachable from \p{q} will not gain
  data \chdesc\ \befores?}\todo{Note that it only needs to move dependencies?}.

\begin{figure}[htb]
  \centering
  \includegraphics[width=\columnwidth]{nrb_merge}
  \caption{Possible dependencies when merging \nrb\ \chdesc\ \p{q}
    into existing \nrb\ \chdesc\ \p{p}.}
  \label{fig:nrb-merge}
\end{figure}

\noindent Algorithm called on \p{q} and \p{p}:\\
Input: \chdesc\ \p{a} and existing \nrb\ \chdesc\ \p{p}.\\
Returns: whether \indirdepends{a}{p} exists. \(\forall\! \p{b}\!: \indirdepends{a}{b}\) and \notindirdepends{b}{p}, creates \indirdepends{p}{b}.

\begin{itemize}
\item If \p{a} is external, return ``no path to \p{p}.''
\item If \p{a} equals \p{p}, return ``path to \p{p}.''
\item Call self on \p{a} and \p{p}.
\item If \p{a} has no path to \p{p}, return ``no path to \p{p}.''
\item For each \p{a} \before\ \p{b}:
  \begin{itemize}
  \item If \p{b} has no path to \p{p}:
    \begin{itemize}
    \item Move \p{b} from a \before\ of \p{a} to a \before\ of \p{p}.
    \end{itemize}
  \end{itemize}
\end{itemize}

\paragraph{\Nrb-\Rb\ \ChDesc\ Merging}
\label{sec:chdescs:merge:nrb:hard-soft}

When creating the first \nrb\ \chdesc\ on a block, \emph{\nrb-\rb\
  \chdesc\ merging} merges all existing (\rb) \chdescs\ into the new
\nrb\ \chdesc.
%
Such an arrangement can arise through a combination of \chdesc\ creates
and block writes; 
%
for example, the block may first obtain an initial (\nrb{}) \chdesc,
%
then gain external \afters\ on its \chdesc,
%
accumulate additional (\rb{}) \chdescs,
%
write the subset of its \chdescs\ with external \afters\ (leaving some
\rb\ \chdescs\ on the block),
%
and then gain a \nrb\ \chdesc.
%
In addition to reducing the number of data \chdescs, \nrb-\rb\
\chdesc\ merging removes the second implicit \nrb\ \chdesc\
dependency, that \rb\ \chdescs\ not explicitly dependent on the
block's \nrb\ \chdesc\ implicitly depend on it.
%
As in \nrb-\nrb\ \chdesc\ merging, \Kudos\ merges such \chdescs\ to
avoid the complications of their implicit dependencies.

\Nrb-\rb\ \chdesc\ merging's implementation first merges all \rb\ \chdescs\
into a \nrb\ \chdesc\ and then \nrb-\nrb\ \chdesc\ merges the new \nrb\
\chdesc\ into the now-existing \nrb\ \chdesc.
%
Our algorithm to transform the dependencies for \nrb-\rb\ \chdesc\
merges (Figure~\ref{algo:merge:hard-soft}) for block $b$
%
chooses a \chdesc\ \p{p} such that
\(\notexists \inset{q}{\PMem[b]}\!: \indirdepends{p}{q}\)
%
and updates its transitive \befores\ to ensure
\(\PDepset{\PSoft[b]} \subseteq \PDepset{p}\).
%
Because any \(\inset{q}{\PMem[b] - p}\) may have \afters, to
preserve dependencies we convert such a \p{q} into \anoop\ \chdesc\
and ensure \depends{q}{p}.

\noindent Algorithm:
\begin{itemize}
\item Choose a \(\inset{p}{\PMem[b]}\!:\
\notexists\! \inset{q}{\PMem[b]}\!:\ \indirdepends{p}{q}\).
\item For each \inset{q}{\PMem[b] - p}:
  \begin{itemize}
  \item Call the \nrb-\nrb\ \before\ move algorithm on \p{q} and \p{p}.
  \item Convert \p{q} into \anoop\ \chdesc.
  \end{itemize}
\item Convert \p{p} into a \nrb\ \chdesc\ (free it's previous data copy).
\end{itemize}

\todo{Note that \nrb-\rb\ merging is rare? Note why it is helpful even though
it is rare?}
\todo{Explain why this preserves dependency semantics? Show possible
dependencies? For the paper, free \chdescs\ instead of convert them
into \noop{}s? (Must modify \nrb-\nrb\ algo usage.)}

\subsubsection{Overlap \ChDesc\ Merging}
\label{sec:chdescs:merge:overlap}
\todo{Note as useful when new may need to be rolled back.}
Bitmap blocks and inode size fields accumulate many nearby and
overlapping mergeable \chdescs\ as data is appended to or truncated
from a file.
%
Many of these and similar \chdescs\ are mergeable and have
dependencies that allow simple (and fast) reasoning to identify many
of the mergeable pairs: two \chdescs\ on block $b$ that overlap no other \chdescs\ in \PMem[b]
and which have no dependency path from the new to the existing \chdesc\
will not induce a block-level cycle and so are writable together.
We know that \textit{later} changes will not cause them to induce a block-level cycle due to
invariant~\ref{cdinvar:add-before} and by not merging if the new \chdesc\
has a before and the before is marked as allowed to violate
invariant~\ref{cdinvar:add-before}.
%
While path existence testing is expensive, a conservative path test
of only a depth of two identifies most mergeable \chdescs. If the new
\chdesc\ has an explicit \before\ that is not the existing \chdesc\ and
this \before\ has a \before, then there may be a path to the existing
\chdesc.
%
To merge two such overlapping \chdescs, add the new \chdesc's explicit
before to the existing \chdesc\ (if any and if not the existing \chdesc).


%%

At the end of \chdesc\ optimizations, say something along the lines:
%
The dynamic optimizations facilitated through \nrb\
\chdescs\ implement the efficiency in systems using soft updates or
journaling\todo{Actually do this for journaling} while expressing
changes modularly through structural descriptions rather than through
internal and semantic file system descriptions.

\todo{Should we talk about why we allow NRBs and merging to be
  disabled? (Debugging simplicity and depend add to \noop\ \chdescs\
  with \afters\ bug catching.)}

\input{chdescdiscuss}
