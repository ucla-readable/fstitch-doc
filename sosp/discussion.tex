\section {Discussion}
\label{sec:discussion}

There are several areas in which we would like to improve our work. The obvious
first area we would like to work on is the performance of \Kudos. We have
already improved the performance by literally several orders of magnitude over
the original implementation, but \Kudos\ is still not as fast as it could be --
and not quite as fast as it needs to be to be a viable option for most computer
systems.

We would also like to identify places where \Kudos\ can be made more flexible,
by adding more features to the system and changing the existing parts that make
these features difficult to add. For instance, we would like to find more
applications which can take advantage of \opgroups, and see what changes might
need to be made to the \opgroup\ interface in order to facilitate adapting those
applications.

There are several ways in which the performance can be improved. For instance,
we create a very large number of \chdescs, and the sheer number of them can
cause problems for any algorithm which needs to traverse parts of the \chdesc\
dependency graph. We already attack this problem by merging \chdescs\ using
several different heuristics to determine when it is safe to do so. However, the
heuristics are fairly conservative, and further improvements will likely uncover
additional merging opportunities.

The merging \Kudos\ currently performs preserves the data exactly, and only
alters the graph structure. When files are deleted, it is often the case that
during the delete operation many blocks are updated but then later freed. It is
not actually necessary to write the updated versions of the blocks at all in
this case; they can merely be discarded. A new type of \chdesc, one which merges
with those already on the block and indicates that the block need not be written
(effectively cancelling them out), could help \Kudos\ take advantage of this
situation.

Another strategy for dealing with the large number of \chdescs\ is to avoid
costly traversals altogether by incrementally calculating and propagating
information about the \chdesc\ graph. We already do this as well, but there are
still some places where we must iterate over potentially large lists of
\chdescs. For instance, when creating a new \chdesc, \Kudos\ needs to identify
other \chdescs\ on the same block which overlap with the new \chdesc. This is
currently one of the larger uses of CPU time in \Kudos. Some optimization has
already been done to improve the performance of this operation, but the problem
has not yet been completely solved.

Finally, there are a few additional features that we would like to add to
\Kudos. For instance, the \LFS\ interface does not currently support sparse
files, nor does the UHFS module. We'd also like to improve the journal module,
to support the stronger data ordering provided by Linux ext3, and to use the
same format for the journal as ext3.
