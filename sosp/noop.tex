% -*- mode: latex; tex-main-file: "paper.tex" -*-

\subsection{Patch Implementation}
\label{sec:patch:noop}

Our \Kudos\ file system implementation creates patch structures
corresponding directly to this abstraction.
%
Functions like \texttt{patch\_\-create\_\-byte} create patches;
%
their arguments include the relevant block, any
direct dependencies, and the new data.
%
Most \patches\ specify this data as a contiguous byte range, including an
offset into the block and the \patch\ length in bytes.
%
The undo data for very small \patches\ (4 bytes or less) is stored in the
\patch\ structure itself; for larger \patches, undo data is stored in
separately allocated memory.
%
In bitmap blocks, changes to individual bits in a word can have independent
dependencies, which we handle with a special bit-flip patch type.


The implementation automatically detects one type of dependency.
%
If two \patches\ $q$ and $p$ affect the same block and have overlapping data
ranges, and $q$ was created after $p$, then \Kudos\ adds an \emph{overlap
dependency} $q \PDDepend p$ to ensure that $q$ is written after $p$.
%
File systems need not detect such
dependencies themselves.


For each block $\PB$, \Kudos\ maintains a list of all \patches\ with
$\PBlock{p} = \PB$.
%
However, committed \patches\ are not tracked; when
\patch\ $p$ commits, \Kudos\ destroys $p$'s data structure and removes all
dependencies $q \PDDepend p$.
%
Thus, a patch whose dependencies have all committed appears like a patch
with no dependencies at all.
%
Each patch $p$ maintains doubly linked lists of its direct dependencies
and ``reverse dependencies'' (that is, all $q$ where $q
\PDDepend p$).


The implementation also supports \emph{\noop} \patches, which have
 no associated data or block.
%
%% \Noop\ \patches\ are useful for representing sets of dependencies and for
%%  preventing other \patches\ from being written.
%
For example, during a journal transaction,
 changes to the main body of the disk should depend on a
 journal commit record that has not yet been created.
%
\Kudos\ makes these \patches\ depend on an \noop\ \patch\
 that is explicitly held in memory.
%
Once the commit record is created, the \noop\ \patch\ is updated to depend on
 the actual commit record and then released.
%
The \noop\ \patch\ automatically commits at the same time as the commit
 record, allowing the main file system changes to follow.
%
%
\Noop\ \patches\ can shrink memory usage by representing quadratic sets of
 dependencies with a linear number of edges: if all $m$ \patches\ in $Q$
 must depend on all $n$ \patches\ in $P$, one could
 %% add $mn$ direct dependencies $p_i \PDDepend q_j$ or
 add an \noop\ \patch\ $e$ and $m+n$ direct dependencies
 $q_i \PDDepend e$ and $e \PDDepend p_j$.
%
This is useful for \patchgroups; see Section~\ref{sec:patchgroup}.
%
However, extensive use of \noop\ \patches\ adds to system time by
 requiring that functions traverse \noop\ \patch\ layers to find true
 dependencies.
%
Our implementation uses \noop\ \patches\ infrequently, and in the
 rest of this paper, patches are nonempty unless explicitly
 stated.


\begin{comment}
To solve this problem, we introduce an additional type of \patch. The
prototypical \patch\ corresponds to some change on disk, but \Kudos\ also
supports \aemphnoop\ \patch\ type, which doesn't change the disk at all.
\Noop\ \patches\ can have \befores, like other \patches, but they don't need to
be written to disk: they are trivially satisfied when all of their \befores\ are
satisfied. Thus, they can be used to ``stand for'' entire sets of other changes.
%
This capability is extremely useful, and is used by most operations on disk
structures so that a single \patch\ can be returned that depends on the whole
change. Likewise, \anoop\ \patch\ can be passed in as a parameter to a disk
operation to make the whole operation depend on a set of other changes. \Noop\
\patches\ allow dependencies between sets with only a linear number of
dependency edges in the \patch\ graph, and without having to pass around arrays
of \patches.
%
The cost is that some functions may have to traverse trees of \noop\ \patches\
to determine true dependencies.
\end{comment}
