\subsection{\ChDesc\ Merging}
\label{sec:patch:merge}

The large number of small, distinct changes made for cases such as
block allocations, inode updates, and directory updates use
significant memory for their \chdesc\ structures and dependencies and
increase the number of \chdescs\ the system processes.
%
\Kudos\ drastically reduces the number of such \chdescs, and thus
\chdesc\ memory usage, by conservatively identifying when a new and an
existing \chdesc\ pair can be written to disk together, which \Kudos\
%
\emph{merges} together into the existing \chdesc\ (updating its disk
change and dependencies) instead of creating the new \chdesc.
%
In this section we present three \chdesc\ merge algorithms. All three
use Invariant~\ref{cdinvar:add-before} to reason about future graph
changes and use fast, conservative checks during \chdesc\ creation; they
differ in their applicable conditions.

\subsubsection{\Nrb\ \ChDesc\ Merging}
\label{sec:chdescs:merge:nrb}

Recall from \S\ref{sec:chdescs:nrb} that a write of any \chdescs\ on block
$B$ must include all \nrb\ \chdescs\ on $B$.
%
This additional requirement is in fact an exquisite optimization
opportunity; it implies that all \nrb\ \chdescs\ on a given block can
be merged.
%
Further, merging can remove the need for the \nrb\ \chdesc\ implicit
dependency rules by ensuring that
%
there is at most one \nrb\ \chdesc\ per block (\nrb-\nrb\ merging)
%
and that all \rb\ \chdescs\ on a given block depend on the \nrb\ \chdesc\
(\nrb-\rb\ merging).
%
We describe these two \chdesc\ merging algorithms and how they
preserve dependency semantics in this section.

\paragraph{\Nrb-\Nrb\ \ChDesc\ Merging}
\label{sec:chdescs:merge:nrb:hard-hard}

\emph{\Nrb-\nrb\ \chdesc\ merging} merges a new \nrb\ \chdesc\ \p{q}
into an existing \nrb\ \chdesc\ \p{p} instead of creating \p{q}.
%
Any two \nrb\ \chdescs\ on the same block may be (and are) merged.
%
Merging all \nrb\ \chdescs\ ensures:
%
\cdinvar{one-nrb}{\(\forall\! B\!: |\PHard[B]| \leq 1\)}
%
\noindent
%
Invariant~\ref{cdinvar:one-nrb} simplifies \nrb\ \chdesc\ handling by
%
removing the implicit dependencies that ensure all \nrb\ \chdescs\
are written together
%
and by removing the need to scan for an existing \nrb\ \chdesc\ when
\nrb-\nrb\ \chdesc\ merging.
%
% Although merging two \chdescs\ will not induce block-level dependency
% cycles, without sufficient care merging could induce \chdesc-level
% dependency cycles.  A trivial example is merging \p{q} into \p{p} when
% \p{q} has an explicit dependency on \p{p}; the combined \p{(p+q)}
% should not and need not depend on itself.
%
To preserve dependency semantics, the merged \p{(p+q)} must depend on
the union of \p{p} and \p{q}'s transitive \befores. Additionally, while the
\chdescs\ can be merged without forming a \chdesc-level dependency cycle,
the merge must ensure that it does not introduce a needless cycle, e.g.
through \anoop\ \chdesc\ \p{e} in \depends{q}{\depends{e}{p}}
\todo{Is cycle avoidance worth mentioning? Is this a good way to mention it?}.

From Invariant~\ref{cdinvar:add-before} and the \nrb\ \chdesc\
creation condition (no external \afters), the only possible
dependencies involving \p{p} and \p{q} are those shown in
Figure~\ref{fig:nrb-merge}\todo{Should we give these deductions or a
  flavor?}.
%
Notice, for example, that any \p{r} such that
\indirdepends{q}{\indirdepends{r}{p}} is \anoop\ \chdesc\todo{This is
  a strong statement. Expand on its implications?}.
%
Our algorithm to transform dependencies for \nrb-\nrb\ \chdesc\ merges
(Figure~\ref{algo:merge:hard-hard}) follows from these possible
dependencies.
%
It updates \p{p}'s transitive \befores\ to ensure
\(\PDepset{q}\todo{Incorrect! Only the non-\noop{}s.} \subseteq
\PDepset{p}\)\todo{Note that Invariant~\ref{cdinvar:add-before}
  ensures that \noop\ \chdescs\ reachable from \p{q} will not gain
  data \chdesc\ \befores?}\todo{Note that it only needs to move dependencies?}.

\begin{figure}[htb]
  \centering
  \includegraphics[width=\columnwidth]{nrb_merge}
  \caption{Possible dependencies when merging \nrb\ \chdesc\ \p{q}
    into existing \nrb\ \chdesc\ \p{p}.}
  \label{fig:nrb-merge}
\end{figure}

\noindent Algorithm called on \p{q} and \p{p}:\\
Input: \chdesc\ \p{a} and existing \nrb\ \chdesc\ \p{p}.\\
Returns: whether \indirdepends{a}{p} exists. \(\forall\! \p{b}\!: \indirdepends{a}{b}\) and \notindirdepends{b}{p}, creates \indirdepends{p}{b}.

\begin{itemize}
\item If \p{a} is external, return ``no path to \p{p}.''
\item If \p{a} equals \p{p}, return ``path to \p{p}.''
\item Call self on \p{a} and \p{p}.
\item If \p{a} has no path to \p{p}, return ``no path to \p{p}.''
\item For each \p{a} \before\ \p{b}:
  \begin{itemize}
  \item If \p{b} has no path to \p{p}:
    \begin{itemize}
    \item Move \p{b} from a \before\ of \p{a} to a \before\ of \p{p}.
    \end{itemize}
  \end{itemize}
\end{itemize}

\paragraph{\Nrb-\Rb\ \ChDesc\ Merging}
\label{sec:chdescs:merge:nrb:hard-soft}

When creating the first \nrb\ \chdesc\ on a block, \emph{\nrb-\rb\
  \chdesc\ merging} merges all existing (\rb) \chdescs\ into the new
\nrb\ \chdesc.
%
Such an arrangement can arise through a combination of \chdesc\ creates
and block writes; 
%
for example, the block may first obtain an initial (\nrb{}) \chdesc,
%
then gain external \afters\ on its \chdesc,
%
accumulate additional (\rb{}) \chdescs,
%
write the subset of its \chdescs\ with external \afters\ (leaving some
\rb\ \chdescs\ on the block),
%
and then gain a \nrb\ \chdesc.
%
In addition to reducing the number of data \chdescs, \nrb-\rb\
\chdesc\ merging removes the second implicit \nrb\ \chdesc\
dependency, that \rb\ \chdescs\ not explicitly dependent on the
block's \nrb\ \chdesc\ implicitly depend on it.
%
As in \nrb-\nrb\ \chdesc\ merging, \Kudos\ merges such \chdescs\ to
avoid the complications of their implicit dependencies.

\Nrb-\rb\ \chdesc\ merging's implementation first merges all \rb\ \chdescs\
into a \nrb\ \chdesc\ and then \nrb-\nrb\ \chdesc\ merges the new \nrb\
\chdesc\ into the now-existing \nrb\ \chdesc.
%
Our algorithm to transform the dependencies for \nrb-\rb\ \chdesc\
merges (Figure~\ref{algo:merge:hard-soft}) for block $B$
%
chooses a \chdesc\ \p{p} such that
\(\notexists \inset{q}{\PMem[B]}\!: \indirdepends{p}{q}\)
%
and updates its transitive \befores\ to ensure
\(\PDepset{\PSoft[B]} \subseteq \PDepset{p}\).
%
Because any \(\inset{q}{\PMem[B] - p}\) may have \afters, to
preserve dependencies we convert such a \p{q} into \anoop\ \chdesc\
and ensure \depends{q}{p}.

\noindent Algorithm:
\begin{itemize}
\item Choose a \(\inset{p}{\PMem[B]}\!:\
\notexists\! \inset{q}{\PMem[B]}\!:\ \indirdepends{p}{q}\).
\item For each \inset{q}{\PMem[B] - p}:
  \begin{itemize}
  \item Call the \nrb-\nrb\ \before\ move algorithm on \p{q} and \p{p}.
  \item Convert \p{q} into \anoop\ \chdesc.
  \end{itemize}
\item Convert \p{p} into a \nrb\ \chdesc\ (free it's previous data copy).
\end{itemize}

\todo{Note that \nrb-\rb\ merging is rare? Note why it is helpful even though
it is rare?}
\todo{Explain why this preserves dependency semantics? Show possible
dependencies? For the paper, free \chdescs\ instead of convert them
into \noop{}s? (Must modify \nrb-\nrb\ algo usage.)}

\subsubsection{Overlap \ChDesc\ Merging}
\label{sec:chdescs:merge:overlap}
\todo{Note as useful when new may need to be rolled back.}
Bitmap blocks and inode size fields accumulate many nearby and
overlapping mergeable \chdescs\ as data is appended to or truncated
from a file.
%
Many of these and similar \chdescs\ are mergeable and have
dependencies that allow simple (and fast) reasoning to identify many
of the mergeable pairs: two \chdescs\ on block $B$ that overlap no other \chdescs\ in \PMem[B]
and which have no dependency path from the new to the existing \chdesc\
will not induce a block-level cycle and so are writeable together.
We know that \textit{later} changes will not cause them to induce a block-level cycle due to
invariant~\ref{cdinvar:add-before} and by not merging if the new \chdesc\
has a before and the before is marked as allowed to violate
invariant~\ref{cdinvar:add-before}.
%
While path existence testing is expensive, a conservative path test
of only a depth of two identifies most mergeable \chdescs. If the new
\chdesc\ has an explicit \before\ that is not the existing \chdesc\ and
this \before\ has a \before, then there may be a path to the existing
\chdesc.
%
To merge two such overlapping \chdescs, add the new \chdesc's explicit
before to the existing \chdesc\ (if any and if not the existing \chdesc).


%%

At the end of \chdesc\ optimizations, say something along the lines:
%
The dynamic optimizations facilitated through \nrb\
\chdescs\ implement the efficiency in systems using soft updates or
journaling\todo{Actually do this for journaling} while expressing
changes modularly through structural descriptions rather than through
internal and semantic file system descriptions.

\todo{Should we talk about why we allow NRBs and merging to be
  disabled? (Debugging simplicity and depend add to \noop\ \chdescs\
  with \afters\ bug catching.)}
