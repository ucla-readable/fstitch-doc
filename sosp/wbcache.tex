\subsection{Write-Back Cache}
\label{sec:modules:wbcache}

Write-back cache \modules\ act as the system's buffer cache, and are responsible
for holding on to the \chdescs\ sent to them until they can be safely sent on
towards the disk. There can be many write-back caches in a configuration at once
(for instance, one for each block device). To a write-back cache, the complex
consistency protocols that other \modules\ want to enforce are nothing more than
sets of dependencies among \chdescs\ -- it has no idea what consistency
protocols it is implementing, if any at all.
% Yet it is the \module\ that ends up doing most of the work to make sure that
% \chdescs\ are written in an acceptable order.

% Despite this, write-back caches are relatively simple; they are only slightly
% more complex than the naive implementation already suggested.
The current write-back cache \module\ is little more than just a front end to
the automatically-maintained ready \chdesc\ lists described in
Section~\ref{sec:patch:readylist}.
%
It uses a FIFO policy for writing dirty blocks, and an LRU policy for evicting
clean blocks (upon being written, a dirty block becomes clean and may then be
evicted).
%
The FIFO policy for writing blocks is really only a heuristic to help the
write-back cache find a block which can be written, however: only ready
\chdescs\ may be written, in order to uphold the in-flight safety property and
thus the disk safety property (\S\ref{sec:patch:dependencies}).
%
Once the write-back cache finds a block with ready \chdescs, all other \chdescs\
on that block are rolled back, the resulting block data is sent to the disk
driver, the ready \chdescs\ are marked \PInfst, and the rolled-back \chdescs\
are rolled forward again. The block itself is also marked \PInfst, so that only
one version of its data will be in flight at a time. (This whole procedure is
basically the buffer cache \textit{Write block} action.)

The simplicity of this algorithm for finding blocks to write is only tolerable
at all because the I/O requests are merged and reordered in the Linux elevator
scheduler. The write-back cache does have one special case meant to help the
elevator scheduler: after writing block $n$, it will check to see if block $n+1$
can be written as well, and continue writing increasing block numbers until some
block is either unwritable or not in the cache. This helps significantly, but
nevertheless we suspect that the simple algorithm here contributes to some of
the I/O delay \Kudos\ incurs in many tests.
