\subsection{Write-Back Cache}
\label{sec:using:wbcache}

Write-back cache \modules\ act as the system's primary cache, and are responsible
for holding on to the \chdescs\ sent to them until they can be safely sent on
towards the disk. There can be many write-back caches in a configuration at once
(for instance, one for each block device). To a write-back cache, the complex
consistency protocols that other \modules\ want to enforce are nothing more than
sets of dependencies among \chdescs\ -- it has no idea what consistency
protocols it is implementing, if any at all. Yet it is the \module\ that ends up
doing most of the work to make sure that \chdescs\ are written in an acceptable
order.

Despite this, write-back caches are relatively simple -- only about 630 lines of
code, including comments. The current write-back cache uses a simple FIFO policy
for writing dirty blocks, and an LRU policy for evicting clean blocks (upon
being written, a dirty block becomes clean and may then be evicted). However,
blocks can't be written unless all the \chdescs\ on them are ready to be sent to
the next \module\ (for example, the disk). A \chdesc\ is ready if all of its
\befores\ are at least as close to the disk as it is. So, when looking for a
block to write, the cache may not be able to write any block it chooses -- but
it writes the oldest dirty block that it can.

The write-back cache may, however, need to write a block which will stay dirty
even after it is written. For instance, \chdescs\ may be arranged in a way that
creates a dependency cycle at the level of blocks, even though the \chdesc\
dependencies themselves are acyclic. The same solution as soft updates uses is
employed to break the cycle: the write-back cache just holds on to the \chdescs\
that cannot yet be written, and forwards the others on to the next \module\ as
it writes the block. It cannot mark the block clean or evict it yet, since it is
still dirty, but progress has been made that can make other blocks writable.

The write-back cache has one other property that makes it useful in \Kudos: it
respects dependencies between one cache and another, so that (for instance)
dependencies between the changes on a file system and its external journal are
properly respected. This is actually not a special case, nor does it require any
extra code -- it is a property that just falls out of the way \chdesc\ graphs
are processed in order to determine which \chdescs\ are ready to move towards
the disk. This property also extends to \opgroups, which
are explained in \S\ref{sec:opgroup}.
