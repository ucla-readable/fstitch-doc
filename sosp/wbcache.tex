% -*- mode: latex; tex-main-file: "paper.tex" -*-

\subsection{Buffer Cache}
\label{sec:modules:wbcache}

The \Kudos\ buffer cache both
%
caches blocks in memory and
ensures that modifications are written to stable storage in a safe order.
%
Modules ``below'' the buffer cache---that is, between its output interface
and the disk---can reorder block writes at will without violating
dependencies, since block writes below the buffer cache contain only
in-flight patches.
%
%% There can be many write-back caches in a configuration at once (for
%% instance, one for each block device). 
%
The buffer cache sees the complex
consistency protocols that other \modules\ want to enforce as nothing more
than sets of dependencies among \patches; it has no idea what consistency
protocols it is implementing, if any.
% Yet it is the \module\ that ends up doing most of the work to make sure that
% \patches\ are written in an acceptable order.

% Despite this, write-back caches are relatively simple; they are only slightly
% more complex than the naive implementation already suggested.
Our prototype buffer cache \module\ 
%% is little more than a front end to
%% the automatically-maintained ready \patch\ lists described in
%% Section~\ref{sec:patch:readylist}.
%
uses a modified FIFO policy to write dirty blocks and an LRU policy to
evict clean blocks.  (Upon being written, a dirty block becomes clean and
may then be evicted.)
%
The FIFO policy used to write blocks is modified only to preserve the
in-flight safety property: a block will not be written if none of its
\patches\ are ready to write.
%
Once the cache finds a block with ready \patches, it extracts all ready
patches $P$ from the block, reverts any remaining \patches\ on that block,
and sends the resulting data to the disk driver.  The ready \patches\ are
marked in-flight and will be committed when the disk driver acknowledges
the write.
%% \patches\ on that block are reverted, the resulting block data is copied
%% and sent to the disk driver, the ready \patches\ are marked in-flight, and
%% the reverted \patches\ are re-applied.
%
The block itself is also marked in flight until the current version
commits, ensuring that the cache will wait until then to write the block
again.


As a performance heuristic, when the cache finds a writable block $n$, it
then checks to see if block $n+1$ can be written as
well.
%
It continues writing increasing block numbers until some block is either
unwritable or not in the cache.
%
\begin{comment}
The block itself is also marked \PInfst, so that only
one version of its data will be in flight at a time. (This whole procedure is
basically the buffer cache \textit{Write block} action.)
\end{comment}
%
This simple optimization greatly improves I/O wait time, since the I/O
requests are merged and reordered in Linux's elevator scheduler.
%
Nevertheless, there may still be important opportunities for further
optimization: for example, since the cache will write a block even if only
one of its \patches\ is ready, it can choose to revert \patches\
unnecessarily when a different order would have required fewer writes.


\begin{comment}
Each \patch\ on a cached block may or may not be visible to a given \module.
For example, \modules\ that respond to user requests generally view the most
current state of every block -- the block with all \patches\ applied. However, a
write-back cache may choose to write some \patches\ on a block while reverting
others, since those others currently have outstanding dependencies. In this
case, \modules\ below the write-back cache (i.e. closer to the disk) should view
those \patches\ in the reverted state. \Kudos\ provides a block revisioning
library function that automatically reverts those \patches\ that should not
be visible at a particular \module, and then re-applies them after that
\module\ is done with the block.
\end{comment}
