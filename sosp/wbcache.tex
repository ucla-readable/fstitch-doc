\subsection{Write-Back Cache}
\label{sec:using:wbcache}

Write-back cache \modules\ act as the system's buffer cache, and are responsible
for holding on to the \chdescs\ sent to them until they can be safely sent on
towards the disk. There can be many write-back caches in a configuration at once
(for instance, one for each block device). To a write-back cache, the complex
consistency protocols that other \modules\ want to enforce are nothing more than
sets of dependencies among \chdescs\ -- it has no idea what consistency
protocols it is implementing, if any at all. Yet it is the \module\ that ends up
doing most of the work to make sure that \chdescs\ are written in an acceptable
order.

Despite this, write-back caches are relatively simple; they are only slightly
more complex than the naive implementation suggested in
Section~\ref{sec:patch:optimizations}. The current write-back cache uses a FIFO
policy for writing dirty blocks, and an LRU policy for evicting clean blocks
(upon being written, a dirty block becomes clean and may then be evicted).
%
The FIFO policy for writing blocks is really only a heuristic to help the
write-back cache find a block which can be written, however, since the in-flight
safety property (\S\ref{sec:patch:dependencies}) must be maintained in order to
provide the disk safety property. When attempting to evict a block $b$, the
write-back cache finds the maximal set of \chdescs\ $P \subseteq \PMem[b]$
satisfying
%
$\PDepset{P} \subseteq P \cup \PDisk$ and $\PHard[b] \subseteq P$.
%
This is efficient, because this $P$ is the ready \chdesc\ list discussed in
Section~\ref{sec:patch:readylist}.
%
Assuming $P \neq \emptyset$, all other \chdescs\ on $b$ (i.e., $\PMem[b] - P$)
are rolled back, the resulting block data is sent to the disk driver, the
\chdescs\ in $P$ are marked \PInfst, and the rolled-back \chdescs\ are rolled
forward again. $b$ itself is also marked \PInfst, so that only one version of
its data will be in flight at a time. (This whole procedure is basically the
\textbf{Write block} action.)

The write-back cache also respects dependencies between one cache and another,
so that (for instance) dependencies between the changes on a file system and its
external journal are properly respected. This is actually not a special case,
nor does it require any extra code -- it is a property that just falls out of
the definition of $P$ above. This property also extends to \opgroups, which are
explained in Section~\ref{sec:opgroup}.
