\section {Evaluation}
\label{sec:evaluation}

We evaluate our prototype implementation of \Kudos\ in two ways: first, we
show that the overall performance is within reason, even though it is slower
than Linux or BSD by a nontrivial amount. We justify this difference, and argue
that it can be improved substantially. Second, we show specific block write
orderings in a variety of situations, demonstrating the effect and utility of
\opgroups\ and the potential for them to improve the efficiency of applications
using them.

\subsection {Optimization Benefits}

See Figures~\ref{f:optdata}
and~\ref{fig:mergereq-cdf}.
\todo{HP vs None untar results are not often near their average. Run
  larger number of tests.}

\begin{figure}
\small
\begin{tabular}{@{}lrrr@{}}
%& \multicolumn{3}{c}{\textbf{Untar}} &
%\multicolumn{3}{c@{}}{\textbf{Delete}} \\
\textbf{Optimization}
        & \textbf{\# patches} & \textbf{Rollback data} & \textbf{System time} \\
\hline
\multicolumn{4}{@{}c@{}}{\textbf{Untar test}\raise2pt\hbox{\strut}} \\
None
        & 537,989		& 458.64~MB             & 4.42~sec \\
Hard patches
        & 537,131               & 205.68~MB             & 4.53~sec \\
Overlap merging
        & 253,839               & 255.45~MB             & 3.58~sec \\
Both optimizations
        & 276,829               & 1.24~MB               & 3.58~sec \\
\hline
\multicolumn{4}{@{}c@{}}{\textbf{Rm test}\raise2pt\hbox{\strut}} \\
None
        & 193,062               & 3.17~MB               & 1.14~sec \\
Hard patches
        & 172,351               & 3.15~MB               & 1.08~sec \\
Overlap merging
        & 87,692                & 1.83~MB               & 0.85~sec \\
Both optimizations
        & 54,076                & 0.64~MB               & 0.75~sec \\
\end{tabular}
\caption{Effectiveness of \Kudos\ optimizations.  The combination of hard
patches and overlap merging removes almost half the patches and 99.73\% of
the rollback data required to untar Linux.}
\label{f:optdata}
\end{figure}


\begin{comment}
\begin{figure}[htb]
\centering
\includegraphics[width=\columnwidth]{opts-patches}
\includegraphics[width=\columnwidth]{opts-rollback}
\begin{tabular}{|l|r|r|} \hline
Optimization & Untar (sys sec) & Delete (sys sec) \\ \hline\hline
None & 4.42 & 1.14 \\ \hline\hline
\Nrb\ \Chdescs{} & 4.53 & 1.08 \\ \hline
Overlap Merge & 3.58 & 0.85 \\ \hline\hline
HP $+$ OM & 3.58 & 0.75 \\ \hline
\end{tabular}
\caption{Effects of \nrb\ \chdescs\ with \nrb\ merging and overlap
merging for the untar and rm tests.}
\label{fig:opts}
\end{figure}
\end{comment}

\begin{comment}
\begin{figure}[htb]
\vspace{-0.5\baselineskip}
\centering{
\includegraphics[width=\hsize]{rb_chdesc_size}
}
\vspace{-0.5\baselineskip}
\caption{\label{fig:patchsize-histo} \Rb\ \chdesc\ size histogram for a sample
workload (extracting a large archive into ext2). All the \chdescs\ larger than
63 bytes have been optimized into \nrb\ \chdescs. \Rb\ \chdescs\ 4 bytes and
smaller account for about 51\% of all \rb\ \chdescs.}
\end{figure}
\end{comment}

\subsection {Microbenchmarks}
We have also looked at the number of block writes that \Kudos\ makes
relative to other systems for several operations. In one test, we
create 100 small files in a directory and measure the number of blocks
written to disk. We do this with and without soft updates. \Kudos\
writes 122 blocks, while FreeBSD writes 135 blocks.

Another test pits Linux ext2 and ext3 against \Kudos\ ext2 with soft updates,
when extracting a large archive. Linux ext2 writes 523248 blocks; Linux ext3
writes 523192; \Kudos\ writes 540472.

We also look at a very small benchmark: removing a file from a
directory and adding a new file to it. We compare the number of blocks
written by \Kudos\ with that of FreeBSD using soft updates. Here
\Kudos\ writes 15 blocks, while FreeBSD writes 83 blocks.

We ran the two small benchmarks with the journal module, in full
data journaling mode. Creating 100 small files with the journal
resulted in 245 blocks written to the disk, but only 126 blocks
without the journal. Deleting and adding a file resulted in writing
31 with the journal module and 18 without it.

\begin{figure}[htb]
\vspace{-0.5\baselineskip}
\centering{
\includegraphics[width=\hsize]{merge_cdf}
}
\vspace{-0.5\baselineskip}
\caption{\label{fig:mergereq-cdf} CDF of disk write request sizes for
  Linux ext2 and \Kudos\ ext2 during a sample workload (extracting a large
  archive into ext2). Linux totals 2021 requests to the disk, while \Kudos\
  totals 4382 requests.}
\end{figure}

\subsection {Macrobenchmarks}

Our first benchmark test is to untar the Linux 2.6.15 source code from
\texttt{linux-2.6.15.tar}, which is already decompressed and cached in RAM. We
did this test on identical machines running Linux 2.6.20.1 (using ext2 and
ext3), FreeBSD 5.4 (using UFS, with and without soft updates), and \Kudos\
running as a Linux 2.6.20.1 kernel module (using ext2, with and without soft
updates). As a second test, we then delete the resulting source tree. The times
are listed in Figure~\ref{fig:macro} and include time to fully sync the changes
to disk.\todo{the BSD rm times are suspicious; did we umount/remount for them?}

% make this into a table
\begin{figure}[htb]
\centering{
\begin{tabular}{|l|r|r|} \hline
System & Untar (sec) & Delete (sec) \\ \hline\hline
\Kudos\ (ext2 SU) & 16.05 & 6.03 \\ \hline
\Kudos\ (ext2 async) & 5.27 & 2.99 \\ \hline\hline
FreeBSD 5.4 (SU) & 20.47 & 4.71 \\ \hline
FreeBSD 5.4 (async) & 9.63 & 3.72 \\ \hline\hline
Linux 2.6.20.1 (ext2) & 6.05 & 0.44 \\ \hline
Linux 2.6.20.1 (ext3) & 12.02 & 0.37 \\ \hline
\end{tabular}
}
\caption{\label{fig:macro} Untar/delete times for \Kudos, FreeBSD, and Linux,
with soft updates (SU) and with async mode.}
\end{figure}

\begin{figure}[htb]
\centering
\begin{tabular}{|l|r|} \hline
System & Time (sec) \\ \hline\hline
\Kudos\ (ext2) & 46.6 \\ \hline % r3266: 46.173 46.555 46.495 46.479 47.371
Linux 2.6.20.1 (ext2) & 16.9 \\ \hline % 10.602 16.983 17.033 16.905 16.798
Linux 2.6.20.1 (ext3) & 19.6 \\ \hline % 16.590 20.533 19.713 19.174 19.129
\end{tabular}
\caption{Postmark. TODO: describe and comment on performance.}
\label{fig:postmark}
\end{figure}

Surprisingly, \Kudos\ is about 22\% faster than FreeBSD using soft updates for
writes, but about 28\% slower for deletions.
%
When compared to Linux ext3, \Kudos\ is about 34\% slower for writes, but many
times slower for deletions.
%
This demonstrates that the overhead of using \chdescs\ in \Kudos\ is not
entirely unreasonable, but has room for significant improvement.
%
While \Kudos\ does still have some CPU usage and I/O delay problems, the its
performance has been steadily improving over time and we expect that it will
continue to improve as we optimize it further.
%
There are more optimization opportunities that can likely bring us much closer
in performance to Linux.

\subsubsection {Postmark}

\subsection {Patchgroups}

\subsubsection {Subversion Case Study}

Complexity of implementation, reliance on specific FS semantics (i.e. ext3)
even with all that jumping through hoops - broken on NTFS, for example.
Straightforward to get correctness with \opgroups; counts only on a simple API.

\subsubsection {UW IMAP Case Study}
\label{sec:evaluation:uwimap}
To assess the \opgroup-enabled UW IMAP mail server
(\S\ref{sec:opgroup:uwimap}) we compare the number of block writes
that \Kudos\ makes relative to FreeBSD to move 100 emails. The test
selects the source mailbox (with 100 messages, sized 2kB each),
creates the new mailbox, copies each of the 100 messages to the new
mailbox, marks each source message for deletion, expunges the marked
messages, requests a check, and logs out. We compare using soft updates;
Figure~\ref{fig:imap-compare} shows the results.

\begin{figure}[htb]
\centering{
\begin{tabular}{|l|r|} \hline
System & \# writes (blocks) \\ \hline\hline
\Kudos\ (with \opgroups) & 919 \\ \hline
\Kudos\ (w/o \opgroups) & 1537 \\ \hline
FreeBSD 5.4 & 2200 \\ \hline
\end{tabular}
}
\caption{\label{fig:imap-compare} Number of block writes to move 100
  IMAP messages.}
\end{figure}

\subsubsection {gzip}
