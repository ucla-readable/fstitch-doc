% -*- mode: latex; tex-main-file: "paper.tex" -*-

\section {Evaluation}
\label{sec:evaluation}
\label{eval}

We evaluate
%
the effectiveness of \patch\ optimizations,
%
the performance of the \Kudos\ implementation relative to Linux ext2
and ext3,
%
the correctness of the \Kudos\ implementation,
%
and the performance of \patchgroups.
%
This evaluation shows
%
that \patch\ optimizations significantly reduce \patch\ memory and CPU
requirements;
%
that a \Kudos\ \patch-based storage system has overall performance
competitive with Linux, though using up to 3.3 times more CPU time;
%
that \Kudos\ file systems are consistent after system crashes;
%
and that a \patchgroup-enabled IMAP server significantly outperforms
the unmodified server.

\subsection{Methodology}

All tests were run on a Dell Precision 380 with a 3.2~GHz Pentium 4
CPU (with hyperthreading disabled), 2~GB of RAM, and a Seagate ST3320620AS 320~GB 7200~RPM SATA2 disk.
%
Tests use a 10~GB file system and the Linux 2.6.20.1 kernel
with the Ubuntu v6.06.1 distribution.
%
Because \Kudos\ is not aware of high memory we disable it for all
configurations, limiting the computer to 912~MB of RAM.
%
Only the PostMark benchmark performs slower due to this cache size limitation.
%
All timing results are the mean over five runs.

To evaluate \patch\ optimizations and \Kudos\ as a whole we ran four
benchmarks.
%
The \emph{untar benchmark} untars and syncs the Linux 2.6.15 source code
from the cached file \texttt{linux-2.6.15.tar} (218~MB).
%
The \emph{delete benchmark}, after unmounting and remounting the file
system following the untar benchmark, deletes the result of the untar
benchmark and syncs.
%
The \emph{PostMark benchmark} emulates the small file workloads seen
on email and netnews servers~\cite{postmark}. We use PostMark v1.5,
configured to create 500 files ranging in size from 500~B to 4~MB;
perform 500 transactions consisting of file reads, writes, creates,
and deletes; delete its files; and finally sync.
%
The modified \emph{Andrew benchmark} emulates a software development
workload.  The benchmark creates a directory hierarchy, copies a
source tree, reads the extracted files, compiles the extracted files,
and syncs. The source code we use for the modified Andrew benchmark is
the Ion window manager, version 2-20040729.

\subsection {Optimization Benefits}

% this table is a command so that we can move its placement without conflicts
\newcommand{\opttable}{
\begin{figure}[t]
\centering
\small
\begin{tabular}{@{}lrrr@{}}
\textbf{Optimization}
        & \textbf{\# \Patches} & \textbf{Undo data} & \textbf{System time} \\ % Malloc Blocks
% Results are from r4121, mean of 5 runs
% Postmark none and hard were run with a 20k dblock cache
% NOTE: also update numbers in \patchoptundo and \patchoptcount
\hline
\multicolumn{4}{@{}c@{}}{\textbf{Untar}\raise2pt\hbox{\strut}} \\
None              &   619,740 &   459.41~MB &  3.33~sec \\ % 789MB 256MB
\Nrb\ \patches\   &   446,002 &   205.94~MB &  2.73~sec \\
Overlap merging   &   111,486 &   254.02~MB &  1.87~sec \\
Both              &    68,887 &     0.39~MB &  1.83~sec \\ % 296MB 256MB
\hline

\multicolumn{4}{@{}c@{}}{\textbf{Delete}\raise2pt\hbox{\strut}} \\
None              &   299,089 &     1.43~MB &  0.81~sec \\ % 50MB 11MB
\Nrb\ \patches\   &    41,113 &     0.91~MB &  0.24~sec \\
Overlap merging   &    54,665 &     0.93~MB &  0.31~sec \\
Both              &     1,800 &     0.00~MB &  0.15~sec \\ % 13MB 11MB
\hline

\multicolumn{4}{@{}c@{}}{\textbf{PostMark}\raise2pt\hbox{\strut}} \\
None              & 4,590,571 & 3,175.28~MB & 23.64~sec \\ % 5,607MB 2,286MB
\Nrb\ \patches\   & 2,544,198 & 1,582.94~MB & 18.62~sec \\
Overlap merging   &   550,442 & 1,590.27~MB & 12.88~sec \\
Both              &   675,308 &     0.11~MB & 11.05~sec \\ % 2,119MB 2,045MB
\hline

\multicolumn{4}{@{}c@{}}{\textbf{Andrew}\raise2pt\hbox{\strut}} \\
None              &    70,932 &    64.09~MB &  4.34~sec \\ % 108MB 39MB
\Nrb\ \patches\   &    50,769 &    36.18~MB &  4.32~sec \\
Overlap merging   &    12,449 &    27.90~MB &  4.20~sec \\
Both              &    10,418 &     0.04~MB &  4.07~sec \\ % 43MB 39MB
\end{tabular}
\caption{Effectiveness of \Kudos\ optimizations.}
\label{f:optdata}
\end{figure}
}

\opttable{}

We evaluate the effectiveness of the \patch\ optimizations discussed in
Section~\ref{sec:patch:optimizations} in terms of
%
the total number of \patches\ created, amount of undo data allocated,
and system CPU time used.
%
Figure~\ref{f:optdata} shows these results for the untar, delete,
PostMark, and Andrew benchmarks for \Kudos\ ext2 in soft updates mode,
with all combinations of using \nrb\ \patches\ and overlap merging.
%
The PostMark results for no optimizations and for just the \nrb\
\patches\ optimization use a smaller maximum \Kudos\ cache size,
80,000 blocks vs. 160,000 blocks, so that the benchmark does not run
our machine out of memory.
%
Optimization effectiveness is similar for journaling configurations.

Both optimizations work well alone, but their combination is particularly
effective at reducing the amount of undo data---which, again, is pure
overhead relative to conventional file systems.
%
Undo data memory usage is reduced by \patchoptundo,
%
the number of \patches\ created is reduced by \patchoptcount,
%
and system CPU time is reduced by up to 81\%.
%
These savings reduce \Kudos\ memory overhead
%
from 145--355\% of the memory allocated for block data
to 4--18\% of that memory, a 95--97\% reduction. For
example, \Kudos\ allocations are reduced from 3,321~MB to 74~MB for
the PostMark benchmark, which sees 2,165~MB of block
allocations.\footnote{Not all the remaining 74~MB is pure \Featherstitch\
overhead; for example, our ext2 implementation contains an inode cache.}

%% benchmark: kudos:block without and with optimizations
% untar: 533:256 to 40:256
% delete: 39:11 to 2:11
% PostMark: 3321:2286 to 74:2045
% Andrew: 69:39 to 4:39

\begin{comment}
\begin{figure}[t]
\vspace{-0.5\baselineskip}
\centering{
\includegraphics[width=\hsize]{rb_patch_size}
}
\vspace{-0.5\baselineskip}
\caption{\label{fig:patchsize-histo} \Rb\ \patch\ size histogram for a sample
workload (extracting a large archive into ext2). All the \patches\ larger than
63 bytes have been optimized into \nrb\ \patches. \Rb\ \patches\ 4 bytes and
smaller account for about 51\% of all \rb\ \patches.}
\end{figure}
\end{comment}

\subsection{Benchmarks and Linux Comparison}
\label{sec:eval:bench}

\newcommand{\safe}[1]{\textbf{#1}}
\newcommand{\unsafe}[1]{#1}

% this table is a command so that we can move its placement without conflicts
\newcommand{\benchtable}{
\begin{figure}[tb]
\centering
\small
\begin{tabular}{@{}l@{~~~~}r@{~~~~}r@{~~~~}r@{~~~~}r@{}}
%\textbf{System} & \multicolumn{1}{@{~~~}c@{~~~}}{\textbf{Untar}} &
%\multicolumn{1}{@{~~~}c@{~~~}}{\textbf{Delete}} &
%\multicolumn{1}{@{~~~}c@{~~~}}{\textbf{PostMark}} &
%\multicolumn{1}{@{~~~}c@{~~~}}{\textbf{Andrew}} \\ \hline
\textbf{System}
	& \textbf{Untar}\hfil\kern0pt
	& \textbf{Delete}\hfil\kern0pt
	& \textbf{PostMark}\hfil\kern0pt
	& \textbf{Andrew}\hfil\kern0pt \\ \hline
% Results are from r4121, mean of 5 runs
\multicolumn{5}{@{}l}{\emph{\Kudos\ ext2}\raise2pt\hbox{\strut}} \\
\safe{soft updates} & \safe{6.4 [1.3]} & \safe{0.8 [0.1]} & \safe{38.3 [10.3]} & \safe{36.9 [4.1]} \\
\safe{meta journal} & \safe{5.8 [1.3]} & \safe{1.4 [0.5]} & \safe{48.3 [14.5]} & \safe{36.7 [4.2]} \\
\safe{full journal} & \safe{11.5 [3.0]} & \safe{1.4 [0.5]} & \safe{82.8 [19.3]} & \safe{36.8 [4.2]} \\
\unsafe{async} & \unsafe{4.1 [1.2]} & \unsafe{0.7 [0.2]} & \unsafe{37.3 [~~6.1]} & \unsafe{36.4 [4.0]} \\
\unsafe{full journal} & \unsafe{10.4 [3.7]} & \unsafe{1.1 [0.5]} & \unsafe{74.8 [23.1]} & \unsafe{36.5 [4.2]} \\ \hline

\multicolumn{5}{@{}l}{\emph{Linux}\raise2pt\hbox{\strut}} \\
\safe{ext3 writeback} & \safe{16.6 [1.0]} & \safe{4.5 [0.3]} & \safe{38.2 [~~3.7]} & \safe{36.8 [4.1]} \\
\safe{ext3 full journal} & \safe{12.8 [1.1]} & \safe{4.6 [0.3]} & \safe{69.6 [~~4.5]} & \safe{38.2 [4.0]} \\
\unsafe{ext2} & \unsafe{4.4 [0.7]} & \unsafe{4.6 [0.1]} & \unsafe{35.7 [~~1.9]} & \unsafe{36.9 [4.0]} \\
\unsafe{ext3 full journal} & \unsafe{10.6 [1.1]} & \unsafe{4.4 [0.2]} & \unsafe{61.5 [~~4.5]} & \unsafe{37.2 [4.1]} \\

%FreeBSD (soft updates) & 23.22 & 15.95 & & \\ 
%FreeBSD (async) & 10.09 & 3.25 & & \\
\end{tabular}
\caption{\label{fig:bench_time} Benchmark times (seconds). System CPU
  times are in square brackets. Safe configurations are \safe{bold},
  unsafe configurations are \unsafe{normal text}.}
\end{figure}
}

We benchmark \Kudos\ and Linux for all four benchmarks, comparing the
effects of different consistency models and comparing patch-based with
non-patch-based implementations.
%% , with both soft updates and journaling dependencies,
%% and Linux, in several consistency modes, for the untar, delete,
%% PostMark, and Andrew benchmarks. 
%
Specifically, we examine Linux ext2 in asynchronous mode; ext3 in
writeback and full journal modes; and \Kudos\ ext2 in
asynchronous, soft updates, metadata journal, and full journal modes.  All
file systems were created with default configurations, and all journaled
file systems used a 64~MB journal.
%
Ext3 implements three different journaling modes, which provide different
consistency guarantees.
The strength of these guarantees is strictly ordered as
``writeback $<$ ordered $<$ full.''
Writeback journaling commits metadata atomically and commits data only
after the corresponding metadata. \Kudos\ metadata journaling is
equivalent to ext3 writeback journaling.
%
Ordered journaling commits data associated with a given transaction
prior to the following transaction's metadata, and is the most
commonly used ext3 journaling mode.
%
In all tests ext3 writeback and ordered journaling modes performed
similarly, and \Kudos\ does not implement ordered mode, so we report
only writeback results.
%
Full journaling commits data atomically.

There is a notable write durability difference between the default
\Kudos\ and Linux ext2/ext3 configurations: \Kudos\ safely presumes a write
is durable after it is on the disk platter, whereas Linux ext2 and
ext3 by default presume a write is durable once it reaches the disk cache.
However, Linux can write safely, by restricting the disk to providing only
a write-through cache, and \Kudos\ can write unsafely by disabling FUA.
%
We distinguish safe (FUA or write-through cache) from unsafe results
when comparing the systems.
%
Although safe results for \Kudos\ and Linux utilize different
mechanisms (FUA vs. a write-through cache), we note that \Kudos\
performs identically in these benchmarks when using either mechanism.

\benchtable{}


The results are listed in
Figure~\ref{fig:bench_time};
%
safe configurations are listed in bold.
%
In general, \Kudos\ performs comparably with Linux
ext2/ext3 when providing similar durability guarantees. Linux
ext2/ext3 sometimes outperforms \Kudos\ (for the PostMark test in
journaling modes), but more often \Kudos\ outperforms Linux.  There are
several possible reasons, including slight differences in block allocation
policy, but the main point is that \Kudos's general mechanism for
tracking dependencies does not significantly degrade total time.
%
Unfortunately, 
\Kudos\ can use up to four times more CPU time than Linux ext2 or
ext3. (\Kudos\ and Linux have similar
system time results for the Andrew benchmark, perhaps because Andrew
creates relatively few patches even in the unoptimized case.)
%
%% \Kudos\ is significantly slower than Linux ext3 at the PostMark
%% benchmark when the systems use journaling, where \Kudos\ uses up to
%% 3.3 times more CPU time than Linux ext3.
%
Higher CPU requirements are an important concern and, despite much progress
in our optimization efforts, remain a weakness.
%
Some of the contributors to \Kudos\ CPU usage are inherent, such as
\patch\ creation, while others are artifacts of the current
implementation, such as creating a second copy of a block to write it to
disk; we have not separated these categories.
%
%
\begin{comment}
Further, while \Kudos\ I/O times are lower than Linux ext2/ext3 I/O
times for the untar and delete benchmarks, we have found that small
block allocation strategy changes can significantly affect I/O time
for many of these benchmarks. This further emphasizes the importance
of the system CPU time difference.
\end{comment}

\begin{comment}
Unlike the untar, delete, and Andrew benchmarks, Linux ext3 writeback
and journal modes outperform \Kudos\ meta journal and full journal
modes, respectively, at PostMark.
\end{comment}

%This demonstrates that the overhead of using \patches\ in \Kudos\ is not
%entirely unreasonable, but has room for significant improvement.

\subsection {Correctness}
\label{sec:eval:correctness}

In order to check that we had implemented the journaling and soft updates
rules correctly,
we implemented a \Kudos\ module which crashes the operating system, without
giving it a chance to synchronize its buffers, at a random time during each
run.
%
In \Kudos\ asynchronous mode, after crashing, fsck nearly always reported
that the file system contained many references to inodes that had been
deleted, among other errors: the file system was corrupt.
%
With our soft updates dependencies, the file system was always soft updates
consistent: fsck reported that inode reference counts were higher than the
correct values (an expected discrepancy after a soft updates crash).
%
With journaling, fsck always reported that the file system was
consistent after the journal replay.

\subsection {\Patchgroups}
\label{sec:evaluation:uwimap}

% #reviewers who want measurements of all case studies: 1
% #reviewers who want measurements of at least svn and imap studies: 2

% this table is a command so that we can move its placement without conflicts
\newcommand{\imaptable}{
\begin{figure}[t]
\centering
\small
\begin{tabular}{@{}lrr@{}}
\textbf{Implementation} & \textbf{Time (sec)}\hfil\kern0pt & \textbf{\# Writes}\hfil\kern0pt \\ \hline

% Results are from r4121, mean of 5 runs
% Single commented out reuslts are from r3933
%% Double commented out results are from r3862

%\multicolumn{5}{@{}c@{}}{\textbf{Unsafe}\raise2pt\hbox{\strut}} \\

% WB
%fsync & Linux & ext2 & 1.5 [0.3] & 2,503 \\
%%Linux ext3 ordered (fsync) & 2.0 [0.3] & 3,025 \\
%fsync & Linux & ext3 journal & 1.8 [0.3] & 2,531 \\

% NOFUA
%\patchgroups & \Kudos & async & 1.2 [0.3] & 18 \\
%%\Kudos\ soft updates (pg) & 11.4 [0.4] & 3,015 \\
%%\Kudos\ meta journal (pg) & 1.4 [0.4] & 33 \\
%\patchgroups & \Kudos & full journal & 1.3 [0.5] & 33 \\ \hline

%\multicolumn{5}{@{}c@{}}{\textbf{Safe}\raise2pt\hbox{\strut}} \\

\multicolumn{3}{@{}l}{\emph{\Kudos\ ext2}\raise2pt\hbox{\strut}} \\

soft updates, fsync per operation & 65.2 [0.3] & 8,083 \\
%%\Kudos\ meta journal (pg) & 51.3 [0.4] & 7,111 \\
full journal, fsync per operation & 52.3 [0.4] & 7,114 \\

%%\Kudos\ async (pg) & 1.6 [0.3] & 2,503 \\
soft updates, \patchgroups & 28.0 [1.2] & 3,015 \\
%%\Kudos\ meta journal (pg) & 1.5 [0.4] & 32 \\
full journal, \patchgroups & 1.4 [0.4] & 32 \\ \hline

%linear & \Kudos & full journal & 1.4 [0.4] & 31 \\

\multicolumn{3}{@{}l}{\emph{Linux ext3}\raise2pt\hbox{\strut}} \\

%fsync & Linux & ext2 & 16.7 [0.3] & 2,503 \\
%%Linux ext3 ordered (fsync) & 23.9 [0.3] & 3,025 \\
full journal, fsync per operation & 19.9 [0.3] & 2,531 \\

full journal, fsync per durable operation & 1.3 [0.3] & 26 \\

\end{tabular}
\caption{\label{fig:imap-compare} IMAP benchmark: move 1,000 messages.
  System CPU times shown in square brackets.
  Writes are in number of requests.  All configurations are safe.}
\end{figure}
}

We evaluate the performance of the \patchgroup-enabled UW IMAP mail
server by benchmarking moving 1,000
messages from one folder to another.
%
To move the messages, the client selects the source mailbox (containing
1,000 2~kB messages), creates a new mailbox, copies each message to
the new mailbox and marks each source message for deletion, expunges
the marked messages, commits the mailboxes, and logs out.

Figure~\ref{fig:imap-compare} shows the results for safe file system
configurations,
%
reporting total time, system CPU time, and the number of disk write
requests (an indicator of the number of required seeks in safe
configurations).
%
We benchmark
%
\Kudos\ and Linux with the unmodified server (sync after each operation),
%
\Kudos\ with the \patchgroup-enabled server (\pgSync\ on durable
operations),
%
and Linux and \Kudos\ with the server modified to assume and take
advantage of fully journaled file systems (changes are effectively
committed in order, so sync only for durable operations).
%
Only safe configurations are listed; unsafe
configurations complete in about 1.5~seconds on either system.
%
\Kudos\ meta and full journal modes perform similarly; we report
only the full journal mode.
%
Linux ext3 writeback, ordered, and full journal modes also perform similarly;
we again report only the full journal mode.
%
Using an fsync per durable operation (\imapCheck\ and \imapExpunge) on
a fully journaled file system performs similarly for \Kudos\ and
Linux; we report the results only for Linux full journal mode.

In all cases \Kudos\ with \patchgroups\ performs better than \Kudos\ with
fsync operations.
%
Fully journaled \Kudos\ with \patchgroups\ performs at least as well
as all other (safe and unsafe) \Kudos\ and all Linux configurations, and is
11--13 times faster than safe Linux ext3 with the unmodified
server.
%
Soft updates dependencies are far slower than journaling for \patchgroups:
as the number of write requests
indicates, each \patchgroup\ on a soft updates file system requires
multiple write requests, such as to update the destination mailbox and
the destination mailbox's modification time. In contrast, journaling
is able to commit a large number of copies atomically using only a
small constant number of requests.
%
The unmodified fsync-based server generates dramatically more requests on
\Kudos\ with full journaling than Linux, possibly indicating a difference
in fsync behavior.
%
The last line of the table shows that synchronizing to disk once per
durable operation with a fully journaled file system performs similarly to
using \patchgroups\ on a journaled file system. However, \patchgroups\
have the advantage that they work equally safely, and efficiently, for
other forms of journaling.



\imaptable{}

With the addition of \patchgroups\ UW IMAP is able to perform mailbox
modifications significantly more efficiently, while preserving mailbox
modification safety. On a metadata or fully journaled file system, UW
IMAP with \patchgroups\ is 97\% faster at moving 1,000 messages than
the unmodified server achieves using fsync to ensure its write
ordering requirements.


\subsection{Summary}
\label{sec:evaluation:summary}

We find
%
that our optimizations greatly reduce system overheads, including
undo data and system CPU time;
%
that \Kudos\ obtains reasonable performance on several benchmarks,
comparable with Linux in most cases despite the additional effort required
to maintain patches;
%
that CPU time remains an optimization opportunity;
%
that applications can effectively define consistency requirements with
\patchgroups\ that apply to many file systems;
%
and that the \Kudos\ implementation correctly
implements soft updates and journaling consistency.
%
Our results indicate that even a patch-based prototype
can implement different consistency models with reasonable cost.
