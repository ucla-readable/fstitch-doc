\section {Evaluation}
\label{sec:evaluation}

We evaluate our prototype implementation of \Kudos\ in two ways: first, we
show that the overall performance is within reason, even though it is slower
than Linux or BSD by a nontrivial amount. We justify this difference, and argue
that it can be improved substantially. Second, we show specific block write
orderings in a variety of situations, demonstrating the effect and utility of
\opgroups\ and the potential for them to improve the efficiency of applications
using them.

\subsection {Overall Performance}

Our first benchmark test is to untar the Linux 2.6.15 source code from
\texttt{linux-2.6.15.tar}, which is already decompressed and cached in RAM. We
did this test on identical machines running Linux 2.6.20.1 (using ext3), FreeBSD
5.4 (using UFS, with and without soft updates), and \Kudos\ running as a Linux
2.6.20.1 kernel module (using UFS, with and without soft updates). As a second
test, we then delete the resulting source tree. The times below include time to
fully sync the changes to disk.

% make this into a table
\begin{figure}[htb]
\centering{
\begin{tabular}{|l|r|r|} \hline
System & Untar (sec) & Delete (sec) \\ \hline\hline
\Kudos\ (SU) & 24.50 & 11.69 \\ \hline
\Kudos\ (async) & 20.14 & 11.27 \\ \hline
FreeBSD 5.4 (SU) & 20.47 & 4.71 \\ \hline
FreeBSD 5.4 (async) & 9.63 & 3.72 \\ \hline\hline
% double line to separate linux
Linux 2.6.20.1 (journal) & 12.90 & 4.90 \\ \hline
\end{tabular}
}
\caption{\label{fig:macro} Untar/delete times for \Kudos, FreeBSD, and Linux,
with soft updates (SU) and with async mode.}
\end{figure}

\Kudos\ is about 20\% slower than FreeBSD using soft updates for
writes, and a little over a factor of two slower for deletions. This
shows that, although it is slow, \Kudos\ is close to running in
acceptable time. While we do have CPU usage and I/O delay problems, we
have only focused on performance for a few weeks\todo{and a year}, in which we were
able to speed up the system enormously.  There are more optimization
opportunities that can likely bring us very close in performance to
existing implementations of soft updates (see \S\ref{sec:discussion}).

\subsection {Block Writes}
We have also looked at the number of block writes that \Kudos\ makes
relative to other systems for several operations. In one test, we
create 100 small files in a directory and measure the number of blocks
written to disk. We do this with and without soft updates. \Kudos\
writes 122 blocks, while FreeBSD writes 135 blocks.

We also look at a very small benchmark: removing a file from a
directory and adding a new file to it. We compare the number of blocks
written by \Kudos\ with that of FreeBSD using soft updates. Here
\Kudos\ writes 15 blocks, while FreeBSD writes 83 blocks.

We ran the two small benchmarks with the journal module, which does
full data journaling. Creating 100 small files with the journal
resulted in 245 blocks written to the disk, but only 126 blocks
without the journal. Deleting and adding a file resulted in writing
31 with the journal module and 18 without it.

From this we gather that our UFS, soft updates, and journal
implementations are correct, and this reinforces our case that we need
to reduce CPU usage to be competitive.

\subsection {UW IMAP Case Study}
\label{sec:evaluation:uwimap}
To assess the \opgroup-enabled UW IMAP mail server
(\S\ref{sec:opgroup:uwimap}) we compare the number of block writes
that \Kudos\ makes relative to FreeBSD to move 100 emails. The test
selects the source mailbox (with 100 messages, sized 2kB each),
creates the new mailbox, copies each of the 100 messages to the new
mailbox, marks each source message for deletion, expunges the marked
messages, requests a check, and logs out. We compare using soft updates,
Figure~\ref{fig:imap-compare} shows the results.

\begin{figure}[htb]
\centering{
\begin{tabular}{|l|r|} \hline
System & \# writes (blocks) \\ \hline\hline
\Kudos\ (with \opgroups) & 919 \\ \hline
\Kudos\ (w/o \opgroups) & 1537 \\ \hline
FreeBSD 5.4 & 2200 \\ \hline
\end{tabular}
}
\caption{\label{fig:imap-compare} Number of block writes to move 100
  IMAP messages.}
\end{figure}
