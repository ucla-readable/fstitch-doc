% -*- mode: latex; tex-main-file: "paper.tex" -*-

\section{Patch Model}
\label{sec:patch}

\makeatletter
\let\emptyset\varnothing
\newcommand{\PState}[1]{\ensuremath{#1.\textit{state}}}
\newcommand{\PBlock}[1]{\ensuremath{#1.\textit{block}}}
\newcommand{\PMemst}{\ensuremath{\textit{mem}}}
\newcommand{\PInfst}{\ensuremath{\textit{flight}}}
\newcommand{\PDiskst}{\ensuremath{\textit{disk}}}
\newcommand{\PSetlim}[1]{\def\@next{#1}\ifx\@next\@empty\else[\@next]\fi}
\newcommand{\PMem}[1][]{\ensuremath{\textit{Mem}\PSetlim{#1}}}
\newcommand{\PInf}[1][]{\ensuremath{\textit{Flight}\PSetlim{#1}}}
\newcommand{\PDisk}[1][]{\ensuremath{\textit{Disk}\PSetlim{#1}}}
\newcommand{\PHard}[1][]{\ensuremath{\textit{\Nrb}\PSetlim{#1}}}
\newcommand{\PSoft}[1][]{\ensuremath{\textit{\Rb}\PSetlim{#1}}}
\newcommand{\PEmpty}[1][]{\ensuremath{\textit{\Noop}\PSetlim{#1}}}
\newcommand{\PDDepset}[1]{\ensuremath{\def\@next{#1}\ifx\@next\@empty\else\@next.\fi\textit{ddeps}}}
\newcommand{\PDepend}{\ensuremath{\leadsto}}
\newcommand{\PDDepend}{\ensuremath{\rightarrow}}
\newcommand{\PDepset}[1]{\ensuremath{\textit{Dep}[#1]}}
\newcommand{\PRDepset}[1]{\ensuremath{\textit{RDep}[#1]}}
\makeatother

Every change to stable storage in a \Kudos\ system is represented by a
\emph{patch}.
%
This section describes the basic patch abstraction using a semi-formal
notation that will be useful for analyzing our optimizations later.
%
Although the patch idea would apply to any stable medium, or to network
file systems, we use terms like ``disk'' and ``disk controller'' throughout
to simplify our terminology.
%
Our patch notation is summarized in Figure~\ref{f:patchnot}.

Each patch $p$ encapsulates four important pieces of state: its
 \emph{block}, its \emph{state}, a set of \emph{direct dependencies}, and
 some \emph{rollback data}.

Patch $p$'s \textbf{block}, written $\PBlock{p}$, is the unit of disk data
 to which $p$ applies.  The disk controller is assumed to write blocks as a
 unit (although not necessarily atomically; see below).  A file system
 change that affects $n$ blocks must be represented using at least $n$
 patches, since a patch by definition can touch only one block.

\begin{figure}
\begin{small}
\begin{tabular}{@{}ll@{}}
$\PBlock{p} = b$ & $p$'s block \\
$\PState{p}$ & $p$'s state, $\in \{\PMemst, \PInfst, \PDiskst\}$ \\
~~~~$\PMem$ & all in-memory patches: $\{p \mid \PState{p} = \PMemst\}$ \\
~~~~$\PMem[b]$ & $\{p \mid \PState{p} = \PMemst \text{ and } \PBlock{p} = b
 \}$ \\
~~~~$\PInf, \PDisk$ & similarly for in-flight and on-disk patches \\
$\PDDepset{p}$ & $p$'s direct dependencies \\
~~~~$p \PDDepend q$ & $p$ directly depends on $q$: $q \in \PDDepset{p}$ \\
~~~~$p \PDepend q$ & $p$ depends on $q$: either $p \PDDepend q$ or there exists \\
       & a patch $x$ so that $p \PDepend x \PDDepend q$ \\
~~~~$\PDepset{p}$ & $p$'s dependencies: $\{ q \mid p \PDepend q \}$ \\
\end{tabular}
\end{small}

\caption{Patch notation.}
\label{f:patchnot}
\end{figure}


Each patch is in one of three \textbf{states}: \emph{in memory}; \emph{in
 flight} to the disk controller, but not yet committed to disk; and
 \emph{on disk}.  The intermediate in-flight state is necessary because
 operating system software loses control of a patch upon writing its block
 to disk.  $p$'s state is written $\PState{p} \in \{\PMemst, \PInfst,
 \PDiskst\}$.  The operating system changes a patch's state from
 \PMemst\ to \PInfst\ by writing its block to the disk controller.  Some
 time later---after possible OS-level disk scheduling, transit over the
 system bus, and a period in the disk's cache---the disk writes the block
 to stable storage and reports success.  When the processor receives this
 notification, it changes the patch's state to \PDiskst.
%
The sets \PMem, \PInf, and \PDisk\ are defined to contain all patches with
 the given state, and the sets $\PMem[b]$, $\PInf[b]$, and $\PDisk[b]$ are
 defined to contain all patches with the given state \emph{on the given
 block} $b$.

Patch $p$'s \textbf{direct dependencies}, written $\PDDepset{p}$, is a set of
 other patches on which $p$ ``depends''.
%
That is, every patch in $\PDDepset{p}$ should be committed to disk either
 before, or at the same time as, $p$ itself.
%
Dependencies are set by the file system based on whatever consistency
 semantics it wants to ensure.
%
For example, a file system with asynchronous writes (and no durability
 guarantees) might write all patches with $\PDDepset{p} = \emptyset$;
%
a file system implementing soft updates would arrange the dependencies
 accordingly;
%
and a file system implementing journaling would write a different set of
 dependencies, where the journal commit record would depend on the journal
 data and the writes to the main body of the file system would in turn
 depend on the commit record.
%
Thus, an upper file system layer defines an initial set of
 dependencies.
%
The rest of \Kudos\ generally preserves this initial set, but can modify
 them as necessary to change or fix dependency semantics.
%
Finally, the buffer cache obeys the constraints they define.


We say $p$ \emph{directly depends on} $q$, and write $p \PDDepend q$, when
 $q \in \PDDepset{p}$.
%
We say $p$ \emph{depends on} $q$, and write $p \PDepend q$, when there
 exists a dependency chain from $p$ to $q$: that is, either $p \PDDepend q$
 or, for some patch $x$, $p \PDepend x \PDDepend q$.
%
Patch $p$'s set of \emph{dependencies} is written $\PDepset{p} = \{ q \mid
 p \PDepend q \}$; this is of course a superset of its direct dependencies.



\textbf{Rollback data.}
%
Soft updates-like consistency orderings may require that the buffer cache
\emph{not} write one or more \patches\ on some block.
%
In particular, a series of file system operations may create dependencies
that enforce a circular order among blocks, even though the dependencies
themselves do not form a cycle~\cite{ganger00soft}.
%
This is problematic since blocks with circular dependencies can never be
written: no block can be written first since each block depends on another.
%
For this reason, each \patch\ carries \emph{rollback} information that gives
the previous version of the data altered by the \patch.
%
If a \patch\ $p$ is not written with its containing block, the system rolls
back the \patch, which swaps the new data and the previous version.
%
Once the block is written, the system will roll the \patch\ forward and, when
allowed, write the block again, this time including the \patch.
%
Rollback information adds greatly to memory and CPU utilization, but it can
often be optimized away, as we show below.




\paragraph{Generalizations}

Levels of the file system





The disk's behavior is encapsulated in the following action,
which might happen at any time:

\begin{tabbing}
\quad \textbf{Commit block.} \\
\qquad Pick some block $b$. \\
\qquad For each $p \in \PInf[b]$, set $\PState{p} \gets \PDiskst$.
\end{tabbing}


We note that soft updates-like consistency protocols demand more from disks
than many journal protocols.
%
Neither type of protocol assumes that data is committed until the disk
reports success.
%
However, soft updates inherently assumes that blocks are written
\emph{atomically}, except in the case of catastrophic media error.
%
In particular, if the disk fails while a block $b$ is in flight, then $b$'s
value on recovery must equal either the old value or the new value.
%
Most journal designs also handle the case where in-flight blocks might be
corrupted on recovery---for instance, perhaps because the memory holding
the new value of the block lost its coherence before the disk stopped
writing~\cite{tso}.
%
However, some disks actually do provide an atomicity guarantee, for
instance by using non-volatile memory to store blocks before they make it
onto disk~\cite{???}.
%
The \Kudos\ core makes no assumptions about block atomicity, instead relying
on software above it to implement a consistency protocol that makes sense
for the given disk.




The \textbf{disk safety property} is simply written
%
\[ \PDepset{\PDisk} \subseteq \PDisk. \]
%
That is, the dependencies of all \patches\ on disk are always also on disk,
and
%
no matter when the system crashes, the disk is consistent in terms of
dependencies.
%
The file system's job is to set up dependencies so that the disk safety
property implies file system correctness.
%
(This might involve compensating for the possibility that in-flight blocks
can be corrupted on recovery.)


The storage system cannot control when the disk controller commits blocks;
instead, it controls when blocks are released to the controller.
%
It therefore maintains the disk safety property by enforcing, instead, the
\textbf{in-flight safety property}: For any block $b$ with $\PInf[b] \neq
\emptyset$,
%
\[ \PDepset{\PInf[b]} \subseteq \PInf[b] \cup \PDisk . \]
%
This, combined with the disk's ``commit block'' behavior above, implies the
disk safety property.
%
The buffer cache's job is to write \patches\ in an order that obeys the
in-flight safety property.
\todo{mention liveness property?}


The buffer cache's required behavior is then encapsulated in the following
action:

\begin{tabbing}
\textbf{Write block.} \\
\quad Pick some block $b$ with $\PMem[b] \neq \emptyset$. \\
\quad Pick some $P \subseteq \PMem[b]$ with $\PDepset{P} \subseteq P \cup
\PDisk$. \\
\quad For each $p \in P$, set $\PState{p} \gets \PInfst$. \\
\quad For each $p \in \PMem[b]-P$, set $\PDDepset{p} \gets \PDDepset{p}
\cup P$.
\end{tabbing}

\noindent
%
The last step forces any rolled-back \patches\ to wait in memory at least
until the current version of the block is written.  (Our implementation
achieves this property without using dependencies.)
%
It is easy to show that this action preserves the in-flight safety property
and, thus, the disk safety property.


% -*- mode: latex; tex-main-file: "paper.tex" -*-

\subsection{Implementation and \noop\ \chdescs}
\label{sec:patch:noop}

\Kudos\ file system implementations create \chdescs\ with one of the
following functions:

\begin{scriptsize}
\begin{alltt}
int \textbf{patch_create_byte}(
    bdesc_t *block, bdev_t *owner,
    uint16_t offset, uint16_t length,
    const void *data, patch_t **head);
int \textbf{patch_create_bit}(
    bdesc_t *block, bdev_t *owner,
    uint16_t offset, uint32_t xor,
    patch_t **head);
int \textbf{patch_create_\noop}(
    bdev_t *owner, patch_t **tail,
    size_t nheads, patch_t * heads[]);
int \textbf{patch_create_diff}(
    bdesc_t *block, bdev_t *owner,
    uint16_t offset, uint16_t length,
    const void *data, patch_t **head);
\end{alltt}
\end{scriptsize}


We first address dependency convenience and memory usage with \emph{\noop\
\chdescs}, which have no associated data or block.
%
This makes it just a means for tracking dependencies.


For example, imagine writing two files \texttt{before.txt}
The dependencies 
\Chdescs\ as so far described can be tedious and inefficient to manage when
dealing with large sets of them corresponding to file system operations. For
instance, if writing some file \texttt{\after.txt} is to depend on writing some
other file \texttt{\before.txt}, it will be inconvenient to keep arrays of all
the \chdescs\ corresponding to the two operations and inefficient to store the
potentially quadratic number of edges in the \chdesc\ graph.

To solve this problem, we introduce an additional type of \chdesc. The
prototypical \chdesc\ corresponds to some change on disk, but \Kudos\ also
supports \aemphnoop\ \chdesc\ type, which doesn't change the disk at all.
\Noop\ \chdescs\ can have \befores, like other \chdescs, but they don't need to
be written to disk: they are trivially satisfied when all of their \befores\ are
satisfied. Thus, they can be used to ``stand for'' entire sets of other changes.
%
This capability is extremely useful, and is used by most operations on disk
structures so that a single \chdesc\ can be returned that depends on the whole
change. Likewise, \anoop\ \chdesc\ can be passed in as a parameter to a disk
operation to make the whole operation depend on a set of other changes. \Noop\
\chdescs\ allow dependencies between sets with only a linear number of
dependency edges in the \chdesc\ graph, and without having to pass around arrays
of \chdescs.
%
The cost is that some functions may have to traverse trees of \noop\ \chdescs\
to determine true dependencies.

Modules can also use \noop\ \chdescs\ to \emph{prevent} changes from being
written. A \emph{managed} \noop\ \chdesc\ must be explicitly satisfied; any
changes that depend on that \noop\ \chdesc\ are delayed until the owning \module\
explicitly satisfies it. This is used, for instance, by the journal \module\
(\S\ref{sec:using:journal}) to prevent a transaction's \chdescs\ from
being written before the journal commits.

\Noop\ \chdescs\ are not included in our formal model of \chdescs\ for simplicity;
they add some additional complexity but do not change the basic ideas.
\todo{Well, they are now... the rules need updating since they have no blocks.}



\subsection{Challenges}
\label{sec:patch:challenges}

A naive implementation of this model scales extremely poorly,
particularly with cache size.
%
Challenges in a \patch-based file system implementation include:

\textbf{Buffer cache graph traversal.}
%
In order to evict and write a block, the buffer cache must choose a block
$b$,
%
and then find a set of \patches\ $P_b \subseteq \PMem[b]$ whose dependencies
satisfy a graph property, namely that $\PDepset{P_b} \subseteq P_b \cup
\PDisk$.
%
It usually makes sense to define $P_b$ maximally---that is, as the
\emph{largest} corresponding set of \patches.
%
In the ideal (and common) case $P_b = \PMem[b]$, which lets the cache reuse
$b$'s memory once $P_b$ is committed to disk.  However, in some cases there
may be no block for which $P_b = \PMem[b]$.
%
It would also be nice if the blocks chosen for writing also maximized the
disk's commit rate, by minimizing seeks and so forth.

A naive implementation might calculate, for each in-memory block $b$, the
largest set of \patches\ $P_b \subseteq \PMem[b]$ with $\PDepset{P_b}
\subseteq P_b \cup \PDisk$, then evict some block close to previously
written blocks and with few rolled-back \patches\ (where $\PMem[b] - P_b$
is small).
%
This, however, would be extraordinarily expensive.
%
Finding $P_b$ requires traversing a dependency graph which might contain
thousands and thousands of nodes.
%
Doing so for each block, once per eviction, would take huge amounts of CPU
time.


\textbf{Rollback memory usage.}
%
Only a small fraction of \patches\ will ever need to be rolled back.
%
For example, most data writes never need to be rolled back in any file
system.
%
If a \patch\ won't be rolled back under any circumstances, the memory and
CPU time spent to preserve the old version is wasted.


\textbf{\Patch\ memory usage.}
%
Patches themselves take up memory and require time to allocate, free, and
traverse.
%
If two \patches\ have redundant dependencies, it would be faster to combine
them.


%% \textbf{Dependency memory usage.}
%% %
%% The $\PDDepset{}$ sets are stored as doubly linked lists; each individual
%% dependency takes up memory.
%% %
%% Important and common dependency relations require many dependencies to
%% express; for example, if \patches\ $p_1,\dots,p_n$ depend, as a group, on
%% \patches\ $q_1,\dots,q_n$, expressing this constraint would require $n^2$
%% total dependencies.


The next section tackles all of these challenges.


\section{Patch Optimizations}

\begin{figure}
\centering

\includegraphics[scale=0.62]{fig/opt_1}

\textbf{a)} Before optimizations

\vskip.5\baselineskip

\includegraphics[scale=0.62]{fig/opt_2}

\textbf{b)} With \nrb\ \patches

\vskip.5\baselineskip

\includegraphics[scale=0.62]{fig/opt_3}

\textbf{c)} With \nrb--\nrb\ merging

\vskip.5\baselineskip

\includegraphics[scale=0.62]{fig/opt_4}

\textbf{d)} With overlap merging

\caption{\Patches\ required to append 4 blocks to an existing file, without
and with optimization.  \Nrb\ \patches\ are shown with heavy borders.}
\end{figure}

% -*- mode: latex; tex-main-file: "paper.tex" -*-

\subsection{\Nrb\ \Patches}
\label{sec:patch:nrb}

The first optimization reduces space overhead by
eliminating undo data.
%
When a \patch\ $p$ is created, \Kudos\ conservatively detects whether $p$
 might require reversion:
%
that is, whether any possible future patches and dependencies could force
 the buffer cache to undo $p$ before making further progress.
%
If no future patches and dependencies could force
 $p$'s reversion, then $p$ does not need undo data.
%
The challenge is to detect this condition without predicting the future.
%
We solve this challenge by restricting dependency creation.


We implemented support for \textbf{hard patches}, which are simply patches
 that lack undo data.
%
Since a hard patch $h$ cannot be reverted, any other patch on its block
 must depend on $h$ (the other patches can't be written without $h$).
%
We enforce this requirement, for example using overlap
 dependencies, and
%
as a result, the buffer cache will write a block's hard patches (if any)
 whenever it writes the block.
%
The system aims to reduce memory usage by making most patches hard.


But which patches can be made hard?
%
Define a \emph{block-level cycle} as a dependency chain of uncommitted
 patches $p_n \PDepend \cdots \PDepend p_1$ where the ends have the same
 block $\PBlock{p_n} = \PBlock{p_1}$, and at least one patch in the middle
 has a different block $\PBlock{p_i} \neq \PBlock{p_1}$.
%
The patch $p_n$ is called the \emph{head} of the block-level cycle.
%
Now assume that a patch $p \in \PMem$ is not the head of any block-level
 cycle.
%
One can then show that the buffer cache can always write a block without
 rolling back $p$.
%
The hard case is where $\PBlock{p}$ cannot itself be written without
 rolling back $p$, which occurs when $p$ has an uncommitted dependency $q$
 on a different block.
%
However, we know that $q$'s uncommitted dependencies, if any, are all on
 blocks other than $p$'s; otherwise there would be a block-level cycle.
%
Since \Featherstitch\ disallows circular dependencies, every
 chain of dependencies starting at $q$ has finite length, and therefore
 contains an uncommitted patch $x$ all of whose dependencies have
 been committed.
%
(If $x$ has in-flight dependencies, simply wait
 for the disk controller to commit them.)
%
Since $x$ is not on $p$'s block, the buffer cache can write $x$ without
 rolling back $p$.


\Featherstitch\ may thus make a patch hard when it can prove that patch
 will never be the head of a block-level cycle.
%
This requires two tricks.
%
First, an API restriction allows us to search for \emph{existing},
 rather than future, block-level cycles:
%
\emph{A \patch's direct dependencies are all supplied at creation time}.
%
After $p$ is created, the system can add new dependencies $q \PDDepend p$,
 but will never add new dependencies $p \PDDepend q$.\footnote{The actual
 rule is somewhat more flexible: modules may add new direct dependencies if
 they guarantee that those dependencies don't produce any new block-level
 cycles.  As one example, if no \patch\ depends on some \noop\ \patch\ $e$,
 then adding a new $e \PDDepend q$ dependency can't produce a cycle.}
%
Since every \patch\ follows this rule, all possible block-level cycles with
 head $p$ are present in the dependency graph when $p$ is created.
%
\Featherstitch\ must still check for these cycles, of course, and
%
actual graph traversals proved expensive.
%
We thus implemented a conservative approximation: \patch\ $p$ is
created as \nrb\ if \emph{no} patches on other blocks depend on uncommitted
 patches on $\PBlock{p}$.
%
That is, given any dependency between uncommitted patches $y \PDepend x$,
 either $\PBlock{x}$ isn't on $p$'s block or \emph{both} $x$ and $y$ are on
 $p$'s block.
%
If this holds, then $p$ cannot head a block-level cycle no matter its
 dependencies.
%
This heuristic works well in practice and, given some bookkeeping, 
 takes $O(1)$ time to check.


\begin{comment}
\Kudos\ further ensures that the dependency structure correctly
represents dependencies on the same block through overlap
dependencies: since \nrb\ \patches\ are considered to cover the entire
block, every succeeding \patch\ will overlap at least one \nrb\ \patch,
and \Kudos\ will automatically add a dependency.
%
(Some cases are handled by other optimizations.)


The buffer cache's ``write block'' behavior must account for \nrb\
\patches, as it \emph{must} write any \nrb\ \patches\ that exist on a
block.
%
Let $\PHard[b]$ be the set of \nrb\ \patches\ on block $b$.
%
Then to write block $b$, the buffer cache must choose some $P \subseteq
\PMem[b]$ with
%
\[ \PDepset{P} \subseteq P \cup \PDisk \text{ and } \PHard[b] \cap \PMem
\subseteq P. \]
%
If no such $P$ exists, then the cache must write a different block.
\end{comment}


Applying \nrb\ \patch\ rules to our example makes 14 of the 23 \patches\ \nrb\
(Figure~\ref{fig:opt}b),
%
reducing the undo data required by slightly more than half.


\begin{comment}
%
To avoid this overhead, \Kudos\ identifies \patches\ that will never
need to be reverted and omits their undo data. We call these \emph{\nrb}
\patches. (The opposite naturally being a \emph{\rb} \patch, when
necessary to differentiate them.)
%
Since a \nrb\ \patch\ cannot be reverted, a write of any \patches\
on block $\PB$ must include all \nrb\ \patches\ on $\PB$. To accordingly
update our formal model we define a new set of \patches, \PHard, which
contains all \nrb\ \patches. We write \PHard[\PB] to restrict the set
to block $\PB$\todo{Introduce \PSoft\ and \PSoft[\PB].}:

\begin{tabbing}
\textbf{Write block.} \\
\quad Pick some block $b$ with $\PMem[b] \neq \emptyset$. \\
\quad Pick some $P \subseteq \PMem[b]$ with $\PDepset{P} \subseteq P \cup
\PDisk$ and $\PHard[\PB] \subseteq P$. \\
\quad Move each $p \in P$ to $\PInf$ (in-flight). \\
\quad For each $p \in \PMem[\PB]-P$, set $\PDDepset{p} \gets \PDDepset{p}
\cup P$.
\end{tabbing}

\paragraph{}
To avoid (expensive) dependency traversals to determine whether a new
\patch\ will need to be reverted,
%
\Kudos\ conservatively identifies \nrb\ \patches\ using only local
dependency information.
%
\Kudos\ detects that a new \patch\ on block $\PB$ may need to be reverted if:
\todo{Which form is easier to read? Can we write \(\PMem - \PMem[\PB] - \PEmpty\) more concisely?}
%
\todo{Actually, our implementation also uses in flight \patches. Can we make
it not?}
%
\[ \PRDepset{\PMem[b]} \cap (\PMem - \PMem[b] - \PEmpty) \ne \emptyset \]
\[ \exists \inset{p}{\PMem[b]}\!:\
   \exists c\!:\ \exists \inset{q}{\PMem[c]}\!:\
   \indirdepends{q}{p} \]
%
This is both a safe and useful indicator because
%
the presence of an external \after\ is a necessary condition for a new
\patch's \before\ to induce a block-level cycle
%
and many blocks have no \patches\ with external \afters\ (e.g. most
file data blocks).

While this algorithm detects whether a \patch\ may need to be reverted,
\Kudos\ must also be sure that no future dependency manipulation
will cause the \patch\ to require being reverted.
%
We introduce Invariant~\ref{cdinvar:add-before} to support such reasoning:
%
\cdinvar{add-before}{All block-level cycles induced through
\patch\ \p{p}'s \befores\ exist when \p{p} is
created\todo{Change this phrasing? ``Once created, a \patch\ will not
gain any \befores\ that induce block-level cycles.''}.}
%
\noindent \Kudos\ ensures this invariant by restricting \before\
additions to \patch\ creation, \noop\ \patches\ with no \afters, or
when the invariant is statically proven to hold for the affected
\patches.
\end{comment}

% -*- mode: latex; tex-main-file: "paper.tex" -*-

\subsection{Ready \Patch\ Lists}
\label{sec:patch:readylist}

\newcommand{\PReady}[1]{\ensuremath{#1.\textit{ready}}}

Another important optimization greatly reduces CPU time spent in the
\Kudos\ buffer cache.
%
The buffer cache's main task is to choose sets of \patches\ $P$ that
satisfy the in-flight safety property $\PDepset{P} \subseteq P \cup
\PDisk$.
%
A naive implementation would simply traverse the dependency graph starting
at these patches, looking for problematic dependencies.
%
Patch merging can reduce the size of these traversals by combining patches
together.
%
Unfortunately, even modest traversals become painfully slow when executed
on every block in a large buffer cache, and in our initial implementation
these traversals were a performance bottleneck for even modest cache
sizes.
% 
\Featherstitch\ therefore precomputes much of the information
required for the buffer cache to choose a set of \patches\ to write.

\Kudos\ explicitly tracks, for each \patch, how many of its
direct dependencies remain uncommitted or in flight.
%
These counts are incremented as \patches\ are added to the system, and
decremented as the system receives commit notifications from the disk.
%
When both counts reach zero, the \patch\ is safe to write, and it is moved
into a \emph{ready list} on its containing block.
%
\begin{comment}
(\Noop\ \patches\ automatically commit when all their dependencies commit.)
\end{comment}
%
The buffer cache, then, can immediately tell whether any of a block's
patches are writable by examining its ready list.

To write a block $\PB$, the buffer cache initially populates the set $P$ with the
contents of the ready list.
%
While moving a patch $p$ into $P$, \Kudos\ checks whether there exist
dependencies $q \PDDepend p$ where $q$ is also on block $\PB$.
%
The system can potentially write $q$ at the same time as $p$, so $q$'s
counts are updated as if $p$ has already committed.
%
This may make $q$ ready, after which it in turn is added to $P$.
%
(This premature accounting is safe because the system won't try to write
$\PB$ again until $p$ actually commits.)


On-line maintenance of the ready counts adds some cost to several \patch\
manipulations, but since it saves so much duplicate work in the buffer
cache the resulting system is more efficient by multiple orders of
magnitude---and in particular, CPU time no longer scales superlinearly with
the size of the cache.


\begin{comment}
For a \module\ like the write-back cache to forward \patches\ in a
dependency-preserving order, the \module\ must find \patches\ whose \befores\
are all ``closer to the disk'' (or are also being forwarded as part of the same
block write). We say that such \patches\ are \emph{ready}. 


Each \patch\ has a count of the number of \befores\ it has at block device
modules just as close to the disk as it currently is, and a count of the number
of \befores\ it has which are in flight. When these counts are both zero, it is
ready. A \patch's \before\ counts are incrementally updated as \befores\ are
added and removed and as \beforing\ \patches\ are moved closer to the disk.

Because \Kudos\ makes sure that the \befores\ of a \patch\ are at least as
close to the disk as it is, only directly reachable \beforing\ \patches\ need to
be included in a \patch's \before\ counts. \Noop\ \patches, with the exception
of managed \noop\ \patches\ (which have an explicit owning block device), add a
wrinkle to this simplifying rule, however. They are considered to be as close to
the disk as their \before\ which is the farthest from the disk, in effect,
propagating the distance to the disk metric through them.

When a \before\ count update changes whether a \patch\ is ready to write, the
\patch's inclusion in its block's ready list is updated. To write a block, a
\module\ thus iterates through the block's ready list, sending \patches\ to the
target block device, until the list is empty. Thus instead of having to
repeatedly traverse \patch\ graphs to determine readiness on demand, we have
this information maintained automatically as it changes. This automatic
maintenance adds some cost to forwarding \patches\ and changing the graph
structure, but since it saves so much duplicate work\footnote{The amount of
duplicate work saved is actually superlinear in the size of the write-back
cache.} it is much more efficient.
\end{comment}

\subsection{\ChDesc\ Merging}
\label{sec:patch:merge}

File operations such as block allocations, inode updates, and directory updates
create a large number of small, distinct \chdescs. Keeping track of many small
\chdescs\ and their dependencies requires significant amounts of memory.
%
\Kudos\ uses \emph{merges} to drastically reduces the number of \chdescs, and
thus \chdesc\ memory usage, by conservatively identifying when a new and an
existing \chdesc\ pair can be written to disk together. Rather than creating
a new \chdesc, Kudos updates the disk change and dependencies to merge the
changes into the existing \chdesc.
%
In this section we present three \chdesc\ merge algorithms. All three
use Invariant~\ref{cdinvar:add-before} to reason about future graph
changes and use fast, conservative checks during \chdesc\ creation; they
differ in their applicable conditions.

\subsubsection{\Nrb\ \ChDesc\ Merging}
\label{sec:chdescs:merge:nrb}

Recall from \S\ref{sec:chdescs:nrb} that a write of any \chdescs\ on block
$b$ must include all \nrb\ \chdescs\ on $b$.
%
This additional requirement is in fact an exquisite optimization
opportunity; it implies that all \nrb\ \chdescs\ on a given block can
be merged.
%
Further, merging can remove the need for the \nrb\ \chdesc\ implicit
dependency rules by ensuring that
%
there is at most one \nrb\ \chdesc\ per block (\nrb-\nrb\ merging)
%
and that all \rb\ \chdescs\ on a given block depend on the \nrb\ \chdesc\
(\nrb-\rb\ merging).
%
We describe these two \chdesc\ merging algorithms and how they
preserve dependency semantics in this section.

\paragraph{\Nrb-\Nrb\ \ChDesc\ Merging}
\label{sec:chdescs:merge:nrb:hard-hard}

\emph{\Nrb-\nrb\ \chdesc\ merging} merges a new \nrb\ \chdesc\ \p{q}
into an existing \nrb\ \chdesc\ \p{p} instead of creating \p{q}.
%
Any two \nrb\ \chdescs\ on the same block may be (and are) merged.
%
Merging all \nrb\ \chdescs\ ensures:
%
\cdinvar{one-nrb}{\(\forall\! b\!: |\PHard[b]| \leq 1\)}
%
\noindent
%
Invariant~\ref{cdinvar:one-nrb} simplifies \nrb\ \chdesc\ handling by
%
removing the implicit dependencies that ensure all \nrb\ \chdescs\
are written together
%
and by removing the need to scan for an existing \nrb\ \chdesc\ when
\nrb-\nrb\ \chdesc\ merging.
%
% Although merging two \chdescs\ will not induce block-level dependency
% cycles, without sufficient care merging could induce \chdesc-level
% dependency cycles.  A trivial example is merging \p{q} into \p{p} when
% \p{q} has an explicit dependency on \p{p}; the combined \p{(p+q)}
% should not and need not depend on itself.
%
To preserve dependency semantics, the merged \p{(p+q)} must depend on
the union of \p{p} and \p{q}'s transitive \befores. Additionally, while the
\chdescs\ can be merged without forming a \chdesc-level dependency cycle,
the merge must ensure that it does not introduce a needless cycle, e.g.
through \anoop\ \chdesc\ \p{e} in \depends{q}{\depends{e}{p}}
\todo{Is cycle avoidance worth mentioning? Is this a good way to mention it?}.

From Invariant~\ref{cdinvar:add-before} and the \nrb\ \chdesc\
creation condition (no external \afters), the only possible
dependencies involving \p{p} and \p{q} are those shown in
Figure~\ref{fig:nrb-merge}\todo{Should we give these deductions or a
  flavor?}.
%
Notice, for example, that any \p{r} such that
\indirdepends{q}{\indirdepends{r}{p}} is \anoop\ \chdesc\todo{This is
  a strong statement. Expand on its implications?}.
%
Our algorithm to transform dependencies for \nrb-\nrb\ \chdesc\ merges
(Figure~\ref{algo:merge:hard-hard}) follows from these possible
dependencies.
%
It updates \p{p}'s transitive \befores\ to ensure
\(\PDepset{q}\todo{Incorrect! Only the non-\noop{}s.} \subseteq
\PDepset{p}\)\todo{Note that Invariant~\ref{cdinvar:add-before}
  ensures that \noop\ \chdescs\ reachable from \p{q} will not gain
  data \chdesc\ \befores?}\todo{Note that it only needs to move dependencies?}.

\begin{figure}[htb]
  \centering
  \includegraphics[width=\columnwidth]{nrb_merge}
  \caption{Possible dependencies when merging \nrb\ \chdesc\ \p{q}
    into existing \nrb\ \chdesc\ \p{p}.}
  \label{fig:nrb-merge}
\end{figure}

\noindent Algorithm called on \p{q} and \p{p}:\\
Input: \chdesc\ \p{a} and existing \nrb\ \chdesc\ \p{p}.\\
Returns: whether \indirdepends{a}{p} exists. \(\forall\! \p{b}\!: \indirdepends{a}{b}\) and \notindirdepends{b}{p}, creates \indirdepends{p}{b}.

\begin{itemize}
\item If \p{a} is external, return ``no path to \p{p}.''
\item If \p{a} equals \p{p}, return ``path to \p{p}.''
\item Call self on \p{a} and \p{p}.
\item If \p{a} has no path to \p{p}, return ``no path to \p{p}.''
\item For each \p{a} \before\ \p{b}:
  \begin{itemize}
  \item If \p{b} has no path to \p{p}:
    \begin{itemize}
    \item Move \p{b} from a \before\ of \p{a} to a \before\ of \p{p}.
    \end{itemize}
  \end{itemize}
\end{itemize}

\paragraph{\Nrb-\Rb\ \ChDesc\ Merging}
\label{sec:chdescs:merge:nrb:hard-soft}

When creating the first \nrb\ \chdesc\ on a block, \emph{\nrb-\rb\
  \chdesc\ merging} merges all existing (\rb) \chdescs\ into the new
\nrb\ \chdesc.
%
Such an arrangement can arise through a combination of \chdesc\ creates
and block writes; 
%
for example, the block may first obtain an initial (\nrb{}) \chdesc,
%
then gain external \afters\ on its \chdesc,
%
accumulate additional (\rb{}) \chdescs,
%
write the subset of its \chdescs\ with external \afters\ (leaving some
\rb\ \chdescs\ on the block),
%
and then gain a \nrb\ \chdesc.
%
In addition to reducing the number of data \chdescs, \nrb-\rb\
\chdesc\ merging removes the second implicit \nrb\ \chdesc\
dependency, that \rb\ \chdescs\ not explicitly dependent on the
block's \nrb\ \chdesc\ implicitly depend on it.
%
As in \nrb-\nrb\ \chdesc\ merging, \Kudos\ merges such \chdescs\ to
avoid the complications of their implicit dependencies.

\Nrb-\rb\ \chdesc\ merging's implementation first merges all \rb\ \chdescs\
into a \nrb\ \chdesc\ and then \nrb-\nrb\ \chdesc\ merges the new \nrb\
\chdesc\ into the now-existing \nrb\ \chdesc.
%
Our algorithm to transform the dependencies for \nrb-\rb\ \chdesc\
merges (Figure~\ref{algo:merge:hard-soft}) for block $b$
%
chooses a \chdesc\ \p{p} such that
\(\notexists \inset{q}{\PMem[b]}\!: \indirdepends{p}{q}\)
%
and updates its transitive \befores\ to ensure
\(\PDepset{\PSoft[b]} \subseteq \PDepset{p}\).
%
Because any \(\inset{q}{\PMem[b] - p}\) may have \afters, to
preserve dependencies we convert such a \p{q} into \anoop\ \chdesc\
and ensure \depends{q}{p}.

\noindent Algorithm:
\begin{itemize}
\item Choose a \(\inset{p}{\PMem[b]}\!:\
\notexists\! \inset{q}{\PMem[b]}\!:\ \indirdepends{p}{q}\).
\item For each \inset{q}{\PMem[b] - p}:
  \begin{itemize}
  \item Call the \nrb-\nrb\ \before\ move algorithm on \p{q} and \p{p}.
  \item Convert \p{q} into \anoop\ \chdesc.
  \end{itemize}
\item Convert \p{p} into a \nrb\ \chdesc\ (free it's previous data copy).
\end{itemize}

\todo{Note that \nrb-\rb\ merging is rare? Note why it is helpful even though
it is rare?}
\todo{Explain why this preserves dependency semantics? Show possible
dependencies? For the paper, free \chdescs\ instead of convert them
into \noop{}s? (Must modify \nrb-\nrb\ algo usage.)}

\subsubsection{Overlap \ChDesc\ Merging}
\label{sec:chdescs:merge:overlap}
\todo{Note as useful when new may need to be rolled back.}
Bitmap blocks and inode size fields accumulate many nearby and
overlapping mergeable \chdescs\ as data is appended to or truncated
from a file.
%
Many of these and similar \chdescs\ are mergeable and have
dependencies that allow simple (and fast) reasoning to identify many
of the mergeable pairs: two \chdescs\ on block $b$ that overlap no other \chdescs\ in \PMem[b]
and which have no dependency path from the new to the existing \chdesc\
will not induce a block-level cycle and so are writable together.
We know that \textit{later} changes will not cause them to induce a block-level cycle due to
invariant~\ref{cdinvar:add-before} and by not merging if the new \chdesc\
has a before and the before is marked as allowed to violate
invariant~\ref{cdinvar:add-before}.
%
While path existence testing is expensive, a conservative path test
of only a depth of two identifies most mergeable \chdescs. If the new
\chdesc\ has an explicit \before\ that is not the existing \chdesc\ and
this \before\ has a \before, then there may be a path to the existing
\chdesc.
%
To merge two such overlapping \chdescs, add the new \chdesc's explicit
before to the existing \chdesc\ (if any and if not the existing \chdesc).


%%

At the end of \chdesc\ optimizations, say something along the lines:
%
The dynamic optimizations facilitated through \nrb\
\chdescs\ implement the efficiency in systems using soft updates or
journaling\todo{Actually do this for journaling} while expressing
changes modularly through structural descriptions rather than through
internal and semantic file system descriptions.

\todo{Should we talk about why we allow NRBs and merging to be
  disabled? (Debugging simplicity and depend add to \noop\ \chdescs\
  with \afters\ bug catching.)}

% -*- mode: latex; tex-main-file: "paper.tex" -*-

\subsection{Other Optimizations}
\label{sec:patch:misc}

Optimizations can only do so much
with bad dependencies.
%
Just as having too few dependencies can compromise system correctness,
having too many dependencies, or the wrong dependencies, can non-trivially
degrade system performance.
%
For example, in both the following \patch\ arrangements, $s$ depends on all
of $r$, $q$, and $p$, but the left-hand arrangement gives the system more
freedom to reorder block writes:

\begin{figure}[htb]
\vskip-.4\baselineskip
\centering
\begin{tabular}{@{}p{.3\hsize}p{.3\hsize}@{}}
\centering \includegraphics[scale=.62]{fig/figures_4} &
\centering \includegraphics[scale=.62]{fig/figures_6}
\end{tabular}
\vskip-1.2\baselineskip
\end{figure}

\noindent%
If $r$, $q$, and $p$ are adjacent on disk, the left-hand
arrangement can be satisfied with two disk requests while the right-hand
one will require four.
%
Although neither arrangement is much harder to code, in several cases
we discovered that one of our file system implementations was performing
slowly because it created an arrangement like the one on the right.

Care must be taken to avoid unnecessary \emph{implicit} dependencies,
and in particular overlap dependencies.
%
For instance, inode blocks contain multiple inodes, and changes to two
inodes should generally be independent; a similar statement holds for
directories.
%% and even sometimes for different fields in a summary block like
%% the superblock.
%
Patches that change one independent field at a time generally give
the best results.
%%  can be obtained from \emph{minimal} \patches\ that change
%% one independent field at a time. 
\Kudos\ will merge these \patches\ when
appropriate, but if they cannot be merged, minimal \patches\ tend to
cause fewer \patch\ reversions and give more flexibility in write
ordering.

File system implementations can generate better dependency arrangements
when they can detect that certain states will never appear on disk.
%
For example, soft updates requires that clearing an inode
depend on nullifications of all corresponding directory entries, which
normally induces dependencies from the inode onto the directory
entries.
%
However, if a directory entry will never exist on disk---for example,
because a patch to remove the entry merged with the patch that created
it---then there is no need to require the corresponding dependency.
%
Similarly, if \emph{all} a file's directory entries will never exist on
disk, the patches that free the file's blocks need not depend on the
directory entries.
%
Leaving out these dependencies can speed up the system by avoiding
block-level cycles, such as those in Figure~\ref{f:soft2hard}, and the
rollbacks and double writes they cause.
%
\begin{comment}
A dependency from the inode clear onto the directory entry clear is
sufficient to ensure this property. However, when a directory entry
will never exist on disk because it is created and removed before the
creation is committed, the inode clear need not depend on the
directory entry clear.
\end{comment}
%
The \Kudos\ ext2 module implements these optimizations,
%%  and another, related
%% one: if a directory entry will never reference an inode, then the patches
%% that clear the corresponding inode bitmap and block bitmap entries need not
%% depend on the patch that deleted that directory entry.
%
significantly reducing disk writes, patch allocations, and undo data required
when files are created and deleted within a short time.
%
Although the optimizations are file system specific, the file system
implements them using general properties, namely, whether two patches
successfully merge.

%% the changes to update a field in an inode structure
%% on disk, a \patch\ spanning the entire inode could be created even though
%% only a single field changed. A later change to a different field in the
%% same inode would appear to overlap the first, possibly creating an
%% unnecessary dependency. Creating \patches\ that correspond more precisely
%% to the changes being made avoids this problem, so a utility function is
%% provided by \Kudos\ to make this operation as convenient as creating a
%% single \patch\ for an entire large structure.

%% \subsubsection{\Patch\ List Ordering}

\begin{comment}
%% Declared uninteresting.
The buffer cache and a few other \modules\ perform better in the
common case that each block's \patches\ are listed in order of creation
time,
%
taking $O(n)$ time to process $n$ patches rather than $O(n^2)$.
%% in the common case that this order is preserved.
\end{comment}

Finally, block allocation policies can have a dramatic effect on the number of
I/O requests required to write changes to the disk. For instance, soft
updates-like dependencies require that data blocks be initialized before
an indirect block references them.   
Allocating an
indirect block in the middle of a range of file data blocks
forces the data blocks to be
written as two smaller I/O requests, since the indirect block
cannot be written at the same time. Allocating the indirect block somewhere
else allows the data blocks to be written in one larger I/O request, at the
cost of (depending on readahead policies) a potential slowdown in read performance.



We often found it useful to examine \patch\ dependency graphs visually.
%% (looking at created \patches\ and the \patch's
\Kudos\ optionally logs \patch\ operations to disk;
a separate debugger inspects and generates graphs from these
logs.
%
Although the graphs could be daunting, they provided some evidence that
patches work as we had hoped: performance problems could
be analyzed by examining general dependency structures, and
sometimes easily fixed by manipulating those structures.


\begin{comment}
Several functions in \Kudos\ iterate over lists of \patches\ looking for either
a single \patch\ or set of \patches\ satisfying some property, or trying to
process all the \patches\ in the list in some order determined by the dependency
graph. It is generally the case that the \patches\ satisfying the property or
the order in which the \patches\ should be processed can be determined very
quickly by keeping the lists sorted. For instance, the library function which
reverts \patches\ needs to perform the revert operations essentially in inverse
creation order, so that reverting a \patch\ which has since been overwritten
by a later \patch\ does the right thing. Keeping the list of all \patches\ on a
block sorted in creation order (which is very easy) makes this an efficient
operation, while it might otherwise take $O(n^2)$ time to execute. Similarly,
many \patch\ merging functions need to find for a given block some \patch\
which has no \befores\ on the same block, and the oldest \patch\ on a block
always satisfies this requirement.
\end{comment}

