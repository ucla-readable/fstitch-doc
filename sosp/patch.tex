% -*- tex-main-file: "paper.tex" -*-

\section{Patch Model}
\label{sec:patch}

\emph{Patches} represent any and all changes to stable storage in a \Kudos\
system.
%
Patches encapsulate both the data to be written and any dependencies among
writes.
%
This section describes the basic patch model, and the optimizations we have
found important so far, using a simple notation that allows us to prove the
correctness of several transformations.
%
However, patches are not merely theoretical objects; each active patch
corresponds to an object in the \Kudos\ implementation.


%% \subsection{Basics}

\Kudos's job is simply to write patches to one or more disks.
%
(The same ideas would apply to any other stable medium, or to network file
systems, but to simplify our terminology, we use the term ``disk''
throughout.)

\makeatletter
\newcommand{\PState}[1]{\ensuremath{#1.\textit{state}}}
\newcommand{\PBlock}[1]{\ensuremath{#1.\textit{block}}}
\newcommand{\PMemst}{\ensuremath{\textit{mem}}}
\newcommand{\PInfst}{\ensuremath{\textit{flight}}}
\newcommand{\PDiskst}{\ensuremath{\textit{disk}}}
\newcommand{\PSetlim}[1]{\def\@next{#1}\ifx\@next\@empty\else[\@next]\fi}
\newcommand{\PMem}[1][]{\ensuremath{\textit{Mem}\PSetlim{#1}}}
\newcommand{\PInf}[1][]{\ensuremath{\textit{Flight}\PSetlim{#1}}}
\newcommand{\PDisk}[1][]{\ensuremath{\textit{Disk}\PSetlim{#1}}}
\newcommand{\PDDepset}[1]{\ensuremath{\def\@next{#1}\ifx\@next\@empty\else\@next.\fi\textit{ddeps}}}
\newcommand{\PDepend}{\ensuremath{\leadsto}}
\newcommand{\PDDepend}{\ensuremath{\rightarrow}}
\newcommand{\PDepset}[1]{\ensuremath{\textit{Dep}[#1]}}
\makeatother

Patches begin in memory.
%
They are eventually written to a disk controller, a state we call \emph{in
flight} because the processor has lost control over when the patches are
committed (the controller might service requests in any order).
%
At some point thereafter, patches are committed to the disk itself; the
disk reports success after the write has committed.
%
Thus, each patch $p$ has a state:
%
\[ \PState{p} \in \{ \PMemst, \PInfst, \PDiskst \}. \]
%
The set $\PMem$ is defined as $\PMem = \{p \mid \PState{p} = \PMemst\}$,
and similarly for $\PInf$ and $\PDisk$.

Patches are written in units called \emph{blocks}, which may cover one or
more sectors.
%
Each patch applies to a single block, written $\PBlock{p}$.
%
For a given block $b$, we define $\PMem[b] = \{p \in \PMem \mid \PBlock{p}
= b \}$, and similarly for $\PInf[b]$ and $\PDisk[b]$.
%
The disk's behavior is encapsulated in the following action,
which might happen at any time:

\begin{tabbing}
\quad \textbf{Commit block.} \\
\qquad Pick some block $b$. \\
\qquad For each $p \in \PInf[b]$, set $\PState{p} \gets \PDiskst$.
\end{tabbing}


We note that soft updates-like consistency protocols demand more from disks
than many journal protocols.
%
Neither type of protocol assumes that data is committed until the disk
reports success.
%
However, soft updates inherently assumes that blocks are written
\emph{atomically}, except in the case of catastrophic media error.
%
In particular, if the disk fails while a block $b$ is in flight, then $b$'s
value on recovery must equal either the old value or the new value.
%
Most journal designs also handle the case where in-flight blocks might be
corrupted on recovery---for instance, perhaps because the memory holding
the new value of the block lost its coherence before the disk stopped
writing~\cite{tso}.
%
However, some disks actually do provide an atomicity guarantee, for
instance by using non-volatile memory to store blocks before they make it
onto disk~\cite{???}.
%
The \Kudos\ core makes no assumptions about block atomicity, instead relying
on software above it to implement a consistency protocol that makes sense
for the given disk.


\paragraph{Dependencies}
%
Each patch $p$ also has a set of \emph{direct dependencies}, $\PDDepset{p}$.
%
These are other patches on which $p$ depends: every $q \in \PDDepset{p}$
must be written to disk before $p$, or atomically with $p$, or the file
system's consistency is at risk.
%
We say $p$ \emph{directly depends on} $q$ and write $p \PDDepend q$ when $q
\in \PDDepset{p}$.
%
The \emph{depends on} relation, written $p \PDepend q$, is $\PDDepend$'s
transitive closure: $p \PDepend q$ iff there exists a path of direct
dependencies $p \PDDepend p_1 \PDDepend \cdots \PDDepend p_n \PDDepend q$.
%
The set $\PDepset{p}$ consists of $p$'s dependencies $\{q \mid p \PDepend
q \}$.
%
Given a set of patches $P = \{p_i\}$, $\PDepset{P} = \bigcup_i
\PDepset{p_i}$.
%
Dependencies link patches into a directed graph; $\PDepset{p}$ is just the
set of $p$'s successors in the graph.


The \textbf{disk safety property} is simply written
%
\[ \PDepset{\PDisk} \subseteq \PDisk. \]
%
That is, the dependencies of all patches on disk are always also on disk,
and
%
no matter when the system crashes, the disk is consistent in terms of
dependencies.
%
The file system's job is to set up dependencies so that the disk safety
property implies file system correctness.
%
(This might involve compensating for the possibility that in-flight blocks
can be corrupted on recovery.)


The storage system cannot control when the disk controller commits blocks;
instead, it controls when blocks are released to the controller.
%
It therefore maintains the disk safety property by enforcing, instead, the
\textbf{in-flight safety property}: For any block $b$ with $\PInf[b] \neq
\emptyset$,
%
\[ \PDepset{\PInf[b]} \subseteq \PInf[b] \cup \PDisk . \]
%
This, combined with the disk's ``commit block'' behavior above, implies the
disk safety property.
%
The buffer cache's job is to write patches in an order that obeys the
in-flight safety property.


\paragraph{Rollback}
%
A simplistic implementation of patches might just keep track of
dependencies; all changes would be applied directly to a block's in-memory
copy.
%
After all, disks write \emph{blocks}---arrays of memory---not patches.
%
However, soft updates-like orderings may require that the buffer cache
\emph{not} write one or more patches on some block.
%
In particular, a series of file system operations may create dependencies
that enforce a circular order among blocks, even though the dependencies
themselves do not form a cycle~\cite{ganger00soft}.
%
This is problematic since blocks with circular dependencies can never be
written: no block can be written first since each block depends on another.


For this reason, each patch carries \emph{rollback} information that gives
the previous version of the data altered by the patch.
%
If a patch $p$ is not written with its containing block, the system rolls
back the patch, which swaps the new data and the previous version.
%
Once the block is written, the system will roll the patch forward and, when
allowed, write the block again, this time including the patch.
%
Rollback information adds greatly to memory and CPU utilization, but it can
often be optimized away, as we show below.


The buffer cache's required behavior is then encapsulated in the following
action:

\begin{tabbing}
\textbf{Write block.} \\
\quad Pick some block $b$ with $\PMem[b] \neq \emptyset$. \\
\quad Pick some $P \subseteq \PMem[b]$ with $\PDepset{P} \subseteq P \cup
\PDisk$. \\
\quad For each $p \in P$, set $\PState{p} \gets \PInfst$. \\
\quad For each $p \in \PMem[b]-P$, set $\PDDepset{p} \gets \PDDepset{p}
\cup P$.
\end{tabbing}

\noindent
%
The last step forces any rolled-back patches to wait in memory at least
until the current version of the block is written.  (Our implementation
achieves this property without using dependencies.)
%
It is easy to show that this action preserves the in-flight safety property
and, thus, the disk safety property.


% -*- mode: latex; tex-main-file: "paper.tex" -*-

\subsection{Implementation and \noop\ \chdescs}
\label{sec:chdescs:noop}

\Kudos\ file system implementations create \chdescs\ with one of the
following functions:

\begin{scriptsize}
\begin{alltt}
int \textbf{patch_create_byte}(
    bdesc_t *block, bdev_t *owner,
    uint16_t offset, uint16_t length,
    const void *data, patch_t **head);
int \textbf{patch_create_bit}(
    bdesc_t *block, bdev_t *owner,
    uint16_t offset, uint32_t xor,
    patch_t **head);
int \textbf{patch_create_\noop}(
    bdev_t *owner, patch_t **tail,
    size_t nheads, patch_t * heads[]);
int \textbf{patch_create_diff}(
    bdesc_t *block, bdev_t *owner,
    uint16_t offset, uint16_t length,
    const void *data, patch_t **head);
\end{alltt}
\end{scriptsize}


We first address dependency convenience and memory usage with \emph{\noop\
\chdescs}, which have no associated data or block.
%
This makes it just a means for tracking dependencies.


For example, imagine writing two files \texttt{before.txt}
The dependencies 
\Chdescs\ as so far described can be tedious and inefficient to manage when
dealing with large sets of them corresponding to file system operations. For
instance, if writing some file \texttt{\after.txt} is to depend on writing some
other file \texttt{\before.txt}, it will be inconvenient to keep arrays of all
the \chdescs\ corresponding to the two operations and inefficient to store the
potentially quadratic number of edges in the \chdesc\ graph.

To solve this problem, we introduce an additional type of \chdesc. The
prototypical \chdesc\ corresponds to some change on disk, but \Kudos\ also
supports \aemphnoop\ \chdesc\ type, which doesn't change the disk at all.
\Noop\ \chdescs\ can have \befores, like other \chdescs, but they don't need to
be written to disk: they are trivially satisfied when all of their \befores\ are
satisfied. Thus, they can be used to ``stand for'' entire sets of other changes.
%
This capability is extremely useful, and is used by most operations on disk
structures so that a single \chdesc\ can be returned that depends on the whole
change. Likewise, \anoop\ \chdesc\ can be passed in as a parameter to a disk
operation to make the whole operation depend on a set of other changes. \Noop\
\chdescs\ allow dependencies between sets with only a linear number of
dependency edges in the \chdesc\ graph, and without having to pass around arrays
of \chdescs.
%
The cost is that some functions may have to traverse trees of \noop\ \chdescs\
to determine true dependencies.

Modules can also use \noop\ \chdescs\ to \emph{prevent} changes from being
written. A \emph{managed} \noop\ \chdesc\ must be explicitly satisfied; any
changes that depend on that \noop\ \chdesc\ are delayed until the owning \module\
explicitly satisfies it. This is used, for instance, by the journal \module\
(\S\ref{sec:using:journal}) to prevent a transaction's \chdescs\ from
being written before the journal commits.

\Noop\ \chdescs\ are not included in our formal model of \chdescs\ for simplicity;
they add some additional complexity but do not change the basic ideas.
\todo{Well, they are now... the rules need updating since they have no blocks.}



\subsection{Challenges}

A naive implementation of this model scales extremely poorly,
particularly with cache size.
%
Challenges in a patch-based file system implementation include:

\textbf{Buffer cache graph traversal.}
%
In order to evict and write a block, the buffer cache must choose a block
$b$,
%
and then find a set of \patches\ $P_b \subseteq \PMem[b]$ whose dependencies
satisfy a graph property, namely that $\PDepset{P_b} \subseteq P_b \cup
\PDisk$.
%
It usually makes sense to define $P_b$ maximally---that is, as the
\emph{largest} corresponding set of \patches.
%
In the ideal (and common) case $P_b = \PMem[b]$, which lets the cache reuse
$b$'s memory once $P_b$ is committed to disk.  However, in some cases there
may be no block for which $P_b = \PMem[b]$.
%
It would also be nice if the blocks chosen for writing also maximized the
disk's commit rate, by minimizing seeks and so forth.

A naive implementation might calculate, for each in-memory block $b$, the
largest set of patches $P_b \subseteq \PMem[b]$ with $\PDepset{P_b}
\subseteq P_b \cup \PDisk$, then evict some block close to previously
written blocks and with few rolled-back \patches\ (where $\PMem[b] - P_b$
is small).
%
This, however, would be extraordinarily expensive.
%
Finding $P_b$ requires traversing a dependency graph which might contain
thousands and thousands of nodes.
%
Doing so for each block, once per eviction, would take huge amounts of CPU
time.


\textbf{Rollback memory usage.}
%
Only a small fraction of patches will ever need to be rolled back.
%
For example, most data writes never need to be rolled back in any file
system.
%
If a patch won't be rolled back under any circumstances, the memory and
CPU time spent to preserve the old version is wasted.


\textbf{\Patch\ memory usage.}
%
Patches themselves take up memory and require time to allocate, free, and
traverse.
%
If two patches have redundant dependencies, it would be faster to combine
them.


%% \textbf{Dependency memory usage.}
%% %
%% The $\PDDepset{}$ sets are stored as doubly linked lists; each individual
%% dependency takes up memory.
%% %
%% Important and common dependency relations require many dependencies to
%% express; for example, if patches $p_1,\dots,p_n$ depend, as a group, on
%% patches $q_1,\dots,q_n$, expressing this constraint would require $n^2$
%% total dependencies.


The next section tackles all of these challenges.


\section{Patch Optimizations}

\begin{figure*}
\centering
\begin{tabular}{@{}p{.5\hsize}p{.5\hsize}@{}}
{\centering\includegraphics[scale=0.8]{fig/opt_1}\endgraf}&
{\centering\includegraphics[scale=0.8]{fig/opt_2}\endgraf}\\
{\centering\textbf{a)} Before optimizations\endgraf}&
{\centering\textbf{b)} With hard patches\endgraf}\\
{\centering\includegraphics[scale=0.8]{fig/opt_3}\endgraf}&
{\centering\includegraphics[scale=0.8]{fig/opt_4}\endgraf}\\
{\centering\textbf{c)} With hard--hard merging\endgraf}&
{\centering\textbf{d)} With overlap merging\endgraf}\\
\end{tabular}
\end{figure*}

\subsection{Ready \ChDesc\ Lists}

For a \module\ like the write-back cache to forward \chdescs\ in a
dependency-preserving order, the \module\ must find \chdescs\ whose \befores\
are all ``closer to the disk'' (or are also being forwarded as part of the same
block write). We say that such \chdescs\ are \emph{ready}. Because the
write-back cache frequently searches for and writes many ready \chdescs,
redundant \chdesc\ graph traversals to calculate \chdesc\ readiness would
severly limit cache size scalabity. \Kudos\ therefore explictly tracks a
\chdesc's \before\ counts through incremental dependency updates, and uses these
counts to maintain a \chdesc\ ready list for each block.

Each \chdesc\ has a count of the number of \befores\ it has at block device
modules just as close to the disk as it currently is, and a count of the number
of \befores\ it has which are in flight. When these counts are both zero, it is
ready. A \chdesc's \before\ counts are incrementally updated as \befores\ are
added and removed and as \beforing\ \chdescs\ are moved closer to the disk.

Because \Kudos\ makes sure that the \befores\ of a \chdesc\ are at least as
close to the disk as it is, only directly reachable \beforing\ \chdescs\ need to
be included in a \chdesc's \before\ counts. \Noop\ \chdescs, with the exception
of managed \noop\ \chdescs\ (which have an explicit owning block device), add a
wrinkle to this simplifying rule, however. They are considered to be as close to
the disk as their \before\ which is the farthest from the disk, in effect,
propagating the distance to the disk metric through them.

When a \before\ count update changes whether a \chdesc\ is ready to write, the
\chdesc's inclusion in its block's ready list is updated. To write a block, a
\module\ thus iterates through the block's ready list, sending \chdescs\ to the
target block device, until the list is empty. Thus instead of having to
repeatedly traverse \chdesc\ graphs to determine readiness on demand, we have
this information maintained automatically as it changes. This automatic
maintenance adds some cost to forwarding \chdescs\ and changing the graph
structure, but since it saves so much duplicate work\footnote{The amount of
duplicate work saved is actually superlinear in the size of the write-back
cache.} it is much more efficient.

