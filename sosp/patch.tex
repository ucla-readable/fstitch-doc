% -*- mode: latex; tex-main-file: "paper.tex" -*-

\section{Patch Model}
\label{sec:patch}

\makeatletter
\let\emptyset\varnothing
\newcommand{\PState}[1]{\ensuremath{#1.\textit{state}}}
\newcommand{\PBlock}[1]{\ensuremath{#1.\textit{block}}}
\newcommand{\PMemst}{\ensuremath{\textit{mem}}}
\newcommand{\PInfst}{\ensuremath{\textit{flight}}}
\newcommand{\PDiskst}{\ensuremath{\textit{disk}}}
\newcommand{\PSetlim}[1]{\def\@next{#1}\ifx\@next\@empty\else[\@next]\fi}
\newcommand{\PMem}[1][]{\ensuremath{\textit{Mem}\PSetlim{#1}}}
\newcommand{\PInf}[1][]{\ensuremath{\textit{Flight}\PSetlim{#1}}}
\newcommand{\PDisk}[1][]{\ensuremath{\textit{Disk}\PSetlim{#1}}}
\newcommand{\PHard}[1][]{\ensuremath{\textit{\Nrb}\PSetlim{#1}}}
\newcommand{\PSoft}[1][]{\ensuremath{\textit{\Rb}\PSetlim{#1}}}
\newcommand{\PEmpty}[1][]{\ensuremath{\textit{\Noop}\PSetlim{#1}}}
\newcommand{\PDDepset}[1]{\ensuremath{\def\@next{#1}\ifx\@next\@empty\else\@next.\fi\textit{ddeps}}}
\newcommand{\PDepend}{\ensuremath{\leadsto}}
\newcommand{\PDDepend}{\ensuremath{\rightarrow}}
\newcommand{\PDepset}[1]{\ensuremath{\textit{Dep}[#1]}}
\newcommand{\PRDepset}[1]{\ensuremath{\textit{RDep}[#1]}}
\makeatother

Every change to stable storage in a \Kudos\ system is represented by a
\emph{patch}.
%
This section describes the basic patch abstraction using a semi-formal
notation that will be useful for analyzing our optimizations later.
%
Although the patch idea would apply to any stable medium, or to network
file systems, we use terms like ``disk'' and ``disk controller'' throughout
to simplify our terminology.
%
Our patch notation is summarized in Figure~\ref{f:patchnot}.

Each patch $p$ encapsulates four important pieces of information: its
 \emph{block}, its \emph{state}, a set of \emph{direct dependencies}, and
 some \emph{rollback data}.

Patch $p$'s \textbf{block}, written $\PBlock{p}$, is the unit of disk data
 to which $p$ applies.  The disk controller is assumed to write blocks as a
 unit (although not necessarily atomically; see below).  A file system
 change that affects $n$ blocks must be represented using at least $n$
 patches, since a patch by definition can touch only one block.

\begin{figure}
\begin{small}
\begin{tabular}{@{}ll@{}}
$\PBlock{p} = b$ & $p$'s block \\
$\PState{p}$ & $p$'s state, $\in \{\PMemst, \PInfst, \PDiskst\}$ \\
~~~~$\PMem$ & all in-memory patches: $\{p \mid \PState{p} = \PMemst\}$ \\
~~~~$\PMem[b]$ & $\{p \mid \PState{p} = \PMemst \text{ and } \PBlock{p} = b
 \}$ \\
~~~~$\PInf, \PDisk$ & similarly for in-flight and on-disk patches \\
$\PDDepset{p}$ & $p$'s direct dependencies \\
~~~~$p \PDDepend q$ & $p$ directly depends on $q$: $q \in \PDDepset{p}$ \\
~~~~$p \PDepend q$ & $p$ depends on $q$: either $p \PDDepend q$ or there exists \\
       & a patch $x$ so that $p \PDepend x \PDDepend q$ \\
~~~~$\PDepset{p}$ & $p$'s dependencies: $\{ q \mid p \PDepend q \}$ \\
\end{tabular}
\end{small}

\caption{Patch notation.}
\label{f:patchnot}
\end{figure}


Each patch is in one of three \textbf{states}: \emph{in memory}; \emph{in
 flight} to the disk controller, but not yet committed to disk; and
 \emph{on disk}.  The intermediate in-flight state is necessary because
 operating system software loses control of a patch upon writing its block
 to disk.  $p$'s state is written $\PState{p} \in \{\PMemst, \PInfst,
 \PDiskst\}$.  The operating system changes a patch's state from
 \PMemst\ to \PInfst\ by writing its block to the disk controller.  Some
 time later---after possible OS-level disk scheduling, transit over the
 system bus, and a period in the disk's cache---the disk writes the block
 to stable storage and reports success.  When the processor receives this
 notification, it changes the patch's state to \PDiskst.
%
The sets \PMem, \PInf, and \PDisk\ are defined to contain all patches with
 the given state, and the sets $\PMem[b]$, $\PInf[b]$, and $\PDisk[b]$ are
 defined to contain all patches with the given state \emph{on the given
 block} $b$.

Patch $p$'s \textbf{direct dependencies}, written $\PDDepset{p}$, is a set of
 other patches on which $p$ ``depends''.
%
That is, every patch in $\PDDepset{p}$ should be committed to disk either
 before, or at the same time as, $p$ itself.
%
Dependencies are set by the file system based on whatever consistency
 semantics it wants to ensure.
%
For example, a file system with asynchronous writes (and no durability
 guarantees) might write all patches with $\PDDepset{p} = \emptyset$;
%
a file system implementing soft updates would arrange the dependencies
 accordingly;
%
and a file system implementing journaling would write a different set of
 dependencies, where the journal commit record would depend on the journal
 data and the writes to the main body of the file system would in turn
 depend on the commit record.
%
Thus, an upper file system layer defines an initial set of
 dependencies.
%
The rest of \Kudos\ generally preserves this initial set, but can modify
 them as necessary to change or fix dependency semantics.
%
Finally, the buffer cache obeys the constraints thus defined.


We say $p$ \emph{directly depends on} $q$, and write $p \PDDepend q$, when
 $q \in \PDDepset{p}$.
%
We say $p$ \emph{depends on} $q$, and write $p \PDepend q$, when there
 exists a dependency chain from $p$ to $q$: that is, either $p \PDDepend q$
 or, for some patch $x$, $p \PDepend x \PDDepend q$.
%
Patch $p$'s set of \emph{dependencies} is written $\PDepset{p} = \{ q \mid
 p \PDepend q \}$; this is of course a superset of its direct dependencies.
%
Given a \emph{set} of patches $P$, we write $\PDepset{P}$ to mean the set's
 combined dependencies $\bigcup_{p\in P} \PDepset{p}$.



\textbf{Rollback data.}
%
Soft updates-like consistency orderings may require that the buffer cache
\emph{not} write one or more \patches\ on some block.
%
In particular, a series of file system operations may create dependencies
that enforce a circular order among blocks, even though the dependencies
themselves do not form a cycle~\cite{ganger00soft}.
%
This is problematic since blocks with circular dependencies can never be
written: no block can be written first since each block depends on another.
%
For this reason, each \patch\ carries \emph{rollback} information that gives
the previous version of the data altered by the \patch.
%
If a \patch\ $p$ is not written with its containing block, the system rolls
back the \patch, which swaps the new data and the previous version.
%
Once the block is written, the system will roll the \patch\ forward and, when
allowed, write the block again, this time including the \patch.
%
Rollback information adds greatly to memory and CPU utilization, but it can
often be optimized away, as we show below.


\subsection{Obeying dependencies}

\Kudos\ must ensure the \textbf{disk safety property}, which states that
 the dependencies of all patches on disk are also on disk:
%
\[ \PDepset{\PDisk} \subseteq \PDisk. \]
%
Thus, no matter when the system crashes, the disk is consistent in terms of
dependencies.
%
The file system's job is to set up dependencies so that the disk safety
property implies file system correctness.

However, \Kudos\ can only determine when patches are handed to the disk
 controller, not when they are written to disk.
%
Disk controller behavior is encapsulated in the following atomic action:

\begin{tabbing}
\textit{Commit block:} \\
\quad Pick some block $b$. \\
\quad For each $p \in \PInf[b]$, set $\PState{p} \gets \PDiskst$.
\end{tabbing}

\noindent
%
\Kudos\ must thus enforce the following \textbf{in-flight
 safety property}: For any block $b$,
%
\[ \PDepset{\PInf[b]} \subseteq \PInf[b] \cup \PDisk . \]
%
Since in-flight blocks are mutually independent (for any $b' \neq b$,
 $\PDepset{\PInf[b]} \cap \PDepset{\PInf[b']} \subseteq \PDisk$), the disk
 controller can safely write them in any order and still preserve the disk
 safety property.


The buffer cache must thus behave according to the following atomic action:

\begin{tabbing}
\textit{Write block:} \\
\quad Pick some block $b$ with $\PMem[b] \neq \emptyset$. \\
\quad Pick some $P \subseteq \PMem[b]$ with $\PDepset{P} \subseteq P \cup
\PDisk$. \\
\quad For each $p \in P$, set $\PState{p} \gets \PInfst$. \\
\quad For each $p \in \PMem[b]-P$, set $\PDDepset{p} \gets \PDDepset{p}
\cup P$.
\end{tabbing}

\noindent
%
Since the disk controller can write in-flight blocks in any order, at most
one version of a block can be in flight at any time.
%
Thus, the last step above forces any rolled-back \patches\ to wait in
memory at least until the in-flight version of the block is written.
%
Likewise, any patch created on a block with in-flight patches must depend
on those patches.
%
(Our implementation achieves this property without using dependencies.)
%
The buffer cache is also expected to write all blocks eventually, a
 liveness property.


This model does not completely define the disk's behavior on system crash,
 in particular with respect to in-flight blocks.
%
Soft updates inherently assumes that blocks are written
\emph{atomically}, except in the case of catastrophic media error:
%
if the disk fails while a block $b$ is in flight, then $b$'s
value on recovery must equal either the old value or the new value.
%
Most journal designs do not rely on this assumption, and can recover
 properly even if in-flight blocks are corrupted on recovery---for instance,
 because the memory holding the new value of the block lost its coherence
 before the disk stopped writing~\cite{tso}.
%
However, some disks actually do provide an atomicity guarantee, for
 instance by using non-volatile memory to store blocks before they make it
 onto disk~\cite{???}.
%
The \Kudos\ core makes no assumptions about block atomicity, instead relying
 on software above it to implement a consistency protocol that makes sense
 for the given disk.


The buffer cache can't handle cross-block dependency cycles since it must
write data in units of blocks.
%
Thus, if $p \PDepend q \PDepend p$, then $p$ and $q$ must be on the same
block; otherwise, neither $p$ nor $q$ could ever get written.
%
Whenever upper layers might generate cross-block dependency cycles,
intervening layers must break those cycles somehow before they reach the
cache.
%
For instance, a journal layer might detect dependency cycles and put the
corresponding patches into a single transaction, but the
\emph{implementation} of that transaction could not contain cross-block
cycles.
%
In practice, the \Kudos\ implementation currently disallows any dependency
cycles, including cycles within a single block.


% -*- mode: latex; tex-main-file: "paper.tex" -*-

\subsection{Implementation and \noop\ \chdescs}
\label{sec:chdescs:noop}

\Kudos\ file system implementations create \chdescs\ with one of the
following functions:

\begin{scriptsize}
\begin{alltt}
int \textbf{patch_create_byte}(
    bdesc_t *block, bdev_t *owner,
    uint16_t offset, uint16_t length,
    const void *data, patch_t **head);
int \textbf{patch_create_bit}(
    bdesc_t *block, bdev_t *owner,
    uint16_t offset, uint32_t xor,
    patch_t **head);
int \textbf{patch_create_\noop}(
    bdev_t *owner, patch_t **tail,
    size_t nheads, patch_t * heads[]);
int \textbf{patch_create_diff}(
    bdesc_t *block, bdev_t *owner,
    uint16_t offset, uint16_t length,
    const void *data, patch_t **head);
\end{alltt}
\end{scriptsize}


We first address dependency convenience and memory usage with \emph{\noop\
\chdescs}, which have no associated data or block.
%
This makes it just a means for tracking dependencies.


For example, imagine writing two files \texttt{before.txt}
The dependencies 
\Chdescs\ as so far described can be tedious and inefficient to manage when
dealing with large sets of them corresponding to file system operations. For
instance, if writing some file \texttt{\after.txt} is to depend on writing some
other file \texttt{\before.txt}, it will be inconvenient to keep arrays of all
the \chdescs\ corresponding to the two operations and inefficient to store the
potentially quadratic number of edges in the \chdesc\ graph.

To solve this problem, we introduce an additional type of \chdesc. The
prototypical \chdesc\ corresponds to some change on disk, but \Kudos\ also
supports \aemphnoop\ \chdesc\ type, which doesn't change the disk at all.
\Noop\ \chdescs\ can have \befores, like other \chdescs, but they don't need to
be written to disk: they are trivially satisfied when all of their \befores\ are
satisfied. Thus, they can be used to ``stand for'' entire sets of other changes.
%
This capability is extremely useful, and is used by most operations on disk
structures so that a single \chdesc\ can be returned that depends on the whole
change. Likewise, \anoop\ \chdesc\ can be passed in as a parameter to a disk
operation to make the whole operation depend on a set of other changes. \Noop\
\chdescs\ allow dependencies between sets with only a linear number of
dependency edges in the \chdesc\ graph, and without having to pass around arrays
of \chdescs.
%
The cost is that some functions may have to traverse trees of \noop\ \chdescs\
to determine true dependencies.

Modules can also use \noop\ \chdescs\ to \emph{prevent} changes from being
written. A \emph{managed} \noop\ \chdesc\ must be explicitly satisfied; any
changes that depend on that \noop\ \chdesc\ are delayed until the owning \module\
explicitly satisfies it. This is used, for instance, by the journal \module\
(\S\ref{sec:using:journal}) to prevent a transaction's \chdescs\ from
being written before the journal commits.

\Noop\ \chdescs\ are not included in our formal model of \chdescs\ for simplicity;
they add some additional complexity but do not change the basic ideas.
\todo{Well, they are now... the rules need updating since they have no blocks.}



\subsection{Discussion}

- Levels of the file system (dependencies at one level can apply to
  different blocks, etc.)



\section{Patch Optimizations}
\label{sec:patch:challenges}

A naive implementation of this model has all the properties we want except
good performance.
%
File system modules can define their own consistency protocols; other
modules can modify them: for instance, the journaling module we describe
below automatically changes soft updates orderings into journal
transactions.
%
Applications can affect consistency orderings, and it is all relatively
easy thanks to the patch abstraction.
%
Unfortunately, that abstraction is extremely expensive.


Consider an unoptimized version of \Kudos\ running an ext2 file system with
soft updates-like dependencies.
%
Figure~\ref{f:opt}a shows the patches generated when an application appends
16384 bytes of data to an existing empty file.
%
This requires allocating four new 4096-byte blocks, writing data to them,
attaching them to the file's inode, changing the inode's file size and
modification time, and updating the ``group descriptor'' and superblock to
account for the allocated blocks.
%
Eight blocks are written during the operation.
%
\Kudos, however, represents the operation with 23 patches and roughly 33000
(!) bytes of rollback data.
%
The patches slow down the buffer cache system by making graph traversals
more expensive.
%
The 4096 bytes of rollback data per patch stored for ``zero data'' and
``write data'' patches, which initialize and write data blocks, is
particularly painful since in this file system, data blocks \emph{never}
need to be rolled back!


This section shows how the actual \Kudos\ system uses generic patch
properties and dependency analysis to reduce the 23 patches and 33000 bytes
of rollback data in Figure~\ref{f:opt}a to the 8 patches and 0 bytes of
rollback data in Figure~\ref{f:opt}d.
%
These optimizations apply transparently to any \Kudos\ file system.


\begin{comment}

Challenges in a \patch-based file system implementation include:

\textbf{Buffer cache graph traversal.}
%
In order to evict and write a block, the buffer cache must choose a block
$b$,
%
and then find a set of \patches\ $P_b \subseteq \PMem[b]$ whose dependencies
satisfy a graph property, namely that $\PDepset{P_b} \subseteq P_b \cup
\PDisk$.
%
It usually makes sense to define $P_b$ maximally---that is, as the
\emph{largest} corresponding set of \patches.
%
In the ideal (and common) case $P_b = \PMem[b]$, which lets the cache reuse
$b$'s memory once $P_b$ is committed to disk.  However, in some cases there
may be no block for which $P_b = \PMem[b]$.
%
It would also be nice if the blocks chosen for writing also maximized the
disk's commit rate, by minimizing seeks and so forth.

A naive implementation might calculate, for each in-memory block $b$, the
largest set of \patches\ $P_b \subseteq \PMem[b]$ with $\PDepset{P_b}
\subseteq P_b \cup \PDisk$, then evict some block close to previously
written blocks and with few rolled-back \patches\ (where $\PMem[b] - P_b$
is small).
%
This, however, would be extraordinarily expensive.
%
Finding $P_b$ requires traversing a dependency graph which might contain
thousands and thousands of nodes.
%
Doing so for each block, once per eviction, would take huge amounts of CPU
time.


\textbf{Rollback memory usage.}
%
Only a small fraction of \patches\ will ever need to be rolled back.
%
For example, most data writes never need to be rolled back in any file
system.
%
If a \patch\ won't be rolled back under any circumstances, the memory and
CPU time spent to preserve the old version is wasted.


\textbf{\Patch\ memory usage.}
%
Patches themselves take up memory and require time to allocate, free, and
traverse.
%
If two \patches\ have redundant dependencies, it would be faster to combine
them.


%% \textbf{Dependency memory usage.}
%% %
%% The $\PDDepset{}$ sets are stored as doubly linked lists; each individual
%% dependency takes up memory.
%% %
%% Important and common dependency relations require many dependencies to
%% express; for example, if \patches\ $p_1,\dots,p_n$ depend, as a group, on
%% \patches\ $q_1,\dots,q_n$, expressing this constraint would require $n^2$
%% total dependencies.


The next section tackles all of these challenges.
\end{comment}


\begin{figure}
\centering

\includegraphics[scale=0.62]{fig/opt_1}

\textbf{a)} Before optimizations

\vskip.5\baselineskip

\includegraphics[scale=0.62]{fig/opt_2}

\textbf{b)} With \nrb\ \patches

\vskip.5\baselineskip

\includegraphics[scale=0.62]{fig/opt_3}

\textbf{c)} With \nrb\ patch merging

\vskip.5\baselineskip

\includegraphics[scale=0.62]{fig/opt_4}

\textbf{d)} With overlap merging

\caption{\Patches\ required to append 4 blocks to an existing file, without
and with optimization.  \Nrb\ \patches\ are shown with heavy borders.}
\label{f:opt}
\end{figure}

\subsection{\Nrb\ \ChDescs}
\label{sec:chdescs:nrb}

Each data\todo{Name?} \chdesc\ contains a copy of its block's previous data to
allow rollback\footnote{Actually, \Kudos\ supports a specialized type of \chdesc\ for
efficiently flipping individual bits using an inline exclusive-or mask instead
of a copy of the previous data, but most \chdescs\ are not of this type.}.
%
In practice, many \chdesc\ are never actually rolled back (e.g. file
data blocks)
%
and the previous data copies nearly double the combined memory usage of
\chdescs\ and cached blocks.
%
To avoid this overhead, \Kudos\ identifies \chdescs\ that will never
need to be rolled back and omits their previous data copies. We call
these \emph{\nrb} \chdescs. (The opposite naturally being a \emph{\rb}
\chdesc, when necessary to differentiate them.)
%
Since a \nrb\ \chdesc\ cannot be rolled back, a write of any \chdescs\
on block $B$ must include all \nrb\ \chdescs\ on $B$. To accordingly
update our formal model we define a new set of \chdescs, \ChNrb, which
contains all \nrb\ \chdescs. We write \ChNrbB{B} to restrict the set
to block $B$\todo{Introduce \ChRb\ and \ChRbB{B}.}:

\paragraph{Write block to disk controller}
For some block $B$: \\
Let \(P \subseteq \ChMemB{B}\) s.t.
\(\BeforeS{P} \subseteq \ChDisk \cup P\) and \(\ChNrbB{B} \subseteq P\) \\
Set \p{p}.state $:=$ \stateinf\ for all \inset{p}{P}

\paragraph{}
To avoid (expensive) dependency traversals to determine whether a new
\chdesc\ will need to be rolled back to write \ChAll,
%
\Kudos\ conservatively identifies \nrb\ \chdescs\ using only local
dependency information.
%
\Kudos\ detects that a new \chdesc\ may need to be rolled back if any
\chdescs\ already on the block have external (on a different
block\todo{Descriptive enough? Mention \noop\ \chdesc\ \after\ recursion?})
\afters.
%
This is both a safe and useful indicator because
%
the presence of an external \after\ is a necessary condition for a new
\chdesc's \before\ to induce a block-level cycle
%
and many blocks have no \chdescs\ with external \afters\ (e.g. most
file data blocks).

While this algorithm detects whether a \chdesc\ may need to be rolled
back to write \ChAll, \Kudos\ must also be sure that no future
dependency manipulation will cause the \chdesc\ to require a rollback.
%
We introduce Invariant~\ref{cdinvar:add-before} to support such reasoning:
%
\cdinvar{add-before}{All block-level cycles induced through
\chdesc\ \p{p}'s \befores\ exist when \p{p} is
created\todo{Change this phrasing? ``Once created, a \chdesc\ will not
gain any \befores\ that induce block-level cycles.''}.}
%
\noindent \Kudos\ ensures this invariant by restricting \before\
additions to \chdesc\ creation, \noop\ \chdescs\ with no \afters, or
when the invariant is statically proven to hold for the affected
\chdescs.

% -*- mode: latex; tex-main-file: "paper.tex" -*-

\subsection{\Nrb\ \Patch\ Merging}
\label{sec:patch:merge}

File operations such as block allocations, inode updates, and directory updates
create many distinct \patches. Keeping track of these
\patches\ and their dependencies requires memory and
CPU time.
%
\Kudos\ therefore \emph{merges} \patches\ when possible, drastically reducing
\patch\ counts and memory usage, by conservatively identifying when a
new \patch\ could always be written at the same time as an existing \patch.
%
Rather than creating a new \patch\ in this case, \Kudos\ updates the data
and dependencies so as to merge the new \patch\ into the existing one.


Two types of patch merging involve hard patches, and the first is trivial
to explain:
%
since all of a block's \nrb\ \patches\ \emph{must} be written at the same
time, they can always safely be merged.
%
\Kudos\ thus ensures that each block contains at most one \nrb\ \patch.
%
If \Kudos\ detects that a new patch $p$ could be created as \nrb\ and $p$'s
block already contains a \nrb\ \patch\ $h$, then
%
the implementation merges $p$ into $h$ by applying $p$'s data to the block
and setting $\PDDepset{h} \gets \PDDepset{h} \cup \PDDepset{p}$.
%
The existing \nrb\ \patch\ $h$ is returned to the caller.
%
This changes $h$'s direct dependency set after $h$'s creation time, but
since $p$ could have been created \nrb, the change cannot introduce any new
block-level cycles.
%
Unfortunately, the merge can create \emph{intra}-block cycles.
%
If some \noop\ \patch\ $e$ has $p \PDepend e \PDepend h$, then after
the merge $h \PDepend e \PDepend h$.
%
\Kudos\ detects and prunes these cyclic
dependencies as $p$'s dependencies are merged into $h$.


\Nrb\ \patch\ merging is able to eliminate 8 of the \patches\ in our running
example, as shown in Figure~\ref{fig:opt}c.


Second, \Featherstitch\ detects when a new \nrb\ patch can be merged with
 a block's existing \emph{\rb} \patches.
%
Block-level cycles may force a patch $p$ to be created as soft.
%
Once those cycles are broken (because the relevant patches commit), $p$
 could be converted to hard; but to avoid unnecessary work,
%
\Kudos\ delays the conversion, performing it only when it detects that a
 new patch on $p$'s block could be created \nrb.
%
Figure~\ref{f:soft2hard} demonstrates this scenario using soft updates-like
 dependencies.


Specifically, consider a new \nrb\ \patch\ $h$ added to a block that
contains some \rb\ \patch\ $p$.
%
Since $h$ is considered to overlap $p$, \Kudos\ adds a direct dependency
$h \PDDepend p$.
%
Since $h$ could be \nrb\ even including this overlap dependency, we know
there are no block-level cycles with head $h$.
%
But as a result, we know that there are no block-level dependency cycles
with head $p$, since any such cycle $p \PDepend \cdots \PDepend p_1$ would
imply an $h$-headed cycle $h \PDDepend p \PDepend \cdots \PDepend p_1$.
%
Thus, $p$ can be transformed into a \nrb\ \patch.  \Kudos\ performs this
transformation and merges $p$ and $h$ via \nrb\ \patch\ merging.


\begin{figure}
\centering
\begin{small}
\begin{tabular}{@{}p{.32\hsize}@{~~}p{.32\hsize}@{~~}p{.32\hsize}@{}}
\centering \includegraphics[width=.93\hsize]{fig/softhard_1} &
\centering \includegraphics[width=.93\hsize]{fig/softhard_2} &
\centering \includegraphics[width=.93\hsize]{fig/softhard_3} \cr
\centering \textbf{a)} Block-level cycle &
\centering \textbf{b)} $d_1$ commits &
\centering \textbf{c)} After merge
\end{tabular}
\end{small}
\caption{\Rb-to-\nrb\ patch merging.  \textbf{a)} Soft updates-like
dependencies among directory data and an inode block.  $i \PDDepend d_1$
because $d_1$ deletes a file whose inode is on $i$; $d_2 \PDDepend i$
because $d_2$ allocates a file whose inode is on $i$. \textbf{b)} Writing
$d_1$ removes the cycle. \textbf{c)} $d_3$, which adds a hard link,
initiates soft-to-hard merging.}
\label{f:soft2hard}
\end{figure}



\begin{comment}

\Kudos\ includes three distinct \patch\ merge algorithms.
%
All three use Invariant~\ref{cdinvar:add-before} to reason about future
graph changes and use fast, conservative checks during \patch\ creation;
they differ in their applicable conditions.


\subsubsection{\Nrb\ \Patch\ Merging}
\label{sec:patches:merge:nrb}

Recall from Section~\ref{sec:patches:nrb} that a write of any \patches\ on
block $b$ must include all \nrb\ \patches\ on $b$.
%
This additional requirement is in fact an exquisite optimization
opportunity; it implies that all \nrb\ \patches\ on a given block can
be merged.
%
Further, merging can remove the need for the \nrb\ \patch\ implicit
dependency rules by ensuring that
%
there is at most one \nrb\ \patch\ per block (\nrb-\nrb\ merging)
%
and that all \rb\ \patches\ on a given block depend on the \nrb\ \patch\
(\nrb-\rb\ merging).
%
We describe these two \patch\ merging algorithms and how they
preserve dependency semantics in this section.

\paragraph{\Nrb-\Nrb\ \Patch\ Merging}
\label{sec:patches:merge:nrb:hard-hard}

\emph{\Nrb-\nrb\ \patch\ merging} merges a new \nrb\ \patch\ \p{q}
into an existing \nrb\ \patch\ \p{p} instead of creating \p{q}.
%
Any two \nrb\ \patches\ on the same block may be (and are) merged.
%
Merging all \nrb\ \patches\ ensures:
%
\cdinvar{one-nrb}{\(\forall\! b\!: |\PHard[b]| \leq 1\)}
%
\noindent
%
Invariant~\ref{cdinvar:one-nrb} simplifies \nrb\ \patch\ handling by
%
removing the implicit dependencies that ensure all \nrb\ \patches\
are written together
%
and by removing the need to scan for an existing \nrb\ \patch\ when
\nrb-\nrb\ \patch\ merging.
%
% Although merging two \patches\ will not induce block-level dependency
% cycles, without sufficient care merging could induce \patch-level
% dependency cycles.  A trivial example is merging \p{q} into \p{p} when
% \p{q} has an explicit dependency on \p{p}; the combined \p{(p+q)}
% should not and need not depend on itself.
%
To preserve dependency semantics, the merged \p{(p+q)} must depend on
the union of \p{p} and \p{q}'s transitive \befores. Additionally, while the
\patches\ can be merged without forming a \patch-level dependency cycle,
the merge must ensure that it does not introduce a needless cycle, e.g.
through \anoop\ \patch\ \p{e} in \depends{q}{\depends{e}{p}}
\todo{Is cycle avoidance worth mentioning? Is this a good way to mention it?}.

From Invariant~\ref{cdinvar:add-before} and the \nrb\ \patch\
creation condition (no external \afters), the only possible
dependencies involving \p{p} and \p{q} are those shown in
Figure~\ref{fig:nrb-merge}\todo{Should we give these deductions or a
  flavor?}.
%
Notice, for example, that any \p{r} such that
\indirdepends{q}{\indirdepends{r}{p}} is \anoop\ \patch\todo{This is
  a strong statement. Expand on its implications?}.
%
Our algorithm to transform dependencies for \nrb-\nrb\ \patch\ merges
(Figure~\ref{algo:merge:hard-hard}) follows from these possible
dependencies.
%
It updates \p{p}'s transitive \befores\ to ensure
\(\PDepset{q}\todo{Incorrect! Only the true \patches.} \subseteq
\PDepset{p}\)\todo{Note that Invariant~\ref{cdinvar:add-before}
  ensures that \noop\ \patches\ reachable from \p{q} will not gain
  data \patch\ \befores?}\todo{Note that it only needs to move dependencies?}.

\begin{figure}[htb]
  \centering
  \includegraphics[width=\columnwidth]{nrb_merge}
  \caption{Possible dependencies when merging \nrb\ \patch\ \p{q}
    into existing \nrb\ \patch\ \p{p}.}
  \label{fig:nrb-merge}
\end{figure}

\noindent Algorithm called on \p{q} and \p{p}:\\
Input: \patch\ \p{a} and existing \nrb\ \patch\ \p{p}.\\
Returns: whether \indirdepends{a}{p} exists. \(\forall\! \p{b}\!: \indirdepends{a}{b}\) and \notindirdepends{b}{p}, creates \indirdepends{p}{b}.

\begin{itemize}
\item If \p{a} is external, return ``no path to \p{p}.''
\item If \p{a} equals \p{p}, return ``path to \p{p}.''
\item Call self on \p{a} and \p{p}.
\item If \p{a} has no path to \p{p}, return ``no path to \p{p}.''
\item For each \p{a} \before\ \p{b}:
  \begin{itemize}
  \item If \p{b} has no path to \p{p}:
    \begin{itemize}
    \item Move \p{b} from a \before\ of \p{a} to a \before\ of \p{p}.
    \end{itemize}
  \end{itemize}
\end{itemize}

\paragraph{\Nrb-\Rb\ \Patch\ Merging}
\label{sec:patches:merge:nrb:hard-soft}

When creating the first \nrb\ \patch\ on a block, \emph{\nrb-\rb\
  \patch\ merging} merges all existing (\rb) \patches\ into the new
\nrb\ \patch.
%
Such an arrangement can arise through a combination of \patch\ creates
and block writes; 
%
for example, the block may first obtain an initial (\nrb{}) \patch,
%
then gain external \afters\ on its \patch,
%
accumulate additional (\rb{}) \patches,
%
write the subset of its \patches\ with external \afters\ (leaving some
\rb\ \patches\ on the block),
%
and then gain a \nrb\ \patch.
%
In addition to reducing the number of data \patches, \nrb-\rb\
\patch\ merging removes the second implicit \nrb\ \patch\
dependency, that \rb\ \patches\ not explicitly dependent on the
block's \nrb\ \patch\ implicitly depend on it.
%
As in \nrb-\nrb\ \patch\ merging, \Kudos\ merges such \patches\ to
avoid the complications of their implicit dependencies.

\Nrb-\rb\ \patch\ merging's implementation first merges all \rb\ \patches\
into a \nrb\ \patch\ and then \nrb-\nrb\ \patch\ merges the new \nrb\
\patch\ into the now-existing \nrb\ \patch.
%
Our algorithm to transform the dependencies for \nrb-\rb\ \patch\
merges (Figure~\ref{algo:merge:hard-soft}) for block $b$
%
chooses a \patch\ \p{p} such that
\(\notexists \inset{q}{\PMem[b]}\!: \indirdepends{p}{q}\)
%
and updates its transitive \befores\ to ensure
\(\PDepset{\PSoft[b]} \subseteq \PDepset{p}\).
%
Because any \(\inset{q}{\PMem[b] - p}\) may have \afters, to
preserve dependencies we convert such a \p{q} into \anoop\ \patch\
and ensure \depends{q}{p}.

\noindent Algorithm:
\begin{itemize}
\item Choose a \(\inset{p}{\PMem[b]}\!:\
\notexists\! \inset{q}{\PMem[b]}\!:\ \indirdepends{p}{q}\).
\item For each \inset{q}{\PMem[b] - p}:
  \begin{itemize}
  \item Call the \nrb-\nrb\ \before\ move algorithm on \p{q} and \p{p}.
  \item Convert \p{q} into \anoop\ \patch.
  \end{itemize}
\item Convert \p{p} into a \nrb\ \patch\ (free it's previous data copy).
\end{itemize}

\todo{Note that \nrb-\rb\ merging is rare? Note why it is helpful even though
it is rare?}
\todo{Explain why this preserves dependency semantics? Show possible
dependencies? For the paper, free \patches\ instead of convert them
into \noop{}s? (Must modify \nrb-\nrb\ algo usage.)}

\end{comment}


\subsection{Overlap Merging}
\label{sec:patches:merge:overlap}

The final type of merging merges \rb\ \patches\ with other \patches,
\nrb\ or \rb, when they overlap.
%
Bitmap blocks, inodes, and directory entries accumulate many nearby
and overlapping \patches\ as data is appended to or truncated from a
file and as files are created and removed.
%
Figure~\ref{fig:opt} shows how even data blocks can collect overlapping
dependencies: actual data writes $d'_j$ overlap, and therefore depend on,
block initialization writes $d_j$, but cannot be made \nrb\ since when they
are created another block (the inode) already depends on the data block.
%
Luckily, simple reasoning can identify many mergeable pairs,
further reducing the number of \patches\ and the amount of undo data
required.


Two overlapping \patches\ $p_1$ and $p_2$, with $p_1 \PDepend p_2$, may be
merged iff it would always be possible to write them at the same time.
%
Here we may reuse the reasoning developed for \nrb\ \patches\ above: it is
always possible to write these \patches\ simultaneously if, assuming that
$p_2$ were \nrb, $p_1$ could also be made \nrb---that is, if $p_1$ will
never be the head of a block-level cycle terminating at $p_2$.
%
The same properties that simplified the creation of \nrb\ \patches\ also help
us check this property: that is, if no block-level cycle $p_1 \PDepend x
\PDepend p_2$ exists when $p_1$ is created, then no such block-level cycle
will ever exist.


As with \nrb\ \patch\ creation, the \Kudos\ implementation checks a simpler
property that requires less graph traversal.
%
It checks that every path starting at $p_1$ fits at least one of the
following cases:

%byte overlap selection details, in case we want to describe them:
% - if overlap one other byte patch, it is target
% - if overlap two byte patches and one is the hard, non-hard is target
% - else fail

%bit overlap merge details, in case we want to describe them:
% try to merge into bits if there are inram bit changes in this word:
% overlap:
% - if overlap one other bit patch (bit-wise), it is target
% - if overlap one other bit patch (word-wise), it is target
% - else fail
% if target overlaps a byte patch, fail
%
% dependency cycle check, look at the single p->q:
% if add_overlap_depend_head_is_ok(target, q), merge
% else fail
% 
% else if there are no bit changes in this word:
% - if there is a hard patch, it is the only patch, and
%    add_overlap_depend_head_is_ok(hard, q), merge into hard
%
% function add_overlap_depend_head_is_ok(overlap, head)
% - if !head, head = overlap, or head notin ram, return true
% - if |Deps(overlap)|, return true
% - if head in Deps(overlap), truen true (check first two befores)
% - if |Deps(head)| = 0, return true
% - let Y = { y | exists x, y: target->x->y and q->y } (branch at most 2, |Y| <= 3)
%   then foreach z in Deps(q) - Y:
%   - if z = target, return false
%   - if Deps(z) > 1, return false
%   - if Deps(z) !<= Y, return false
% return true

\begin{xcompactitemize}
\item $p_1 \PDDepend p_2$.
\item $p_1 \PDDepend h$, where $h$ is the \nrb\ \patch\ on $\PBlock{p_1}$.
\item $p_1 \PDDepend q$, where $q \not\in \PMem$.
% This rule is covered by the next (but we do do this check for speed)
%\item $p_1 \PDDepend q$, where $q$ depends on no other \patch.
\item $p_1 \PDDepend q$, where $p_2 \PDDepend q$.
\item $p_1 \PDDepend q$, where $\PDepset{q} \subseteq \PDepset{p_2}$
  and $|\PDepset{q}| \leq 2$.
\item Has length at most 10, traverses no node with more than 10
  direct dependencies, and does not traverse $p_2$.\todo{Yuck\ldots}
\end{xcompactitemize}

\noindent
%
If all paths fit, then there are no block-level cycles from $p_1$
to $p_2$, $p_1$ and $p_2$ can have the same lifetime, and $p_1$ can be
merged into $p_2$ where they overlap.
%
(This may require growing $p_2$ to cover $p_1$'s data range.)
%
It also simplifies the implementation somewhat to limit overlap
merging to the case when $p_1$ does not overlap with any other
\patches, and to limit $p \PDDepend q$ existence checks to just $p$'s
and $q$'s two oldest and newest dependencies.
%
The rules to overlap merge two bit \patches\ are similar.



In our running example, overlap merging is able to combine all remaining
\rb\ \patches\ with their \nrb\ counterparts, reducing the number of \patches\
to the minimum of 8 and the amount of undo data to the minimum of 0.
%
In our experiments, we observe that \nrb\ \patches\ and our \patch\ merging
optimizations reduce the amount of memory allocated for undo data in
soft updates and journaling orderings by \patchoptundo.


\begin{comment}
%
If the only dependency between $p_1$ and $p_2$ is direct---that is, no path
$p_1 \PDepend x \PDepend p_2$ exists for any $x \not\in \{p_1,
p_2\}$---then it will always be possible to write $p_1$ and $p_2$ at the
same time.
%
Specifically, it is possible to write $p_1$ 


Many of these and similar \patches\ are mergeable and have
dependencies that allow simple (and fast) reasoning to identify many
of the mergeable pairs: two \patches\ on block $b$ that overlap no other \patches\ in \PMem[b]
and which have no dependency path from the new to the existing \patch\
will not induce a block-level cycle and so are writable together.
We know that \textit{later} changes will not cause them to induce a block-level cycle due to
invariant~\ref{cdinvar:add-before} and by not merging if the new \patch\
has a before and the before is marked as allowed to violate
invariant~\ref{cdinvar:add-before}.
%
While path existence testing is expensive, a conservative path test
of only a depth of two identifies most mergeable \patches. If the new
\patch\ has an explicit \before\ that is not the existing \patch\ and
this \before\ has a \before, then there may be a path to the existing
\patch.
%
To merge two such overlapping \patches, add the new \patch's explicit
before to the existing \patch\ (if any and if not the existing \patch).


%%

At the end of \patch\ optimizations, say something along the lines:
%
The dynamic optimizations facilitated through \nrb\
\patches\ implement the efficiency in systems using soft updates or
journaling\todo{Actually do this for journaling} while expressing
changes modularly through structural descriptions rather than through
internal and semantic file system descriptions.

\todo{Should we talk about why we allow NRBs and merging to be
  disabled? (Debugging simplicity and depend add to \noop\ \patches\
  with \afters\ bug catching.)}
\end{comment}

\subsection{Ready \ChDesc\ Lists}

For a \module\ like the write-back cache to forward \chdescs\ in a
dependency-preserving order, the \module\ must find \chdescs\ whose \befores\
are all ``closer to the disk'' (or are also being forwarded as part of the same
block write). We say that such \chdescs\ are \emph{ready}. Because the
write-back cache frequently searches for and writes many ready \chdescs,
redundant \chdesc\ graph traversals to calculate \chdesc\ readiness would
severly limit cache size scalabity. \Kudos\ therefore explictly tracks a
\chdesc's \before\ counts through incremental dependency updates, and uses these
counts to maintain a \chdesc\ ready list for each block.

Each \chdesc\ has a count of the number of \befores\ it has at block device
modules just as close to the disk as it currently is, and a count of the number
of \befores\ it has which are in flight. When these counts are both zero, it is
ready. A \chdesc's \before\ counts are incrementally updated as \befores\ are
added and removed and as \beforing\ \chdescs\ are moved closer to the disk.

Because \Kudos\ makes sure that the \befores\ of a \chdesc\ are at least as
close to the disk as it is, only directly reachable \beforing\ \chdescs\ need to
be included in a \chdesc's \before\ counts. \Noop\ \chdescs, with the exception
of managed \noop\ \chdescs\ (which have an explicit owning block device), add a
wrinkle to this simplifying rule, however. They are considered to be as close to
the disk as their \before\ which is the farthest from the disk, in effect,
propagating the distance to the disk metric through them.

When a \before\ count update changes whether a \chdesc\ is ready to write, the
\chdesc's inclusion in its block's ready list is updated. To write a block, a
\module\ thus iterates through the block's ready list, sending \chdescs\ to the
target block device, until the list is empty. Thus instead of having to
repeatedly traverse \chdesc\ graphs to determine readiness on demand, we have
this information maintained automatically as it changes. This automatic
maintenance adds some cost to forwarding \chdescs\ and changing the graph
structure, but since it saves so much duplicate work\footnote{The amount of
duplicate work saved is actually superlinear in the size of the write-back
cache.} it is much more efficient.

% -*- mode: latex; tex-main-file: "paper.tex" -*-

\subsection{Other Optimizations}
\label{sec:patch:misc}

Even with these optimizations, there is only so much
that can be done with bad sets of dependencies.
%
Just as having too few dependencies can compromise system correctness,
having too many dependencies, or the wrong dependencies, can non-trivially
degrade the system performance.
%
For example, in both the following \patch\ arrangements, $p$ depends on all
of $q$, $r$, and $s$, but the left-hand arrangement gives the system more
freedom to reorder block writes:

\begin{figure}[htb]
\centering
\begin{tabular}{@{}p{.45\hsize}p{.45\hsize}@{}}
\centering \includegraphics[width=.5\hsize]{fig/figures_4} &
\centering \includegraphics[width=.5\hsize]{fig/figures_6}
\end{tabular}
\vskip-\baselineskip
\end{figure}

\noindent%
For example, if $q$, $r$, and $s$ are adjacent on disk, the left-hand
arrangement can be satisfied with two disk requests while the right-hand
one will require four.
%
Although the arrangements have similar coding difficulty, in several cases
we discovered that one of our file system implementation was performing
slowly because it created an arrangement like the one on the right.

Care must also be taken to avoid unnecessary \emph{implicit} dependencies,
and in particular overlap dependencies.
%
For instance, inode blocks contain multiple inodes, and changes to two
inodes should generally be independent; a similar statement holds for
directories and even sometimes for different fields in a summary block like
the superblock.
%
The best results can be obtained from \emph{minimal} \patches\ that change
one independent field at a time; \Kudos\ will merge these \patches\ when
appropriate, but if they cannot be merged, minimal \patches\ will lead to
fewer rollbacks and more flexibility in write ordering.


%% the changes to update a field in an inode structure
%% on disk, a \patch\ spanning the entire inode could be created even though
%% only a single field changed. A later change to a different field in the
%% same inode would appear to overlap the first, possibly creating an
%% unnecessary dependency. Creating \patches\ that correspond more precisely
%% to the changes being made avoids this problem, so a utility function is
%% provided by \Kudos\ to make this operation as convenient as creating a
%% single \patch\ for an entire large structure.

%% \subsubsection{\Patch\ List Ordering}

The buffer cache and a few other \modules\ perform better in the
common case that the \patches\ on a block are listed in order of creation
time.
%
This improves their performance from $O(n^2)$ to $O(n)$ in the number of
\patches.
%% in the common case that this order is preserved.

Finally, block allocation policies can have a dramatic effect on the number of
I/O requests required to write changes to the disk. For instance, with
dependencies implementing soft updates, indirect block data cannot be written
until the referenced blocks have been at least initialized. By allocating an
indirect block in the middle of a range of data blocks for a file, the data
blocks must be written as two smaller I/O requests since the indirect block
cannot be written at the same time. Allocating the indirect block somewhere
else allows the data blocks to be written in one larger I/O request.


\begin{comment}
Several functions in \Kudos\ iterate over lists of \patches\ looking for either
a single \patch\ or set of \patches\ satisfying some property, or trying to
process all the \patches\ in the list in some order determined by the dependency
graph. It is generally the case that the \patches\ satisfying the property or
the order in which the \patches\ should be processed can be determined very
quickly by keeping the lists sorted. For instance, the library function which
rolls \patches\ back needs to perform the rollbacks essentially in inverse
creation order, so that rolling back a \patch\ which has since been overwritten
by a later \patch\ does the right thing. Keeping the list of all \patches\ on a
block sorted in creation order (which is very easy) makes this an efficient
operation, while it might otherwise take $O(n^2)$ time to execute. Similarly,
many \patch\ merging functions need to find for a given block some \patch\
which has no \befores\ on the same block, and the oldest \patch\ on a block
always satisfies this requirement.
\end{comment}

