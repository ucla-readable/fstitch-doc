% -*- mode: latex; tex-main-file: "paper.tex" -*-

\paragraph{Journaling}
\label{sec:using:journal}

\begin{comment}
% This paragraph is basically replicated in sec:patch:examples
In a journaling file system, changes to disk structures are grouped into
\emph{transactions} that commit atomically.
%
Any change in a transaction is first copied into an on-disk journal.
%
A single \emph{commit record} block is written to the journal once the
transaction's changes are stably committed there.
%
This record commits the transaction itself, allowing its changes to be
written into the main body of the file system in any order.
%
If the system crashes, the journal information is used to recover the main
file system from its possibly-incomplete state.
%
Once all the transaction's changes are written, a \emph{completion record}
is written to the journal to mark the transaction as complete; that portion
of the journal may now be reused.
\end{comment}

The \Kudos\ journal module sits below a regular file system and transforms
incoming \patches\ into journal transactions.
%
Data \patches\ are copied into the journal; a commit record depends on the
journal \patches; and the original file system \patches\ depend in turn on the
commit record.
%
The dependencies among the original \patches\ are removed since the journal
itself provides consistency for each high-level file system operation by
replaying outstanding transactions on recovery.
\footnote{However, it does take care to ensure that the user-specified
dependencies described in \S\ref{sec:patchgroup} are not violated.}
%
The journal format is similar to ext3's~\cite{tweedie98journaling}: a
transaction contains a list of block numbers, the data to be written to
those blocks, and finally a single commit record.
%
%Figure~\ref{fig:journal} shows how the journal module transforms the
%\patches\ in Figure~\ref{fig:softupdate} into journaling semantics.
%
Note that the journal must modify existing \patches' direct dependencies;
this is allowed since the new dependencies will never introduce a
block-level cycle.

Like ext3, transactions are required to commit in sequence. Therefore the
journal \module\ sets each commit record to depend on the previous, and each
completion record to depend on the previous completion record. This allows
multiple outstanding transactions in the journal, which benefits performance,
but ensures that in the event of a crash, the journal will contain only
contiguous sequential transactions.

Since the commit record is created at the end of the transaction, the journal
\module\ uses a special \noop\ \patch\ explicitly held in memory to prevent
file system changes from being written to the disk until the transaction is
complete. This \noop\ \patch\ is set to depend on the previous transaction's
completion record, which prevents merging between transactions while allowing
merging within a transaction. This temporary dependency is removed when the
real commit record is created and its dependencies are set up as described
above.

\begin{comment}
Almost all of this description of journaling translates directly into \patch\
dependencies. The incoming \patches\ must be rearranged to implement the new
structure, but for the most part this transformation is straightforward. There
are two special considerations which the journal \module\ must address, however.

First, with soft updates, \patches\ can always be written to the disk when
flushing the cache, while the journal must be able to ``lock'' \patches\ into
the cache while transactions are in progress. To accomplish this, the journal
\module\ uses a managed \noop\ \patch, as outlined in
Section~\ref{sec:patch:noop}.

Second, the commit record is created at the end of the transaction, but the file
system changes created during the transaction must be made to depend on it.
Ordinarily this is not permitted (see \S\ref{sec:patch:nrb}). The condition for
violating this rule is a static proof that no cycle can result from doing so
(immediately or in the future); we have determined this to be the case for the
journal \module\ by hand.
\end{comment}

\begin{comment}
Due to this design, the journal \module\ is completely independent of any
specific file system. It is a block device \module\ that automatically journals
whatever file system is stored on it. In fact, the incoming \patches\ need not
be arranged for soft updates, or for that matter in any particular configuration
at all.
\end{comment}

% Is it important to specify how we figure out where transaction boundaries
% are? It seemed confusing to one reviewer due to this section preceeding the
% modules section.

Our journal module prototype can run in full data journal mode, where every
updated block is written to the journal, or in metadata-only mode, where only
blocks containing file system metadata are written to the journal. It can
tell which blocks are which by looking for a special flag on each \patch.
