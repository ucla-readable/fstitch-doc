% -*- mode: latex; tex-main-file: "paper.tex" -*-

\subsection{Journal}
\label{sec:modules:journal}

The journal module sits below a regular file system, such as ext2, and transforms
incoming \patches\ into patches implementing journal transactions.
%
File system blocks are copied into the journal; a commit record depends on the
journal \patches; and the original file system \patches\ depend in turn on the
commit record.
%
Any soft updates-like dependencies among the original \patches\ are removed,
since they are not needed when the journal handles consistency; however, the
journal does care to ensure that user-specified dependencies, such as
\patchgroups, are not violated.
%
%% itself provides consistency for each high-level file system operation by
%% replaying outstanding transactions on recovery.
%% \footnote{However, it does take care to ensure that the user-specified
%% dependencies described in \S\ref{sec:patchgroup} are not violated.}
%
The journal format is similar to ext3's~\cite{tweedie98journaling}: a
transaction contains a list of block numbers, the data to be written to
those blocks, and finally a single commit record.
%
Although the journal modifies existing \patches' direct dependencies, it
ensures that the new dependencies will never introduce a block-level
cycle.

Like ext3, transactions are required to commit in sequence. Therefore the
journal \module\ sets each commit record to depend on the previous commit record, and each
completion record to depend on the previous completion record. This allows
multiple outstanding transactions in the journal, which benefits performance,
but ensures that in the event of a crash, the journal will contain only
contiguous sequential transactions.

Since the commit record is created at the end of the transaction, the journal
\module\ uses a special \noop\ \patch\ explicitly held in memory to prevent
file system changes from being written to the disk until the transaction is
complete. This \noop\ \patch\ is set to depend on the previous transaction's
completion record, which prevents merging between transactions while allowing
merging within a transaction. This temporary dependency is removed when the
real commit record is created and its dependencies are set up as described
above.

%Due to this design, the journal \module\ is completely independent of any
%specific file system. It is a block device \module\ that automatically journals
%whatever file system is stored on it. In fact, the incoming \patches\ need not
%be arranged for soft updates, or for that matter in any particular configuration
%at all.

% Is it important to specify how we figure out where transaction boundaries
% are? It seemed confusing to one reviewer due to this section preceeding the
% modules section.

Our journal module prototype can run in full data journal mode, where every
updated block is written to the journal, or in metadata-only mode, where only
blocks containing file system metadata are written to the journal. It can
tell which blocks are which by looking for a special flag on each \patch\ set
by the UHFS module.

We provide several other modules that modify dependencies, including an
``asynchronous mode'' module that removes all dependencies, allowing the
buffer cache to write blocks in any order.
%
This implements similar semantics as existing file systems like ext2 in
asynchronous write mode.
