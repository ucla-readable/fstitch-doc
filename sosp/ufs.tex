\subsection{UFS}
\label{sec:modules:ufs}

\Kudos\ supports the version of UFS (Unix File System, the modern incarnation of
the Fast File System~\cite{mckusick84fast}) used in 4.2 BSD. UFS is of
particular interest because it is the only file system that has been extended
with both soft updates and journaling.~\cite{seltzer00journaling} We chose UFS1
over UFS2, as UFS1 is well established and more widely supported.

The UFS \module\ provides a good demonstration of some of the flexibility of the
\module\ interfaces in \Kudos. For instance, UFS uses 2KB \emph{fragments} to
store small files efficiently. Once a file gets big enough to require the use of
indirect blocks, UFS changes its allocation policy and starts allocating 16KB
\emph{blocks}, where a block is made up of 8 aligned and contiguous fragments.
Our UFS \module\ implements this by using fragments as the basic block size. For
large files, the UFS \module\ internally allocates a block, but returns only the
first fragment in that block at the \LFS\ level. The next 7 allocation calls
will simply return the subsequent fragments in the already allocated block. In
this way, the UFS \module\ can stay consistent internally without special
support from other \Kudos\ modules.
