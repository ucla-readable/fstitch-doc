\begin{abstract}

% The abstract will call them "change descriptors" always.
We propose a file system implementation architecture, called \emph{\Kudos},
where \emph{change descriptor} structures represent any and all changes to
stable storage.
%
%%  File systems generate change descriptors for all writes, then
%% send them to block devices for eventual commit. Explicit dependencies between
%% change descriptors let \Kudos\ \modules\ preserve necessary file system
%% invariants without understanding the file system itself. Change descriptors can
%% implement many consistency mechanisms, including soft updates and journaling.
%
\Kudos\ is decomposed into fine-grained \modules\ which generate, consume,
 forward, and manipulate change descriptors. 
%
The uniform change descriptor abstraction allows modules to impose and
 follow arbitrary file system consistency policies: a collection of
 loosely-coupled modules cooperates to implement strong and possibly
 complex guarantees, even though each individual module does a relatively
 small part of the work.
%
%% A particular innovation of the
%% \module\ design is the separation of the low-level specification of on-disk
%% layout from higher-level file system-independent code, which operates on
%% abstract disk structures. 
%
For example, by observing and modifying \chdesc\ constraints, our
 journaling \module\ can automatically add journaling to any file system.
%
Additionally, a new system call interface gives applications some direct
 control over change descriptors. We have used this interface to
improve the UW IMAP server, removing inefficient and unnecessary calls to
\texttt{fsync()} while preserving the integrity of mail messages.
%
We have implemented \Kudos\ as a Linux kernel module. Our current
implementation is competitive with FreeBSD soft updates for number of
blocks written, and allows several novel configurations like ext2 with
soft updates or correct UFS soft updates over a loopback device.

\end{abstract}
