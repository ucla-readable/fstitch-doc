\begin{abstract}

File system data consistency inherently relies on careful write ordering,
 where one block cannot be written until another is safely committed to
 stable storage.
%
Previous file systems have enforced write orderings in system-dependent
 ways, either with rules specialized for each file system
 structure~\cite{ganger00soft} or with a journal, which enforces a
 particular consistency protocol~\cite{xxx}.
%
We present a general \emph{\chdesc} abstraction that can represent any
 write ordering in a file system agnostic manner.
%
A \chdesc-based file system implementation expresses dependencies among
 writes, but not how those dependencies should be enforced.
%
Each file system \module\ can examine, preserve, and even modify write
 orderings, as we demonstrate with a \module\ that automatically changes
 soft-updates orderings into journal transactions.
%
Generalized file system dependencies are naturally exportable to user
 level, allowing applications to specify their own consistency protocols
 for the file system to follow.

We present the \chdesc\ abstraction, describe a number of important
 optimizations for \chdesc-based file systems, and present a Linux kernel
 implementation of a storage subsystem that uses \chdescs\ to enforce
 consistency.
%
Our prototype is competitive with FreeBSD soft updates and ext3 journaling
 [[??]] for number of blocks written, and allows several novel
 configurations, such as ext2 with soft updates or correct dependency
 enforcement within a file system image file.


\begin{comment}

We propose a file system implementation architecture, called \emph{\Kudos},
where structures called \emph{\chdescs} represent any and all changes to
stable storage.
%
%%  File systems generate \chdescs\ for all writes, then
%% send them to block devices for eventual commit. Explicit dependencies between
%% \chdescs\ let \Kudos\ \modules\ preserve necessary file system
%% invariants without understanding the file system itself. \Chdescs\ can
%% implement many consistency mechanisms, including soft updates and journaling.
%
\Kudos\ is decomposed into fine-grained \modules\ which generate, consume,
 forward, and manipulate \chdescs.
%
The uniform abstraction of \chdescs\ allows modules to impose and
 follow arbitrary file system consistency policies: a collection of
 loosely-coupled modules cooperates to implement strong and possibly
 complex guarantees, even though each individual module does a relatively
 small part of the work.
%
%% A particular innovation of the
%% \module\ design is the separation of the low-level specification of on-disk
%% layout from higher-level file system-independent code, which operates on
%% abstract disk structures. 
%
For example, by observing and modifying \chdesc\ constraints, our
 journaling \module\ can automatically add journaling to any file system.
%
Additionally, a new system call interface gives applications some direct
 control over \chdescs. We have used this interface to
improve the UW IMAP server, removing inefficient and unnecessary calls to
\texttt{fsync()} while preserving the integrity of mail messages.
%
We have implemented \Kudos\ as a Linux kernel module. Our current
implementation is competitive with FreeBSD soft updates for number of
blocks written, and allows several novel configurations like ext2 with
soft updates or correct UFS soft updates over a loopback device.

\end{comment}

\end{abstract}
