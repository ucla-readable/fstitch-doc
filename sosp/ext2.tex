\subsection{ext2}
\label{sec:using:ext2}

\Kudos\ currently has implementations of three base file system types: a simple
file system without inodes called josfs, Linux ext2, and 4.2 BSD UFS. File
system \modules\ in \Kudos\ are supposed to generate soft updates dependencies
by default; other dependency arrangements are achieved by transforming these.
We briefly describe ext2 as an example of a complete and relatively complex file
system.

The ext2 \module\ is implemented at the \LFS\ interface. This keeps properties
specific to ext2 (such as the ext2 on-disk format and rules governing block
allocation) hidden within the file system \module. The ext2 \module\ creates
\chdescs\ for all its changes to the disk, and connects them to form
configurations that achieve soft updates consistency. To the best of our
knowledge, this is the first implementation of ext2 to provide soft updates
consistency guarantees. Unlike FreeBSD's soft updates implementation, once the
dependencies have been hooked up, the ext2 \module\ no longer needs to concern
itself with the \chdescs\ it has created. The block device subsystem will track
and enforce the \chdesc\ orderings.

In order to implement soft updates ordering, each \LFS\ call to the ext2
\module\ must connect the \chdescs\ it creates according to the rules for soft
updates~\cite{ganger00soft}. The UHFS \module\ (\S\ref{sec:modules:uhfs})
connects the resulting \chdesc\ subgraphs together in soft updates order,
completing the implementation of soft updates using \chdescs.
