% -*- mode: latex; tex-main-file: "paper.tex" -*-

\subsection{Other Optimizations}
\label{sec:patch:misc}

Optimizations can only do so much
with bad dependencies.
%
Just as having too few dependencies can compromise system correctness,
having too many dependencies, or the wrong dependencies, can non-trivially
degrade system performance.
%
For example, in both the following \patch\ arrangements, $s$ depends on all
of $r$, $q$, and $p$, but the left-hand arrangement gives the system more
freedom to reorder block writes:

\begin{figure}[htb]
\vskip-.4\baselineskip
\centering
\begin{tabular}{@{}p{.3\hsize}p{.3\hsize}@{}}
\centering \includegraphics[scale=.62]{fig/figures_4} &
\centering \includegraphics[scale=.62]{fig/figures_6}
\end{tabular}
\vskip-1.2\baselineskip
\end{figure}

\noindent%
If $r$, $q$, and $p$ are adjacent on disk, the left-hand
arrangement can be satisfied with two disk requests while the right-hand
one will require four.
%
Although neither arrangement is much harder to code, in several cases
we discovered that one of our file system implementations was performing
slowly because it created an arrangement like the one on the right.

Care must be taken to avoid unnecessary \emph{implicit} dependencies,
and in particular overlap dependencies.
%
For instance, inode blocks contain multiple inodes, and changes to two
inodes should generally be independent; a similar statement holds for
directories.
%% and even sometimes for different fields in a summary block like
%% the superblock.
%
Patches that change one independent field at a time generally give
the best results.
%%  can be obtained from \emph{minimal} \patches\ that change
%% one independent field at a time. 
\Kudos\ will merge these \patches\ when
appropriate, but if they cannot be merged, minimal \patches\ tend to
cause fewer \patch\ reversions and give more flexibility in write
ordering.

File system implementations can generate better dependency arrangements
when they can detect that certain states will never appear on disk.
%
For example, soft updates requires that clearing an inode
depend on nullifications of all corresponding directory entries, which
normally induces dependencies from the inode onto the directory
entries.
%
However, if a directory entry will never exist on disk---for example,
because a patch to remove the entry merged with the patch that created
it---then there is no need to require the corresponding dependency.
%
Similarly, if \emph{all} a file's directory entries will never exist on
disk, the patches that free the file's blocks need not depend on the
directory entries.
%
Leaving out these dependencies can speed up the system by avoiding
block-level cycles, such as those in Figure~\ref{f:soft2hard}, and the
rollbacks and double writes they cause.
%
\begin{comment}
A dependency from the inode clear onto the directory entry clear is
sufficient to ensure this property. However, when a directory entry
will never exist on disk because it is created and removed before the
creation is committed, the inode clear need not depend on the
directory entry clear.
\end{comment}
%
The \Kudos\ ext2 module implements these optimizations,
%%  and another, related
%% one: if a directory entry will never reference an inode, then the patches
%% that clear the corresponding inode bitmap and block bitmap entries need not
%% depend on the patch that deleted that directory entry.
%
significantly reducing disk writes, patch allocations, and undo data required
when files are created and deleted within a short time.
%
Although the optimizations are file system specific, the file system
implements them using general properties, namely, whether two patches
successfully merge.

%% the changes to update a field in an inode structure
%% on disk, a \patch\ spanning the entire inode could be created even though
%% only a single field changed. A later change to a different field in the
%% same inode would appear to overlap the first, possibly creating an
%% unnecessary dependency. Creating \patches\ that correspond more precisely
%% to the changes being made avoids this problem, so a utility function is
%% provided by \Kudos\ to make this operation as convenient as creating a
%% single \patch\ for an entire large structure.

%% \subsubsection{\Patch\ List Ordering}

\begin{comment}
%% Declared uninteresting.
The buffer cache and a few other \modules\ perform better in the
common case that each block's \patches\ are listed in order of creation
time,
%
taking $O(n)$ time to process $n$ patches rather than $O(n^2)$.
%% in the common case that this order is preserved.
\end{comment}

Finally, block allocation policies can have a dramatic effect on the number of
I/O requests required to write changes to the disk. For instance, soft
updates-like dependencies require that data blocks be initialized before
an indirect block references them.   
Allocating an
indirect block in the middle of a range of file data blocks
forces the data blocks to be
written as two smaller I/O requests, since the indirect block
cannot be written at the same time. Allocating the indirect block somewhere
else allows the data blocks to be written in one larger I/O request, at the
cost of (depending on readahead policies) a potential slowdown in read performance.



We often found it useful to examine \patch\ dependency graphs visually.
%% (looking at created \patches\ and the \patch's
\Kudos\ optionally logs \patch\ operations to disk;
a separate debugger inspects and generates graphs from these
logs.
%
Although the graphs could be daunting, they provided some evidence that
patches work as we had hoped: performance problems could
be analyzed by examining general dependency structures, and
sometimes easily fixed by manipulating those structures.


\begin{comment}
Several functions in \Kudos\ iterate over lists of \patches\ looking for either
a single \patch\ or set of \patches\ satisfying some property, or trying to
process all the \patches\ in the list in some order determined by the dependency
graph. It is generally the case that the \patches\ satisfying the property or
the order in which the \patches\ should be processed can be determined very
quickly by keeping the lists sorted. For instance, the library function which
reverts \patches\ needs to perform the revert operations essentially in inverse
creation order, so that reverting a \patch\ which has since been overwritten
by a later \patch\ does the right thing. Keeping the list of all \patches\ on a
block sorted in creation order (which is very easy) makes this an efficient
operation, while it might otherwise take $O(n^2)$ time to execute. Similarly,
many \patch\ merging functions need to find for a given block some \patch\
which has no \befores\ on the same block, and the oldest \patch\ on a block
always satisfies this requirement.
\end{comment}
