\subsection{Miscellaneous}
\label{sec:patch:misc}

\subsubsection{Unnecessary \ChDesc\ Dependencies}
Even with \Kudos\ dynamically optimizing \chdesc\ graphs, there is only so much
that can be done while preserving the semantics of the dependencies specified by
the \module\ which created the \chdescs. It is obvious that having too few
dependencies compromises the correctness of the system; it is perhaps less
obvious but no less true that having too many dependencies can nontrivially
degrade the performance of the system.

In Figure~\ref{fig:chdescarrange}, we have two possible arrangements for three
\chdescs. The \noop\ \chdesc\ represents a root node that can reach all
other \chdescs. In the parallel arrangement on the right, \Kudos\ has the
freedom to write \chdescs\ $C_1$, $C_2$, and $C_3$ to disk in any order. In the
serial arrangement on the left, there exists only one valid write ordering.
Depending on the arrangement of other \chdescs, \Kudos\ may have to perform
additional rollbacks and write some blocks more times than would otherwise be
necessary in order to write these \chdescs\ to disk. Even if that is not
necessary, \Kudos\ will still have less flexibility in choosing an order to
write the blocks, potentially increasing disk seek times due to suboptimal
ordering.

% {{{ fig:chdescarrange
\begin{figure}[htb]
  \centering
  \includegraphics[width=192pt]{fig/figures_5}
  \caption{\label{fig:chdescarrange} \Chdesc\ dependencies, when
  not strictly needed, restrict the possible choices for write ordering.
  This results in suboptimal write ordering and more scans through the
  \chdescs\ for \Kudos. On the right, \chdescs\ C1, C2, and C3 can be written
  in any order. Only one ordering is possible on the left.}
\end{figure}
% }}}

Even when no unnecessary dependencies are explicitly created, they can still be
created implicitly and care must be taken to avoid them. When \chdescs\ overlap,
the later \chdesc\ is made to depend on the first so that it cannot precede the
change it is changing further to disk. As a result, however, unnecessary
dependencies may result if the \chdescs\ are larger than the data they are
actually changing. In Figure~\ref{fig:overlap}, to update a field in an inode
structure on disk, a \chdesc\ spanning the entire inode could be created even
though only a single field changed. A later change to a different field would
appear to overlap the first, creating an unnecessary dependency. Creating
\chdescs\ which correspond more precisely to the changes being made avoids this
problem, so a utility function is provided by \Kudos\ to make this operation as
convenient as creating a single \chdesc\ for an entire large structure.

% {{{ fig:overlap
\begin{figure}[htb]
  \centering
  \includegraphics[width=64pt]{fig/figures_4}
  \caption{\label{fig:overlap} On inode 17, the gray regions represent
  modified fields that do not overlap. If \chdesc\ A and \chdesc\ B are
  exactly the size of the gray regions, then there is no implicit dependency.
  However, making \chdescs\ for the entire inode data structure will make
  one \chdesc\ depend on the other because they overlap.}
\end{figure}
% }}}

\subsubsection{\ChDesc\ List Ordering}
Several functions in \Kudos\ iterate over lists of \chdescs\ looking for either
a single \chdesc\ or set of \chdescs\ satisfying some property, or trying to
process all the \chdescs\ in the list in some order determined by the dependency
graph. It is generally the case that the \chdescs\ satisfying the property or
the order in which the \chdescs\ should be processed can be determined very
quickly by keeping the lists sorted. For instance, the library function which
rolls \chdescs\ back needs to perform the rollbacks essentially in inverse
creation order, so that rolling back a \chdesc\ which has since been overwritten
by a later \chdesc\ does the right thing. Keeping the list of all \chdescs\ on a
block sorted in creation order (which is very easy) makes this an efficient
operation, while it might otherwise take $O(n^2)$ time to execute. Similarly,
many \chdesc\ merging functions need to find for a given block some \chdesc\
which has no \befores\ on the same block, and the oldest \chdesc\ on a block
always satisfies this requirement.
