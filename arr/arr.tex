\documentclass[12pt]{article}
\usepackage{geometry}
\usepackage{times}

\usepackage{url,hyperref,ifpdf}
\ifpdf
  \usepackage[pdftex]{graphicx}
  \usepackage[monochrome]{color}
  \DeclareGraphicsExtensions{.jpg,.pdf,.mps,.png}
  \pdfinfo
  { 
    /Title (The KudOS Architecture for File Systems)
    /Author (Andrew de los Reyes, Chris Frost, Mike Mammarella, Lei Zhang, {adlr,frost,mikem,leiz}@cs.ucla.edu)
  }
  \pdfcompresslevel=9
\else
  \usepackage{graphicx}
  \DeclareGraphicsExtensions{.eps,.jpg,.mps,.png}
\fi

% A list of stuff not to hyphenate
\hyphenation{KudOS}

\geometry{top=1in,bottom=1in,left=1in,right=1in}

\pagestyle{empty}

% reduce pre-pargraph spacing in a hacky way
\newcommand{\preparagraphspacing}{\vspace{-0.2in}}

\begin{document}

\noindent\textbf{THE KUDOS ARCHITECTURE FOR FILE SYSTEMS}

\noindent{Andrew de los Reyes, Chris Frost, Eddie Kohler, Mike Mammarella, and Lei Zhang}

\noindent \{adlr,frost,kohler,mikem,leiz\}@cs.ucla.edu

\noindent{Computer Science Department, University of California, Los Angeles}

\preparagraphspacing
\paragraph{Introduction:}

Our research aims to make file systems support more flexible and efficient ways
for applications to implement custom stable storage consistency policies:
guaranteed orderings of disk writes containing the application's data.
Additionally, we are working to make file system software itself more modular,
adaptable, and extensible.

\preparagraphspacing
\paragraph{Methods and Results:}

In the KudOS system, \emph{change descriptor} structures represent any and all
changes to stable storage. They are inspired by the soft updates system in BSD's
FFS, but generalized so that they are not specific to any particular file
system. Each change descriptor stores the old state of a disk block and the
change's \emph{dependencies} -- other change descriptors that must be committed
before it is safe to commit this change. The file system software is decomposed
into many fine-grained modules which generate, consume, forward, and manipulate
change descriptors. Change descriptors can be arranged to implement many
consistency policies, including soft updates and journaling: merely by
rearranging the change descriptors, our journaling module can automatically add
journalling to any file system.

The ability to work with change descriptors is also extended outside the file
system software in the kernel, so that applications can set up their own
user-defined dependencies -- allowing them to define consistency protocols for
the file system to follow. In contrast, existing systems offer little help to
applications with custom consistency and performance requirements, imposing
either high overhead (data journaling) or not guaranteeing data consistency
(soft updates, for example, ensures metadata consistency only).

By using user-defined change descriptors, applications can specify minimal and
precise data dependencies for order-sensitive data. This could improve the
performance and correctness of applications requiring specific consistency
semantics. For instance, a mail server could ensure that a message is not lost
when it is moved from one folder to another by creating a change descriptor
dependency that tells the file system not to remove the message from its
original folder until it has been added to the new one.

%\begin{figure}[htb]
%\begin{center}
%\includegraphics[width=5in]{blah}
%\caption{Example ridiculously large change descriptor graph}
%\label{fig}
%\end{center}
%\end{figure}

KudOS modules can be relatively independent from one another, because change
descriptors help to isolate the logic of all the modules, while still allowing
them to communicate important data ordering constraints. For instance, the
journaling module's ordering requirements can be set up without understanding
the file system that is being journaled, since the file system's changes to the
disk are represented as change descriptors. The resulting modified change
descriptors could then be sent through a loop device, allowing dependency
information to be maintained across a boundary that would otherwise lose that
information. 

\preparagraphspacing
\paragraph{Conclusion:}

... Put some introductory conclusion sentences here. ...

Extending change descriptors into userspace allows very flexible management of
disk write orderings by applications, which we hope will allow simpler and more
efficient application-level consistency policies. There are many issues to
consider when allowing applications this kind of control, however. For instance,
we must prevent misbehaving or malicious applications from creating excessive
numbers of change descriptors, or setting up dependencies such that certain
change descriptors stay in the system indefinitely.

\end{document}
